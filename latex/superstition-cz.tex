\documentclass{book}
\usepackage{microtype}
\usepackage[a5paper]{geometry}
\usepackage[hidelinks]{hyperref}
\usepackage{csquotes}
\usepackage[czech]{babel}
\usepackage[T1]{fontenc}

\title{Nejnebezpečnější pověra}
\author{Larken Rose}
\date{2012}

\begin{document}

\maketitle

\vspace*{\fill}

\begin{center}
  \emph{Nejnebezpečnější pověra}, druhá edice \\
  Copyright 2012, Larken Rose \\
  ISBN 978-1-62407-169-0 (originální tištěná edice)
\end{center}

\vspace{\fill}

\begin{center}
  \textbf{Poznámka o copyrightu} \\
  \emph{\enquote{Copyright} je obvykle implicitní hrozbou (\enquote{Nekopíruj to, jinak!}). Doufám, že každý, komu se tato kniha líbí, si ode mne koupí další kopie, ale pokud ji někdo zkopíruje bez mého svolení, jen proto se necítím oprávněn použít proti takové osobě sílu, ať už sám, nebo prostřednictvím \enquote{státu.} Kdyby někdo začal prodávat \enquote{falešné} kopie, to už je jiný příběh. Knihu jsem si ale opatřil autorskými právy především proto, aby si ji nemohl opatřit někdo jiný a tím mi pomocí státního násilí zabránit v jejím šíření.} \\
  -- Larken Rose
\end{center}

\tableofcontents

\begin{center}
  \emph{Tato kniha a ostatní díla Larkena Rose jsou dostupné na} www.LarkenRose.com
\end{center}

\newpage

\vspace*{\fill}

\begin{center}
  \textbf{Věnování} \\
  \emph{Tato kniha je věnována dvěma lidem: prvnímu člověku, který kvůli přečtení této knihy neuposlechne příkazu, aby někomu ublížil, a člověku, kterému se v důsledku toho nic nestane.}
\end{center}

\vspace*{\fill}

\newpage

\section{Než začnete číst}

To, co se dočtete v této knize, bude s největší pravděpodobností v přímém rozporu s tím, co vás učili vaši rodiče a učitelé, co vám říkaly církve, média a vláda, a s většinou toho, čemu jste vy, vaše rodina a vaši přátelé vždy věřili. Přesto je to pravda, jak uvidíte, pokud si dovolíte o této otázce objektivně uvažovat. Nejenže je to pravda, ale možná je to také ta nejdůležitější pravda, kterou kdy uslyšíte.

Stále více lidí tuto pravdu objevuje, ale k tomu je třeba překonat mnohé předsudky a hluboce zakořeněné pověry, odložit celoživotní indoktrinaci a poctivě a čestně prozkoumat některé nové myšlenky. Pokud to uděláte, zažijete dramatickou změnu ve svém pohledu na svět. Zpočátku se budete téměř jistě cítit nepříjemně, ale z dlouhodobého hlediska se vám to vyplatí. A pokud se dostatečný počet lidí rozhodne tuto pravdu vidět a přijmout ji, nejenže to drasticky změní způsob, jakým tito lidé vidí svět; drasticky to změní i svět samotný, a to k lepšímu.

Ale kdyby taková jednoduchá pravda mohla změnit svět, nevěděli bychom o ní všichni a neprosadili bychom ji už dávno? Pokud by lidé byli čistě rasou myslících, objektivních bytostí, pak ano. Ale historie ukazuje, že většina lidských bytostí raději doslova zemře, než aby objektivně přehodnotila systémy víry, v nichž byla vychována. Průměrný člověk, který si přečte v novinách o válce, útlaku a nespravedlnosti, se bude divit, proč taková bolest a utrpení existují, a bude si přát, aby skončily. Pokud mu však někdo naznačí, že k tomuto utrpení přispívají jeho \emph{vlastní} názory, téměř jistě takový návrh bez rozmýšlení odmítne a možná dokonce napadne toho, kdo takový návrh vyslovil.

Takže, čtenáři, pokud pro tebe tvá víra a pověry -- z nichž mnohé sis sám nevybral, ale pouze zdědil jako nezpochybnitelné \enquote{obnošené} přesvědčení -- mají větší význam než pravda a spravedlnost, pak prosím přestaň číst a dej tuto knihu někomu jinému. Pokud jsi naopak ochotný zpochybnit některé ze svých dlouho zažitých, předpojatých názorů, pokud to může snížit utrpení druhých, pak si tuto knihu přečtěte. A \emph{pak} ji dej někomu jinému.

\chapter{Nejnebezpečnější pověra}

\section{Počínaje pointou}

Kolik milionů lidí se dívalo na kruté hrůzy historie s nesčetnými příklady nelidského zacházení člověka s člověkem a nahlas přemýšlelo, jak se něco takového mohlo stát? Pravdou je, že většina lidí by nechtěla vědět, jak se to děje, protože sami nábožensky lpí právě na víře, která to umožňuje. Drtivou většinu utrpení a nespravedlnosti na světě, a to jak v současnosti, tak i tisíce let nazpět, lze přímo přičíst jediné \emph{myšlence}. Není to chamtivost, nenávist ani žádná z dalších emocí či idejí, které se obvykle viní ze zla ve společnosti. Většina násilí, krádeží, přepadení a vražd ve světě je naopak důsledkem pouhé \emph{pověry} -- víry, která, ačkoli je téměř všeobecně rozšířená, je v rozporu se všemi důkazy a rozumem (ačkoli ti, kdo tuto víru zastávají, to tak samozřejmě nevidí). \enquote{Pointa} této knihy je snadno vyjádřitelná, i když pro většinu lidí je obtížné ji přijmout, nebo dokonce o ní klidně a racionálně uvažovat:

\textbf{Víra v autoritu, která zahrnuje veškerou víru ve stát, je iracionální a vnitřně rozporná; odporuje civilizaci a morálce a představuje nejnebezpečnější a nejničivější pověru, jaká kdy existovala. Víra v autoritu je spíše úhlavním nepřítelem lidstva, než aby byla hybnou silou řádu a spravedlnosti.}

Téměř všichni jsou ovšem vychováváni k přesnému opaku: že poslouchání \enquote{autority} je ctnost (alespoň ve většině případů), že respektování a dodržování \enquote{státních zákonů} nás činí civilizovanými a že nerespektování \enquote{autority} vede pouze k chaosu a násilí. Lidé jsou vlastně tak důkladně vycvičeni, aby si poslušnost spojovali s \enquote{dobrem,} že napadání konceptu \enquote{autority} bude pro většinu lidí znít jako naznačování, že neexistuje nic takového jako dobro a zlo, že není třeba dodržovat žádné normy chování, že není třeba mít vůbec žádnou morálku. To však \emph{není} to, co je zde obhajováno -- spíše naopak.

Mýtus autority je totiž třeba zbořit právě proto, že \emph{existuje} něco jako dobro a zlo, že \emph{záleží} na tom, jak se k sobě lidé chovají, a že by se lidé \emph{měli} vždy snažit žít morálně. Navzdory neustálé autoritářské propagandě, která tvrdí opak, se úcta k \enquote{autoritě} a úcta k lidem navzájem vylučují a diametrálně liší. Důvodem, proč nemít úctu k mýtu autority, je to, abychom \emph{mohli} mít úctu k lidskosti a spravedlnosti.

Je zde ostrý kontrast mezi tím, co nás učí jako účel \enquote{autority} (vytvořit mírumilovnou, civilizovanou společnost), a skutečnými výsledky \enquote{autority} v praxi. Prolistujte si jakoukoli učebnici dějepisu a zjistíte, že většina nespravedlností a ničení, k nimž došlo na celém světě, nebyla výsledkem toho, že lidé \enquote{porušili zákon,} ale spíše výsledkem toho, že lidé \emph{poslouchali} a \emph{vynucovali} \enquote{zákony} různých \enquote{států.} Zlo, které bylo spácháno navzdory \enquote{autoritě,} je zanedbatelné ve srovnání se zlem, které bylo spácháno \emph{ve jménu} \enquote{autority.}

Přesto se děti stále učí, že mír a spravedlnost jsou výsledkem autoritářské vlády a že navzdory zjevnému zlu, kterého se autoritářské režimy na celém světě v historii dopustily, jsou stále morálně povinny respektovat a poslouchat současnou \enquote{vládu} své země. Učí je, že \enquote{dělat, co se ti řekne,} je synonymem dobrého člověka a že \enquote{hrát podle pravidel} je synonymem správného jednání. Naopak, být morálním člověkem vyžaduje převzít osobní odpovědnost za posuzování dobra a zla a řídit se vlastním svědomím, což je opakem respektování a poslouchání \enquote{autorit.}

Důvod, proč je tak důležité, aby lidé tuto skutečnost pochopili, je ten, že hlavní nebezpečí, které představuje mýtus autority, se \emph{nenachází} v myslích \enquote{státních} vládců, ale v myslích těch, kteří \emph{jsou} ovládáni. Jeden zlý jedinec, který rád vládne ostatním, je pro lidstvo triviální hrozbou, pokud spousta dalších lidí nepovažuje takovou nadvládu za legitimní, protože je jí dosaženo prostřednictvím \enquote{státních zákonů.} Zvrácená mysl Adolfa Hitlera sama o sobě nepředstavovala pro lidstvo téměř žádnou hrozbu. Byly to miliony lidí, kteří Hitlera vnímali jako \enquote{autoritu,} a cítili se proto povinni poslouchat jeho příkazy a plnit jeho nařízení, kdo ve skutečnosti způsobil škody napáchané Třetí říší. Jinými slovy, problém není v tom, že \emph{zlí} lidé věří v autoritu; problém je v tom, že v podstatě \emph{dobří} lidé věří v autoritu, a v důsledku toho nakonec obhajují a dokonce páchají násilné, nespravedlivé a utlačující činy a dokonce i vraždy.

Průměrný etatista (ten, kdo věří ve stát) bude sice naříkat nad všemi způsoby, jimiž byla \enquote{autorita} použita jako nástroj zla, a to i v jeho vlastní zemi, ale stále bude trvat na tom, že je možné, aby \enquote{stát} byl silou dobra, a stále si bude představovat, že \enquote{autorita} může a musí zajistit cestu k míru a spravedlnosti.

Lidé se mylně domnívají, že mnoho užitečných a legitimních věcí, které přinášejí lidské společnosti prospěch, vyžaduje existenci \enquote{státu.} Je například dobré, aby se lidé organizovali za účelem vzájemné obrany, aby spolupracovali na dosažení společných cílů, aby hledali způsoby, jak spolupracovat a vycházet spolu v míru, aby vymýšleli dohody a plány, které lidem lépe umožní existovat a prosperovat ve vzájemně výhodném a nenásilném stavu civilizace. Ale to \emph{není} to, co je \enquote{stát.} Navzdory tomu, že \enquote{státy} vždy tvrdí, že jednají ve prospěch lidí a společného dobra, pravdou je, že \enquote{stát} je ze své podstaty vždy v přímém rozporu se zájmy lidstva. \enquote{Autorita} není ušlechtilá myšlenka, která se někdy mýlí, ani v zásadě platný koncept, který je někdy zkažený. Od shora dolů, od začátku do konce, je samotný \emph{koncept} \enquote{autority} protilidský a strašlivě destruktivní.

Většina lidí samozřejmě takové tvrzení těžko spolkne. Není snad stát nezbytnou součástí lidské společnosti? Není to mechanismus, díky němuž je civilizace možná, protože nás, nedokonalé lidi, nutí chovat se spořádaně a mírumilovně? Není právě uzákonění společných pravidel a zákonů tím, co nám umožňuje spolu vycházet, civilizovaně řešit spory a spravedlivě a nenásilně obchodovat a jinak komunikovat? Neslyšeli jsme snad vždy, že nebýt \enquote{právního státu} a společného respektu k \enquote{autoritám,} nebyli bychom o nic lepší než banda hloupých, násilnických bestií, žijících ve stavu věčného konfliktu a chaosu?

Ano, to nám bylo řečeno. A ne, nic z toho není pravda. Ale snažit se vymanit z věčných lží, snažit se vydestilovat pravdu z džungle hluboce zakořeněných nepravd může být nesmírně obtížné, nemluvě o tom, že je to nepříjemné.

\section{Přehled}

Na následujících stránkách čtenář projde několika etapami, aby plně pochopil, proč je víra v autoritu skutečně nejnebezpečnější pověrou v dějinách světa. Nejprve bude pojem \enquote{autority} destilován na jeho nejzákladnější podstatu, aby mohl být definován a objektivně zkoumán.

Ve druhé části se ukáže, že \emph{koncept sám o sobě} je fatálně chybný, že základní předpoklad veškerého \enquote{státu} je naprosto neslučitelný s logikou a morálkou. Ve skutečnosti se ukáže, že \enquote{stát} je čistě \emph{náboženská} víra -- na víře založené přijetí nadlidské, mytologické entity, která nikdy neexistovala a existovat nebude. (Od čtenáře se neočekává, že přijme takové překvapivé tvrzení bez dostatečných důkazů a rozumného zdůvodnění, které budou poskytnuty.)

Ve třetí části bude ukázáno, proč je víra v autoritu, včetně veškeré víry ve stát, strašlivě nebezpečná a destruktivní. Konkrétně bude ukázáno, jak víra v autoritu dramaticky ovlivňuje jak \emph{vnímání}, tak \emph{jednání} různých kategorií lidí, což vede doslova miliardy jinak dobrých a mírumilovných lidí k tomu, že schvalují nebo páchají násilné a nemorální agresivní činy. Ve skutečnosti to dělá každý, kdo věří ve stát, ačkoli si to naprostá většina neuvědomuje a vehementně by to popírala.

Ve čtvrté části se čtenář konečně dozví, jak by mohl vypadat život \emph{bez} víry v autoritu. V rozporu s obvyklým předpokladem, že absence \enquote{státu} by znamenala chaos a zkázu, se ukáže, že po opuštění mýtu autority se mnohé změní, ale mnohé také zůstane stejné. Ukáže se, proč spíše než víra ve stát, která je příznivá a nezbytná pro mírovou společnost, jak se téměř všichni učili, je tato víra zdaleka největší \emph{překážkou} vzájemně prospěšné organizace, spolupráce a mírového soužití. Stručně řečeno, ukáže se, proč skutečná civilizace může existovat a bude existovat až poté, co bude vymýcen mýtus autority.

\section{Identifikace nepřítele}

Od útlého dětství se učíme podřizovat se vůli \enquote{autorit,} poslouchat nařízení těch, kteří tak či onak získali moc a vládu. Od počátku je dítě hodnoceno, ať už explicitně, nebo implicitně, nejprve podle toho, jak dobře poslouchá své rodiče, pak podle toho, jak dobře poslouchá své učitele, a nakonec podle toho, jak dobře poslouchá \enquote{státní zákony.} Ať už implicitně, nebo explicitně, společnost je prosycena myšlenkou, že poslušnost je ctnost a že dobří lidé jsou ti, kteří dělají to, co jim \enquote{autorita} nařídí. V důsledku této myšlenky se pojmy morálka a poslušnost v myslích většiny lidí natolik zamotaly, že jakýkoli útok na pojem \enquote{autorita} bude většině lidí připadat jako útok na morálku samotnou. Jakýkoli návrh, že \enquote{stát} je ze své podstaty nelegitimní, bude znít jako návrh, že by se všichni měli chovat jako bezcitná, krutá zvířata, která žijí podle práva silnějšího.

Potíž je v tom, že systém víry průměrného člověka se opírá o změť nejasných, často protichůdných pojmů a předpokladů. Pojmy jako morálka a zločin, právo a legislativa, vládci a občané často používají lidé, kteří tyto pojmy nikdy racionálně nezkoumali. Abychom pochopili podstatu \enquote{autority} a \enquote{státu,} musíme začít přesným vymezením toho, co tyto pojmy znamenají.

Co je to za věc, které říkáme \enquote{stát?} Je to organizace, která lidem říká, co mají dělat. Ale to samo o sobě není úplná definice, protože různé jiné osoby a organizace, které nenazýváme \enquote{státem,} také říkají lidem, co mají dělat. \enquote{Stát} však pouze nenavrhuje a nežádá, ale přikazuje. Na druhou stranu by se dalo říci, že příkazy vydávají také inzerenti a kazatelé, ale ti se za \enquote{stát} nepovažují. Na rozdíl od \enquote{příkazů} kazatelů a inzerentů jsou příkazy \enquote{státu} vymáhány pod hrozbou trestu, použitím síly proti těm, kteří se nepodřídí. Ale ani to nám neposkytuje úplnou definici, protože pouliční násilníci a tyrani také prosazují své příkazy, ale nikdo je nenazývá \enquote{státem.}

Charakteristickým rysem \enquote{státu} je to, že se má za to, že má morální \emph{právo} vydávat a vynucovat příkazy. Jeho příkazy se nazývají \enquote{zákony} a neuposlechnutí jeho příkazů se nazývá \enquote{zločin.} Stručně řečeno, určujícím faktorem, který z něčeho dělá \enquote{stát,} je vnímaná legitimita a správnost moci a vlády, kterou má nad ostatními -- jinými slovy jeho \enquote{autorita.}

\enquote{Autoritu} lze shrnout jako \textbf{právo vládnout}. Není to jen \emph{schopnost} násilně ovládat druhé, kterou má do určité míry téměř každý. Je to údajné morální \emph{právo} násilně ovládat druhé. Pouliční gang se od \enquote{státu} liší tím, jak je vnímán lidmi, které ovládá. Nepovolená vniknutí, loupeže, vydírání, přepadení a vraždy páchané obyčejnými násilníky vnímá téměř každý jako nemorální, neoprávněné a zločinné. Jejich oběti sice jejich požadavkům vyhoví, ale nikoli z pocitu morální povinnosti uposlechnout, pouze ze strachu. Pokud by se zamýšlené oběti pouličního gangu domnívaly, že se mohou postavit na odpor, aniž by jim hrozilo nebezpečí, učinily by tak bez sebemenšího pocitu viny. Nevnímají pouličního násilníka jako nějakého legitimního, právoplatného vládce, nepředstavují si ho jako \enquote{autoritu.} Kořist, kterou násilník vybírá, nenazývají \enquote{daněmi} a jeho hrozby nenazývají \enquote{zákony.}

Naproti tomu, příkazy vydávané těmi, kdo nosí nálepku \enquote{státu,} většina těch, na něž jsou jejich příkazy zaměřeny, vnímá zcela jinak. Většina lidí vnímá moc a vládu, kterou \enquote{státní zákonodárci} vykonávají nad všemi ostatními, jako platnou a legitimní, \enquote{legální} a dobrou. Většina těch, kdo se těmto příkazům podřizují tím, že \enquote{dodržují zákony,} a odevzdávají své peníze tím, že \enquote{platí daně,} tak nečiní pouze ze strachu z trestu v případě neuposlechnutí, ale také z pocitu \emph{povinnosti} poslouchat. Nikdo není hrdý na to, že ho okradl pouliční gang, ale mnozí nosí nálepku \enquote{zákona dbalého daňového poplatníka} jako čestný odznak. Je to dáno výhradně tím, jak poddaní vnímají ty, kteří jim dávají příkazy. Pokud jsou vládnoucí vnímáni jako autorita, pak jsou z definice vnímáni jako ti, kteří mají morální právo takové příkazy vydávat, což následně implikuje morální povinnost lidí tyto příkazy poslouchat. Označit se za \enquote{zákona dbalého daňového poplatníka} znamená \emph{chlubit se} svou loajální poslušností vůči \enquote{státu.}

V minulosti si některé církve nárokovaly právo trestat kacíře a jiné hříšníky, ale v dnešním západním světě je pojem \enquote{autority} téměř vždy spojen s vládou. Ve skutečnosti v dnešní době se oba z těchto pojmů navzájem implikují: \enquote{Autorita} údajně vychází z nařízení (\enquote{zákonů}) \enquote{státu} a \enquote{stát} je organizace, která má podle představ právo vládnout, tj. \enquote{autorita.}

Je třeba rozlišovat mezi tím, zda je příkaz ospravedlněn na základě \emph{situace}, a tím, zda je ospravedlněn na základě toho, kdo příkaz vydal. V této knize se zabýváme pouze druhým typem \enquote{autority,} ačkoli se tento termín občas používá v jiném smyslu, který má tendenci toto rozlišení zamlžovat. Když například někdo tvrdí, že měl \enquote{autoritu} zastavit lupiče, aby vrátil kabelku staré paní, nebo říká, že měl \enquote{autoritu} vyhnat vetřelce ze svého pozemku, netvrdí, že má nějaká zvláštní práva, která jiní nemají. Říká pouze, že se domnívá, že určité \emph{situace} ospravedlňují vydání rozkazu nebo použití síly.

Naproti tomu koncept \enquote{státu} je o tom, že \emph{někteří lidé} mají zvláštní právo vládnout. A tato představa, představa, že někteří lidé -- například v důsledku \enquote{voleb} nebo jiných politických rituálů -- mají morální právo ovládat ostatní v situacích, kdy by většina lidí neměla, je konceptem, o němž se zde hovoří. Má se za to, že pouze ti, kdo řídí \enquote{stát,} mají právo vydávat \enquote{zákony;} má se za to, že pouze oni mají právo vybírat \enquote{daně;} má se za to, že pouze oni mají právo vést války, regulovat určité záležitosti, udělovat licence k různým činnostem atd. Když se v této knize hovoří o \enquote{víře v autoritu,} má se na mysli právě tento význam: myšlenka, že někteří lidé mají morální právo násilně ovládat druhé, a že v důsledku toho mají tito druzí morální povinnost je poslouchat.

Je třeba zdůraznit, že \enquote{autorita} je vždy v očích pozorovatele. Pokud ovládaný věří, že ten, kdo ho ovládá, na to má právo, pak ovládaný vnímá ovládajícího jako autoritu. Pokud ovládaný nevnímá nadvládu jako oprávněnou, pak ovládajícího nevnímá jako autoritu, ale jednoduše jako tyrana nebo násilníka. Chapadla víry v autoritu zasahují do všech aspektů lidského života, ale společným jmenovatelem je vždy \emph{vnímaná legitimita} moci, kterou má nad druhými. Každý \enquote{zákon} a \enquote{daň,} (federální, státní i komunální) každé volby a kampaň, každá licence a povolení, každá politická debata a hnutí -- zkrátka vše, co má co do činění s vládou, od banální městské vyhlášky až po \enquote{světovou válku} -- se zcela opírá o myšlenku, že někteří lidé získali morální právo -- tím či oním způsobem, v té či oné míře -- vládnout nad ostatními.

Nejde jen o zneužití \enquote{autority} nebo o spor o \enquote{dobrým státem} a \enquote{špatným státem,} ale o zkoumání základního koncpetu \enquote{autority.} To, zda je \enquote{autorita} chápána jako absolutní, nebo zda má své podmínky či omezení, může mít vliv na to, kolik škody tato \enquote{autorita} napáchá, ale nemá to vliv na to, zda je základní koncept racionální. Například americká ústava je představována tak, že vytvořila \enquote{autoritu,} která měla, alespoň teoreticky, značně omezené právo vládnout. Přesto se stále snažila vytvořit \enquote{autoritu} s právem dělat věci (např. \enquote{zdaňovat} a \enquote{regulovat}), které průměrný občan nemá právo dělat sám. Ačkoli předstírala, že dává právo vládnout pouze v určitých specifických záležitostech, stále si nárokovala propůjčení určité \enquote{autority} vládnoucí třídě, a jako taková je terčem následující kritiky \enquote{autority} stejně, jako by byla \enquote{autorita} nejvyššího diktátora.

(Pojem \enquote{autority} se někdy používá ve smyslu, který nemá s tématem této knihy nic společného. Například člověk, který je odborníkem v nějakém oboru, je často označován jako \enquote{autorita.} Stejně tak některé vztahy připomínají \enquote{autoritu,} ale nezahrnují žádné právo vládnout. Na vztah zaměstnavatel-zaměstnanec se často pohlíží, jako by existoval \enquote{šéf} a \enquote{podřízený.} Avšak bez ohledu na to, jak moc je zaměstnavatel panovačný nebo povýšený, nemůže pracovníky povolat do služby nebo je uvěznit za neposlušnost. Jedinou pravomocí, kterou skutečně má, je pravomoc ukončit dohodu propuštěním zaměstnance. A stejnou moc má i zaměstnanec, protože může dát výpověď. Totéž platí i pro jiné vztahy, které mohou připomínat \enquote{autoritu,} například řemeslník a jeho učeň, sensei bojových umění a jeho žák nebo trenér a sportovec, kterého trénuje. Takové scénáře zahrnují ujednání založená na vzájemné, dobrovolné dohodě, v níž má každá ze stran možnost z ujednání vystoupit. Takový vztah, kdy jedna osoba dobrovolně dovolí druhé, aby řídila její činnost v naději, že bude mít prospěch ze znalostí nebo dovedností druhé osoby, není typem \enquote{autority,} který je předmětem této knihy.)

\section{Nic takového neexistuje}

Většina lidí věří, že \enquote{stát} je nezbytný, i když zároveň uznávají, že \enquote{státní moc} často vede ke korupci a zneužívání. Vědí, že \enquote{stát} může být neefektivní, nespravedlivý, nerozumný a utlačující, ale stále věří, že \enquote{autorita} může být silou dobra. Neuvědomují si, že problém nespočívá jen v tom, že \enquote{stát} přináší horší výsledky nebo že \enquote{autorita} je často zneužívána. Problém je v tom, že samotný \emph{koncept} je naprosto iracionální a vnitřně rozporný. Není to nic jiného než pověra postrádající jakoukoli logickou či důkazní oporu, kterou lidé zastávají pouze v důsledku neustálé sektářské indoktrinace, jejímž cílem je zakrýt logickou absurditu tohoto konceptu. Nezáleží na míře nebo způsobu použití; pravda je taková, že \enquote{autorita} \emph{vůbec neexistuje a nemůže existovat}, a neuznání této skutečnosti vedlo miliardy lidí k tomu, že věří věcem a dělají věci, které jsou strašlivě destruktivní. Nic takového jako dobrá \enquote{autorita} nemůže existovat -- ve skutečnosti nic takového jako \enquote{autorita} \emph{vůbec} neexistuje. Ačkoli to může znít divně, lze to snadno dokázat.

Stručně řečeno, \textbf{stát neexistuje}. Nikdy neexistoval a nikdy existovat nebude. Politici jsou skuteční, vojáci a policisté, kteří prosazují vůli politiků, jsou skuteční, budovy, které obývají, jsou skutečné, zbraně, kterými vládnou, jsou velmi skutečné, ale jejich údajná \enquote{autorita} není. A bez této \enquote{autority,} bez \emph{práva} dělat to, co dělají, nejsou ničím jiným než bandou násilníků. Pojem \enquote{stát} znamená \emph{legitimitu} -- znamená výkon \enquote{autority} nad určitým lidem nebo místem. Způsob, jakým lidé mluví o těch, kdo jsou u moci, nazývají jejich příkazy \enquote{zákony,} označují neuposlechnutí těchto příkazů za \enquote{zločin} a tak dále, implikuje právo \enquote{státu} vládnout a odpovídající povinnost jejích poddaných poslouchat. Bez \emph{práva} vládnout (\enquote{autority}) není důvod nazývat tento subjekt \enquote{státem} a všichni politici a jejich žoldáci se stávají naprosto neodlišitelnými od obřího syndikátu organizovaného zločinu, jejich \enquote{zákony} nemají o nic větší platnost než výhrůžky lupičů a zlodějů aut. A takový je ve skutečnosti každý \enquote{stát:} nelegitimní banda násilníků, zlodějů a vrahů, která se vydává za právoplatný vládnoucí orgán.

(Důvodem, proč se pojmy \enquote{stát} a \enquote{autorita} v této knize objevují v uvozovkách, je skutečnost, že nikdy neexistuje legitimní právo vládnout, takže stát a autorita vlastně nikdy neexistují. V této knize se tyto pojmy vztahují pouze na lidi a gangy, o nichž se mylně \emph{předpokládá}, že mají právo vládnout.)

Veškeré politické diskuse hlavního proudu -- všechny debaty o tom, co by mělo být \enquote{legální} a \enquote{nelegální,} kdo by měl být dosazen k moci, jaká by měla být \enquote{národní politika,} jak by měl \enquote{stát} řešit různé problémy -- to vše je naprosto iracionální a naprostá ztráta času, protože to vše vychází z falešného předpokladu, že jedna osoba může mít právo vládnout druhé, že \enquote{autorita} vůbec může existovat. Celá debata o tom, jak by se měla používat \enquote{autorita} a co by měl dělat \enquote{stát,} je přesně tak užitečná jako debata o tom, jak by měl Ježíšek zvládnout Vánoce. Je však neskonale nebezpečnější. Na druhou stranu, odstranění tohoto nebezpečí -- vlastně největší hrozby, jaké kdy lidstvo čelilo -- nevyžaduje změnu základní podstaty člověka, přeměnu veškeré nenávisti na lásku ani provedení jakékoli jiné drastické změny stavu vesmíru. Místo toho to vyžaduje pouze to, aby si lidé uvědomili a následně opustili jednu konkrétní pověru, jednu iracionální lež, které se téměř všichni naučili věřit. V jistém smyslu by se většina problémů světa vyřešila přes noc, kdyby všichni udělali něco obdobného, jako kdyby přestali věřit v Ježíška.

Jakýkoliv nápad nebo návrh řešení problému, který závisí na existenci \enquote{státu,} což zahrnuje naprosto vše, co spadá do oblasti politiky, je ze své podstaty neplatný. Použijeme-li analogii, dva lidé by mohli vést užitečnou a racionální diskusi o tom, zda je pro jejich město lepší vyrábět elektřinu z jaderné energie nebo z vodních elektráren. Pokud by však někdo navrhl, že lepší variantou by bylo vyrábět elektřinu pomocí kouzelného skřítčího prachu, jeho připomínky by byly a měly by být odmítnuty jako směšné, protože skutečné problémy nelze řešit pomocí mýtických entit. Přesto téměř všechny moderní diskuse o společenských problémech nejsou ničím jiným než sporem o to, který typ kouzelného skřítčího prachu lidstvo zachrání. Veškerá politická diskuse se opírá o nezpochybnitelný, ale falešný předpoklad, který všichni berou na vědomí jen proto, že vidí a slyší všechny ostatní opakovat tento mýtus: představu, že může existovat něco takového jako legitimní \enquote{stát.}

Problémem rozšířených mylných představ je právě to, že jsou rozšířené. Když nějaké přesvědčení -- i to nejnesmyslnější a nejnelogičtější -- zastává většina lidí, nebude věřícím připadat nerozumné. Setrvávat v této víře jim bude připadat snadné a bezpečné, zatímco zpochybňovat ji bude nepříjemné a velmi obtížné, ne-li nemožné. Dokonce ani hojné důkazy o strašlivě ničivé síle mýtu autority, na téměř nepochopitelné úrovni a táhnoucí se tisíce let zpět, nestačily k tomu, aby více než hrstka lidí byť jen začala zpochybňovat tento základní koncept. A tak se lidé, věřící, že jsou osvícení a moudří, nadále potácejí v jedné kolosální katastrofě za druhou, což je důsledek jejich neschopnosti zbavit se \emph{nejnebezpečnější pověry}: víry v autoritu.

\section{Odnože pověry}

Z konceptu \enquote{autority} vyrůstá rozsáhlý soubor terminologie. Všechny tyto termíny mají společné to, že implikují určitou legitimitu jedné skupiny lidí, která násilně ovládá jinou skupinu. Zde je jen několik příkladů:

\textbf{\enquote{Stát:}} Jak již bylo zmíněno, \enquote{stát} je prostě označení pro organizaci nebo skupinu lidí, kteří si představují, že mají právo vládnout. Mnoho dalších termínů, které popisují části \enquote{státu} (jako \enquote{prezident,} \enquote{kongresman,} \enquote{soudce} a \enquote{zákonodárce}), posiluje domnělou legitimitu vládnoucí třídy.

\textbf{\enquote{Zákon:}} Výrazy \enquote{zákon} a \enquote{právní předpis} mají zcela odlišný význam než slova \enquote{hrozba} a \enquote{příkaz.} Rozdíl opět závisí na tom, zda si ti, kdo takové \enquote{zákony} vydávají a nařizují, představují, že k tomu mají \emph{právo}. Pokud pouliční gang vydává příkazy všem ve svém okolí, nikdo takové příkazy nenazývá \enquote{zákony.} Pokud však \enquote{stát} vydává příkazy prostřednictvím \enquote{legislativního} procesu, téměř každý je nazývá \enquote{zákony.} Ve skutečnosti je každý autoritářský \enquote{zákon} příkaz podpořený hrozbou postihu vůči těm, kteří se mu nepodřídí. Ať už se jedná o \enquote{zákon} proti páchání vražd nebo proti stavbě terasy bez stavebního povolení, není to ani návrh, ani žádost, ale příkaz podpořený hrozbou násilí, ať už v podobě nucené konfiskace majetku (tj. pokuty), nebo únosu lidské bytosti (tj. uvěznění). To, co by se dalo nazvat \enquote{vydíráním,} kdyby to dělal běžný občan, se nazývá \enquote{zdaněním,} když to dělají lidé, kteří si představují, že mají právo vládnout. To, co by se za normálních okolností považovalo za obtěžování, napadení, únos a další trestné činy, je považováno za \enquote{regulaci} a \enquote{vymáhání práva,} pokud je provádějí ti, kdo tvrdí, že představují \enquote{autoritu.}

Používání pojmu \enquote{zákon} k popisu přirozených vlastností vesmíru, jako jsou \emph{zákony} fyziky a matematiky, samozřejmě nemá nic společného s principem \enquote{autority.} Kromě toho existuje ještě jeden pojem, nazývaný \enquote{přirozené právo,} který se od zákonného \enquote{práva} (tj. \enquote{legislativy}) velmi liší. Koncept přirozeného práva spočívá v tom, že existují normy dobra a zla vlastní lidstvu, které nezávisí na žádné lidské \enquote{autoritě} a které ve skutečnosti nahrazují všechny lidské \enquote{autority.} Ačkoli byl tento koncept v nepříliš vzdálené minulosti předmětem mnoha diskusí, dnes už Američany jen zdřídka uslyšíme používat pojem \enquote{právo} v tomto kontextu. Tento koncpet není tím, co je v této knize myšleno pod pojmem \enquote{právo.}

\textbf{\enquote{Zločin:}} Odvozeninou pojmu \enquote{právo} je pojem \enquote{zločin.} Výraz \enquote{spáchání trestného činu} má samozřejmě negativní konotaci. To, že pojem \enquote{porušení zákona} je morálně špatný, implikuje, že příkaz, který je neuposlechnut, je ze své podstaty legitimní, a to pouze na základě toho, kdo příkaz vydal. Pokud pouliční gang řekne majiteli obchodu: \enquote{Dej nám polovinu svého zisku, nebo ti ublížíme,} nikdo nebude považovat majitele obchodu za \enquote{zločince,} pokud se postaví proti takovému vydírání. Pokud však stejný požadavek vznesou ti, kdo nesou označení \enquote{stát,} přičemž tento požadavek se nazývá \enquote{zákon} a \enquote{daně,} pak by stejného majitele obchodu téměř každý považoval za \enquote{zločince,} pokud by se mu odmítl podřídit.

Pojmy \enquote{zločin} a \enquote{zločinec} samy o sobě ani nenaznačují, jaký \enquote{zákon} je porušován. Pomalá jízda na červenou na prázdné křižovatce je \enquote{zločin} stejně jako je \enquote{zločin} vražda souseda. Před sto lety bylo \enquote{zločinem} učit otroka číst; v Německu čtyřicátých let bylo \enquote{zločinem} ukrývat Židy před SS. V Pensylvánii je \enquote{zločinem} spát venku v nebo na lednici. Spáchat \enquote{zločin} znamená doslova neuposlechnout příkazů politiků a \enquote{zločinec} je každý, kdo tak učiní. Takové pojmy mají opět zjevně negativní konotaci. Většina lidí nechce být nazývána \enquote{zločincem} a myslí to jako urážku, pokud někoho jiného nazvou \enquote{zločincem.} Opět to předpokládá, že \enquote{orgán,} který vydává a prosazuje \enquote{zákony,} má na to \emph{právo}.

\textbf{\enquote{Zákonodárci:}} \enquote{Zákonodárci} mají zvláštní paradox, neboť se má za to, že mají právo vydávat příkazy, vybírat \enquote{daně,} regulovat chování a jinak násilně ovládat lidi, ale pouze pokud tak činí prostřednictvím \enquote{legislativního} procesu. Lidé v \enquote{zákonodárných sborech} jsou vnímáni jako ti, kdo mají právo vládnout, ale \emph{jen} tehdy, pokud svou domnělou \enquote{autoritu} uplatňují prostřednictvím určitých přijatých politických rituálů. Když tak učiní, jsou pak \enquote{zákonodárci} představováni jako osoby, které mají výlučné právo vydávat příkazy a najímat lidi k jejich prosazování -- což je právo, které nemají žádní jiní jedinci. Jinak řečeno, široká veřejnost si upřímně představuje, že morálka je pro \enquote{zákonodárce} \emph{jiná} než pro všechny ostatní. Požadování peněz pod pohrůžkou násilí je nemorální krádež, když to dělá většina lidí, ale když to dělají politici, je to vnímáno jako \enquote{zdanění.} Šéfování lidem a násilné ovládání jejich chování je považováno za obtěžování, zastrašování a napadání, když to dělá většina lidí, ale když to dělají politici, je to považováno za \enquote{regulaci} a \enquote{vymáhání práva.} Říká se jim \enquote{zákonodárci,} nikoli \enquote{příkazodárci,} protože jejich příkazy -- pokud jsou prováděny určitými \enquote{legislativními} postupy -- jsou považovány za přirozeně legitimní. Jinými slovy, jsou vnímáni jako \enquote{autorita} a poslušnost jejich legislativním příkazům je považována za morální imperativ.

\textbf{\enquote{Strážci zákona:}} Jeden z nejčastějších příkladů \enquote{autority,} se kterým se mnoho lidí denně setkává, nese označení \enquote{policie} nebo \enquote{strážci zákona.} Chování \enquote{strážců zákona} a způsob, jakým je ostatní vnímají a jak s nimi jednají, zcela jasně ukazuje, že nejsou vnímáni \emph{jen} jako lidé, ale jako zástupci entity zvané \enquote{autorita,} na kterou se podle přesvědčení vztahují velmi odlišné morální normy.

Předpokládejme například, že někdo řídil auto, aniž by byl připoután bezpečnostním pásem. Kdyby si toho všiml jiný průměrný občan, donutil řidiče zastavit a požadoval po něm vysokou částku peněz, řidič by byl pobouřen. Bylo by to považováno za vydírání, obtěžování, případně napadení a únos. Ale když někdo, kdo tvrdí, že jedná jménem \enquote{státu,} udělá přesně totéž blikajícími světly (a pronásleduje dotyčného, pokud nezastaví) a pak mu dá \enquote{pokutu,} většina lidí takové jednání považuje za naprosto legitimní.

Ve zcela reálném smyslu lidé, kteří nosí odznaky a uniformy, nejsou všemi ostatními vnímáni jako obyčejní lidé. Jsou vnímáni jako odnož abstraktní věci zvané \enquote{autorita.} V důsledku toho se správnost chování \enquote{policistů} a oprávněnost jejich činů měří podle daleko jiných měřítek než chování všech ostatních. Jsou posuzováni spíše podle toho, jak dobře prosazují \enquote{zákon,} než podle toho, zda jejich individuální jednání odpovídá běžným normám správnosti a nesprávnosti, které platí pro všechny ostatní. Tento rozdíl vyjadřují sami \enquote{strážci zákona,} kteří své jednání často obhajují slovy jako: \enquote{Já zákon nevytvářím, já ho jen vymáhám.} Zjevně očekávají, že budou posuzováni pouze podle toho, jak věrně plní vůli \enquote{zákonodárců,} a nikoli podle toho, zda se chovají jako civilizované, rozumné lidské bytosti.

\textbf{\enquote{Země} a \enquote{národy :}} Pojmy \enquote{právo} a \enquote{zločin} jsou zřejmými odnožemi pojmů \enquote{stát} a \enquote{autorita,} ale mnoho dalších slov v anglickém jazyce je buď změněno vírou v autoritu, nebo existují výhradně kvůli této víře. Například \enquote{země} nebo \enquote{národ} je čistě politický pojem. Hranice kolem \enquote{země} je podle definice hranicí vymezující území, nad nímž si jedna konkrétní \enquote{autorita} nárokuje právo vládnout, čímž se toto místo odlišuje od území, nad nímž si právo vládnout nárokují \emph{jiné} \enquote{autority.}

Zeměpisné lokality jsou samozřejmě velmi reálné, ale pojem \enquote{země} neoznačuje pouze místo. Vždy odkazuje na politickou \enquote{jurisdikci} (další termín vycházející z víry v \enquote{autoritu}). Když lidé mluví o lásce ke své zemi, málokdy jsou schopni vůbec definovat, co to znamená, ale nakonec jediné, co slovo \enquote{země} může znamenat, není místo nebo lidé nebo nějaký abstraktní princip či pojem, ale pouze území, na kterém si určitá parta nárokuje právo vládnout. V kontextu této skutečnosti je pojem lásky k vlasti poněkud zvláštní myšlenkou; vyjadřuje jen o málo víc než psychologickou náklonnost k ostatním subjektům, které ovládá stejná vládnoucí třída -- což vůbec není to, co si většina lidí představuje pod pojmem národní loajalita a vlastenectví. Lidé mohou pociťovat lásku k určité kultuře nebo k určitému místu a lidem, kteří tam žijí, nebo k nějakému filozofickému ideálu a zaměňovat to za lásku k vlasti, ale v konečném důsledku je \enquote{země} prostě oblast, ve které si určitý \enquote{stát} nárokuje právo vládnout. Tohle je to, co určuje hranice, a jsou to právě tyto hranice, které definují \enquote{zemi.}

\section{Pokusy racionalizovat iracionální}

Lidé, kteří se považují za vzdělané, svobodomyslné a pokrokové, se nechtějí považovat za otroky pána, nebo dokonce za poddané vládnoucí třídy. Z tohoto důvodu bylo provedeno mnoho racionalizace a zastírání ve snaze popřít základní povahu \enquote{státu} jako vládnoucí třídy. Bylo vytvořeno mnoho verbální gymnastiky, zavádějící terminologie a mytologie, které se snaží zastřít skutečný vztah mezi \enquote{vládou} a jejich poddanými. Tato mytologie se učí děti jako \enquote{občanská nauka,} přestože většina z ní je zcela nelogická a odporuje všem poznatkům. Následující text se zabývá několika oblíbenými typy propagandy, které se používají k zamlžování podstaty \enquote{autority.}

\section{Mýtus souhlasu}

V moderním světě je otroctví téměř všeobecně odsuzováno. Vztah domnělé \enquote{autority} k jejímu podřízenému je však do značné míry vztahem otrokáře (majitele) k otrokovi (majetku). Protože nechtějí připustit a schválit to, co se rovná otroctví, jsou ti, kdo věří v autoritu, vyškoleni, aby si zapamatovali a opakovali zjevně nepřesnou rétoriku, která má zakrýt skutečnou povahu situace. Jedním z příkladů je fráze \enquote{souhlas občanů.}

Existují dva základní způsoby interakce mezi lidmi: na základě vzájemné dohody, nebo pod výhružkami či násilím od jedné osoby ke druhé. První způsob lze označit jako \enquote{souhlas} -- obě strany dobrovolně a ochotně souhlasí s tím, co se má udělat. Druhý způsob lze označit jako \enquote{vládnutí} -- jedna osoba ovládá druhou. Protože tyto dva pojmy -- souhlas a vládnutí -- jsou protikladné, nedává smysl, aby někdo vládl se \enquote{souhlasem občanů.} Pokud existuje vzájemný souhlas, nejedná se o \enquote{vládu;} pokud existuje vládnutí, neexistuje souhlas. Někteří budou tvrdit, že většina neboli lid jako celek dal souhlas k tomu, aby se mu vládlo, i když mnozí jednotlivci tak neučinili. Takový argument však staví pojem souhlasu na hlavu. Nikdo, jednotlivec ani skupina, nemůže dát souhlas k tomu, aby se něco dělo někomu \emph{jinému}. To prostě není to, co \enquote{souhlas} znamená. Říci: \enquote{Dávám \emph{svůj} souhlas k tomu, abyste vy byli okradeni.} odporuje logice. Přesto je to základem kultu \enquote{demokracie:} představa, že většina může dát souhlas také za menšinu. To není \enquote{souhlas občanů;} je to násilné ovládání občanů se \enquote{souhlasem} třetí strany.

I kdyby byl někdo natolik hloupý, že by někomu skutečně řekl: \enquote{Souhlasím s tím, abyste mě násilím ovládali,} v okamžiku, kdy vládce musí \enquote{ovládaného} k něčemu nutit, již zjevně nejde o \enquote{souhlas.} Před tímto okamžikem neexistuje žádné \enquote{ovládání} -- pouze dobrovolná spolupráce. Přesnější vyjádření formulace tohoto konceptu odhaluje jeho vnitřní schizofrenii: \enquote{Souhlasím s tím, abyste mě k něčemu nutili, ať už s tím souhlasím, nebo ne.}

Ve skutečnosti však nikdo nikdy nesouhlasí s tím, aby si \enquote{stát} dělal, co chce. A tak, aby vykonstruovali \enquote{souhlas} tam, kde žádný není, přidávají věřící v autoritu do mytologie další, ještě bizarnější krok: pojem \enquote{implicitního souhlasu.} Tvrdí, že pouhým životem ve městě, státu nebo zemi člověk \enquote{souhlasí} s tím, že bude dodržovat jakákoli pravidla, která vydávají lidé, kteří tvrdí, že mají právo vládnout tomuto městu, státu nebo zemi. Myšlenka je taková, že pokud se někomu pravidla nelíbí, může město, stát nebo zemi zcela opustit, a pokud se rozhodne neodejít, znamená to, že souhlasí s tím, aby ho vládci této jurisdikce ovládali.

Ačkoli je tato myšlenka neustále omílána jako svatá pravda, odporuje zdravému rozumu. Nedává to o nic větší smysl, než když zloděj zastaví v neděli řidiče a řekne mu: \enquote{Tím, že v neděli jedete touto čtvrtí, mi dáváte souhlas k tomu vzít si vaše auto.} Jedna osoba zjevně nemůže rozhodovat o tom, co se považuje za to, že někdo jiný s něčím \enquote{souhlasí.} Dohoda je, když si dva nebo více lidí sdělí vzájemné svolení nějakou dohodu uzavřít. Pouhým narozením na určitém území nikdo nedává k ničemu souhlas, stejně jako jej nedává bydlením ve vlastním domě, o kterém nějaký král nebo politik prohlásil, že patří do říše, které vládne. Jedna věc je, když někdo řekne: \enquote{Jestli chceš jet v mém autě, nesmíš kouřit,} nebo \enquote{Do mého domu můžeš vstoupit, jen když si sundáš boty.} Něco jiného je snažit se druhým lidem říkat, co mohou dělat na svém vlastním pozemku. Ten, kdo má právo stanovit pravidla pro určité místo, je z definice vlastníkem tohoto místa. To je základem myšlenky soukromého vlastnictví: že může existovat \enquote{vlastník,} který má výlučné právo rozhodovat o tom, co se s tímto majetkem a na tomto majetku bude dělat. Vlastník domu má právo ostatní z domu vykázat a v důsledku toho má právo říkat návštěvníkům, co mohou a co nemohou dělat, dokud jsou uvnitř.

A to objasňuje základní předpoklady, který stojí za myšlenkou implicitního souhlasu. Říci někomu, že jeho jedinou platnou volbou je buď opustit \enquote{zemi,} nebo se podřídit jakýmkoli příkazům, které politici vydají, logicky znamená, že vše v \enquote{zemi} je majetkem politiků. Pokud člověk rok co rok platí za dům nájem, nebo si ho dokonce sám postavil, a jeho volba je stále buď poslechnout politiky, nebo odejít, znamená to, že jeho dům \emph{a} čas a úsilí, které do něj investoval, jsou majetkem politiků. A to, že čas a úsilí jednoho člověka právem patří druhému, je definice otroctví. Přesně to znamená teorie \enquote{implicitního souhlasu:} že každá \enquote{země} je obrovskou otrokářskou plantáží a že všechno a všichni v ní jsou majetkem politiků. A pán samozřejmě nepotřebuje souhlas svého otroka.

Věřící ve \enquote{stát} nikdy nevysvětlí, jak je možné, že několik politiků získalo právo jednostranně si přivlastnit tisíce čtverečních kilometrů země, kde již žili jiní lidé, jako své území, které mohou ovládat a využívat podle svého uvážení. Nebylo by to nic jiného, než kdyby nějaký šílenec prohlásil: \enquote{Tímto prohlašuji Severní Ameriku za své právoplatné panství, takže každý, kdo zde žije, musí dělat, co řeknu. Pokud se vám to nelíbí, můžete odejít.}

S přístupem \enquote{poslouchej, nebo vypadni} je také praktický problém, který spočívá v tom, že vystoupením by se jedinec pouze přemístil na nějakou \emph{jinou} obří otrokářskou plantáž, do jiné \enquote{země.} Konečným výsledkem je, že každý na zemi je otrokem a jedinou volbou je, pod jakým pánem bude žít. To zcela vylučuje skutečnou svobodu. Přesněji řečeno, to není to, co znamená \enquote{souhlas.}

Víra, že politikům patří všechno, se ještě výrazněji projevuje v pojetí imigračních \enquote{zákonů.} Představa, že člověk potřebuje povolení politiků, aby mohl vkročit kamkoli do celé země -- představa, že může být \enquote{zločin,} když někdo překročí neviditelnou hranici, z jedné autoritářské jurisdikce do druhé -- znamená, že celá země je majetkem vládnoucí třídy. Pokud občan \enquote{ilegálního imigranta} nesmí zaměstnat, nesmí s ním obchodovat a \enquote{ilegála} dokonce nesmí ani pozvat do svého domu, pak tento jednotlivý občan nevlastní nic a politici vlastní všechno.

Teorie \enquote{implicitního souhlasu} je nejen logicky chybná, ale zjevně ani nepopisuje realitu. Jakýkoli \enquote{stát,} který by měl souhlas svých poddaných, by nepotřeboval \enquote{vymáhat právo.} K vynucování dochází pouze tehdy, pokud někdo s něčím nesouhlasí. Každý, kdo má otevřené oči, vidí, že \enquote{stát} běžně dělá spoustě lidí věci proti jejich vůli. Vědět o nesčetném množství výběrčích daní, pochůzkářů, inspektorů a regulátorů, pohraničníků, protidrogových agentů, státních zástupců, soudců, vojáků a všech ostatních žoldáků \enquote{státu,} a přesto tvrdit, že \enquote{stát} dělá to, co dělá, se \emph{souhlasem} \enquote{občanů,} je naprosto směšné. Každý jednotlivec, pokud je k sobě alespoň trochu upřímný, ví, že těm, kteří jsou u moci, je jedno, zda souhlasí s dodržováním jejich \enquote{zákonů.} Příkazy politiků budou vymáhány, v případě potřeby i hrubou silou, ať už se souhlasem jednotlivce, nebo bez něj.

\section{Další mytologie}

Kromě mýtu o \enquote{souhlasu občanů} se často opakují i další politická rčení a dogmatická rétorika, přestože jsou zcela nepravdivá. Ve Spojených státech se například lidé učí -- a věrně i opakují -- výroky jako \enquote{Stát \emph{jsme} my} a \enquote{Stát je náš sluha} a \enquote{Vláda nás \emph{zastupuje}.} Takové aforismy jsou očividně hrubě nepravdivé, přestože je neustále papouškují jak vládci, tak poddaní.

Jedním z nejbizarnějších a nejbludnějších (ale velmi častých) tvrzení je, že \enquote{stát \emph{jsme} my, všichni lidé.} Děti se ve škole učí tuto absurditu opakovat, přestože si všichni plně uvědomují, že politici vydávají příkazy a požadavky a všichni ostatní se jim buď podřídí, nebo jsou potrestáni. Ve Spojených státech existuje vládnoucí třída a třída poddaných a rozdíly mezi nimi jsou mnohé a zřejmé. Jedna skupina přikazuje, druhá poslouchá. Jedna skupina požaduje obrovské sumy peněz, druhá skupina platí. Jedna skupina říká druhé skupině, kde mohou žít, kde mohou pracovat, co mohou jíst, co mohou pít, čím mohou jezdit, pro koho mohou pracovat, jakou práci mohou vykonávat atd. Jedna skupina bere a utrácí biliony dolarů z toho, co vydělá druhá skupina. Jedna skupina se skládá výhradně z ekonomických parazitů, zatímco úsilí druhé skupiny vytváří veškeré bohatství.

V tomto systému je zcela zřejmé, kdo poroučí a kdo poslouchá. Lidé ani náhodou nejsou \enquote{státem} a věřit v opak vyžaduje zásadní zapírání. Ve snaze, aby tato lež zněla racionálně, se však používají i jiné mýty. Například se také tvrdí, že \enquote{stát pracuje pro nás, je to náš služebník.} Takové tvrzení opět ani zdaleka neodpovídá zjevné realitě; je to jen o málo víc než sektářská mantra, blud záměrně naprogramovaný do obyvatelstva, aby překroutil jejich pohled na realitu. A většina lidí o tom ani nepochybuje. Většina si nikdy nepoloží otázku: Pokud \enquote{stát} pracuje pro nás, pokud je naším zaměstnancem, proč rozhoduje o tom, kolik jí budeme platit? Proč náš \enquote{zaměstnanec} rozhoduje o tom, co pro nás bude dělat? Proč nám náš \enquote{zaměstnanec} říká, jak máme žít? Proč náš \enquote{zaměstnanec} vyžaduje naši poslušnost při plnění jakýchkoli svévolných příkazů, které vydává, a posílá na nás ozbrojené vymahatele, pokud neuposlechneme? Není možné, aby \enquote{stát} byl služebníkem už z jeho samotné podstaty. Jednoduše a konkrétně řečeno, pokud vám někdo může poroučet a brát vám peníze, není vaším služebníkem; a pokud tyto věci \emph{nemůže} dělat, nevládne vám a není \enquote{státem.} Jakkoli omezená, \enquote{stát} je organizace, o níž se má za to, že má právo násilně ovládat chování svých poddaných prostřednictvím \enquote{zákonů,} což činí všeobecně přijímanou rétoriku o \enquote{státních úřednících} zcela směšnou. Představa, že by \emph{vládce} mohl být někdy \emph{služebníkem} těch, nad nimiž vládne, je zjevně absurdní. Přesto se tato nemožnost v hodinách občanské nauky vykládá jako nezpochybnitelná svatá pravda.

Ještě rozšířenější lží, která se snaží zakrýt vztah mezi \enquote{státem} a veřejností, je pojem \enquote{zastupitelská demokracie.} Tvrdí se, že lidé si volbou určitých osob do mocenských pozic \enquote{vybírají své vůdce} a že ti, kteří jsou ve funkci, jen zastupují vůli lidu. Opět platí, že nejenže toto tvrzení vůbec neodpovídá realitě, ale i abstraktní teorie, ze které vychází, je ze své podstaty chybná.

V reálném světě takzvaní \enquote{zastupitelé} neustále dělají věci, které jejich poddaní nechtějí: zvyšují \enquote{daně,} zahajují válečné konflikty, prodávají moc a vliv tomu, kdo jim dá nejvíce peněz, a tak dále. Každý daňový poplatník si snadno vybaví příklady věcí financovaných z jeho peněz, které mu vadí, ať už jde o dotace obrovským korporacím, sociální dávky určitým jednotlivcům, akce porušující lidská práva nebo jen celkově plýtvající, zkorumpovanou a neefektivní byrokratickou mašinérii \enquote{státu.} Neexistuje nikdo, kdo by mohl upřímně říci, že \enquote{stát} dělá vše, co chce, a nic, co nechce.

I teoreticky je koncept \enquote{zastupitelské demokracie} z podstaty chybný, protože \enquote{stát} nemůže zastupovat lid jako celek, pokud všichni nechtějí přesně to samé. Protože různí lidé chtějí, aby \enquote{stát} dělal různé věci, bude \enquote{stát} vždy postupovat proti vůli alespoň části lidí. I kdyby \enquote{stát} dělal přesně to, co chce většina poddaných (což se ve skutečnosti nikdy nestane), stále by nesloužil lidu jako celku; násilím by diskriminoval menšiny ve prospěch většin.

Navíc ten, kdo zastupuje někoho jiného, nemůže mít \emph{více} práv než ten, koho zastupuje. Pokud například jeden člověk nemá právo vloupat se do domu svého souseda a ukrást mu cennosti, pak nemá ani právo určit zástupce, který by to udělal za něj. Zastupovat někoho znamená jednat jeho jménem. Skutečný zástupce může dělat pouze to, na co má osoba, kterou zastupuje, právo. V případě \enquote{státu} však lidé, o nichž politici tvrdí, že je zastupují, nemají právo dělat \emph{cokoli}, co dělají politici: ukládat \enquote{daně,} přijímat \enquote{zákony} atd. Průměrní občané nemají právo násilím omezovat svobody svých sousedů, říkat jim, jak mají žít, a trestat je, pokud neuposlechnou. Když tedy \enquote{stát} dělá takové věci, nezastupuje nikoho ani nic jiného než sám sebe.

Zajímavé je, že i ti, kteří mluví o \enquote{zastupitelské demokracii,} odmítají přijmout jakoukoli osobní odpovědnost za činy těch, které volili. Pokud jimi zvolený kandidát přijme škodlivý \enquote{zákon,} zvýší \enquote{daně} nebo rozpoutá válku, voliči nikdy necítí stejnou vinu nebo stud, jaký by cítili, kdyby takové věci dělali oni sami osobně nebo kdyby k tomu najali či pověřili někoho jiného. Tato skutečnost ukazuje, že ani ti nejzapálenější voliči ve skutečnosti nevěří rétorice o \enquote{zastupitelské demokracii} a nepovažují politiky za své zástupce. Terminologie neodpovídá realitě a jediným účelem této rétoriky je zastřít skutečnost, že vztah mezi každou \enquote{vládou} a jejími poddanými je stejný jako vztah mezi pánem a otrokem. Jeden pán může své otroky bičovat méně přísně než jiný; jeden pán může svým otrokům dovolit, aby si ponechali více z toho, co vyprodukují; jeden pán se může o své otroky lépe starat -- ale nic z toho nemění základní, základní povahu vztahu pána a otroka. Ten, kdo má právo vládnout, je pán; ten, kdo má povinnost poslouchat, je otrok. A to platí i tehdy, když se lidé rozhodnou popsat situaci pomocí nepřesné rétoriky a klamavých eufemismů, jako je \enquote{zastupitelská demokracie,} \enquote{souhlas občanů} a \enquote{vůle lidu.}

Celá představa \enquote{demokracie} a \enquote{vlády lidí pro lidi} je sice hezká politická rétorika, ale je logicky nesmyslná. Vládnoucí třída nemůže sloužit ani zastupovat ty, kterým vládne, stejně jako nemůže otrokář sloužit ani zastupovat své otroky. Jediný způsob, jak to může udělat, je přestat být otrokářem, tedy osvobodit své otroky. Stejně tak jediným způsobem, jak by se vládnoucí třída mohla stát služebníkem lidu, je \emph{přestat být} vládnoucí třídou, vzdát se veškeré své moci. \enquote{Stát} nemůže sloužit lidu, pokud nepřestane být \enquote{státem.}

Dalším příkladem iracionální etatistické doktríny je koncept \enquote{právního státu.} Podle této myšlenky je vláda pouhých \emph{lidí} špatná, protože slouží těm, kdo mají zlovolnou touhu po moci, zatímco \enquote{vláda práva,} jak zní teorie, spočívá v tom, že lidstvu jsou rovným dílem vnucována objektivní, rozumná pravidla. Chvilka zamyšlení odhalí absurditu tohoto mýtu. Navzdory tomu, že se o \enquote{právu} často hovoří jako o nějakém svatém, neomylném souboru pravidel, která spontánně vyplývají z podstaty vesmíru, ve skutečnosti je \enquote{právo} pouhým souborem příkazů vydávaných a vynucovaných \emph{lidmi} ze \enquote{státu.} Rozdíl mezi \enquote{vládou práva} a \enquote{vládou lidí} by byl pouze tehdy, kdyby takzvané \enquote{zákony} psalo něco \emph{jiného} než lidé.

\section{Tajemná přísada}

Ve snaze ospravedlnit existenci vládnoucí třídy (\enquote{státu}) etatisté často popisují naprosto rozumné, legitimní a užitečné věci a pak je prohlašují za \enquote{stát.} Mohou tvrdit: \enquote{Jakmile lidé spolupracují, aby vytvořili organizovaný systém vzájemné obrany, je to stát.} Nebo mohou tvrdit: \enquote{Když lidé kolektivně rozhodují o tom, jak budou v jejich městě fungovat věci jako silnice, obchod a vlastnická práva, je to stát.} Nebo mohou tvrdit: \enquote{Když lidé spojí své zdroje, aby dělali věci kolektivně, místo aby každý jednotlivec musel dělat všechno sám za sebe, to je stát.} Ani jedno z těchto tvrzení není pravdivé.

Taková tvrzení mají za cíl, aby \enquote{stát} zněl jako přirozená, legitimní a užitečná součást lidské společnosti. Všem těmto tvrzením však zcela uniká základní povaha \enquote{státu.} \enquote{Stát} není organizace, spolupráce nebo vzájemná dohoda. Nespočet skupin a organizací -- supermarkety, fotbalové týmy, automobilky, lukostřelecké kluby atd. -- se zapojují do kooperativních, vzájemně výhodných kolektivních akcí, ale nenazývají se \enquote{státem,} protože si nepředstavují, že mají právo vládnout. A to je právě ta tajná složka, která z něčeho dělá \enquote{autoritu:} domnělé \emph{právo} násilně ovládat druhé.

\enquote{Státy} nevznikají jen ze supermarketů nebo fotbalových týmů, ani z lidí, kteří se připravují a zajišťují svou vzájemnou obranu. Mezi otázkou \enquote{Jak se můžeme účinně bránit?} a tvrzením \enquote{Mám právo vám vládnout} je zásadní rozdíl! Navzdory tomu, co mohou tvrdit učebnice občanské nauky, \enquote{státy} nejsou výsledkem ani ekonomiky, ani základní lidské interakce. Nevznikají jen v důsledku toho, že jsou lidé civilizovaní a organizovaní. Jsou zcela produktem mýtu, že \enquote{někdo musí vládnout.} Bez pověry o autoritě by se žádná spolupráce nebo organizace nikdy nestala \enquote{státem.} K tomu, aby se \emph{poskytovatel služeb}, ať už jde o jídlo, přístřeší, informace, ochranu nebo cokoli jiného, proměnil v právoplatného \emph{vládce}, je zapotřebí drastické změny ve vnímání veřejnosti. Systém organizace se nemůže zázračně stát \enquote{státem,} stejně jako se ochranka nemůže zázračně stát králem.

A tato skutečnost souvisí s dalším tvrzením etatistů, že odstranění \enquote{státu} by vedlo k tomu, že by se k moci dostaly násilnické gangy, které by se následně staly novým \enquote{státem.} Násilná nadvláda se však přirozeně nestává \enquote{státem} stejně jako mírová spolupráce. Pokud si nový gang nebude představovat, že má \emph{právo} vládnout, nebude vnímán jako \enquote{stát.} Schopnost ovládat moderní obyvatelstvo -- zejména ozbrojené -- ve skutečnosti zcela závisí na vnímání \emph{legitimity} budoucích vládců. Vládnout dnes jakékoli populaci značné velikosti pouze hrubou silou by vyžadovalo obrovské množství zdrojů (zbraně, špehové, žoldáci atd.), a to natolik, že by to bylo téměř nemožné. Představa bandy nelítostných násilníků, kteří ovládnou zemi, může být zábavným filmem, ale ve skutečnosti se to nemůže stát v zemi, která je vybavena dokonce jen základními komunikačními prostředky a střelnými zbraněmi. Jediný způsob, jak dnes ovládnout velkou populaci, je, že budoucí vládce nejprve přesvědčí lidi, že má morální \emph{právo} nad jim vládnout; nadvládu může získat pouze tehdy, pokud se mu nejprve podaří vtlouct svým zamýšleným obětem do hlavy mýtus autority, a tím je přesvědčit, že je legitimní a správný \enquote{stát.} A pokud se mu to podaří, bude k získání a udržení moci zapotřebí jen velmi málo skutečné síly. Pokud však jeho režim někdy ztratí legitimitu v očích svých obětí, nebo ji od začátku ani nikdy nezíská, samotná hrubá síla mu žádnou trvalou moc nezajistí.

Zkrátka, ani gangy, ani spolky se nikdy nemohou stát \enquote{státem,} pokud lidé nevěří, že někdo má právo jim vládnout. Stejně tak, jakmile se lidé jako celek osvobodí od mýtu autority, nebudou potřebovat žádnou revoluci, aby byli svobodní; \enquote{stát} prostě přestane existovat, protože jediné místo, kde kdy existoval, je v myslích těch, kteří věří v pověru o autoritě. Opět platí, že politici a žoldáci, kteří vymáhají své hrozby, jsou velmi skuteční, ale bez vnímané \emph{legitimity} jsou uznáváni jako banda mocichtivých násilníků, nikoli jako \enquote{stát.}

Je třeba také zmínit, že někteří tvrdili (včetně Thomase Jeffersona v Deklaraci nezávislosti), že je možné a žádoucí mít \enquote{stát,} který nedělá nic jiného než chrání práva jednotlivců. Ale organizace, která by dělala pouze toto, by nebyla \enquote{státem.} Každý jednotlivec má právo bránit sebe i ostatní proti útočníkům. Uplatňování tohoto práva, a to i prostřednictvím velmi organizované, rozsáhlé operace, by nebylo \enquote{státem,} stejně jako organizovaná, rozsáhlá výroba potravin nepředstavuje \enquote{stát.} Aby něco bylo \enquote{státem,} musí to podle definice dělat něco, na co průměrní lidé nemají právo. \enquote{Stát,} který má stejná práva jako všichni ostatní, není \enquote{státem,} stejně jako není \enquote{státem} průměrný člověk na ulici.

\section{Výmluva na nezbytnost}

Výmluva, ke které se etatisté (lidé věřící ve stát) nakonec často uchylují, je, že lidstvo \emph{vyžaduje} \enquote{stát,} že společnost potřebuje vládce, že někdo musí vládnout, jinak by zde byl neustálý chaos a krvavé nepokoje. Ale nutnost, ať už skutečná, nebo falešná, nemůže zhmotnit mýtickou entitu. Právo vládnout nevznikne jen proto, že ho údajně \enquote{potřebujeme,} abychom měli mírumilovnou společnost. Nikdo nebude tvrdit, že Ježíšek musí být skutečný, protože ho potřebujeme, aby Vánoce fungovaly. Pokud \enquote{autorita} neexistuje a nemůže existovat, jak bude prokázáno níže, je tvrzení, že ji \enquote{potřebujeme,} nejen zbytečné, ale zjevně i nepravdivé. Pouhou silou vůle nemůžeme něco vykouzlit k existenci. Pokud vyskočíte z letadla bez padáku, vaše \enquote{potřeba} padák nezhmotní. Ze stejného důvodu, pokud je nemožné, aby jeden člověk získal právo vládnout nad druhým, a nemožné, aby jeden člověk získal povinnost podřídit se druhému (jak je dokázáno níže), pak je tvrzení, že se takové věci \enquote{musí} stát, prázdným argumentem.

\chapter{Vyvrácení mýtu}

\section{Opuštění mýtu}

Stále více lidí dnes věří, že \enquote{stát} není nutný a že lidská společnost by v praxi fungovala mnohem lépe bez něj. Jiní tvrdí, že bez ohledu na to, co \enquote{funguje} lépe, je společnost bez násilného státu jedinou morální volbou, protože je to jediná volba, která neobhajuje iniciování násilí proti nevinným lidem. Ačkoli jsou takové argumenty platné a přínosné, ve skutečnosti existuje zásadnější fakt, který tyto diskuse činí bezpředmětnými: \enquote{autorita,} ať už morální či nikoli, a ať už \enquote{funguje} či nikoli, nemůže existovat. A to není pouhé konstatování toho, co by \emph{mělo být}, ale popis toho, co \emph{je}. Pokud \enquote{autorita} nemůže existovat -- což bude logicky prokázáno níže -- je jakákoli debata o tom, zda ji \enquote{potřebujeme} nebo jak dobře funguje v praktické rovině, zbytečná.

Smyslem této knihy tedy není, že by \enquote{stát} měl být zrušen, ale že \enquote{stát} -- tedy \emph{legitimní} vládnoucí třída, která má \enquote{autoritu} -- \emph{neexistuje a nemůže existovat} a že neschopnost uznat tuto skutečnost vedla k nezměrnému utrpení a nespravedlnosti. Dokonce i většina těch, kteří uznávají \enquote{stát} jako obrovskou hrozbu pro lidstvo, mluví o jejím odstranění, jako by skutečně existovala. Mluví, jako by existovala volba mezi tím, zda \enquote{stát} mít, nebo nemít. Ale neexistuje. \enquote{Stát} je logicky nemožný. Problémem ve skutečnosti není \enquote{stát,} ale \emph{víra} ve \enquote{stát.} Analogicky, ten, kdo si uvědomí, že Ježíšek není skutečný, nezačne křížovou výpravu za zrušení Ježíška nebo za jeho vystěhování ze severního pólu. Prostě v něj přestane věřit. Rozdíl je v tom, že víra v Ježíška způsobuje jen málo škody, zatímco víra v mýtické monstrum zvané \enquote{autorita} vede k nepředstavitelné bolesti a utrpení, útlaku a nespravedlnosti.

Nejde o to, že bychom se měli snažit vytvořit svět bez \enquote{autorit,} ale o to, že by bylo vhodné, aby lidé přijali skutečnost, že svět bez \enquote{autorit} je \emph{všechno, co kdy existovalo}, a že lidstvo by na tom bylo mnohem lépe a lidé by se chovali mnohem racionálněji, morálněji a civilizovaněji, kdyby byla tato skutečnost přijata širokou veřejností.

\section{Proč je mýtus lákavý}

Než se ukáže, že \enquote{autorita} nemůže existovat, je třeba se krátce zmínit o tom, proč by někdo něco takového chtěl. Je zřejmé, proč ti, kdo usilují o nadvládu nad druhými, chtějí, aby \enquote{stát} existoval: poskytuje jim snadný, údajně legitimní mechanismus, jehož prostřednictvím mohou druhé násilně ovládat. Ale proč by ji měl chtít někdo jiný -- proč by ji měli chtít ti, kteří jsou \emph{ovládáni}?

Myšlení etatistů obvykle začíná rozumnou obavou, ale končí šíleným \enquote{řešením.} Průměrný člověk, který se dívá na svět s vědomím, že na něm žijí miliardy lidských bytostí, z nichž mnohé jsou hloupé nebo nepřátelské, chce přirozeně mít nějakou jistotu, že bude chráněn před všemi nedbalostmi a zlovůlí, které mohou ostatní napáchat. Většina věřících ve stát to otevřeně popisuje jako důvod, proč je \enquote{stát} potřeba: protože lidem nelze věřit, protože je v lidské přirozenosti krást, rvát se atd. Etatisté často tvrdí, že bez mocenského orgánu, bez \enquote{státu,} který by stanovoval a vynucoval společenská pravidla pro každého, by každý spor skončil krveprolitím, spolupráce by byla omezená nebo žádná, obchod by se téměř zastavil, každá by byl \enquote{sám za sebe} a lidstvo by degradovalo na lovce a sběrače nebo na svět z filmu \enquote{Mad Max.}

V debatě mezi etatismem a anarchismem (nebo voluntaryismem) se často nesprávně předpokládá, že jde o to, zda jsou lidé ze své podstaty dobří a důvěryhodní, a proto nepotřebují být ovládáni, nebo jsou ze své podstaty špatní a nedůvěryhodní, a proto potřebují \enquote{stát,} který by je hlídal. Ve skutečnosti, ať už jsou lidé všichni dobří, všichni špatní, nebo něco mezi tím, je víra v autoritu stále iracionální pověrou. Ale nejoblíbenější výmluva pro \enquote{stát} -- že lidé jsou špatní a je třeba jim vládnout -- bezděčně odhaluje šílenství vlastní každému etatismu.

Pokud jsou lidé tak neopatrní, hloupí a zlomyslní, že jim nelze věřit, že sami udělají správnou věc, jak by se situace zlepšila, kdybychom vzali \emph{podskupinu} těch samých neopatrných, hloupých a zlomyslných lidí a dali jim společenské \emph{povolení} násilně ovládat všechny ostatní? Proč si někdo myslí, že přeskupení a reorganizace skupiny nebezpečných bestií je zcivilizuje? Odpověď naznačuje mytologickou povahu víry v autoritu. Autoritáři neusilují pouze o jiné uspořádání lidských bytostí, ale o zapojení jakési \emph{nadlidské} entity s právy, která lidské bytosti nemají, a s \emph{ctnostmi}, které lidské bytosti nemají, a kterou lze použít k udržení všech nedůvěryhodných lidí na uzdě. Tvrdit, že lidské bytosti jsou tak nedokonalé, že je třeba jim vládnout -- což je běžný refrén etatistů -- znamená, že to vládnutí musí provádět něco \emph{jiného} než lidské bytosti. Ale ať studujete \enquote{stát} jakkoli usilovně, zjistíte, že jej vždy řídí výhradně \emph{lidé}. Tvrdit, že \enquote{stát} je nutný, protože \emph{lidé} jsou nedůvěryhodní, je stejně iracionální jako tvrdit, že když někoho napadne roj včel, řešením je vytvořit autoritářskou hierarchii \emph{mezi včelami} a některým z nich přidělit povinnost bránit \emph{ostatním} včelám škodit. Ať už jsou včely jakkoli nebezpečné, takové \enquote{řešení} je směšné.

To, co věřící od \enquote{státu} skutečně chtějí, je obrovská, nezastavitelná moc, která bude použita pro dobro. Neexistuje však žádný kouzelnický ani politický trik, který by byl schopen zaručit, že nastane spravedlnost, že zvítězí \enquote{ti dobří} nebo že nevinní budou chráněni a bude o ně postaráno. Obrovský, nadlidský, magický spasitel, na jehož potřebě etatisté trvají, aby zachránil lidstvo před jemi samými, \emph{neexistuje}. Minimálně na této planetě jsou lidé to nejvyšší, co známe -- nad nimi už není nic, co by jim vládlo a nutilo chovat se správně, a halucinování o takové nadlidské entitě ji nečiní skutečnou a situaci nijak nepomáhá.

\section{Náboženství státu}

\enquote{Stát} není ani vědecký pojem, ani racionální sociologický konstrukt, ani logická, praktická metoda lidské organizace a spolupráce. Víra ve stát není založena na rozumu, ale na víře. Ve skutečnosti je víra ve stát náboženstvím, které se skládá ze souboru dogmatických rčení a iracionálních doktrín, které odporují důkazům i logice a které si věřící metodicky zapamatovali a neustále je opakují. Stejně jako jiná náboženství popisuje evangelium \enquote{státu} nadlidskou, nadpřirozenou entitu, která stojí nad obyčejnými smrtelníky a vydává přikázání rolníkům, pro něž je bezvýhradná poslušnost morálním imperativem. Neuposlechnutí přikázání (\enquote{porušení zákona}) je považováno za hřích a věřící se vyžívají v trestání nevěřících a hříšníků (\enquote{zločinců}), přičemž jsou zároveň velmi hrdí na svou vlastní loajalitu a pokornou podřízenost svému bohu (jako \enquote{zákona dbalí daňoví poplatníci}). A zatímco smrtelníci mohou svého pána pokorně prosit o laskavosti a o povolení dělat určité věci, je považováno za rouhačské a pobuřující, když si někdo z pokorných rolníků představuje, že je způsobilý rozhodovat o tom, které \enquote{zákony} státního boha má dodržovat a které je v pořádku ignorovat. Jejich mantra zní: \enquote{Můžete se snažit zákon změnit, ale dokud je to zákon, musíme ho všichni dodržovat!}

Náboženská povaha víry v autoritu je všem na očích, kdykoli lidé slavnostně stojí s rukama na srdci a nábožensky proklamují svou nehynoucí víru a loajalitu k vlajce a \enquote{státu} (\enquote{republice}). Těm, kteří odříkávají slib věrnosti a cítí přitom hlubokou hrdost, jen zřídkakdy dojde, že ve skutečnosti přísahají věrnost systému podřízení a autoritářské nadvládě. Stručně řečeno, slibují, že budou dělat, co se jim řekne, a chovat se jako loajální poddaní svých pánů. Kromě zjevně nepřesné věty na konci o \enquote{svobodě a spravedlnosti pro všechny} je celý slib o podřízenosti \enquote{státu,} který tvrdí, že zastupuje kolektiv, jako by to samo o sobě byl nějaký velký a vznešený cíl. Slib a myšlení a emoce, které má vyvolávat, by se stejně dobře uplatnily v jakémkoli tyranském režimu v dějinách. Je to spíše slib poslušnosti a snadné ovladatelnosti, podřízení se \enquote{republice} než slib konat správné věci. Mnoho dalších vlasteneckých rituálů a písní, stejně jako otevřeně náboženská úcta vzdávaná dvěma kusům pergamenu -- Deklaraci nezávislosti a Ústavě USA -- také ukazují, že lidé nepovažují \enquote{stát} pouze za praktickou nutnost; považují ji za boha, kterého je třeba chválit a uctívat, ctít a poslouchat. Hlavním faktorem, který dnes odlišuje víru ve stát od jiných náboženství, je to, že lidé skutečně věří v boha zvaného \enquote{stát.} Ostatní bohové, o nichž lidé tvrdí, že v ně věří, a církve, které navštěvují, jsou dnes ve srovnání s nimi jen prázdnými rituály a polohlasně papouškovanými pověrami. Pokud jde o jejich každodenní život, bohem, ke kterému se lidé skutečně modlí, aby je ochránil před neštěstím, porazil jejich nepřátele a zahrnul je požehnáním, je \enquote{stát.} Je to \enquote{stát,} jehož přikázání lidé nejčastěji respektují a poslouchají.

Kdykoli dojde ke konfliktu mezi \enquote{státem} a naukou menších bohů -- například \enquote{zaplať svůj podíl} (daně) versus \enquote{nepokradeš} nebo \enquote{služba vlasti} (vojenská služba) versus \enquote{nezabiješ} -- příkazy \enquote{státu} nahradí všechna učení ostatních náboženství. Politici, velekněží církve \enquote{státu} -- mluvčí a představitelé \enquote{státu,} kteří shora vydávají posvátný \enquote{zákon} -- dokonce otevřeně prohlašují, že je lidem dovoleno praktikovat jakékoli náboženství, pokud se nedostanou do rozporu s nejvyšším náboženstvím tím, že neuposlechnou \enquote{zákon} -- tedy diktát boha zvaného \enquote{stát.}

Snad nejvýmluvnější je, že když průměrnému člověku navrhnete, že Bůh možná neexistuje, bude pravděpodobně reagovat méně emotivně a nepřátelsky, než když nadhodíte myšlenku života bez \enquote{státu.} To naznačuje, ke kterému náboženství jsou lidé citově silněji vázáni a ve které náboženství skutečně pevněji věří. Ve skutečnosti věří ve \enquote{stát} tak hluboce, že ji jako víru vůbec neuznávají. Důvodem, proč tolik lidí reaguje na myšlenku bezstátní společnosti (\enquote{anarchie}) urážkami, apokalyptickými předpověďmi a emocionálními záchvaty vzteku, a nikoliv klidným uvažováním, je to, že jejich víra ve stát není výsledkem pečlivého, racionálního zvážení důkazů a logiky. Je to v každém ohledu náboženská víra, které věří jen díky dlouhodobé indoktrinaci. A téměř nic není pro uctívače státu existenciálně děsivější než úvahy o tom, že \enquote{stát} -- jejich spasitel a ochránce, učitel a pán -- ve skutečnosti neexistuje a nikdy neexistoval.

Mnoho politických rituálů má otevřeně náboženský podtext. Velkolepé budovy připomínající katedrály, okázalost při inauguracích a jiných \enquote{státních} ceremoniálech, tradiční kostýmy a odvěké rituály, způsob, jakým se zachází s příslušníky vládnoucí třídy a jak se o nich mluví (např. \enquote{čestní}), to vše dodává takovým jednáním atmosféru posvátnosti a úcty, která mnohem více připomíná náboženské obřady než praktický způsob kolektivní organizace.

Bylo by užitečné mít nějaké morálně nadřazené, všemocné božstvo, které by chránilo nevinné a bránilo nespravedlnosti. A právě takový má být, jak doufají etatisté, \enquote{stát:} moudrý, nestranný, vševědoucí a všemocný \enquote{konečný rozhodčí,} který bude nadřazen sobeckým rozmarům člověka a neomylně rozdávat spravedlnost a právo. Nic takového však neexistuje a ani existovat nemůže a existuje mnoho důvodů, proč je pošetilé hledat ve \enquote{státu} řešení lidské nedokonalosti. Lidé mohou říkat, že chtějí, aby \enquote{stát} prosazoval objektivní pravidla civilizovaného chování, ale ve skutečnosti si každý etatista přeje, aby \enquote{autorita} prosazovala jeho \emph{vlastní} představu o spravedlnosti a morálce. Etatista si však neuvědomuje, že v okamžiku, kdy existuje \enquote{autorita,} již není na etatistovi, aby rozhodoval o tom, co se považuje za morální nebo spravedlivé -- \enquote{autorita} si bude nárokovat právo dělat to za něj. A tak se vyznavači autority znovu a znovu pokoušeli vytvořit všemocnou sílu dobra tím, že některé lidi jmenovali vládci, aby rychle zjistili, že jakmile je pán na trůně, je mu jedno, co jeho otroci doufali, že s mocí, kterou mu dali, udělá. A to se stalo všemožným etatistům s velmi odlišnými názory a programy. Socialisté tvrdí, že \enquote{stát} je potřeba k \enquote{spravedlivému} přerozdělování bohatství; objektivisté tvrdí, že \enquote{stát} je potřeba k ochraně práv jednotlivců; konstitucionalisté tvrdí, že \enquote{stát} je potřeba pouze k plnění těch úkolů, které jsou uvedeny v ústavě; věřící v demokracii tvrdí, že \enquote{stát} je potřeba k plnění vůle většiny; mnozí křesťané tvrdí, že \enquote{stát} je potřeba k prosazování Božích zákonů; a tak dále. A v každém případě je lid nakonec zklamán, protože \enquote{autorita} vždy změní plán tak, aby sloužil zájmům lidí u moci. Jakmile se \enquote{státu} ujme skupina vládců, nezáleží na tom, co si masy předsevzaly, že se svou mocí udělají. Tuto skutečnost prokázal každý \enquote{stát} v dějinách. Jakmile si lid vytvoří vládce, lid už z definice nevládne.

Očekávat něco jiného je absurdní i bez všech historických příkladů. Očekávat, že pán bude sloužit otrokovi -- očekávat, že moc bude využívána výhradně ve prospěch toho, kdo je ovládán, nikoliv toho, kdo ovládá -- je absurdní. Ještě šílenější je, že etatisté tvrdí, že jmenování vládců je jediným způsobem, jak překonat nedokonalost a nedůvěryhodnost člověka. Etatisté se dívají na svět plný cizích lidí, kteří mají pochybné motivy a pochybnou morálku, a bojí se toho, co by někteří z těchto lidí mohli udělat. To je samo o sobě zcela oprávněná obava. Ale jako ochranu před tím, co by někteří z těchto lidí mohli udělat, etatisté obhajují to, že některým z těchto pochybných lidí dají obrovskou moc a společenské \emph{povolení} vládnout všem ostatním v marné naději, že se tito lidé nějakým zázrakem rozhodnou využít svou nově nabytou moc pouze k dobru. Jinými slovy, etatista se dívá na své spoluobčany a říká si: \enquote{Nevěřím ti, že jsi můj přítel, ale věřím ti, že jsi můj pán.}

Bizarní je, že téměř každý etatista připouští, že politici jsou více nečestní, zkorumpovaní, zákeřní a sobečtí než většina lidí, ale přesto trvá na tom, že civilizace může existovat pouze tehdy, pokud tito obzvláště nedůvěryhodní lidé dostanou moc a právo násilně ovládat všechny ostatní. Věřící ve stát skutečně věří, že jediné, co je může ochránit před vadami lidské povahy, je vzít \emph{některé} z těchto vadných lidí -- ve skutečnosti některé z těch nejvadnějších -- a jmenovat je bohy s právem ovládat celé lidstvo v absurdní naději, že pokud jim bude dána taková obrovská moc, budou ji tito lidé využívat pouze k dobru. A skutečnost, že se to v dějinách světa nikdy nestalo, etatistům nebrání trvat na tom, že se to \enquote{musí} stát, aby byla zajištěna mírová civilizace.

(\emph{Osobní poznámka autora: To vše říkám jako bývalý oddaný etatista, který po většinu svého života nejenže přijímal sobě odporující a klamné racionalizace, na nichž je založen mýtus státu, ale sám tuto mytologii vehementně šířil. Své vlastní autoritářské indoktrinaci jsem neunikl rychle a pohodlně, ale pouštěl jsem se této pověry pomalu a neochotně, s velkým duševním \enquote{kopáním a řevem} na cestě. Zmiňuji se o tom jen proto, aby bylo zřejmé, že když mluvím o víře v autoritu jako o naprosto iracionální a šílené, útočím tím na své vlastní předchozí přesvědčení stejně jako na přesvědčení kohokoli jiného.})

Jiný pohled na věc je, že etatisté se obávají, že různí lidé mají různá přesvědčení, různé názory, různá měřítka morálky. Vyjadřují obavy typu: \enquote{Co když nebude existovat stát a někdo si bude myslet, že je v pořádku mě zabít a ukrást mi věci?.} Ano, pokud existují protichůdné názory -- jako že vždy existovaly a vždy existovat budou -- mohou vést ke konfliktu. Autoritářské \enquote{řešení} spočívá v tom, že namísto toho, aby každý sám rozhodoval o tom, co je správné a co by měl dělat, by měla existovat centrální \enquote{autorita,} která vytvoří jeden soubor pravidel, který bude vynucován na všech. Etatisté samozřejmě doufají, že \enquote{autorita} bude vydávat a prosazovat \emph{správná} pravidla, ale nikdy nevysvětlí, jak a proč by se to mělo stát. Vzhledem k tomu, že nařízení \enquote{státu} píší pouhé lidské bytosti -- obvykle výjimečně zkorumpované lidské bytosti toužící po moci -- proč by měl někdo očekávat, že jejich \enquote{pravidla} budou lepší než \enquote{pravidla,} která by si každý jednotlivec zvolil sám?

Víra ve stát nezajistí, aby se všichni shodli; pouze vytváří příležitost k drastickému přerůstání osobních neshod do rozsáhlých válek a masového útlaku. Ani přítomnost \enquote{autority,} která spor řeší, nijak nezaručuje, že zvítězí \enquote{správná} strana. Přesto etatisté mluví, jako by \enquote{stát} byl spravedlivý, rozumný a racionální v situacích, kdy by jednotlivci takoví nebyli. To opět ukazuje, že věřící ve stát si představují \enquote{autoritu} jako nadlidské ctnosti, kterým je třeba důvěřovat více než ctnostem obyčejných smrtelníků. Historie ukazuje opak. Pokřivený smysl pro morálku jednoho člověka nebo několika málo lidí může vést k vraždě jednoho člověka nebo dokonce desítek lidí, ale tentýž pokřivený smysl pro morálku klidně jen několika lidí, když se dostanou k mašinérii zvané \enquote{stát,} může vést k vraždě milionů lidí. Etatista chce, aby jeho představa \enquote{dobrých pravidel} byla vnucena všem centrální \enquote{autoritou,} ale nemá žádný způsob, jak to uskutečnit, a nemá žádný důvod očekávat, že se tak stane. Při hledání všemocného \enquote{klaďase,} který by zachránil situaci, etatisté vždy nakonec vytvoří všemocné záporáky. Znovu a znovu vytvářejí obrovská, nezastavitelná \enquote{státní} monstra v naději, že budou bránit nevinné, jen aby zjistili, že se tato monstra stávají pro nevinné mnohem větší hrozbou než hrozby, před nimiž je měli chránit.

Ironií je, že etatisté ve své snaze zajistit spravedlnost pro všechny ve skutečnosti prosazují legitimizaci zla. Pravdou je, že jediné, co víra v autoritu dělá a co může dělat, je \emph{vnášet do společnosti více nemorálního násilí}. To není nešťastná náhoda ani vedlejší účinek v zásadě dobré myšlenky. Je to truismus vycházející z podstaty víry v autoritu, což lze snadno logicky dokázat.

\section{Víc nemorálního násilí}

Téměř všichni souhlasí s tím, že někdy je fyzická síla oprávněná a někdy ne. Ačkoli se lidé mohou přít o detaily a \enquote{šedé zóny,} obecně se uznává, že \emph{agresivní} síla -- iniciování násilí proti jiné osobě -- je neoprávněná a nemorální. Patří sem krádeže, napadení a vraždy, ale i nepřímé formy agrese, jako je vandalismus a podvod. Naproti tomu použití síly na obranu nevinných je všeobecně přijímáno jako oprávněné a morální, dokonce ušlechtilé. Oprávněnost použití síly je dána situací, v níž je použita, nikoli tím, \emph{kdo} ji používá. Zjednodušeně řečeno, sílu, kterou má právo použít každý, lze označit za \enquote{dobrou sílu} a sílu, kterou nemá právo použít nikdo, za \enquote{špatnou sílu.} (Čtenář může použít svá vlastní měřítka, a logika zde bude stále platit.)

Kvůli víře v autoritu si představujeme, že zástupci \enquote{státu} mají právo použít sílu nejen v situacích, kdy by takové právo měl každý, ale i v jiných situacích, například při použití síly k výběru \enquote{daní.} Je logické, že pokud má každý právo použít \enquote{dobrou sílu,} ale \enquote{zákon} údajně opravňuje představitele \enquote{státu} použít sílu i v \emph{jiných} situacích, pak \enquote{zákon} není nic jiného než pokus o \emph{legitimizaci} špatné síly. Stručně řečeno, \enquote{autorita} je povolení páchat zlo -- dělat věci, které by byly uznány za nemorální a neoprávněné, kdyby je dělal kdokoli jiný.

Tuto skutečnost samozřejmě nechápe ani nadšený volič, který si hrdě vyvěsí na dvorek kamaňovou ceduli, ani občan s dobrými úmysly, který \enquote{kandiduje.} Kdyby to věděli, pochopili by, že \enquote{demokracie} není nic jiného než většinově schválené nemorální násilí a že nemůže napravit společnost ani být nástrojem svobody či spravedlnosti. Navzdory mytologii, která tvrdí, že hlas člověka je jeho \enquote{hlasem} a že právo volit činí lidi svobodnými, je pravda taková, že jediné, co \enquote{demokracie} dělá, je legitimizace agrese a neoprávněného násilí. Logika tohoto jevu je tak jednoduchá a zřejmá, že je zapotřebí obrovského množství propagandy, aby se lidé naučili ji \emph{nevidět}. Pokud má každý právo použít ve své podstatě spravedlivou sílu a představitelé \enquote{státu} mohou použít \enquote{sílu} i v \emph{jiných} situacích, pak to, co \enquote{stát} přidává společnosti, je ze své podstaty nemorální násilí.

Problém spočívá v tom, že lidé se učí, že když je násilí \enquote{legální} a páchá ho \enquote{autorita,} mění se z nemorálního násilí na spravedlivé \enquote{vymáhání práva.} Základním předpokladem, na němž staví veškerý \enquote{stát,} je myšlenka, že to, co by bylo morálně špatné pro průměrného člověka, může být morálně \emph{správné}, pokud to dělají zástupci \enquote{autority,} což naznačuje, že normy morálního chování, které platí pro lidské bytosti, \emph{neplatí} pro zástupce \enquote{státu} (opět narážka na to, že věc zvaná \enquote{stát} je nadlidská). Ve své podstatě spravedlivá síla, která je, jak se většina lidí obecně shoduje, omezena na obrannou sílu, nevyžaduje žádný \enquote{zákon} nebo zvláštní \enquote{autoritu,} aby byla platná. \enquote{Zákon} a \enquote{stát} jsou potřeba jedině ke snaze legitimizovat nemorální sílu. A to je přesně to, co \enquote{stát} společnosti dává, to jediné co dává: více z podstaty nespravedlivého násilí. Nikdo, kdo tuto jednoduchou pravdu chápe, by nikdy netvrdil, že \enquote{stát} je pro lidskou civilizaci nezbytný.

Představa, že \enquote{zákon} vytvořený člověkem obchází obvyklá pravidla civilizovaného chování, má docela děsivé důsledky. Pokud \enquote{stát} není omezen základní lidskou morálkou, což samotný pojem \enquote{autorita} předpokládá, jakými normami či principy by mělo být jednání \enquote{státu} vůbec omezeno? Jestliže platí 30\% \enquote{zdanění,} proč by nemělo platit 100\% \enquote{zdanění?} Je-li \enquote{legální} krádež legitimní a spravedlivá, proč by nemohlo být legitimní a spravedlivé \enquote{legalizované} mučení a vražda? Pokud nějaká \enquote{kolektivní potřeba} vyžaduje, aby společnost měla instituci, která má výjimku z morálky, proč by měla mít nějaká omezení toho, co může dělat? Pokud je vyhlazení celé rasy, zákaz náboženství nebo násilné zotročení milionů lidí považováno za nezbytné pro \enquote{obecné dobro,} podle jakých morálních norem by si mohl někdo stěžovat, když už přijal předpoklad \enquote{autority?} Veškerá víra ve stát spočívá na myšlence, že \enquote{obecné dobro} ospravedlňuje \enquote{legální} zahájení násilí proti nevinným v té či oné míře. A jakmile byla tato premisa přijata, neexistuje žádný objektivní morální standard, který by omezoval chování \enquote{státu.} Historie to ukazuje až příliš jasně.

Téměř každý přijímá mýtus, že lidé nejsou dostatečně důvěryhodní, morální a moudří, aby mohli v míru existovat bez \enquote{státu,} který by je držel na uzdě. Dokonce i mnozí, kteří souhlasí s tím, že v ideální společnosti by nebyli žádní vládci, často zastávají názor, že lidské bytosti nejsou na takovou společnost \enquote{připraveny.} Takové názory vycházejí ze zásadního nepochopení toho, co je \enquote{autorita} a co společnosti přináší. Představa \enquote{státu} jako \enquote{nutného zla} (jak jej popsal Patrick Henry) naznačuje, že existence \enquote{státu} omezuje násilnou, agresivní povahu lidských bytostí, zatímco ve skutečnosti činí pravý opak: víra v autoritu legitimizuje a \enquote{legalizuje} agresi.

Bez ohledu na to, jak hloupé nebo moudré jsou lidské bytosti, jak zlovolné nebo ctnostné mohou být, říkat, že lidské bytosti nejsou \enquote{připraveny} na bezstátní společnost nebo že jim nelze \enquote{věřit,} že by mohly existovat bez \enquote{autority,} které se podřizují, znamená tvrdit, že mírová civilizace může existovat pouze tehdy, pokud existuje obrovská, mocná mašinérie, která do společnosti vnáší obrovské množství nemorálního násilí. Etatisté samozřejmě násilí jako nemorální neuznávají, protože pro ně násilí nepáchají obyčejní smrtelníci, ale zástupci boha známého jako \enquote{stát,} a bohové mají právo dělat věci, které smrtelníci nemají. Když se toto téměř všeobecně rozšířené přesvědčení -- že je nutné zavést do společnosti nemorální násilí, aby se zabránilo lidem páchat nemorální násilí -- popíše přesně a doslovně, odhalí se jako zjevně absurdní mýtus, kterým je. Ale každý, kdo věří v mýtus státu, musí věřit přesně tomu. Nevěří tomu v důsledku racionálního myšlení a logiky; přijímají to jako prvek víry, protože je to součástí nezpochybnitelné doktríny \enquote{státní} církve.

\section{Kdo jim dal právo?}

Existuje několik způsobů, jak ukázat, že mytologie, kterou se veřejnost učí o \enquote{státu,} je vnitřně rozporná a iracionální. Jedním z nejjednodušších způsobů je položit si otázku: Jak někdo získá právo vládnout druhému? Staré pověry tvrdily, že někteří lidé byli výslovně určeni bohem nebo skupinou bohů, aby vládli ostatním. Různé legendy vyprávějí o nadpřirozených událostech (Dáma v jezeře, Meč v kameni atd.), které určovaly, kdo bude mít právo vládnout ostatním. Naštěstí lidstvo tyto hloupé pověry z větší části překonalo. Bohužel je nahradily nové pověry, které jsou ještě méně racionální.

Staré mýty přinejmenším připisovaly nějaké \enquote{vyšší moci} úkol jmenovat určité lidi vládci nad ostatními -- něco, co by božstvo mohlo alespoň teoreticky udělat. Nová zdůvodnění \enquote{autority} však tvrdí, že stejného úžasného výkonu lze dosáhnout i bez nadpřirozené pomoci. Stručně řečeno, navzdory všem složitým rituálům a spletitým zdůvodněním spočívá veškerá moderní víra ve stát na představě, že obyčejní smrtelníci mohou prostřednictvím určitých politických procedur udělit některým lidem různá práva, která na počátku nikdo z lidí neměl. Bláznivost takové představy by měla být zřejmá. Neexistuje žádný rituál nebo dokument, kterým by nějaká skupina lidí mohla někomu jinému delegovat právo, které nikdo z této skupiny nemá. A tato samozřejmá pravda sama o sobě boří jakoukoli možnost legitimního \enquote{státu.}

Průměrný člověk věří, že \enquote{stát} má právo dělat věci, na které průměrný jednotlivec právo nemá. Je tedy zřejmé, že otázka zní: Jak a od koho získali lidé, kteří pracují pro \enquote{stát,} taková práva? Například -- ať už tomu říkáte \enquote{krádež} nebo \enquote{zdanění} -- jak získali \enquote{státní} zaměstnanci právo násilně odebírat majetek těm, kteří si na něj vydělali? Žádný volič takové právo nemá. Jak tedy mohli voliči takové právo politikům udělit? Dnes je veškerý etatismus založen výhradně na předpokladu, že \emph{lidé mohou delegovat práva, která nemají}. Dokonce i americká ústava předstírala, že dává \enquote{Kongresu} právo \enquote{zdaňovat} a \enquote{regulovat} určité věci, ačkoli autoři ústavy sami žádné takové právo neměli, a proto nemohli takové právo nikomu jinému dát.

Protože každý člověk má právo \enquote{vládnout} sám sobě (jakkoli je tato myšlenka schizofrenní), může, alespoň teoreticky, k tomu zmocnit někoho jiného. Ale právo vládnout ostatním lidem je právem, které nemá a nemůže jím ani nikoho jiného pověřit. A kdyby \enquote{stát} vládl pouze těm jednotlivcům, z nichž každý dobrovolně delegoval své právo vládnout sám sobě, nebyl by to stát.

A počet zúčastněných osob nemá na logiku vliv. Tvrdit, že většina může někomu udělit právo, které nikdo z jednotlivců v této většině nemá, je stejně iracionální jako tvrdit, že tři lidé, z nichž žádný nemá auto ani peníze na jeho koupi, mohou někomu darovat auto. Jednoduše řečeno, nemůžete někomu dát něco, co nemáte. A tato jednoduchá pravda sama o sobě vylučuje \emph{veškerý} \enquote{stát,} protože pokud mají ti, kdo řídí \enquote{stát,} pouze ta práva, která mají ti, kdo je zvolili, pak \enquote{stát} ztrácí jedinou složku, která jej činí \enquote{státem:} právo vládnout nad ostatními (\enquote{autoritu}). Pokud má stejná práva a pravomoci jako všichni ostatní, není důvod nazývat jej \enquote{státem.} Pokud politici nemají o nic víc práv než vy, všechny jejich požadavky a příkazy, všechny jejich politické rituály, sbírky \enquote{zákonů,} soudy atd. se nepodobají ničemu jinému než symptomům hluboké bludné psychózy. Nic z toho, co dělají, nemůže mít žádnou legitimitu, stejně jako kdybyste totéž dělali vy sami, ledaže by nějakým způsobem získali práva, která vy nemáte. A to je nemožné, protože nikdo na světě a žádná skupina lidí na světě jim taková nadlidská práva nemohla dát.

Žádný politický rituál nemůže změnit morálku. Žádné volby nemohou udělat ze zlého činu dobrý čin. Pokud je pro vás špatné něco dělat, pak je špatné, aby to dělali ti, kdo pro stát pracují.

A pokud stejná morálka, která platí pro vás, platí i pro \enquote{státní} zaměstnance -- pokud ti, kteří jsou ve \enquote{veřejné funkci,} nemají více práv než vy -- pak \enquote{stát} přestává být státem. Budeme-li je posuzovat podle stejných měřítek jako ostatní smrtelníky, pak ti, kdo nesou označení \enquote{stát,} jsou jen bandou násilníků, teroristů, zlodějů a vrahů a jejich činy postrádají jakoukoli legitimitu či \enquote{autoritu.} Není to nic jiného než banda glázlů, kteří trvají na tom, že různé dokumenty a rituály jim dávají \emph{právo} být grázly. Bohužel jim to věří i většina jejich obětí.

\section{Určování morálky}

Pojem \enquote{autorita} závisí na pojmech dobra a zla (tj. morálky). Mít \enquote{autoritu} neznamená pouze mít \emph{schopnost} násilně ovládat jiné lidi, což je něco, co mají nesčetní násilníci, zloději a gangy, kteří nejsou označováni jako \enquote{autorita;} znamená to mít \emph{právo} ovládat jiné lidi, což znamená, že poddaní mají morální \emph{povinnost} poslouchat, a to nejen proto, aby se vyhnuli trestu, ale také proto, že taková poslušnost (\enquote{dodržování zákona}) je morálně \emph{dobrá} a neposlušnost (\enquote{porušování zákona}) je morálně \emph{špatná}. Aby tedy existovalo něco jako \enquote{autorita,} musí existovat něco jako dobro a zlo. (Na tom, jak kdo definuje dobro a zlo nebo co považuje za zdroj morálky, pro účely této diskuse nijak zvlášť nezáleží. Použijte své vlastní definice a logika bude platit i nadále.) Pojem \enquote{autorita} sice vyžaduje existenci dobra a zla, ale zároveň se zcela vylučuje s existencí dobra a zla. Toto zdánlivě podivné tvrzení dokáže jednoduchá analogie.

Matematické zákony jsou objektivní, neměnnou součástí reality. Když ke dvěma jablkům přičtete dvě jablka, získáte čtyři jablka. Ti, kdo studují matematiku, se snaží pochopit více o realitě, dozvědět se o tom, co už je. Ten, kdo by vstoupil na pole matematiky s jasným cílem změnit matematické zákony, by byl považován za blázna, a to právem. Představte si, jak absurdní by bylo, kdyby nějaký profesor matematiky prohlásil: \enquote{Tímto vyhlašuji, že od nynějška se dvě plus dvě bude rovnat pěti.} Přesto k takovému šílenství dochází pokaždé, když politici přijímají \enquote{zákony.} Nejenže pozorují svět a snaží se co nejlépe určit, co je správné a co ne -- to by měl a musí udělat každý jednotlivec sám za sebe. Ne, oni tvrdí, že \emph{mění} morálku tím, že vydávají nějaké nové nařízení. Jinými slovy, podobně jako šílený profesor matematiky, který si myslí, že pouhým prohlášením může \emph{přimět}, aby se dvě plus dvě rovnalo pěti, politici mluví a jednají, jako by byli \emph{zdrojem} morálky, jako by měli moc \emph{vymýšlet} (prostřednictvím \enquote{zákonů}), co je správné a co špatné, jako by se nějaký čin mohl stát špatným jen proto, že ho prohlásili za \enquote{nezákonný.}

Ať už jde o matematiku, morálku nebo cokoli jiného, je obrovský rozdíl mezi snahou \emph{poznat}, co je pravda, a snahou \emph{diktovat}, co je pravda. To první je užitečné, to druhé je šílené. A to druhé je to, co \enquote{státní} zaměstnanci předstírají každý den. Politici ve svých \enquote{zákonech} nevyjadřují pouze to, jak by se podle nich měli lidé chovat, a to na základě univerzálních norem morálky. Každý má právo říci: \enquote{Myslím si, že dělat tuhle věc je špatné a dělat tamtu věc je dobré,} ale nikdo by takové názory nenazval \enquote{zákony.} Místo toho je poselství politiků: \enquote{My \emph{děláme} tuhle věc špatnou a \emph{děláme} tuhle věc dobrou.} Stručně řečeno, každý \enquote{zákonodárce} trpí hluboce bludným božským komplexem, který ho vede k přesvědčení, že prostřednictvím politických rituálů má spolu se svými kolegy \enquote{zákonodárci} skutečně moc \emph{změnit} dobro a zlo, a to pouhým nařízením.

Smrtelníci nemohou změnit morálku stejně jako nemohou změnit zákony matematiky. Jejich chápání něčeho se může změnit, ale nemohou svým nařízením změnit podstatu vesmíru. Nikdo rozumný by se o to ani nepokoušel. Přesto se za to každý nový \enquote{zákon} přijatý politiky vydává: \emph{změna} toho, co představuje morální chování. A jakkoli je tato představa absurdní, je nezbytným prvkem víry ve stát: představa, že masy jsou morálně povinny poslouchat \enquote{tvůrce zákonů} -- že neuposlechnutí (\enquote{porušení zákona}) je morálně špatné -- nikoli proto, že příkazy politiků náhodou odpovídají objektivním pravidlům morálky, ale proto, že jejich příkazy \emph{diktují} a \emph{určují}, co je morální a co ne.

Pochopení prostého faktu, že obyčejní smrtelníci nemohou z dobra udělat zlo a ze zla dobro, samo o sobě způsobuje, že se mýtus státu rozpadá. Každý, kdo plně pochopí tuto jedinou jednoduchou pravdu, nemůže nadále věřit ve stát, protože pokud politici postrádají takovou nadpřirozenou moc, jejich příkazy nemají žádnou přirozenou legitimitu a přestávají být \enquote{autoritou.} Pokud dobro není tím, co politici říkají, že je -- pokud dobro a zlo skutečně nevychází z rozmarů politikobohů -- pak nikdo nemůže mít žádnou morální povinnost respektovat nebo poslouchat příkazy politiků a jejich \enquote{zákony} se stávají naprosto neplatnými a irelevantními. Stručně řečeno, pokud vůbec existuje něco jako dobro a zlo, ať už si tyto pojmy definujete jakkoli, pak jsou \enquote{státní zákony} vždy nelegitimní a bezcenné.

Každý člověk je (z definice) morálně povinen dělat to, co považuje za správné. Pokud mu nějaký \enquote{zákon} nařizuje, aby jednal jinak, je tento \enquote{zákon} ze své podstaty nelegitimní a měl by být neuposlechnut. A pokud se \enquote{zákon} náhodou shoduje s tím, co je správné, je \enquote{zákon} prostě irelevantní. Důvodem, proč se například zdržet vraždy, je to, že vražda je ze své podstaty špatná. To, zda nějací politici přijali \enquote{zákon,} který prohlašuje vraždu za špatnou -- zda ji \enquote{postavili mimo zákon} -- nemá na morálnost tohoto činu žádný vliv. \enquote{Legislativa,} ať už říká cokoli, nikdy není \emph{důvodem}, proč je něco dobré nebo špatné. V důsledku toho jsou i \enquote{zákony} zakazující zlé činy, jako je napadení, vražda a krádež, nelegitimní. Lidé by se sice takových činů neměli dopouštět, ale proto, že tyto činy jsou samy o sobě špatné, nikoli proto, že \enquote{zákony} vytvořené člověkem říkají, že jsou špatné. A pokud neexistuje povinnost dodržovat \enquote{zákony} politiků, pak z definice nemají žádnou \enquote{autoritu.}

Vrátíme-li se k analogii s profesorem matematiky, pak pokud by profesor autoritativně prohlásil, že pouhým dekretem dosáhne toho, že dvě plus dvě se bude rovnat pěti, každý rozumný člověk by tento dekret považoval za nesprávný a bludný. Kdyby naopak profesor prohlásil, že se chystá \emph{přimět}, aby se dva plus dva rovnaly čtyřem, bylo by takové prohlášení stále hloupé a nesmyslné, i když se dva plus dva \emph{rovná} čtyřem. Profesorovo prohlášení není \emph{důvodem}, proč se to rovná čtyřem. Tak či onak, profesorovo prohlášení by nemělo žádný vliv na schopnost lidí sčítat dvě a dvě. A tak je to i se \enquote{zákony} politiků: bez ohledu na to, zda se skutečně shodují s objektivním dobrem a zlem, nikdy nemají \enquote{autoritu,} protože nikdy nejsou \emph{zdrojem} dobra a zla, nikdy nikomu \emph{neukládají} povinnost chovat se určitým způsobem, a neměly by tedy mít žádný vliv na to, co kterýkoli jednotlivec posoudí jako morální či nemorální.

Vezměme si příklad \enquote{zákony} zakazující drogy. Domnívat se, že je špatné použít násilí proti někomu za to, že si dal pivo (což je \enquote{legální}), ale je dobré, aby \enquote{strážci zákona} použili násilí proti někomu, kdo kouří trávu (protože je to \enquote{nelegální}), logicky znamená, že politici skutečně mají schopnost měnit morálku -- vzít dvě v podstatě stejná chování a z jednoho udělat nemorální čin, který dokonce ospravedlňuje násilnou odplatu. Navíc pokud někdo připustí legitimitu \enquote{zákonů} (příkazů politiků), musí také připustit, že pití alkoholu bylo jeden den naprosto morální, ale druhý den -- v den, kdy byla uzákoněna \enquote{prohibice} -- bylo nemorální. Poté, o nemnoho let později, to bylo jeden den nemorální a druhý den morální -- v den, kdy byla prohibice zrušena.

Dokonce ani bohové většiny náboženství si nenárokují moc neustále měnit a revidovat svá přikázání, pravidelně \emph{měnit} to, co je správné a co ne. Takovou moc si nárokují pouze politici. Každý akt \enquote{legislativy} zahrnuje takové šílenství: představu, že jeden den by nějaký čin mohl být naprosto přípustný a hned druhý den -- v den, kdy by byl \enquote{zakázán} -- by byl nemorální.

\section{Nelze neusuzovat}

Téměř všichni jsou vedeni k tomu, že pro civilizaci je nejdůležitější respekt k \enquote{zákonu} a že dobří lidé jsou ti, kteří \enquote{hrají podle pravidel,} což znamená, že plní příkazy vydané \enquote{státem.} Ve skutečnosti jsou však morálka a poslušnost často přímými protiklady. Bezmyšlenkovité podřizování se jakékoli \enquote{autoritě} pro lidstvo představuje největší zradu, jaká může existovat, protože se snaží odhodit svobodnou vůli a individuální úsudek, které nás činí lidmi a činí nás schopnými morálky, ve prospěch slepé poslušnosti, která redukuje lidské bytosti na nezodpovědné roboty. Víra v autoritu -- myšlenka, že jedinec má někdy povinnost ignorovat svůj vlastní úsudek a rozhodovací proces ve prospěch poslušnosti někomu jinému -- není jen špatná myšlenka; je vnitřně rozporná a absurdní. Hluboké šílenství, které s tím souvisí, lze shrnout následovně:

\enquote{\emph{Myslím, že je dobré dodržovat zákony. Jinými slovy, usuzuji, že bych měl dělat to, co mi zákonodárci přikazují. Jinými slovy, usuzuji, že místo abych sám rozhodoval o tom, co bych měl dělat, měl bych se podřídit vůli těch, kteří jsou ve vládě. Jinými slovy, usuzuji, že je lepší, aby se mé jednání řídilo úsudkem lidí u moci, než mým osobním úsudkem. Jinými slovy, usuzuji, že je správné, abych se řídil úsudkem druhých, a špatné, abych se řídil svým vlastním úsudkem. Jinými slovy, usuzuji, že bych neměl usuzovat.}}

Pokaždé, když dojde ke konfliktu mezi vlastním svědomím a tím, co přikazuje \enquote{zákon,} existují pouze dvě možnosti: buď by se člověk měl řídit svým svědomím bez ohledu na to, co říká takzvaný \enquote{zákon,} nebo je povinen poslouchat \enquote{zákon,} i když to znamená dělat to, co on osobně považuje za špatné. Bez ohledu na to, zda je úsudek jednotlivce chybný, či nikoli, je schizofrenní šílenství, když člověk věří, že je pro něj \emph{dobré} dělat to, co považuje za špatné. Přesto je to základem víry v autoritu. Pokud člověk chápe skutečnost, že každý jedinec je povinen vždy a všude dělat to, co považuje za správné, pak nemůže mít žádnou morální povinnost poslouchat nějakou vnější \enquote{autoritu.} Opět platí, že pokud se \enquote{zákon} shoduje s úsudkem jednotlivce, je \enquote{zákon} irelevantní. Pokud je naopak \enquote{zákon} v rozporu s jeho individuálním úsudkem, pak je třeba na \enquote{zákon} pohlížet jako na nelegitimní. Ať tak či onak, \enquote{zákon} nemá žádnou \enquote{autoritu.}

(Povinnost poslouchat \enquote{autoritu} není totéž to stejné, jako kdžyž lidé dobrovolně přizpůsobí své chování zájmu mírového soužití. Například člověk si může myslet, že má plné právo pouštět si hudbu na svém dvorku, ale přesto se může rozhodnout, že to na žádost svého souseda neudělá. Nebo může člověk změnit způsob oblékání, mluvení a chování, když navštíví nějakou jinou kulturu nebo prostředí, kde by jeho obvyklé chování mohlo ostatní pohoršovat. Existuje mnoho faktorů, které mohou ovlivnit něčí názor na to, co by měl nebo neměl dělat. Přiznat si, že \enquote{autorita} je mýtus není vůbec totéž jako nezajímat se o to, co si myslí ostatní. Souhlasit s různými zvyky, etickými zásadami a dalšími společenskými normami v zájmu toho, abychom spolu vycházeli a vyhnuli se konfliktům, je často naprosto racionální a užitečné. Racionální není, když se někdo cítí morálně zavázán dělat něco, co sám o sobě za daných okolností nepovažuje za správné.)

Řečeno na rovinu, víra v autoritu slouží jako mentální berlička pro lidi, kteří se snaží uniknout odpovědnosti kterou nesou za to, že jsou myslícími lidmi. Je to pokus přenést odpovědnost za rozhodování na někoho jiného: na ty, kteří si nárokují \enquote{autoritu.} Snaha vyhnout se odpovědnosti tím, že \enquote{prostě plníme rozkazy,} je však hloupá, protože každý člověk se musí \emph{rozhodnout} udělat to, co mu bylo řečeno. Dokonce i to, co se jeví jako slepá poslušnost, je stále výsledkem toho, že se člověk \emph{rozhodl} být poslušný. Nevybrat si nic není možné. Nebo, jak to vyjádřila kapela Rush ve své písni \enquote{Free Will:} \enquote{\emph{I když se rozhodneš nerozhodnout se, stejně ses rozhodnul.}}

Výmluva \enquote{pouze jsem plnil rozkazy} se elegantně vyhýbá skutečnosti, že se dotyčný musel nejprve rozhodnout, že bude poslouchat \enquote{autoritu.} I když nějaká \enquote{autorita} prohlásí: \enquote{Musíš mě poslouchat,} jak tvrdí nespočet protichůdných \enquote{autorit,} jedinec si stejně musí vybrat, které z nich, pokud vůbec nějaké, uvěří. Skutečnost, že většina lidí o takových věcech příliš nepřemýšlí, nic nemění na tom, že měli možnost neuposlechnout, a jsou tedy za své činy plně odpovědní -- právě té odpovědnosti, které je chtěli \enquote{autority} zbavit. Není možné neusuzovat, není možné se nerozhodnout. Aby člověk předstíral, že někdo nebo něco jiného rozhodlo za něj -- že se na rozhodnutí nijak nepodílel, a tudíž nenese žádnou odpovědnost za výsledek -- je naprosto šílené. Loajální poslušnost vůči \enquote{autoritám,} ačkoli je mnohými vykreslována jako velká ctnost, není ve skutečnosti ničím jiným než ubohým pokusem uniknout odpovědnosti za to, že je člověk, a redukovat se na nemyslící, amorální, programovatelný stroj.

Každý se vždy rozhoduje sám a je za to osobně odpovědný. Dokonce i ti, kteří halucinují \enquote{autoritu,} se stále rozhodují pro víru a poslušnost a jsou stále odpovědní za to, že tak učinili. \enquote{Autorita} je pouhý klam, díky němuž si lidé představují, že je možné vyhnout se odpovědnosti tím, že pouze udělají, co jim bylo řečeno. Nebo, abychom to vyjádřili osobněji:

Vaše činy jsou vždy určovány výhradně vaším vlastním úsudkem a vašimi vlastními rozhodnutími. Snažit se své chování přičítat nějaké vnější síle, například \enquote{autoritě,} je zbabělé a nepoctivé. \emph{Vy} jste se rozhodli a nesete za to odpovědnost. I kdybyste jen hloupě poslechli nějakou samozvanou \enquote{autoritu,} \emph{vy} jste se tak rozhodli. Tvrzení, že za vás rozhodovalo něco mimo vás -- tvrzení, že jste neměli na výběr; že jste se museli podřídit \enquote{autoritě} -- je zbabělá lež.

Neexistuje žádná zkratka k určení pravdy, ať už jde o morálku nebo cokoli jiného. Až příliš často se základem lidského systému víry stává toto: \enquote{Abych věděl, co je pravda, stačí se zeptat mé neomylné autority; a já vím, že moje autorita má vždycky pravdu, protože mi \emph{říká}, že má vždycky pravdu.} Samozřejmě bude vždy existovat nespočet konkurenčních, protichůdných \enquote{autorit} a každá z nich se bude prohlašovat za zdroj pravdy. Není tedy jen dobré, aby lidé sami posuzovali, co je pravda a co ne; je to zcela nevyhnutelné. Dokonce i ti, kteří považují za velkou ctnost mít systém přesvědčení -- politický, náboženský nebo jiný -- založený na \enquote{víře,} si neuvědomují, že pouze jednotlivec se může rozhodnout, \emph{čemu} bude věřit. Ať už si to chce připustit, nebo ne, vždy je tím, kdo rozhoduje v konečném důsledku; vždy se na základě vlastního úsudku rozhoduje, čemu věřit a co dělat.

\chapter{Účinky pověry}

\section{Účinky mýtu}

Lidé se po celé věky podléhali nejrůznějším pověrám a falešným domněnkám, z nichž mnohé byly relativně neškodné. Například když většina lidí věřila, že Země je placatá, neměla tato fakticky nesprávná představa téměř žádný vliv na to, jak lidé žili svůj každodenní život nebo jak se k sobě navzájem chovali. Stejně tak, pokud děti věří v zubní vílu nebo v to, že čápi rodí děti, nestanou se z nich v důsledku přijetí těchto mýtů šiřitelé zla. Na druhou stranu v průběhu let představovaly jiné falešné domněnky a mýty pro lidstvo skutečné nebezpečí. Mohlo jít o prosté nedorozumění mezi lékaři, které je vedlo k tomu, že zkoušeli \enquote{léky,} které pro jejich pacienty představovaly větší hrozbu než nemoci, které se snažili léčit. Drastičtějším příkladem jsou některé kultury, které přinášely lidské oběti v naději, že si tak získají přízeň svých imaginárních bohů.

Nic jiného se však nepřibližuje míře destrukce -- duševní, emocionální i fyzické -- k níž došlo na celém světě a v celé zaznamenané historii v důsledku víry v autoritu. Tím, že mýtus autority dramaticky mění to, jak lidé vnímají svět, mění i jejich myšlení a jednání. Víra v legitimitu vládnoucí třídy (\enquote{státu}) totiž vede téměř každého k tomu, že buď schvaluje, nebo páchá činy zla, aniž by si to uvědomoval. Po přesvědčení, že \enquote{autorita} je skutečná a že díky ní některé lidské bytosti získaly morální právo iniciovat násilí a páchat akty agrese vůči ostatním (prostřednictvím takzvaných \enquote{zákonů}), je každý demokrat, každý republikán, každý volič a každý, kdo obhajuje \enquote{stát} v jakékoli podobě, zastáncem násilí a bezpráví. Samozřejmě to tak nevidí, protože jejich víra v autoritu pokřivila a zvrátila jejich vnímání reality.

Potíž je v tom, že když nám něco změní vnímání reality, člověk si toho málokdy všimne. Například pro člověka, který nosí barevné kontaktní čočky, může svět vypadat úplně jinak, i když samotné čočky nevidí. Totéž platí pro mentální \enquote{čočky.} Každý člověk si myslí, že svět je skutečně takový, jak ho vidí. Každý může ukázat na ostatní a tvrdit, že jsou mimo realitu, ale téměř nikdo si nemyslí, že jeho vlastní vnímání je zkreslené, i když mu to ostatní tvrdí. Výsledkem jsou miliardy lidí, kteří na sebe navzájem ukazují prstem a říkají si, jak jsou bludní a pomýlení, přičemž téměř nikdo z nich není ochoten nebo dokonce schopen poctivě prozkoumat mentální \enquote{čočky,} které zkreslují jejich vlastní vnímání.

Vše, čemu byl člověk vystaven, zejména v mládí, má vliv na jeho pohled na svět. To, co ho naučili rodiče, co se naučil ve škole, jak viděl chování lidí, kultura, ve které vyrůstal, náboženství, ve kterém byl vychováván, to vše vytváří dlouhodobý soubor mentálních \enquote{čoček,} které ovlivňují jeho pohled na svět. Existuje nespočet příkladů toho, jak pouhé rozdíly v perspektivě vedly k hrozivým následkům. Sebevražedný atentátník, který úmyslně zabíjí civilisty, si představuje, že dělá správnou věc. Téměř každý na obou stranách každé války si představuje, že je v právu. Nikdo si nepředstavuje, že je ten zlý. Vojenské konflikty jsou výhradně výsledkem rozdílů v perspektivě, které jsou důsledkem mentálních \enquote{čoček,} jež byly vycvičeny u vojáků na obou stranách. Mělo by být samozřejmé, že kdyby tisíce v zásadě dobrých lidí viděly svět takový, jaký je, nesnažily by se zoufale zabíjet jeden druhého. Ve většině případů není problém ve skutečném zlu nebo zlovůli, ale prostě v neschopnosti vidět věci tak, jak jsou.

Vezměme si jako analogii někoho, kdo požil silný halucinogen a v důsledku toho nabyl přesvědčení, že jeho nejlepší přítel je ve skutečnosti zákeřné mimozemské monstrum v přestrojení. Z pohledu toho, kdo má halucinace, je násilný útok na jeho přítele naprosto rozumný a oprávněný. Problém v případě člověka, jehož vnímání reality bylo takto zkresleno, nespočívá v tom, že by byl nemorální, hloupý nebo zlovolný. Problém je v tom, že nevidí věci tak, jak ve skutečnosti jsou, a v důsledku toho jsou rozhodnutí a činy, které se mu zdají naprosto přiměřené, ve skutečnosti strašlivě destruktivní. A když takovou halucinaci sdílí mnoho lidí, jsou následky ještě mnohem horší.

Když mají všichni stejně mylné vnímání reality -- když všichni věří něčemu nepravdivému, dokonce i něčemu zjevně absurdnímu -- nepřipadá jim to nepravdivé nebo absurdní. Když nepravdivou nebo nelogickou myšlenku neustále opakují a posilují téměř všichni, málokdy někoho napadne ji začít zpochybňovat. Ve skutečnosti se většina lidí stává doslova neschopnou ji zpochybnit, protože se časem v jejich myslích upevní jako samozřejmost -- předpoklad, který nepotřebuje racionální základ a není třeba ho analyzovat nebo přehodnocovat, protože všichni vědí, že je pravdivý. Ve skutečnosti však každý člověk prostě předpokládá, že je to pravda, protože si nedokáže představit, že by všichni ostatní -- včetně všech těch vážených, známých a vzdělaných lidí z rádia a televize -- mohli všichni věřit něčemu nepravdivému. Co je jednomu průměrnému jedinci do toho, že pochybuje o něčem, co všichni ostatní zřejmě naprosto bez problémů přijímají jako nezpochybnitelnou pravdu?

Takto hluboce zakořeněná víra je pro ty, kdo jí věří, neviditelná. Když si mysl vždy myslela něco jedním způsobem, bude si představovat důkazy a halucinovat zkušenosti, které tuto myšlenku podporují. Před tisíci lety by lidé s jistotou prohlašovali, že je prokázáno, že Země je placatá, a říkali by to se stejnou jistotou a upřímností, s jakou my nyní prohlašujeme, že je kulatá. Představa, že svět je obrovská kulovitá věc, která se vznáší ve vesmíru a není k ničemu připoutána, jim připadala zjevně směšná. A jejich naprosto falešný předpoklad, že svět je plochý, by jim připadal jako vědecký, samozřejmý fakt.

Stejně tak je to s vírou v autoritu a \enquote{stát.} Většině lidí připadá \enquote{stát} jako samozřejmá realita, stejně racionální a samozřejmá jako gravitace. Málokdo tento pojem někdy objektivně zkoumal, protože k tomu nikdy neměl důvod. \enquote{Každý ví,} že \enquote{stát} je skutečný, nutný, legitimní a nevyhnutelný. Všichni předpokládají, že tomu tak je, a mluví, jako by tomu tak bylo, tak proč by to někdo zpochybňoval? Nejenže lidé málokdy dostanou důvod pojem \enquote{stát} zkoumat, ale mají i velmi přesvědčivou psychologickou motivaci jej nezkoumat. Pro někoho je nesmírně nepříjemné a znepokojivé, dokonce existenciálně děsivé, když zpochybňuje jeden ze základních předpokladů, na nichž je celý jeho pohled na realitu a celý jeho morální kodex po celý život založen. Pro člověka, jehož vnímání a úsudek byly zkresleny pověrou autority (a to se týká téměř každého), nebude snadné ani příjemné uvažovat o možnosti, že celý jeho systém víry je založen na lži a že mnohé z toho, co v důsledku víry v tuto lež po celý život dělal, škodilo jemu samotnému, jeho přátelům a rodině a lidstvu obecně. Víra v autoritu a \enquote{stát} zkrátka pokřivuje vnímání téměř každého člověka, zkresluje jeho úsudek a vede ho k tomu, že říká a dělá věci, které jsou často iracionální, nesmyslné, kontraproduktivní, pokrytecké, nebo dokonce strašlivě destruktivní a odporně zlé. Věřící v mýtus to tak samozřejmě nevidí, protože ho vůbec nevnímají jako víru. Jsou pevně přesvědčeni, že \enquote{autorita} je skutečná, a na základě tohoto falešného předpokladu usuzují, že jejich výsledné vnímání, myšlení, názory a činy jsou naprosto rozumné, ospravedlnitelné a správné, stejně jako Aztékové nepochybně věřili, že jejich lidské oběti jsou rozumné, ospravedlnitelné a správné. Pověra, která je schopna přimět jinak slušné lidi, aby považovali dobro za zlo a zlo za dobro -- což je přesně to, co víra v autoritu dělá -- představuje skutečnou hrozbu pro lidstvo.

Pověra o autoritě ovlivňuje vnímání a jednání různých lidí různým způsobem, ať už jde o \enquote{zákonodárce,} kteří si představují, že mají právo vládnout, \enquote{strážce zákona,} kteří si představují, že mají právo a povinnost prosazovat příkazy \enquote{zákonodárců,} poddané, kteří si představují, že mají morální povinnost poslouchat, nebo pouhé pozorovatele, kteří jen neutrálně přihlížejí. Působení víry v autoritu na tyto různé skupiny vede dohromady k takové míře útlaku, nespravedlnosti, krádeží a vražd, která by jinak prostě nemohla existovat a neexistovala.

\chapter{Účinky na pány}

\section{Svaté právo politiků}

V této zemi jsou na vrcholu gangu zvaného \enquote{stát} kongresmani, prezidenti a \enquote{soudci.} (V jiných zemích se vládci nazývají jinak, například \enquote{králové,} \enquote{císaři} nebo \enquote{poslanci.}) A přestože stojí na vrcholu autoritářské organizace, nejsou vnímáni jako \enquote{autorita} sama o sobě (jako kdysi králové). Stále si představují, že jednají jménem něčeho jiného, než jsou oni sami -- jakési abstraktní entity zvané \enquote{stát.} V důsledku víry v autoritu si představují, že mají právo dělat jménem \enquote{státu} věci, na které nikdo z nich nemá právo jako jednotlivec. Legitimita jejich jednání se neměří podle toho, co dělají, ale podle toho, jak to dělají. V očích většiny lidí jsou činnosti, které politici vykonávají ve své \enquote{oficiální funkci,} a příkazy, které vydávají prostřednictvím přijatých politických rituálů, posuzovány podle zcela jiných měřítek než jejich činnosti jakožto soukromých osob.

Pokud se kongresman vloupe do domu svého souseda a vezme si 1000 dolarů, je považován za zločince. Pokud naopak spolu se svými kolegy politiky uvalí \enquote{daň} a požaduje od téhož souseda stejných 1000 dolarů, je to považováno za legitimní. To, co by bylo ozbrojenou loupeží, by pak téměř všichni považovali za legitimní \enquote{zdanění.} Nejenže by kongresman nebyl považován za gaunera, ale všichni \enquote{daňoví podvodníci,} kteří by se \emph{vzpírali} jeho vyděračským požadavkům, by byli považováni za \enquote{zločince.}

Víra v autoritu však nemění jen to, jak jsou \enquote{zákonodárci} vnímáni masami, ale také to, jak \enquote{zákonodárci} vnímají sami sebe. Mělo by být zřejmé, že pokud člověk nabude přesvědčení, že má morální právo vládnout ostatním, bude mít toto přesvědčení významný vliv na jeho chování. Pokud je přesvědčen, že má právo pod hrozbou trestu požadovat část příjmu všech lidí (za předpokladu, že tak učiní prostřednictvím uznávaných \enquote{zákonných} postupů), bude tak téměř jistě to bude dělat. Pokud je přesvědčen, že má právo násilně určovat rozhodnutí svých sousedů -- že je morální a legitimní, aby tak postupoval -- téměř jistě to udělá. A přinejmenším zpočátku tak může dokonce jednat s těmi nejlepšími úmysly.

Jednoduché mentální cvičení umožňuje nahlédnout do toho, jak a proč politici jednají. Zamyslete se nad tím, co byste udělali vy, kdybyste se stali králem světa. Kdybyste byli u moci, jak byste věci zlepšili? Než budete číst dál, pečlivě tuto otázku zvažte.

Na otázku, co by udělali, kdyby byli ve vedení, téměř nikdo neodpoví: \enquote{Nechal bych lidi na pokoji.} Místo toho si většina lidí začne představovat, jak by mohli schopnost \emph{ovládat} lidi využít jako nástroj pro dobro, pro zlepšení lidstva. Pokud člověk vychází z předpokladu, že takové ovládání může být legitimní a spravedlivé, jsou možnosti téměř nekonečné. Mohli bychom vytvořit zdravější zemi tím, že bychom lidi nutili jíst výživnější potraviny a pravidelně cvičit. Chudým by se dalo pomoci tím, že bohaté donutíme, aby jim dávali peníze. Lidé by mohli být ve větším bezpečí, kdyby byli nuceni platit za silný obranný systém. Mohli bychom zajistit větší spravedlnost a soucit ve společnosti tím, že donutíme lidi, aby se chovali tak, jak by se chovat měli.

Ačkoli si lze představit mnoho pozitivních přínosů pro společnost, pokud by se \enquote{státní} moc využívala k dobrým účelům, stejně snadno si lze představit i možnost tyranie a útlaku -- nebo spíš \emph{nevyhnutelnost} tyranie a útlaku. Jakmile se někdo domnívá, že má právo ovládat ostatní, je jen málo pravděpodobné, že se rozhodne tuto moc nepoužívat. A bez ohledu na to, jaké ušlechtilé úmysly měl na začátku, ve skutečnosti nakonec použije násilí a hrozbu násilí, aby vnutil svou vůli ostatním. Dokonce i zdánlivě dobročinné cíle, jako je \enquote{dávání chudým,} nejprve vyžadují, aby \enquote{stát} násilím odebral bohatství jinému. Jakmile někdo -- jakkoli ctnostný a s dobrými úmysly -- přijme premisu, že \enquote{legální} agrese je legitimní, a jakmile dostane otěže moci a s nimi i domnělé právo vládnout, šance, že se tento člověk \emph{nerozhodne} násilně ovládat své bližní, je téměř nulová. Míra nátlaku a násilí, které způsobí ostatním, se může lišit, ale v té či oné míře se stane tyranem, protože jakmile někdo skutečně uvěří, že má právo vládnout (byť jen \enquote{omezeně}), nebude se na ostatní dívat a jednat s nimi jako se sobě rovnými. Bude na ně pohlížet a jednat s nimi jako s poddanými.

A to v případě, že dotyčný začal s dobrými úmysly. Mnozí z těch, kdo usilují o \enquote{vysokou funkci,} to od začátku dělají z čistě sobeckých důvodů, protože touží po bohatství a moci pro sebe a libují si v ovládání ostatních lidí. Získání \enquote{vysoké funkce} je pro takové lidi samozřejmě prostředkem k dosažení obrovské moci, kterou by jinak neměli. Příklady megalomanů, kteří využívají fasádu \enquote{autority} k páchání ohavných zvěrstev, jsou po celém světě i v dějinách tak běžné a známé, že je snad ani není třeba zmiňovat. Dosazení zlých lidí do pozic \enquote{autority} (např. Stalin, Lenin, Mao, Hitler, Mussolini, Pol Pot) vedlo k okrádání, přepadávání, šikanování, terorizování, mučení a přímému vraždění téměř nepochopitelného množství lidských bytostí. Je to tak zřejmé, až je skoro hloupé to vůbec zmiňovat: předání moci špatným lidem představuje nebezpečí pro lidstvo.

Ale dát moc \emph{dobrým} lidem -- lidem, kteří alespoň zpočátku hodlají svou moc využívat k dobru -- může být stejně nebezpečné, protože věřit, že má právo vládnout, nutně vyžaduje, aby věřil, že je osvobozen od základní morálky. Když si někdo představuje, že je legitimním \enquote{zákonodárcem,} bude se snažit použít sílu \enquote{zákona} k ovládání svých bližních a nebude při tom cítit žádnou vinu.

Ironií je, že ačkoli jsou \enquote{zákonodárci} na samém vrcholu autoritářské hierarchie, ani oni nepřijímají osobní odpovědnost za to, co \enquote{stát} dělá. Dokonce i oni mluví, jako by \enquote{zákon} byl něco jiného než příkazy, které vydávají. Je například velmi nepravděpodobné, že by se nějaký politik cítil oprávněn najmout ozbrojené násilníky, aby vtrhli do domu jeho souseda, odvlekli ho a zavřeli do klece za údajný přečin kouření marihuany. Přesto mnozí politici právě toto prostřednictvím protidrogové \enquote{legislativy} prosazují. Zdá se, že necítí žádný stud ani vinu za to, že jejich \enquote{legislativa} vedla k tomu, že miliony lidí, kteří nepáchali násilí, byly násilím odebrány svým přátelům a rodinám a donuceny žít v klecích po dlouhá léta -- někdy až do konce života. Když mluví o násilných činech, za něž jsou přímo odpovědní -- a \enquote{protidrogové zákony} jsou jen jedním z příkladů -- používají \enquote{zákonodárci} výrazy jako \enquote{zákony země,} jako by oni sami byli pouhými pozorovateli a \enquote{země,} \enquote{národ} nebo \enquote{lid} byly těmi, kdo takové násilí způsobili.

Úroveň psychického odstupu politiků od toho, co osobně a přímo způsobili svými \enquote{zákony,} vskutku hraničí s šílenstvím. Rozkazují armádám \enquote{výběrčích daní,} aby násilím zabavili majetek vydělaný stovkami milionů lidí. Přijímají jeden dotěrný \enquote{zákon} za druhým a za pomoci hrozeb násilím ovládají každý aspekt života milionů lidí, které nikdy neviděli a o kterých nic nevědí. A poté, co jsou přímo zodpovědní za pravidelné iniciování násilí vůči téměř každému, kdo žije v okruhu stovek či tisíců kilometrů od nich, jsou upřímně šokováni a uraženi, když některá z jejich obětí vyhrožuje, že proti nim použije násilí. Považují za opovrženíhodné, když obyčejný rolník jen vyhrožuje tím, co oni, politici, denně dělají milionům lidí. Zároveň si zřejmě ani nevšimnou milionů lidí, kteří jsou vězněni, kterým je ukraden majetek, kterým je zruinován finanční život, jejichž svoboda a důstojnost jsou napadány, kteří jsou obtěžováni, napadáni a někdy i vražděni \enquote{státními} násilníky, a to v přímém důsledku právě těch \enquote{zákonů,} které tito politici vytvořili.

Když mladí muži a ženy umírají po tisících v aktuální válečné hře politiků, politici o nich mluví jako o \enquote{obětech za svobodu,} ačkoli o nic takového nejde. Politici dokonce používají scény s vojáky v rakvích -- což je důsledek, který lze přímo přičíst tomu, co tito politici udělali -- jako šou, aby veřejnosti ukázali, jak jsou starostliví a soucitní. Ti samí lidé, kteří poslali mladé lidi zabíjet nebo umírat, pak o tom, co se stalo, mluví, jako by sami byli pouhými pozorovateli, a říkají věci jako \enquote{zemřeli za svou zemi} a \enquote{v každé válce jsou oběti,} jako by se válka stala sama od sebe.

A samozřejmě tisíce a tisíce lidí na \enquote{druhé straně} -- poddaných nějaké jiné \enquote{autoritě,} občanů nějaké jiné \enquote{země} -- kteří jsou zabíjeni ve válkách vedených politiky, nejsou téměř vůbec zmíněni. Jsou jen příležitostnou statistikou ve večerních zprávách. A politici nikdy nepřijmou ani špetku odpovědnosti za plošnou, rozsáhlou, vlečnou bolest a utrpení, duševní i fyzické, které jejich válečné štvaní způsobilo tisícům či milionům lidských bytostí. O hloubce jejich popírání a naprostého vyhýbání se osobní odpovědnosti svědčí opět skutečnost, že pokud se některá z obětí válečných her politiků rozhodne zaútočit na zdroj, a to přímým útokem na ty, kteří vydali rozkaz k útoku, \emph{všichni} politici, dokonce i ti, kteří tvrdí, že jsou proti válce, a všechny mluvící hlavy v televizi vyjadřují šok a rozhořčení nad tím, že by někdo něco tak odporného udělal. Je tomu tak proto, že v očích \enquote{zákonodárců} -- díky úžasné síle mýtu autority, který zcela pokřivuje a zkresluje jejich vnímání reality -- jsou věci, které mají za následek smrt tisíců nevinných, \enquote{nešťastná daň za válku,} ale když se některá z jejich obětí pokusí udeřit zpět na zdroj, je to \enquote{terorismus.}

Je dost špatné, když ti, kdo jen plní rozkazy, odmítají osobní odpovědnost za své činy (o tom se píše níže), ale když ti, kdo skutečně vydávají rozkazy a vymýšlejí je, odmítají jakoukoli odpovědnost za to, co jejich rozkazy přímo způsobily, je to naprosté šílenství. Přesto to \enquote{zákonodárci} vždy dělají, a to na všech úrovních. Ať už jde o federální vládu, nebo o nějaké místní zastupitelstvo města či obce, pokaždé, když \enquote{zákonodárce} na něco uvalí \enquote{daň} nebo zavede nějaké nové \enquote{zákonné} omezení, používají politici hrozbu násilí k ovládání lidí. Kvůli své nezlomné víře v mýtus autority však nevidí, že tohle opravdu dělají, a nikdy nepřijmou osobní odpovědnost za to, že svým bližním vyhrožují a vydírají je.

\chapter{Účinky na vymahatele}

\section{Plnění příkazů}

\enquote{Zákonodárci} vydávají příkazy, ale jsou to jejich věrní vymahatelé, kdo je plní. Miliony a miliony jinak slušných a civilizovaných lidí tráví den za dnem obtěžováním, vyhrožováním, vydíráním, ovládáním a jiným utlačováním ostatních, kteří nikomu neublížili ani nikoho neohrozili. Protože je však jednání těchto \enquote{strážců zákona} považováno za \enquote{legální} a protože se domnívají, že jednají jménem \enquote{autority,} představují si, že za své činy nenesou žádnou odpovědnost. Ve skutečnosti ani nepovažují své vlastní činy za \emph{vlastní} činy. Mluví a jednají, jako by jejich mysl a tělo nějakým způsobem ovládla jakási neviditelná entita zvaná \enquote{zákon} nebo \enquote{st8t.} Říkají věci jako: \enquote{Hej, já zákony nevytvářím, já je jen vymáhám; nezáleží to na mně.} Mluví a jednají, jako by pro ně nebylo možné dělat nic jiného než bezmocně vykonávat vůli moci zvané \enquote{autorita,} a že tedy nejsou za své činy osobně zodpovědní o nic víc, než je loutka zodpovědná za to, co ji loutkář nutí dělat.

Když \enquote{strážci zákona} jednají ve své \enquote{úřední} funkci a jsou zdánlivě bezmocně posedlí duchem \enquote{autority,} chovají se tak, jak by se jinak nikdy nechovali, a dělají věci, které by sami uznali za necivilizované, násilné a zlé, kdyby je dělali sami od sebe, aniž by jim to nařizovala \enquote{autorita.} Příklady takového jednání se objevují po celém světě, každou hodinu a každý den, a to nejrůznějšími způsoby. Voják může zastřelit úplně cizího člověka, jehož jediným prohřeškem bylo, že se po vyhlášeném zákazu vycházení procházel ve vojensky okupované zóně. Skupina těžce ozbrojených mužů může někomu vykopnout dveře a odvléct ho pryč nebo zastřelit muže před očima jeho ženy a dětí, protože pěstoval rostlinu, kterou politici prohlásili za zakázanou (\enquote{nelegální}). Byrokrat by mohl vyplnit papír, který by finanční instituci nařídil, aby z něčího bankovního účtu vybrala tisíce dolarů ve jménu \enquote{výběru daní.} Jiný byrokrat může na někoho poslat ozbrojené násilníky, když zjistí, že si dovolil postavit terasu na vlastním pozemku se souhlasem sousedů, ale bez \enquote{státního} souhlasu (v podobě \enquote{stavebního povolení}). Dopravní policista může někoho zastavit a vydírat (prostřednictvím \enquote{pokuty}) za rozbité zadní světlo. Agent TSA může někoho prohledat a prohrabat mu osobní věci, aniž by měl sebemenší důvod k podezření, že dotyčný udělal nebo se chystá udělat něco špatného. \enquote{Soudce} může nařídit ozbrojeným násilníkům, aby někoho zavřeli na týdny, měsíce nebo roky do klece za cokoli, od projevení pohrdání soudcem přes řízení bez písemného souhlasu politiků (v podobě řidičského průkazu) až po účast na jakémkoli druhu vzájemně dobrovolného, ale politiky neschváleného (\enquote{nezákonného}) obchodu.

Tyto příklady a doslova miliony dalších, které by bylo možné uvést, jsou akty agrese spáchané pachateli, kteří by se jich nedopustili, kdyby k nim nedostali pokyn od domnělé \enquote{autority.} Stručně řečeno, k většině případů krádeží, napadení a vražd dochází jen proto, že \enquote{autorita} někomu \emph{přikázala}, aby kradl, útočil nebo zabíjel. Většinou by lidé, kteří takové příkazy plní, sami od sebe takové zločiny nespáchali. Kolik ze 100 000 lidí, kteří pracují pro daňový úřad, se podílelo na obtěžování, vydírání a krádežích \emph{předtím}, než se stali agenty daňového úřadu? Málo, pokud vůbec někdo. Kolik vojáků obtěžovalo, vyhrožovalo nebo zabíjelo lidi, které neznali, \emph{předtím} než vstoupili do armády? Málo, pokud vůbec. Kolik policistů pravidelně zastavovalo, vyslýchalo a unášelo nenásilné lidi \emph{předtím}, než se stali \enquote{strážci zákona?} Velmi málo. Kolik \enquote{soudců} nechalo zavírat lidi do klecí za nenásilné chování \emph{předtím}, než byli jmenováni do \enquote{soudu?} Pravděpodobně žádný.

Když se takové agresivní činy stanou \enquote{legálními} a jsou prováděny ve jménu \enquote{vymáhání práva,} ti, kdo je páchají, si představují, že jsou ze své podstaty legitimní a platné, i když si uvědomují, že kdyby tytéž činy páchali sami, a nikoli jménem \enquote{autority,} jednalo by se o zločiny a byly by nemorální. Ačkoli samozřejmě existují významnější i méně významná kolečka v soukolí \enquote{státní} mašinérie, od podřadných úředníčků až po ozbrojené žoldáky, všichni mají společné dvě věci: 1) způsobují druhým nepříjemnosti způsobem, jakým by to sami neudělali, a 2) nepřijímají žádnou osobní odpovědnost za své činy, když jsou v režimu \enquote{strážce zákona.} Nic to neukazuje tak zřetelně jako skutečnost, že když je správnost nebo morálnost jejich jednání zpochybněna, jejich odpovědí je téměř vždy nějaká variace na \enquote{já jen dělám svou práci.} Ze všech takových prohlášení vyplývá zřejmý důsledek: \enquote{Nejsem zodpovědný za své činy, protože 'autorita' mi \emph{řekla}, abych to udělal.} Jediný způsob, jak to dává alespoň trochu smysl, je ten, že dotyčný doslova nedokáže odmítnout udělat něco, co mu domnělá \enquote{autorita} nařídila. Bohužel je strašnou pravdou, že většina lidí se v důsledku autoritářské indoktrinace zdá být psychicky \emph{neschopná} neuposlechnout příkazů domnělé \enquote{autority.} Většina lidí, kteří mají na výběr mezi tím, co vědí, že je správné, a tím, co vědí, že je špatné, udělá to druhé, když jim to nařídí domnělá \enquote{autorita.} Nic to neprokazuje jasněji než výsledky psychologických experimentů, které v 60. letech 20. století prováděl Dr. Stanley Milgram.

\section{Milgramovy experimenty}

Stručně řečeno, cílem Milgramových studií bylo zjistit, do jaké míry jsou obyčejní lidé ochotni působit bolest cizím lidem jen proto, že jim to řekla \enquote{autorita.} Kompletní popis experimentů a jejich výsledků najdete v knize Dr. Milgrama \emph{Obedience to Authority} (Poslušnost vůči autoritě). Následující text je stručným shrnutím jeho experimentů a zjištění.

Subjekty byly požádány, aby se dobrovolně přihlásily k experimentu, o němž jim bylo řečeno, že testuje lidskou paměť. Pod dohledem vědce (\enquote{autority}) byla jedna osoba připoutána k židli a napojena na elektrody a druhá -- skutečný subjekt studie -- seděla před přístrojem generujícím šoky. Osobě sedící před \enquote{elektrošokovým} přístrojem bylo řečeno, že cílem je otestovat, zda šok, který dostane druhá osoba při špatné odpovědi na otázku týkající se zapamatování, ovlivní její schopnost zapamatovat si věci. Skutečným cílem však bylo otestovat, do jaké míry osoba před přístrojem způsobí bolest nevinnému cizinci jen proto, že jí to nařídila domnělá \enquote{autorita.} Stroj na \enquote{šoky} měl řadu spínačů s napětím až 450 voltů a \enquote{zkoušející} měl zvýšit napětí a dát další šok pokaždé, když \enquote{zkoušený} odpověděl špatně. \enquote{Zkoušený} při testech byl ve skutečnosti herec, který vůbec nedostával šoky, ale při daných úrovních napětí vydával výkřiky bolesti, protesty o srdečních potížích, žádosti o zastavení experimentu, výkřiky o milost a nakonec mlčení (předstíral bezvědomí nebo smrt). A přístroj na \enquote{šoky} byl na horním konci řady spínačů zřetelně označen štítky s nebezpečím.

Výsledky experimentu šokovaly i doktora Milgrama. Stručně řečeno, značná většina subjektů, téměř dvě ze tří, pokračovala v experimentu až do konce a způsobovala zcela cizí osobě šoky, které považovala za nesnesitelně bolestivé, ne-li smrtelné, a to navzdory výkřikům agónie, volání o milost, dokonce i bezvědomí nebo smrti (předstírané) oběti. Sám Dr. Milgram stručně shrnul závěr, ke kterému je třeba dospět:

\enquote{\emph{S otupující pravidelností bylo vidět, jak se dobří lidé podřizují požadavkům autority a vykonávají bezcitné a kruté činy. ... Značná část lidí dělá to, co se jim řekne, bez ohledu na obsah činu a bez omezení svědomí, pokud vnímají, že příkaz pochází od legitimní autority.}}

Za zmínku stojí, že při pokusech nebylo vyhrožováno, že \enquote{zkoušející} bude potrestán za neuposlechnutí, ani nebyla slíbena žádná zvláštní odměna za poslušnost. Zjištění tedy nebyla výsledkem toho, že by někdo škodil někomu jinému, aby si \enquote{zachránil krk} nebo aby z toho jinak profitoval. Místo toho výsledky ukazují, že i bez jakéhokoli příslibu odměny nebo trestu způsobí průměrný člověk nevinnému cizímu člověku nesnesitelnou bolest, dokonce i smrt, jen proto, že mu to nařídila domnělá \enquote{autorita.}

Tento bod nelze přeceňovat: existuje určitá víra, která vede v podstatě dobré lidi k tomu, aby dělali špatné věci, dokonce i odporně zlé věci. Dokonce i zvěrstva Hitlerovy Třetí říše byla výsledkem nikoli milionů zlých lidí, ale velmi malé hrstky skutečně zlých lidí, kteří získali pozice \enquote{autority,} a milionů poslušných lidí, kteří pouze dělali to, co jim domnělá \enquote{autorita} nařídila.

Hannah Arendtová ve své knize o Hitlerově nejvyšším úředníkovi Adolfu Eichmannovi (někdy nazývaném \enquote{architekt holocaustu}) použila výraz \enquote{banalita zla,} aby poukázala na skutečnost, že většina zla není výsledkem osobní zloby nebo nenávisti, ale pouze výsledkem slepé poslušnosti -- jednotlivci se vzdávají vlastní svobodné vůle a úsudku ve prospěch bezmyšlenkovitého podřízení se pomyslné \enquote{autoritě.}

Zajímavé je, že jak kniha Arendtové, tak experimenty doktora Milgrama mnoho lidí urazily. Důvod je prostý: lidé, kteří byli naučeni respektovat \enquote{autoritu} a byli naučeni, že poslušnost je ctnost a že spolupráce s \enquote{autoritou} je to, co nás činí civilizovanými, neradi slyší pravdu, která zní, že skutečně zlí lidé se vší svou zlobou a nenávistí představují pro lidstvo mnohem menší \emph{hrozbu} než v zásadě \emph{dobří} lidé, kteří věří v autoritu. Každý, kdo poctivě zkoumá výsledky experimentů doktora Milgrama, nemůže této skutečnosti uniknout. Ale kromě obecného poučení, které si z Milgramových experimentů lze odnést -- že většina lidí bude úmyslně ubližovat jiným lidem, pokud jim to nařídí domnělá \enquote{autorita} -- stojí za zmínku několik dalších poznatků z Milgramovy práce:

1) Mnoho pokusných osob vykazovalo známky stresu, viny a úzkosti, když působily bolest druhým, a přesto v tom pokračovaly. Tato skutečnost svědčí o tom, že se nejednalo o pouhé odporné sadisty čekající na záminku, aby mohli ubližovat druhým; nedělali to rádi. Navíc to ukazuje, že tito lidé \emph{věděli}, že dělají něco špatného, a přesto to dělali, protože jim to \enquote{autorita} nařídila. Některé pokusné osoby protestovaly, prosily, aby mohly přestat, nekontrolovatelně se třásly, dokonce plakaly, a přesto většina z nich pokračovala až do konce experimentu. Závěr může být jen stěží zřejmější: \emph{Víra v autoritu nutí dobré lidi páchat zlo.}

2) Zdá se, že výše příjmu, úroveň vzdělání, věk, pohlaví a další demografické faktory měly na výsledky malý nebo žádný vliv. Statisticky řečeno, bohatá, kulturní a vzdělaná mladá žena uposlechne autoritářský příkaz ublížit někomu jinému stejně ochotně jako negramotný, chudý, manuálně pracující muž. Jediným společným faktorem všech, kteří pokračovali až do konce experimentu, je to, že věřili v \enquote{autoritu} (samozřejmě). Znovu připomínám, že poučení, které z toho plyne, ať už je jakkoli znepokojivé, je logicky nevyhnutelné: \emph{Bez ohledu na téměř všechny ostatní faktory víra v autoritu mění dobré lidi v pachatele zla.}

3) Průměrný člověk, když je mu experiment popsán, nepočítaje v to výsledky, odhadne, že soucit a svědomí většiny lidí by jim zabránily pokračovat v celém experimentu. Profesionální psychiatři předpovídali, že jen asi jeden z tisíce by se podřídil až do konce experimentu, zatímco ve skutečnosti to bylo asi 65 \%. A když se průměrného člověka, který ve skutečnosti nebyl testován, zeptáme, zda by on osobně došel až do konce studie, kdyby byl testován, obvykle trvá na tom, že ne. Přesto to většina udělá. Znovu opakuji, že toto sdělení je znepokojivé, ale nesporné: \emph{Téměř každý nesmírně podceňuje míru, v jaké může být víra v autoritu, dokonce i vlastní, využita k přesvědčení dobrých lidí, aby páchali zlo.}

4) Dr. Milgram také zjistil, že některé pokusné osoby byly v rozporu s rozumem odhodlány svést vinu za výsledky své slepé poslušnosti na oběť: na toho, kdo dostával šoky. Jinými slovy, někteří z těch, kteří trestali, si díky své zvrácené mentalitě představovali, že trestaný si za své utrpení může sám. S ohledem na to by nemělo překvapit, že když jsou policisté přistiženi při napadení nevinných občanů nebo když jsou vojáci přistiženi při terorizování či vraždění civilistů nebo když jsou vězeňští dozorci přistiženi při mučení vězňů, jejich obhajoba často spočívá v tom, že vinu svalují na \emph{oběť}, bez ohledu na to, jak moc kvůli tomu musí autoritářští agresoři pokřivit pravdu a logiku.

Zajímavé je, že i když v Norimberském procesu nebylo \enquote{pouhé plnění rozkazů} přijato jako platná omluva pro to, co nacisté dělali, je to stále standardní odpověď vojáků, policistů, výběrčích daní, úředníků a dalších představitelů \enquote{autority,} kdykoli je zpochybněna morálnost jejich chování. V Milgramových experimentech i v nesčetných případech skutečného zneužití moci se ti, kdo úmyslně ubližují druhým, jednoduše vrátí ke standardní výmluvě a tvrdí, že nejsou osobně zodpovědní, protože pouze plní rozkazy. V Milgramových experimentech se několik subjektů dokonce přímo zeptalo \enquote{autority,} kdo z nich je zodpovědný za to, co se děje. Když postava \enquote{autority} řekla, že je to on, kdo je zodpovědný, většina subjektů pokračovala bez další debaty, zřejmě spokojena s představou, že cokoli se od té chvíle stane, není jejich vina a oni nebudou bráni k odpovědnosti. Opět je těžké uniknout poselství:

\emph{Víra v autoritu umožňuje v zásadě dobrým lidem distancovat se od zlých činů, které sami páchají, a zbavuje je pocitu osobní odpovědnosti.}

5) Když bylo ponecháno na \enquote{zkoušejícím,} jaké napětí použije, jen velmi zřídkakdy překročil 150 voltů, což byl bod, kdy ten, kdo předstíral, že je šokován, řekl, že nechce pokračovat. Je velmi důležité poznamenat, že až do tohoto bodu -- a téměř všechny pokusné osoby se do tohoto bodu dostaly -- \enquote{trestaný} vydával bolestivé skřeky, ale nežádal o ukončení experimentu. V důsledku toho mohl ten, kdo šoky prováděl, zcela oprávněně tvrdit, že trestaný s tímto uspořádáním souhlasil a až do tohoto okamžiku byl stále dobrovolným účastníkem. Zajímavé je, že z těch několika subjektů, které nevydržely až do konce, jich mnoho přestalo, jakmile \enquote{trestaný} řekl, že chce skončit. To by se dalo nazvat \enquote{libertariánskou linií,} protože jakmile \enquote{trestaný} požádá, aby byl odvázán, pro trestajícího znamená pokračování iniciování násilí vůči druhému -- což je přesně to, proti čemu libertariáni vystupují.

Bohužel ti, kteří se zastaví na \enquote{libertariánské hranici,} tvoří jen malou menšinu populace. Pokud jde o zbytek, zjištění jsou znepokojivě jasná: z lidí, kteří by na příkaz \enquote{autority} šokovali někoho, kdo klidně řekl: \enquote{Už to nechci dělat,} by většina pokračovala v působení bolesti, i kdyby oběť křičela v agónii. Je to proto, že většina lidí je zlá? Ne. Je to proto, že byli naučeni dělat, co se jim řekne, a byli indoktrinováni nejnebezpečnější pověrou ze všech: vírou v autoritu.

Je třeba poznamenat, že ani Dr. Milgram neunikl vlastní indoktrinaci kultem uctívání \enquote{autority.} Dokonce i on se jen tak mimochodem a bez většího komentáře vyjádřil, že \enquote{\emph{nemůžeme mít společnost bez určité struktury autority}.} Chabě se pokusil obhájit výuku poslušnosti vůči \enquote{autoritě} slovy: \enquote{\emph{Poslušnost je často racionální. Je rozumné řídit se příkazy lékaře, dodržovat dopravní značky a vyklidit budovu, když nás policie informuje o hrozbě teroristického útoku}.} Žádný z těchto příkladů však ve skutečnosti nevyžaduje ani neospravedlňuje víru v autoritu. Navzdory tomu, jak lidé často mluví, lékaři nevydávají \enquote{rozkazy.} Jsou \enquote{autoritami} v tom smyslu, že jsou znalí v oboru medicíny, ale ne v tom smyslu, že by měli nějaké právo vládnout. Co se týče ostatních příkladů, hlavním důvodem, proč dodržovat pravidla silničního provozu nebo proč opustit budovu, v níž je bomba, není to, že poslušnost vůči \enquote{autoritě} je ctnost, ale to, že alternativou je zranění nebo smrt. Kdyby nějaký neautoritativní člověk v divadle vytáhl zpod sedadla bombu, podržel ji všem na očích a řekl: \enquote{Bomba! Vypadněme odsud!,} zůstali by všichni ostatní na svých místech, protože tato osoba nebyla vnímána jako \enquote{autorita?} Jistěže ne. A kdyby \enquote{stát} zrušil \enquote{zákon,} který říká, po které straně silnice má každý jezdit, začali by lidé náhodně kličkovat? Samozřejmě že ne. Jezdili by dál po správné straně, protože by do sebe nechtěli narazit. Ačkoli tedy i doktor Milgram lpěl na tom, že víra v autoritu je někdy nutná a dobrá, neuvedl žádný racionální argument, který by takové tvrzení podpořil. O síle mýtu autority svědčí, že ani člověk, který byl svědkem toho, čeho byl svědkem doktor Milgram, se stále nedokázal této pověry zcela vzdát.

Poté, co Dr. Milgram zveřejnil svá zjištění, byli mnozí šokováni a zděšeni tím, do jaké míry jsou normální lidé ochotni způsobit bolest nebo smrt nevinným cizím lidem, když jim to nařídí domnělá \enquote{autorita.} Podobné testy provedené po experimentech doktora Milgrama přinesly podobné výsledky, které některé lidi stále šokují. Výsledky by však skutečně neměly překvapit nikoho, kdo se podíval na to, jak je většina lidských bytostí vychovávána.

\section{Výuka slepé poslušnosti}

Účelem škol je údajně výuka čtení, psaní, matematiky a dalších akademických oborů. Ale poselství, které instituce \enquote{vzdělávání} ve skutečnosti vyučují, je mnohem účinnější než jakékoli užitečné znalosti nebo dovednosti, a to myšlenka, že podřízenost a slepá poslušnost \enquote{autoritě} jsou ctnosti. Stačí se zamyslet nad prostředím, v němž většina lidí stráví většinu svých formativních let. Rok co rok žijí studenti ve světě, v němž:

-- Dostávají souhlas, pochvaly a odměny za to, že jsou tam, kde jim \enquote{autorita} řekne, aby byli, když jim \enquote{autorita} řekne, aby tam byli. Za to, že jsou kdekoli jinde, dostávají nesouhlas, výčitky a tresty. (K tomu patří i to, že jsou do školy nuceni.)

-- Dostávají souhlas, pochvaly a odměny za to, že dělají to, co jim \enquote{autorita} nařídí. Za to, že dělají cokoli jiného nebo že nedělají to, co jim \enquote{autorita} nařizuje, dostávají nesouhlas, výčitky a tresty.

-- Dostávají souhlas, pochvaly a odměny za to, že mluví, kdy a jak jim \enquote{autorita} řekne, aby mluvili, a za to, že mluví kdykoli jindy, jakýmkoli jiným způsobem nebo o jiném tématu, než o jakém jim \enquote{autorita} řekne, aby mluvili, nebo za to, že nemluví, když jim \enquote{autorita} řekne, aby mluvili, dostávají nesouhlas, výčitky a tresty.

-- Dostávají souhlas, pochvaly a odměny za to, že opakují myšlenky, které \enquote{autorita} prohlásí za pravdivé a důležité, a za to, že ústně nebo v písemném testu nesouhlasí s názory těch, kdo se prohlašují za \enquote{autoritu,} nebo že přemýšlejí či píší o jiných tématech, než o jakých jim \enquote{autorita} říká, že mají přemýšlet nebo psát, dostávají nesouhlas, výčitky a tresty.

-- Dostávají souhlas, pochvaly a odměny za to, že \enquote{autoritě} okamžitě sdělí jakýkoli problém nebo osobní konflikt, na který narazí, a za to, že se snaží vyřešit jakýkoli problém nebo urovnat jakýkoli spor sami, dostávají nesouhlas, výčitky a tresty.

-- Dostávají souhlas, pochvaly a odměny za to, že dodržují jakákoli pravidla, ať už jsou jakkoli svévolná, která se jim \enquote{autorita} rozhodne vnutit. Za nedodržování těchto pravidel dostávají nesouhlas, výčitky a tresty. Tato pravidla se mohou týkat téměř čehokoli, včetně toho, jaké nosit oblečení, jaký mít účes, jaký mít výraz obličeje, jak sedět na židli, co mít na stole, jakým směrem se dívat a jaká slova používat.

-- Dostávají souhlas, pochvaly a odměny za to, že řeknou \enquote{autoritě,} když jiný žák nedodrží \enquote{pravidla,} a za to, že tak neučiní, dostávají nesouhlas, výčitky a tresty.

Studenti jasně a okamžitě vidí, že v jejich světě existují dvě odlišné třídy lidí, páni (\enquote{učitelé}) a poddaní (\enquote{studenti}), a že pravidla správného chování se pro obě skupiny dramaticky liší. Páni neustále dělají věci, které poddaným zakazují: komandují lidi, ovládají ostatní pomocí výhrůžek, berou jim majetek atd. Tento neustálý a zjevný dvojí metr učí poddané, že pro pány platí zcela jiná morální pravidla než pro poddané. Poddaní musí dělat vše, co jim pánové nařídí, a pouze to, co jim pánové nařídí, zatímco pánové si mohou dělat prakticky cokoli se jim zachce.

Není to tak dávno, co se páni dokonce běžně dopouštěli fyzického násilí (tj. \enquote{tělesných trestů}) vůči těm poddaným, kteří rychle a bezvýhradně nedělali, co se jim řeklo, a zároveň jim říkali, že je naprosto nepřípustné, aby kdy použili fyzické násilí, a to i v sebeobraně, zejména v sebeobraně proti pánům. Naštěstí se pravidelné, zjevné fyzické násilí ze strany \enquote{učitelů} stalo neobvyklým. Nicméně, i když se násilí stalo méně zjevným, základní metody autoritářského ovládání a trestání zůstávají.

Ve školním prostředí může \enquote{autorita} libovolně měnit pravidla, může trestat celou skupinu za to, co udělá jeden student, a může kdykoli vyslýchat nebo prohledávat kteréhokoli studenta -- nebo všechny studenty. \enquote{Autorita} není nikdy považována za osobu, která by měla povinnost zdůvodnit nebo vysvětlit studentům pravidla, která stanoví, nebo cokoli jiného, co dělá. A \enquote{autoritu} vůbec nezajímá, zda má student dobrý důvod myslet si, že by bylo lepší, kdyby svůj čas trávil někde jinde, dělal něco jiného nebo přemýšlel o něčem jiném. \enquote{Známky,} které student dostává, způsob, jakým se s ním zachází, signály, které se mu vysílají -- písemné, ústní i jiné -- to vše závisí na jediném faktoru: na jeho schopnosti a ochotě bezvýhradně podřídit své vlastní touhy, úsudek a rozhodnutí přáním \enquote{autority.} Pokud to dělá, je považován za \enquote{dobrého.} Pokud tak nečiní, je považován za \enquote{špatného.}

Tento způsob indoktrinace nebyl náhodný. Školství ve Spojených státech, a vlastně i ve většině světa, bylo záměrně vytvořeno podle pruského systému \enquote{vzdělávání,} který byl \emph{navržen} s jasným \emph{cílem} vychovat z lidí poslušné nástroje vládnoucí třídy, snadno ovladatelné a rychle bezmyšlenkovitě poslouchající, zejména pro vojenské účely. Jak vysvětloval Johann Fichte, jeden z tvůrců pruského systému, cílem této metody bylo \enquote{zformovat} studenta tak, aby \enquote{prostě nemohl chtít jinak} než to, co po něm \enquote{autorita} \emph{chce}. V té době se otevřeně přiznávalo, že tento systém je prostředkem k psychickému zotročení obyvatelstva vůli vládnoucí třídy. A přesně to dělá i nadále, a to po celém světě, včetně USA.

Důvodem, proč většina lidí dělá to, co jim \enquote{autorita} nařídí, bez ohledu na to, zda je příkaz morální nebo racionální, je to, že přesně k tomu byli vycvičeni. Všechno v autoritářském \enquote{školství} (a autoritářské výchově), dokonce i ta moderní verze, která předstírá, že je starostlivá a otevřená, neustále vtlouká mladým lidem do hlavy představu, že jejich úspěch, jejich dobrota, jejich hodnota jako lidských bytostí se měří podle toho, jak dobře poslouchají \enquote{autoritu.}

Divíte se, že většina dospělých místo toho, aby na základě logiky a důkazů dospěla k vlastním závěrům, hledá \enquote{autoritu,} která by jim řekla, co si mají myslet? Divíte se, že když muž s odznakem začne štěkat příkazy, většina dospělých se bez otázek nesměle podřídí, i když se ničeho nedopustili? Divíte se, že se většina dospělých podvolí jako ovce jakémukoli výslechu a prohlídce, které jim \enquote{strážci zákona} chtějí provést? Divíte se, že mnoho dospělých běží za nejbližším \enquote{orgánem,} aby vyřešili jakýkoli problém nebo urovnali spor? Divíte se, že většina dospělých splní jakýkoli příkaz, ať už je jakkoli iracionální, nespravedlivý nebo nemorální, pokud si toho, kdo příkaz vydává, představují jako \enquote{autoritu?} Překvapuje vás něco z toho ve světle skutečnosti, že téměř všichni prošli mnohaletým záměrným výcvikem, aby se takto chovali?

Experimenty doktora Milgrama jasně ukázaly, že lidé, které produkuje i naše moderní, údajně osvícená společnost, dokonce i ve starých dobrých Spojených státech -- této údajné baště svobody a spravedlnosti -- jsou většinou bezcitnými, nezodpovědnými a nemyslícími nástroji toho, kdo si nárokuje právo vládnout. Když jsou lidé záměrně vychováváni k tomu, aby se pokorně podřizovali bestii zvané \enquote{autorita} -- když jsou učeni, že je důležitější poslouchat než soudit -- proč bychom se měli vůbec divit vydírání, útlaku, terorismu a masovým vraždám, které jsou páchány jen proto, že to samozvaná \enquote{autorita} nařídila? Celé lidské dějiny ukazují tento smrtící vzorec tak jasně, jak jen to je možné: Všichni, kteří se o to pokoušejí, jsou v tomto ohledu velmi opatrní a zřejmí: několik zlých vládců + mnoho poslušných poddaných = rozsáhlé bezpráví a útlak.

\section{Vytváření zrůd}

Měli bychom se také zmínit o psychologické studii, která byla provedena na Stanfordově univerzitě v roce 1971 a v jejímž rámci bylo zřízeno jakési falešné vězení, kde desítky studentů byly určeny jako falešní vězni a další jako falešní dozorci. Experiment musel být předčasně ukončen již po šesti dnech, protože ti, kterým byla svěřena \enquote{autorita} (dozorci), se stali vůči svým vězňům šokujícím způsobem bezcitnými, hrubými a sadistickými.

Je třeba poznamenat, že týrání, kterého se \enquote{dozorci} dopouštěli, šlo dokonce nad rámec toho, co jim bylo nařízeno těmi, kdo experiment prováděli, což bylo \emph{určeno} k ponižování a dehonestování vězňů. To ukazuje, že osobní zlomyslné nebo sadistické sklony jedince jsou významným faktorem, který k takovému zneužívání přispívá, ale že většina lidí takové sklony otevřeně projevuje pouze tehdy, když se jim dostane postavení \enquote{autority,} o němž se domnívají, že jim k tomu dává svolení. Stejný jev můžeme pozorovat u nejrůznějších druhů zneužívání moci, ať už jde o úředníka, který si dělá zálusk na moc, vojáka nebo policistu, který rád šikanuje nebo napadá civilisty, nebo o jakéhokoli jiného úředníka, který rád panuje nad ostatními. Ty ukazují, že víra v autoritu nejenže umožňuje, aby se v zásadě dobří lidé stali nástroji útlaku a nespravedlnosti, ale také zvýrazňuje a dramaticky zesiluje jakýkoli potenciál zloby, nenávisti, sadismu a lásky k panování, který tito lidé mohou mít. Pověra o autoritě začíná tím, že z průměrných lidí dělá pouhé \emph{pachatele} zla (což Arendtová popsala jako \enquote{banalitu zla}), ale pak pokračuje tím, že z takových lidí dělá \emph{vnitřně} zlé lidi, protože je přesvědčuje, že mají právo, nebo dokonce povinnost, zneužívat a utlačovat jiné lidi. To se projevuje v chování vojáků, policistů, státních zástupců, soudců, dokonce i úředníčků. Každý, jehož práce spočívá v obtěžování, vydírání, vyhrožování, donucování a ovládání slušných lidí, se dříve či později stane přinejmenším bezcitným, ne-li přímo sadistickým. Člověk se nemůže neustále chovat jako zrůda, aniž by se jí nakonec nestal.

Další důležitou věcí, kterou je třeba si uvědomit, jak ukazují nesčetné příklady zneužití moci, je to, že ačkoli víra v autoritu může vést lidi k tomu, aby ubližovali druhým, tatáž víra často nedokáže \emph{omezit} rozsah, v jakém představitelé \enquote{autority} ubližují druhým lidem. Například mnozí jedinci, kteří by sami od sebe nikdy neutiskovali nevinného člověka, se stávají \enquote{policisty,} čímž získávají \enquote{legální} pravomoc páchat určitou míru útlaku. Přesto v mnoha případech nakonec daleko překročí \enquote{legální} útlak, který jsou \enquote{oprávněni} páchat, a stanou se z nich sadistická, mocí posedlá monstra. Totéž, možná ještě více, platí o vojácích.

Důvodem, proč je tolik válečných veteránů nakonec hluboce emocionálně traumatizováno, možná není ani tak přemýšlení o tom, čeho byli svědky, jako spíše přemýšlení o tom, co sami udělali. Vysoký počet sebevražd mezi válečnými veterány tuto tezi podporuje. Nedává smysl, aby si někdo přál vlastní smrt jen proto, že \emph{viděl} něco strašného. Mnohem větší smysl dává, když si někdo přeje vlastní smrt, protože sám něco strašného \emph{udělal} a vlastně se něčím strašným \emph{stal}.

Důvod, proč víra v autoritu může vést lidi k páchání zla, ale nakonec nemůže omezit zlo, které páchají, je prostý. Nehledě na jakákoli \enquote{technická} omezení, která by pro představitele \enquote{autority} měla existovat, hlavní koncept, kterému je vymahatel naučen, a hlavní koncept, který musí přijmout, aby mohl vykonávat svou práci, je, že jako představitel \enquote{autority} stojí nad prostým lidem a má morální právo jej násilím ovládat. Stručně řečeno, učí se, že jeho odznak a jeho postavení z něj činí právoplatného pána všech \enquote{obyčejných} lidí. Jakmile je o této lži přesvědčen, je třeba očekávat, že bude průměrným občanem pohrdat a chovat se k němu s despektem, a to stejným způsobem -- a ze stejného důvodu -- jako se otrokář chová ke svým otrokům nikoli jako k lidským bytostem, ale jako k majetku, na jehož pocitech a názorech nezáleží o nic víc než na pocitech a názorech pánova dobytka nebo jeho nábytku.

Je velmi výmluvné, že mnozí moderní \enquote{strážci zákona} se rychle rozzlobí, dokonce se stanou násilnými, když běžný občan prostě promluví s \enquote{policistou} jako rovný s rovným, místo aby nasadil tón a chování podřízeného podřízeného. Tato reakce je opět naprosto stejná -- a má stejnou příčinu -- jako reakce otrokáře na \enquote{vzpurného} otroka, který s ním mluví jako se sobě rovným. Existuje spousta příkladů, zachycených v četných videích policejního týrání na internetu, kdy údajní představitelé \enquote{autority} propadají zuřivosti a uchylují se k otevřenému násilí jen proto, že někdo, koho oslovili, s nimi mluvil tak, jak by jeden dospělý mluvil s druhým, místo aby mluvil tak, jak by poddaný mluvil s pánem. Státní žoldáci tento nedostatek ústupků označují za to, že někdo má \enquote{postoj.} V jejich očích se někdo, kdo s nimi jedná jako s obyčejnými smrtelníky, jako by byl na stejné úrovni jako všichni ostatní, rovná projevu neúcty k jejich údajné \enquote{autoritě.}

Stejně tak každý, kdo nesouhlasí se zadržením, výslechem nebo prohlídkou ze strany \enquote{strážců zákona,} je státními žoldáky automaticky vnímán jako nějaký potížista, který má co skrývat. Skutečným důvodem, proč takový nedostatek \enquote{spolupráce} autoritářským vymahatelům vadí, je opět to, že se rovná tomu, že se k nim lidé chovají jako k obyčejným lidem, místo aby se k nim chovali jako k nadřazeným bytostem, za které se sami považují. Kdyby například někoho konfrontoval cizí člověk \emph{bez} odznaku, který by ho začal vyslýchat zjevně obviňujícím způsobem a požádal by, aby mu mohl prohledat kapsy, auto a dům, nejenže by to obviněný téměř jistě odmítl, ale pravděpodobně by byl takovou žádostí i pobouřen. \enquote{Samozřejmě, že se mi nemůžete hrabat ve věcech! Kdo si myslíte, že jste?} Když však cizí lidé s odznaky vznášejí takové požadavky, jsou to oni, kdo se urazí, když cíle jejich dotěrného, neoprávněného obtěžování, obviňování a prohlídek vznesou námitky a odmítnou \enquote{spolupracovat.} I když \enquote{policisté} dobře vědí, že čtvrtý a pátý dodatek Ústavy USA výslovně nařizují, že osoba nemá žádnou \enquote{zákonnou} povinnost odpovídat na otázky nebo souhlasit s prohlídkou, takovou \enquote{nespolupráci} -- tj. neschopnost bezvýhradně se podřídit každému rozmaru a požadavku vymahatele -- \enquote{policisté} stále považují za známku toho, že osoba musí být nějaký zločinec a nepřítel státu. Z pohledu \enquote{strážců zákona} by se k představitelům \enquote{autority} choval stejně jako ke všem ostatním jen podlý ubožák.

Znovu opakuji, že většina těchto lidí se na svět dívá jinak, než se stali \enquote{strážci zákona.} Při svém autoritářském výcviku v oblasti vymáhání práva jsou výslovně učeni jednat s lidmi jako s podřízenými, snažit se vždy získat nadvládu nad všemi a vším v okamžiku, kdy dorazí na místo činu, a všem říkat, kam mají jít, co mají dělat, kdy mohou mluvit atd. Nejenže jim říkají, že mají právo všem poroučet, což by bylo dost nebezpečné; jsou vycvičeni v tom, že musí v každé situaci použít cokoli -- příkazy, zastrašování nebo přímé násilí -- aby přiměli všechny přítomné podřídit se jejich \enquote{autoritě,} a učí je, že je zločin, pokud se někdo bezvýhradně nepodřídí jejich vůli, což charakterizují jako \enquote{neuposlechnutí zákonného rozkazu.}

Je také velmi příznačné, že je zvykem, že policie se hned po příjezdu na místo činu ujistí, že nikdo jiný není ozbrojen žádnou zbraní, a každého, kdo ji má, odzbrojí, a to ještě předtím, než se dozví cokoli dalšího o tom, kdo jsou lidé a co se děje, a dokonce bez ohledu na to, zda jsou lidé \enquote{legálně} ozbrojeni. Zřejmým účelem tohoto postupu je okamžitě vytvořit obrovskou nerovnováhu moci, kdy pouze \enquote{strážci zákona} mají možnost násilím vnutit svou vůli ostatním. Představte si, jakou aroganci vyžaduje průměrný občan, který přijde na nějaké místo činu, nezná situaci a zúčastněné osoby a jeho první myšlenkou je: \enquote{Nikdo nesmí mít zbraň, jen já.} Tohle je pro něj jako první myšlenka.

Stručně řečeno, \enquote{strážci zákona} jsou \emph{vycvičeni} k tomu, aby byli despotickými megalomany a aby se ke všem ostatním chovali jako k dobytku. A protože lidská povaha je taková, jaká je, každý, kdo se takto běžně chová k ostatním -- tak, jak se to vyžaduje od \enquote{strážců zákona} -- se naučí ostatními pohrdat a chovat se k nim s opovržením, neúctou a nepřátelstvím. Ať už je jedinec na začátku jakkoli dobrého či špatného srdce, způsob, jak v něm probudit to \emph{nejhorší}, je dát mu \enquote{autoritu} nad ostatními.

(\emph{Osobní poznámka autora: Několik bývalých policistů mi osobně řeklo, že odešli od policie poté, co si všimli, že tato práce a jejich údajná \enquote{autorita} z nich pomalu dělá zrůdy -- jeden z nich použil přesně toto slovo.})

Ve skutečnosti se mnozí \enquote{strážci zákona} snaží být \enquote{hodnými chlapíky} a snaží se chovat k ostatním s úctou. Nakonec však nemohou s ostatními jednat jako rovný s rovným a přitom být \enquote{strážci zákona.} Mohou se chovat přátelsky, a dokonce se za to omlouvat (např. \enquote{Promiňte, ale budu vás muset požádat, abyste...}), ale jejich práce od nich stále vyžaduje, aby násilně ovládali a vydírali ostatní, a to nejen ty, kteří někomu skutečně ublížili. Policista nemůže jednat s ostatními jako se sobě rovnými, aniž by přišel o práci. Představte si policistu, který by zastavoval v dopravě, prohledával místa, zadržoval a vyslýchal lidi nebo používal fyzickou sílu proti někomu \emph{pouze} v situacích, kdy byste se \emph{vy} cítili oprávněni takové věci dělat sami, aniž by vám nějaký odznak nebo \enquote{zákon} říkal, že můžete.

Totéž platí pro \enquote{státní} vyšetřovatele, \enquote{státní} zástupce a soudce. Každý \enquote{státní} zaměstnanec, který by odmítl vyšetřovat, stíhat nebo odsoudit někoho za \enquote{zločin} bez oběti, by rychle přišel o práci. O tom, které \enquote{zákony} budou prosazovat, nerozhodují zástupci \enquote{autority.} Pokud existují morálně nelegitimní \enquote{zákony} (jako že vždy existují), jsou všechny složky autoritářských \enquote{orgánů činných v trestním řízení} povinny je prosazovat, a tím napomáhat vydírání a útlaku nevinných lidí. I když je velká část jeho činnosti zaměřena na skutečné zločince -- ty, kteří se dopustili agresivních činů vůči ostatním -- každý \enquote{strážce zákona} se v rámci své práce musí sám dopouštět agresivních činů. Existují i takoví, kteří nedělají téměř nic \emph{jiného} než iniciují násilí, například \enquote{výběrčí daní,} protidrogoví agenti a imigrační agenti. Díky tomu je téměř ve všech případech doslova nemožné pracovat pro \enquote{stát,} aniž by se člověk dopouštěl nemorálních agresivních činů. Být \enquote{strážcem zákona} a být morální osobou se téměř vždy vzájemně vylučuje.

Jakkoli zdvořile mohou vykonávat svou práci a navzdory tomu, že jdou i po skutečných zločincích (takových, kteří mají oběti), jsou \enquote{strážci zákona} vždy profesionálními agresory, kteří si násilím a hrozbou násilí podřizují lidi vůli politiků. A každý, kdo to dělá, pokud už neměl určitou míru pohrdání a nenávisti k bližnímu, si ji téměř jistě vypěstuje. Jinak řečeno, i ten nejhodnější a nejpřátelštější otrokář, pokud bude nadále věřit v oprávněnost otroctví a bude ho nadále praktikovat, se bude dopouštět zla a bude škodit lidem, které si představuje jako svůj oprávněný majetek. A přirozeně si vůči obětem své agrese vypěstuje určitý stupeň pohrdání a bude se k nim chovat opovržlivě.

Schopnost víry v autoritu způsobit škodu a zároveň \emph{neschopnost} tuto škodu \emph{omezit}, jakmile si pán představuje, že má právo vládnout nad svými \enquote{podřízenými,} lze pozorovat nejen na individuální bázi, ale i ve velkém měřítku. Většina debat a spisů, které vedly k ratifikaci americké ústavy, se soustředila na \emph{omezení} pravomocí, které bude federální vláda mít, a na diskusi o tom, co všechno \emph{nesmí} dělat. Listina práv je například seznam věcí, které má vláda USA podle ústavy \emph{zakázáno} dělat. Devátý a desátý dodatek z něj ve skutečnosti činí otevřený seznam, takže by federální \enquote{vláda} teoreticky neměla dělat nic jiného než to, k čemu ji Ústava výslovně \enquote{opravňovala.} Nicméně s výjimkou třetího dodatku je \enquote{Listina práv} shodou okolností také seznamem práv, která federální agenti porušují každý den, a to v každém státě federace. Ve skutečnosti, ať už na úrovni jednotlivce, nebo na úrovni státu, když někomu řeknete: \enquote{Máš právo vládnout ostatním, ale pouze v těchto mezích,} dříve nebo později to povede k tomu, že tato osoba bude ovládat ostatní, aniž by uznávala \emph{jakékoli} meze své moci.

Z dlouhodobého hlediska nic takového jako \enquote{omezená vláda} neexistuje a ani existovat nemůže, protože jakmile je někdo ostatními akceptován jako právoplatný pán a věří, že má morální právo vládnout, nebude nic a nikdo \enquote{nad ním,} kdo by ho mohl omezit. Uvnitř \enquote{státu} se může vyšší \enquote{autorita} rozhodnout omezit nižší \enquote{autoritu,} ale logika a zkušenost ukazují, že autoritářská hierarchie, brána jako celek, se nikdy nebude omezovat dlouho. Proč by to dělala? Proč by pán někdy stavěl své vlastní zájmy pod zájmy svých otroků?

Ústava USA je toho dokonalým příkladem: kus pergamenu, který údajně uděloval velmi omezenou \enquote{autoritu} určitým lidem, ale který naprosto selhal v tom, aby těmto lidem zabránil překročit tyto hranice a vytvořit něco, co se nakonec rozrostlo v nejmocnější autoritářské impérium v dějinách. A tento problém nelze vyřešit tím, že uvnitř téže autoritářské struktury ustanovíme další skupinu pánů (např. \enquote{soudní systém}), jejichž předpokládaným účelem je prosazovat omezení první skupiny pánů. \enquote{Dělba moci} a \enquote{brzdy a protiváhy} a \enquote{řádný proces} postrádají smysl, pokud jsou jak páni, tak ti, kteří jsou určeni k jejich omezování, součástí téže autoritářské organizace.

\section{Démonizace oběti}

Je důležité zdůraznit, že v Milgramových experimentech si pokusné osoby myslely, že trestají nevinné cizince. Neexistovalo žádné obvinění, že trestaný je špatný člověk nebo že udělal něco nemorálního. Mělo by být zřejmé, že pokud průměrný člověk na příkaz \enquote{autority} způsobí bolest nevinnému člověku, způsobí ji -- s menším váháním a pocitem viny -- i někomu, o kom si myslí, že si takovou bolest \emph{zaslouží}.

Americká armáda (a pravděpodobně i mnoho dalších armád) provedla mnoho výzkumů, aby zjistila, co lze udělat pro překonání přirozeného odporu vojáka k zabíjení, aby zabíjel na rozkaz. A jedním z nejúčinnějších způsobů, jak toho dosáhnout, je démonizovat a dehumanizovat toho, koho má zastřelit. V moderních válkách \enquote{státy} obou stran neustále krmí své vojáky propagandou, jejímž cílem je vykreslit \enquote{nepřítele} jako bandu bezcitných, krutých, sadistických a nelidských monster. Paradoxně se tak stává sebenaplňujícím se proroctvím, protože taková propaganda dělá z \emph{obou} stran bandy bezcitných monster, které se horlivě snaží vyhladit nepřátele, jež nepovažují za plně lidské.

Podobná taktika se používá i při \enquote{vymáhání práva.} Nájemní žoldáci \enquote{státu} jsou mnohem ochotnější páchat na někom křivdy a útlak, pokud byl tento člověk nejprve odlidštěn a démonizován. Dokonce i používaná terminologie -- ze strany pánů, vymahatelů i všech ostatních -- představuje velmi účinnou formu ovládání mysli, která mění vnímání reality jak ze strany vymahatelů, tak ze strany jejich cílů, a tím ovlivňuje chování obou skupin. Takové termíny posilují předpoklad, že poslušnost \enquote{autoritě} je ctnost a neposlušnost je hřích.

Děje se to doslova tak, že jedna skupina lidí vydá příkaz a její vymahatelé ho vnutí masám tím, že trestají neuposlechnutí. Tak to dělá mafie, pouliční gangy, šikana na školních dvorech a všechny \enquote{státy.} Rozdíl je v tom, že když to dělá \enquote{stát,} používá nejen hrozby, ale také indoktrinaci, a to jak vymahatelů, tak široké veřejnosti. Tam, kde je poselství většiny násilíků obvykle přímé a upřímné (\enquote{Dělej, co říkám, nebo ti ublížím}), zahrnuje \enquote{státní} poselství velkou míru psychologie a ovládání mysli, které je nezbytné k tomu, aby se státní žoldáci cítili spravedlivě, když způsobují útlak ostatním. Kontroloři \enquote{státu} se představují jako \enquote{zákonodárci,} kteří mají \emph{právo} \enquote{řídit} společnost, své příkazy představují jako \enquote{zákony} a každého, kdo se jim nepodřídí, vykreslují jako \enquote{zločince.} A na rozdíl od mafiánských \enquote{těžkooděnců} jsou ti, kdo vykonávají odplatu vůči všem, kdo se politikům nepodvolí, líčeni nikoli pouze jako najatí násilníci, ale jako ušlechtilí \enquote{strážci zákona,} kteří spravedlivě chrání společnost před všemi necivilizovanými, opovrženíhodnými \enquote{porušovateli zákona.}

Taková propaganda vede nejen k tomu, že autoritářští vymahatelé provádějí násilí na nevinných lidech, ale také k tomu, že jsou na to hrdí. Prostřednictvím autoritářské indoktrinace jsou přesvědčeni, že \enquote{zločince} staví před \enquote{spravedlnost,} a tím udržují \enquote{právo a pořádek} ve prospěch společnosti. Ve skutečnosti však nejčastěji používají násilí k tomu, aby všechny přinutili poslouchat jakékoli příkazy, které politici vydají, ať už jsou jakkoli nemorální, svévolné, sociálně či ekonomicky destruktivní nebo přímo idiotské.

Je velký rozdíl v konotacích dvou pojmů \enquote{strážce zákona} a \enquote{násilík politiků.} Není však rozdíl v tom, co znamenají doslova. Přesvědčením vymahatelů, že násilí, které používají, představuje spravedlivé a ušlechtilé \enquote{vymáhání práva,} lze však změnit jejich vnímání tak, že budou ochotně a hrdě vnucovat vůli vládnoucí třídy svým bližním. Takových příkladů je tolik, kolik je \enquote{zákonů,} ale všechny spadají do jedné ze dvou kategorií: \emph{zákazy} (kdy politici prohlašují, že jejich poddaní nesmějí dělat určité věci) a \emph{příkazy} (kdy politici prohlašují, že jejich poddaní \emph{musí} dělat určité věci). Pro demonstraci postačí jeden příklad.

\textbf{Zákaz:} Správci vydají nařízení, že jejich poddaní nesmějí držet marihuanu. Tento zákaz je prohlášen za \enquote{zákon} a každý, kdo ho nedodrží, je považován za \enquote{zločince.} Správci pak vynakládají obrovské množství peněz (které seberou svým poddaným na základě jiného \enquote{zákona}) na zaplacení žoldáků, zbraní, obrněných vozidel, věznic atd., a to pouze za účelem zajetí všech, kteří jsou přistiženi při neuposlechnutí jejich \enquote{zákona.}

Vezměme si perspektivu \enquote{policisty,} který má za úkol vymáhat dodržování tohoto \enquote{zákona} a který zjistí, že někdo prodává marihuanu ochotným zákazníkům.

Kdyby tento \enquote{policista} dokázal objektivně zvážit situaci, aniž by jeho vnímání zkresloval mýtus autority, okamžitě by viděl, že jeho \enquote{práce} je nejen nemorální, ale i naprosto idiotská a pokrytecká -- jeho \enquote{prací} je někoho fyzicky zajmout za účelem zavřít ho na dlouhou dobu do klece, a to za něco, co nebylo ani podvodné, ani násilné. Ve skutečnosti až do chvíle, kdy se objevil policista, všichni zúčastnění -- pěstitel, dealer, prodejce, kupující i uživatel -- komunikovali pokojně a dobrovolně. Navíc pokud policista někdy požil alkohol, provinil by se morálně totožným způsobem jako \enquote{zločinec.} Přesto se bude považovat za statečného, spravedlivého a ušlechtilého \enquote{strážce zákona,} když se zúčastní polovojenského ozbrojeného vpádu do domu dotyčného a násilím \enquote{zločince} zajme a odvleče od jeho přátel a rodiny. Pak si policista půjde domů dát pivo a samozřejmě by nereagoval vlídně na nikoho, kdo by se mu v tom pokusil násilím zabránit. Jediný rozdíl -- který ve skutečnosti není vůbec žádný -- spočívá v tom, že politici si vymysleli příkaz týkající se jedné látky ovlivňující mysl (marihuany) a ne druhé (alkoholu). Výsledkem je, že \enquote{policista} bude skutečně věřit, že užívání jedné látky ovlivňující mysl je dobré, zdravé, celoamerické chování, zatímco užívání jiné je pochybné, nemorální a \enquote{zločinné,} a dokonce ospravedlňuje násilné napadení a únos \enquote{pachatelů.}

\textbf{Požadavek:} Kontrolor vydá \enquote{zákon,} podle něhož musí každý jeho poddaný, který vlastní majetek, odvádět kontrolorům každý rok platbu ve výši dvou procent z hodnoty jeho majetku. Tento požadavek se nazývá \enquote{daň z majetku} a je prohlášen za \enquote{zákon} a každý, kdo se mu nepodřídí, je \enquote{zločinec} a \enquote{daňový podvodník.} Správci pak zřídí organizaci \enquote{výběrčích daní,} kteří vyhledají každého, kdo neuposlechne, a buď od něj násilím vymohou peníze, nebo ho násilím vystěhují z jeho majetku, zabaví ho a předají správcům.

Kdyby to ovšem někdo udělal bez veškeré autoritářské propagandy, nazvalo by se to vydíráním: \enquote{Musíš mi platit hromadu peněz, každý rok, jinak tě nenechám bydlet ve tvém vlastním domě.} To by se stalo. A jen málokdo, včetně těch, kteří dnes pracují jako \enquote{výběrčí daní,} by se chtěl stát součástí takového vyděračského systému. Když se však přesně totéž dělá \enquote{legálně,} průměrní lidé nejenže přijmou práci jako součást takového vydírání, ale dají najevo opovržení každým, kdo se tomu brání. Na ty, kteří se pak snaží \emph{nebýt} okrádáni, se pohlíží jako na chamtivé \enquote{daňové podvodníky,} kteří nechtějí platit svůj \enquote{spravedlivý podíl.} A ti, jejichž úkolem je násilně odebírat peníze nebo majetek takovým \enquote{daňovým podvodníkům,} to obvykle dělají s pocitem spravedlnosti, protože skutečně věří, že \enquote{autorita zákona} dokáže vzít to, co je obvykle nemorálním činem -- krádež, lichvářství a vydírání -- a přeměnit to v něco spravedlivého a legitimního. Páchají tedy masové loupeže, mají z toho dobrý pocit a cítí opovržení vůči svým obětem. To je síla nejnebezpečnější pověry.

Etatisté často tvrdí, že zdanění není krádež, protože \enquote{státy} používají daňové příjmy na věci, které slouží \enquote{obecnému blahu,} takže jde jen o to, že lidé platí za zboží a služby, které dostávají. Takový argument ignoruje základní povahu situace. Na jednoduchém příkladu je dvojí metr zřejmý. Předpokládejme, že k vám přijde cizí člověk a řekne vám, že vám posekal trávník nebo že vám doma nechal nějakou věc, a nyní po vás bude požadovat, abyste mu dali 1 000 dolarů, ačkoli jste s žádnou takovou dohodou nikdy nesouhlasili. Je zřejmé, že by se jednalo o vydírání a vy byste neměli povinnost zaplatit, i kdyby vám skutečně posekal trávník nebo vám něco nechal. Nikdo nemá právo vám bez vašeho souhlasu poskytnout nějakou věc nebo službu -- když jste o ni nežádali a nechtěli jste ji koupit -- a pak si od vás násilím vzít to, co prohlásí za hodnotu této věci nebo služby. A přesto přesně to každý \enquote{stát} na všech úrovních vždy dělá.

Když jsou cíle autoritářské agrese úspěšně démonizovány a dehumanizovány, neexistuje v podstatě žádná hranice pro míru násilí a nespravedlnosti, které jsou lidé věřící v autoritu ochotni páchat. Kdo by snad ještě doufal, že svědomí amerických vojáků a \enquote{strážců zákona} může omezit míru bezpráví, které jsou ochotni páchat na zcela cizích lidech, má k dispozici spoustu skutečných příkladů, které dokazují opak. Asi nejznámějším je masakr v My Lai během války ve Vietnamu, kde američtí vojáci nejen zavraždili stovky neozbrojených civilistů, většinou žen a dětí, ale některé z nich také sexuálně napadli a mučili, přičemž někteří vojáci se podle vlastních výpovědí otevřeně radovali z utrpení a smrti svých obětí. Tohle dělali \emph{američtí} vojáci v důsledku své loajality k mýtu autority v kombinaci s démonizací a dehumanizací svých obětí. Sami vojáci to vyjádřili naprosto bez obalu: jeden z nich řekl, že \enquote{jen plnili rozkazy,} a jiný, že většina amerických vojáků tam \enquote{\emph{nepovažovala Vietnamce za lidi}.} (Je třeba poznamenat, že někteří američtí vojáci se s malým úspěchem snažili masakr zastavit nebo omezit.) Ačkoli se jedná o jeden z nejznámějších příkladů válečných zvěrstev spáchaných americkými vojáky, rozhodně není jediný. Ve skutečnosti se příklady sadistického chování některých amerických vojáků stále objevují. Zatímco v Milgramových experimentech by některé pokusné osoby daly najevo -- slovně nebo svým chováním -- že se cítí špatně, když způsobují újmu nevinnému cizinci, \enquote{strážci zákona} a vojáci, kteří jsou nejprve naučeni pohrdat \enquote{nepřítelem,} plní autoritářské příkazy ještě ochotněji, často způsobem, který ukazuje, že je pro ně \emph{slast} způsobovat svým obětem bolest a smrt.

Jasně to ukázaly snímky z irácké věznice Abú Ghrajb, na nichž bylo vidět, že američtí vojáci, muži i ženy, nejenže prováděli psychické a fyzické mučení, ale projevovali potěšení a zábavu nad utrpením svých obětí a s radostí pózovali před kamerou, zatímco ponižovali, napadali, mučili a znásilňovali své vězně. (Bushova i Obamova administrativa zabránila zveřejnění mnoha fotografických důkazů tohoto mučení, protože se obávala, že by tyto snímky mohly ovlivnit mínění Američanů i cizinců o armádě a \enquote{zemi.}) Ačkoli bylo takové mučení prováděno na příkaz nejvyšších úrovní \enquote{státu,} je důležité poznamenat, že ti, kdo tyto příkazy \enquote{autority} vykonávali, zjevně projevovali sadistické potěšení z bolesti a utrpení, které působili jiným lidským bytostem. Někdo, koho vnímali jako \enquote{autoritu,} jim řekl, že nenávidět a ubližovat \enquote{nepříteli} je ušlechtilé a spravedlivé. A tak to dělali a užívali si to.

Stejný přístup a mentalitu lze pozorovat při různých akcích \enquote{orgánů činných v trestním řízení,} jako byl útok na Ruby Ridge v roce 1992 a zásah, zásah a nakonec masakr u Waco v Texasu v roce 1993. Ani v jednom případě nešlo o to, že by \enquote{autorita} šla po někom, kdo skutečně někomu ublížil nebo někoho ohrožoval. Místo toho šlo v obou případech o polovojenské útoky založené na údajném držení \enquote{nelegálních} střelných zbraní. Při incidentu ve Waco nakonec zemřelo osmdesát lidí, včetně mužů, žen a dětí, kteří byli mimo jiné týdny psychicky a fyzicky mučeni nedostatkem spánku a plynem CS. Oběti byly démonizovány, a to jak pro veřejnost, tak pro pracovníky \enquote{orgánů činných v trestním řízení,} a \enquote{státní} agresoři projevovali jak pohrdání svými oběťmi, tak nadšení z myšlenky na jejich zabití. Stejný obecný přístup lze vidět na desítkách videí \enquote{policejní brutality,} která zachycují policisty nadšeně šikanující a dokonce fyzicky napadající lidi, kteří nikoho neohrožují a kteří se ani nebrání a nekladou odpor. Je to přímý důsledek přesvědčení \enquote{strážců zákona,} že všichni ostatní jsou pod jejich úroveň a že jako představitelé \enquote{autority} mají právo na to, aby se k nim všichni ostatní chovali jako k nadřízeným, plazili se před nimi a bezvýhradně plnili jejich příkazy. Stejný vzorec lze pozorovat i u \enquote{výběrčích daní} a dalších byrokratů.

Do jaké míry víra v autoritu skutečně \emph{vytváří} sadistické sklony a do jaké míry pouze uvolňuje sklony, které tu již byly, není moc důležité. Podstatné je, že mýtus autority tím, že předstírá, že zbavuje jednotlivce odpovědnosti za jeho vlastní činy, a tím, že mu přikazuje, aby ubližoval druhým, a říká mu, že ubližovat určitému cíli je nejen přípustné, ale i \emph{cnostné}, mění miliony průměrných, jinak slušných lidí v monstra a sadistické pachatele zla. Jakékoli faktory, které normálně nutí lidi chovat se slušně a nenásilně -- ať už jsou to vnitřní ctnosti jedince, jeho oddanost morálním zásadám nebo náboženskému přesvědčení, nebo prostě jeho obavy z toho, co by si o něm mohli myslet ostatní nebo co by mu mohli udělat -- jsou snadno poraženy a překonány vírou v autoritu. Stručně řečeno, nejúčinnějším způsobem, jak umlčet lidskost a slušnost každého jednotlivce, je naučit ho respektovat a poslouchat \enquote{autoritu.}

\section{Co znamená odznak}

Ti, kdo tvrdí, že jednají pod \enquote{státní autoritou,} se obvykle snaží dát najevo, že tak činí. Když si voják oblékne vojenský oděv, napochoduje do formace nebo nasedne do vojenského vozidla; když si policista oblékne uniformu a nasedne do auta s nápisem \enquote{POLICIE;} když \enquote{státní} agent v civilu -- ať už z FBI, IRS, U.S. Marshals nebo z jakékoli jiné agentury -- ukáže svůj \enquote{odznak} nebo oznámí svůj \enquote{oficiální} titul, činí tím velmi konkrétní prohlášení, které lze shrnout následovně:

\enquote{\emph{Nejednám jako myslící, odpovědná a nezávislá lidská bytost a nemělo by se se mnou tak jednat. Nejsem osobně zodpovědný za své činy, protože nejednám z vlastní svobodné vůle ani z vlastního úsudku o tom, co je správné a co ne. Místo toho jednám jako nástroj něčeho nadlidského, něčeho, co má právo vám vládnout a ovládat vás. Jako takový mohu dělat věci, které vy dělat nemůžete. Mám práva, která vy nemáte. Musíte dělat, co vám řeknu, podřizovat se mým příkazům a chovat se ke mně jako ke svému nadřízenému, protože nejsem obyčejná lidská bytost. Povznesl jsem se nad ni. Díky své bezvýhradné poslušnosti a loajalitě vůči svým pánům jsem se stal součástí nadlidské entity zvané }stát\enquote{ a jednám na základě její }autority.\enquote{ V důsledku toho se na mě nevztahují pravidla lidské morálky a mé činy by neměly být posuzovány podle obvyklých měřítek lidského chování}.}

Toto bizarní, mystické, sektářské přesvědčení zastává \emph{každý} \enquote{strážce zákona} na světě. Je strašně nebezpečné, aby si někdo představoval, že má výjimku ze základních pravidel dobra a zla, a přesto si právě to každý představitel \enquote{státu} představuje. Přestože vojáci a \enquote{strážci zákona} obvykle s velkou hrdostí vystavují své \enquote{služební} uniformy, ve skutečnosti tím veřejně dávají najevo, že trpí bludy, mají zcela pokřivený a dementní pohled na realitu a zradili to, co z nich udělalo lidi: svobodnou vůli a s ní spojenou osobní odpovědnost. Každý člověk, který tvrdí, že jedná jménem \enquote{autority,} tím dává najevo, že přijal naprosto směšnou lež: že jeho postavení, jeho odznak, jeho funkce dramaticky \emph{změní} to, jaké chování je morální a jaké je nemorální. Tato myšlenka je zjevně šílená, ale málokdy je jako taková rozpoznána, protože i oběti vymahatelů tento blud sdílejí.

\section{Vznešené motivy, zlé činy}

Je důležité znovu zdůraznit, že ti, kteří se stanou \enquote{strážci zákona} a vojáky, tak většinou činí z touhy bojovat za spravedlnost. Nicméně kvůli víře v autoritu jsou jejich ušlechtilé úmysly často nakonec zneužity k poškozování nevinných a ochraně viníků. Protože policista má \enquote{vymáhat právo} a voják má plnit rozkazy, jejich vlastní hodnoty a záměry jsou přebity záměry těch, kteří rozkazy vydávají. Bez ohledu na vojenskou náborovou propagandu nabádající mladé muže a ženy, aby vstoupili do armády a bojovali za pravdu a spravedlnost, je skutečným úkolem vojáka zabít toho, koho mu pánové přikážou zabít. Je to tak jednoduché. Kolik Američanů by se samo od sebe rozhodlo jít do cizí země a zabíjet tam úplně cizí lidi? Velmi málo. Kolik Američanů by se na vlastní pěst, kdyby se ocitli v cizí zemi, cítilo oprávněno chodit od dveří ke dveřím, vyslýchat cizí lidi se zbraní v ruce, vnikat do jejich domovů a prohledávat je, protože si myslí, že se v oblasti \emph{mohou} nacházet nějací skutečně zlí lidé? Jen málokdo. To jsou činy, o nichž by mu smysl pro morálku téměř každého jednotlivce řekl, že jsou špatné. Ale když se někdo dobrovolně přidá k autoritářské armádě, záměrně vypíná svůj vlastní úsudek a svědomí ve prospěch toho, aby prostě dělal, co se mu řekne.

Ačkoli vojáci někdy používají legitimní sílu, například v boji proti agresorům a útočníkům, sami se běžně chovají jako agresoři a útočníci. Jinak by \enquote{státní} armáda fungovat nemohla. Představte si armádu, která chodí od domu k domu a zdvořile žádá každého majitele domu o povolení přejít přes jeho pozemek. Pouhé označení situace jako \enquote{válka} způsobuje, že si věřící ve stát představují, že se na ni nevztahují obvyklé normy lidského chování. Pod záminkou nutnosti vojáci vnikají na cizí pozemek, kradou, zastrašují, vyhrožují, napadají, vyslýchají, mučí a vraždí. A dělají to dokonce i proti lidem, které považují za své spojence. Vojenská invaze a okupace Iráku žoldáky americké \enquote{federace,} která byla údajně provedena na obranu iráckého lidu, byla příkladem rozsáhlé agrese a nátlaku -- a byla tedy nemorální -- i když vytlačila režim, který se provinil ještě horší mírou zastrašování a vraždění (režim Saddáma Husajna). Přesto je údajná špatnost nepřítele často uváděna jako ospravedlnění autoritářského nátlaku. Ve skutečnosti se dnes i v historii rozsáhlé násilí na nevinných vždy odehrávalo ve jménu \enquote{boje za svobodu} nebo \enquote{boje proti nespravedlnosti.} Dokonce i když nacisté napadli Polsko, zinscenovali nejprve řadu akcí pod falešnou vlajkou a propagandistických kousků, souhrnně známých jako \enquote{operace Himmler,} aby mohli předstírat, že invaze je ospravedlnitelným aktem sebeobrany. Pravdou je, že i když je zlo nepřátelského režimu snadno viditelné, takže se celkový boj zdá být pro jednu stranu spravedlivý, násilí páchané autoritářskými armádami není nikdy zaměřeno pouze na skutečné agresory na druhé straně. Struktura a metodika hierarchických armád způsobuje, že nevinní jsou \emph{vždy} tak či onak obětí, a to nejen náhodou, ale i záměrně. Smečková mentalita, která je tak velkou součástí vlastenectví, to činí nevyhnutelným.

Za druhé světové války američtí vojáci považovali za nepřítele \enquote{skopčáky} (Němce) a \enquote{japončíky} (Japonce), místo aby za nepřítele považovali jednotlivce, kteří se skutečně dopouštěli agresivních činů proti nevinným lidem -- což je koncepce, která by vyžadovala, aby každý voják neustále používal své vlastní individuální vnímání a morální úsudek při hodnocení každé situace, s níž se setkává, což je neslučitelné s autoritářským velením. Samozřejmě, že z lidí, kteří odpovídají definici \enquote{skopčáků} nebo \enquote{japončíků,} se mnozí na konfliktu nijak nepodíleli (kromě toho, že ho financovali placením \enquote{daní,} jak je uvedeno níže). Ale na obou stranách každé války se \enquote{státní} armády a jimi používaná propaganda vždy zaměřují na obecnou \emph{kategorii} lidí a démonizují ji, a nikoli pouze jednotlivce, kteří násilí skutečně iniciovali. Výsledkem je, že obrovské demografické skupiny nakonec dostanou rozkaz podmanit si nebo vyhladit jedna druhou, takže v žádné válce mezi \enquote{národy} není ani jedna strana nikdy tou \enquote{dobrou,} protože obě armády vždy používají násilí proti nevinným lidem i proti jiným vojákům.

Snad jedním z nejodpornějších příkladů bylo svržení atomových bomb na Nagasaki a Hirošimu, které představovaly dva zdaleka nejhorší individuální teroristické činy a masové vraždy v dějinách. Dohromady měly za následek smrt přibližně dvou set tisíc civilistů -- což je asi sedmdesátkrát horší než počet obětí útoků na Světové obchodní centrum z 11. září 2001. Přiznaným cílem bylo způsobit strach, bolest a smrt obyvatelstvu celé země, aby byla vládnoucí třída této země donucena podřídit se vůli jiné vládnoucí třídy. Ironií je, že to dokonale odpovídá definici \enquote{terorismu,} kterou používá \enquote{vláda} Spojených států, až na to, že tato definice příhodně \emph{vylučuje} činy, které jsou \enquote{legální} a/nebo je páchají \enquote{státy.} Pokud \enquote{státní} zaměstnanci obhajují a provádějí násilné činnosti, jejichž cílem je \enquote{\emph{zastrašit nebo donutit civilní obyvatelstvo}} nebo \enquote{\emph{ovlivnit politiku státu zastrašováním nebo donucováním},} pak jsou považovány za legitimní a spravedlivé. Pokud totéž dělá někdo jiný, jedná se o \enquote{terorismus.} (Viz § 2331 hlavy 18 zákoníku Spojených států.)

Jen na okraj, existence jaderných zbraní je zcela důsledkem víry v autoritu. Na rozdíl od mnoha jiných zbraní je nelze použít k čistě obranným účelům. Jediným důvodem, proč byla jaderná bomba vůbec vynalezena a vyrobena, byla autoritářská, nacionalistická, smečková myšlenka, že je možné a spravedlivé vést válku s celou zemí, a že tedy nevybíravé vyhlazení tisíců lidí najednou může být ospravedlnitelné.

Být členem \enquote{státní} armády vyžaduje, aby se člověk podílel na protilidských činech, i kdyby jen nepřímo, bez ohledu na to, jaké ušlechtilé pohnutky ho vedly ke vstupu do ozbrojených sil. Důvod je prostý: jednat na základě vlastního vnímání a úsudku a řídit se vlastním svědomím a vlastním smyslem pro dobro a zlo je naprosto neslučitelné s členstvím v jakékoli \enquote{státní} armádě. Výsledkem je bohužel to, že \emph{obě} strany každé války jsou špatné, protože obě iniciují násilí proti nevinným. Zároveň mají obě strany každé války pravdu v tom, že každá z nich odsuzuje \emph{druhou} stranu za zahájení násilí proti nevinným. Stručně řečeno, dokud budou existovat vojáci ochotní podřídit se deklarované \enquote{autoritě,} a dokonce i vraždit, když jim to autorita nařídí, trvalý mír nebude možný. Ti, kdo bojují za jakýkoli \enquote{stát,} i když se domnívají, že \enquote{bojují za svou zemi,} nemohou nikdy dosáhnout svobody a spravedlnosti, protože vládnoucí třída ze své podstaty nikdy nechce svobodu a spravedlnost, a to ani pro své vlastní poddané, jinak by přestala existovat. Ať jsou jejich motivy jakkoli ušlechtilé a jejich činy jakkoli odvážné, nakonec jediné, čeho mohou \enquote{státní} vojáci kdy dosáhnout, je zotročení a nadvláda.

Mnoho Američanů má dojem, že američtí vojáci mohou (nebo dokonce musí) neuposlechnout jakýkoli nelegální nebo nemorální rozkaz, což Američanům umožňuje představu, že američtí \enquote{státní} žoldnéři se zásadně liší od žoldnéřů jiných režimů. Zrnko pravdy za tímto přesvědčením je, že američtí vojáci jsou (alespoň teoreticky) povinni neuposlechnout \enquote{nelegální} rozkaz, ale toto pravidlo neříká nic o morálce. A protože vládci určují, co je \enquote{legální,} v konečném důsledku toto pravidlo mnoho neznamená. V boji představuje téměř vše, co každá armáda dělá, násilnou agresi a téměř každý rozkaz, který voják dostane, je nemorální (i když \enquote{legální}) rozkaz, ať už jde o vniknutí na soukromý pozemek, vyhození mostu do povětří, zablokování silnice, odzbrojení civilistů, zadržování a vyslýchání náhodně vybraných lidí nebo zabíjení zcela neznámých lidí. Voják, který by takový rozkaz neuposlechl, by byl téměř jistě postaven před válečný soud. Myšlenka, že by člověk měl někdy neuposlechnout rozkaz -- ať už proto, že je nemorální nebo \enquote{nelegální} -- je sice v zásadě v pořádku, ale je v přímém rozporu s celým pojetím \enquote{autority} a s metodami používanými k výcviku vojáků, aby byli slepě poslušnými nástroji jakéhokoli režimu, kterému slouží.

Dokonce i když pravidla nasazení stanoví, že se má střílet pouze v případě, že se na ně střílí, je to stále často neoprávněné. Když se někdo dopustí agrese, má cíl této agrese právo použít jakoukoli sílu, která je nezbytná k zastavení agresora. To znamená, že v mnoha situacích je střelba na vojáky -- včetně amerických -- ze své podstaty oprávněná. Zabít někoho za to, že se bránil agresorům, je vražda, i když jsou agresory američtí vojáci. A téměř každý voják se běžně dopouští nemorálních agresivních činů v domnění, že příkazy \enquote{autority} mu to umožňují. Kdyby nějaký voják skutečně bral vážně myšlenku, že by měl neuposlechnout nemorální rozkaz, první, co by udělal, by byl odchod z armády.

Ti, kdo jednají jako žoldáci \enquote{státu,} i když to dělají s těmi nejlepšími úmysly, budou vždy součástí mašinérie, která páchá agresi stejně často nebo častěji než brání nevinné. Vzhledem k tomu téměř každý voják v boji dělá věci, které by ospravedlnily použití obranného násilí proti němu. Nicméně, jak to invazní armády vždy dělají, američtí vojenští velitelé označují každého, kdo se brání jejich agresi, za \enquote{nepřátelského bojovníka,} \enquote{povstalce} nebo \enquote{teroristu.} Když je agrese páchána ve jménu \enquote{autority,} mnozí pak považují jakýkoli akt sebe\emph{obrany} proti takové agresi za hřích. Ačkoli se američtí autoritáři mohou nad takovým návrhem pohoršovat, pravdou je, že mnoho tisíc lidí na celém světě mělo dobrý důvod střílet na americké vojáky.

Když je člověk, který nikomu neublížil ani nikoho neohrožoval, ve svém vlastním domě, hledí si svého a těžce ozbrojení násilíci vyrazí jeho dveře, namíří na něj a jeho rodinu samopaly, vyhrožují a rozkazují, má majitel domu absolutní právo chránit sebe a svou rodinu jakýmikoli prostředky, včetně zabití ozbrojených vetřelců. Průměrný Američan, pokud by byl \emph{obětí} takového útoku cizích žoldáků, by se cítil naprosto oprávněný použít jakékoli násilí, aby útočníky odrazil, ale pokud by takové útoky \emph{prováděli} jeho spoluobčané v cizí zemi, tentýž Američan, který je prorostlý uctíváním \enquote{autority} a smečkovou mentalitou, bude \enquote{podporovat vojáky} a bude jásat, když američtí vojáci zavraždí majitele domu, který se pokusí násilím bránit takové agresi a násilnictví.

Autoritářské vojenské akce nejsou nikdy čistě obranné. Když \enquote{státy} vyhlašují válku, nikdy to není na obranu nevinných nebo zachování svobody, ačkoli to je vždy \emph{proklamovaný} účel. Když \enquote{státy} vedou válku, je to vždy proto, aby ochránily nebo rozšířily území či jiné zdroje, které daný \enquote{stát} ovládá. Vládnoucí třída ze své podstaty ani nechce, aby její \emph{vlastní} poddaní byli svobodní, natož aby byli svobodní poddaní nějakého cizího vládce. V důsledku toho se sice často říká, že ten, kdo zemře v boji, položil život za svou zemi, ale ve skutečnosti jsou ti, kdo umírají ve válce, jen prostředky, které tyrani vynakládají v různých válkách o území s jinými, konkurenčními bandami tyranů. Lidé jsou krmeni propagandou o hrdinství, obětavosti a vlastenectví, aby zakryli skutečnost, že \enquote{státy} nikdy nevstupují do válek, aby sloužily spravedlnosti nebo svobodě. Dělají to proto, aby posloužily své vlastní moci. Při objektivním zkoumání historie je to zřejmé.

Dokonce i jeden z nejzjevněji ospravedlnitelných vojenských počinů v dějinách -- boj Spojenců ve druhé světové válce proti mocnostem Osy -- sice vedl k porážce \emph{třetího} nejhoršího masového vraha v dějinách (Adolfa Hitlera), ale také k tomu, že vládci spojeneckých zemí v podstatě darovali polovinu Evropy ještě horšímu masovému vrahovi (Josifu Stalinovi). Motivem většiny amerických vojáků, kteří ve válce bojovali, byla nepochybně ochrana dobra před zlem; motivy těch, kteří jim veleli, a tedy i skutečné výsledky úsilí statečných vojáků, však nebyly ničím jiným než autoritářským dobýváním a mocí.

Za druhé světové války bylo možné alespoň naznačit (s jistou dávkou představivosti) možnost invaze do Spojených států, a tím tvrdit, že šlo o akt sebeobrany, protože v sázce byla \enquote{národní bezpečnost.} Ale většina amerických vojenských operací se vůbec netýkala přímého ohrožení USA. V korejské válce zemřelo třicet tisíc Američanů. Nikdo si nepředstavoval, že by Severní Korea chtěla napadnout USA. 50 tisíc Američanů zemřelo ve válce ve Vietnamu. Nikdo si nepředstavoval, že Severní Vietnam napadne USA. Nikdo si nepředstavoval, že armády Iráku nebo Afghánistánu napadnou USA. Záminkou pro takové konflikty byl vždy nejasný důvod, jako je \enquote{boj proti komunismu,} nebo ještě neurčitější záminka \enquote{války proti terorismu} (což je o to ironičtější, že teroristické taktiky byly a jsou běžně používány americkými silami).

Smutnou ironií je, že americká vládnoucí třída je díky legitimitě, kterou si její oběti představují, \emph{jedinou} bandou, která je skutečně schopna dobývat a podrobovat si americký lid. Právě gigantická vojenská mašinérie a všechny válečné hry, do nichž se zapojila, místo aby poskytla alespoň špetku skutečné ochrany americké veřejnosti, jsou tím, co \emph{vytvořilo} většinu existujících zahraničních hrozeb a co se stále používá jako záminka k ospravedlnění útlaku Američanů jejich \emph{vlastním} \enquote{státem,} mimo jiné prostřednictvím orwellovsky pojmenovaného \enquote{PATRIOT Act} (Vlasteneckého zákona). Oblíbená samolepka na nárazníku, která říká \enquote{Pokud miluješ svou svobodu, poděkuj veteránovi,} je pokračujícím příznakem smečkové, stát uctívající propagandy, kterou vládnoucí třídy krmí své poddané, aby páni měli i nadále pěšáky pro své sadistické, destruktivní mocenské hry. I když otrokář bojuje, aby zabránil nějakému jinému otrokáři v krádeži jeho otroků, stále není přítelem samotných otroků.

Je zcela pochopitelné, že někdo, kdo riskoval svůj život, prošel peklem, ublížil nebo zabil jiné lidské bytosti, možná i nevinné, a utrpěl v důsledku toho fyzické nebo emocionální trauma, se zdráhá přijmout, že veškerá jeho odvaha, utrpení a škody, které způsobil ostatním, nakonec posloužily jen plánům megalomanů. Nicméně i některé z nejznámějších vojenských osobností v historii nakonec dospěly k poznání, že \enquote{státy} se do války nepouštějí za nějakým ušlechtilým účelem, ale kvůli zisku a moci. Generálmajor Smedley Butler, který byl v době své smrti v roce 1940 nejvíce vyznamenaným příslušníkem americké námořní pěchoty v dějinách, napsal knihu s názvem \enquote{War Is a Racket} (Válka je kšeft), v níž kritizoval vojensko-průmyslový komplex a uvedl, že válka \enquote{\emph{je vedena ve prospěch několika málo lidí na úkor mnoha},} a dokonce zašel tak daleko, že svou \emph{vlastní} vojenskou \enquote{službu} označil za jednání \enquote{\emph{svalovce vysoké třídy},} \enquote{vyděrače} a \enquote{gangstera.} Podobně generál Douglas MacArthur vyslovil názor, že vojenská expanze je poháněna \enquote{\emph{uměle vyvolanou psychózou válečné hysterie}} a \enquote{\emph{neustálou propagandou strachu}.} Generál MacArthur také řekl následující: \enquote{\emph{Vládnoucí mocnosti nás udržují v neustálém stavu strachu -- udržují nás v neustálém tupém vlasteneckém zápalu s výkřiky o vážném národním ohrožení. Vždy se objevilo nějaké strašné zlo, které nás mělo pohltit, pokud bychom se za ním slepě nesjednotili poskytnutím požadovaných přemrštěných částek. Přesto se při zpětném pohledu zdá, že k těmto katastrofám nikdy nedošlo, že nikdy nebyly zcela reálné.}}

Samozřejmě, že kritika války jako kšeftu, z něhož má prospěch pouze vládnoucí třída, neznamená, že vládnoucí třída na druhé straně není také zlá nebo že by se jí nemělo odporovat. Zvěrstva páchaná vymahately režimů Stalina, Maa, Hitlera, Lenina, Pol Pota a mnoha dalších byla mimořádně závažná a použití \emph{obranného} násilí proti aktům agrese páchaným představiteli těchto režimů bylo jistě oprávněné. Ale autoritářská válka staví pěšáka proti pěšákovi v rozsáhlých krvavých bojích, které pokrývají obrovské geografické oblasti a vždy se při nich stává obětí civilní obyvatelstvo, zatímco vládnoucí třídy na obou stranách tomu z bezpečné vzdálenosti přihlížejí.

Dalším důkazem toho, že ve válce nikdy nejde o ideály nebo zásady, je skutečnost, že americká \enquote{vláda} často vedla válku proti tyranům, které dosadila, jako byl Manuel Noriega nebo Saddám Husajn. Ještě křiklavějším příkladem toho, že ve válce nejde o principy, je skutečnost, že na začátku druhé světové války byli Josef Stalin a jeho Sovětský svaz zapřisáhlými nepřáteli Spojených států. Na konci války byl tento psychotický masový vrah americkými \enquote{státními} propagandisty označován jako \enquote{strýček Joe} a bylo s ním zacházeno jako s ušlechtilým spojencem. Stalinovy zločiny proti lidskosti, které měly za následek desítky milionů mrtvých, zůstaly v USA v té době téměř nezmíněny. Ve světle této skutečnosti je absurdní tvrdit, že se americká \enquote{vláda} rozhodla vstoupit do druhé světové války na základě nějakého morálního principu nebo proto, aby porazila zlo.

Je důležité si uvědomit, co se děje a co se neděje v tradiční mezinárodní válce. Soupeřící vládnoucí třídy, včetně amerických vládců, spokojeně přihlížejí tomu, jak se jejich pěšáci navzájem po tisících vyvražďují, ale oficiální politikou mnoha \enquote{států,} včetně USA, již dlouho není pokoušet se zabíjet cizí \enquote{vládce} -- tj. ty, kteří jsou nejvíce zodpovědní za to, že k válce dochází. Ve skutečnosti je nejmorálnějším, nejracionálnějším a nejschůdnějším prostředkem obrany proti jakékoli invazní \enquote{autoritě} vražda těch, kteří jí velí. Zaměřit se na \enquote{vlády} namísto jejich loajálních vymahatelů by lidstvu skvěle posloužilo, nejenže by to mnohem rychleji ukončilo většinu násilných konfliktů, ale vytvořilo by to obrovský odstrašující prostředek pro všechny megalomany, kteří by byli v pokušení konflikty vůbec rozpoutat. Přesto mezi většinou tyranů na vysoké úrovni existuje otevřená, vzájemná a trvalá dohoda, že je sice v pořádku zahrávat si s životy svých poddaných, ale že se jen zřídkakdy zaměří na sebe navzájem.

A tak znovu a znovu nastupují na bojiště obrovské počty vojáků, aby se navzájem zabíjeli, zatímco skuteční nepřátelé lidstva -- vládci na \emph{obou} stranách -- zůstávají v bezpečí. Životy vojáků s dobrými úmysly, statečných \enquote{státních} vymahatelů, kteří loajálně plní rozkazy až do hořkého konce, jsou tak naprosto promarněny v úsilí, které v konečném důsledku nikomu nepřinese skutečnou svobodu a spravedlnost. A pokud se vojákovi podaří rozpoznat a zaměřit se na ty, kteří jsou za nespravedlnost a útlak nejvíce zodpovědní -- na ty, kteří na obou stranách každé války nosí nálepku \enquote{státu} -- je odsouzen jako zrádce a terorista.

\section{Hrdost páchat zlo}

Ať už jde o vojáka, nebo o úředníka nižšího stupně, úkolem všech \enquote{strážců zákona} je násilně vnutit veřejnosti vůli vládnoucí třídy. Přesto si většina z nich představuje, že tím \enquote{slouží lidu.} Představa, že někomu \enquote{sloužíme} tím, že proti němu zahájíme násilí, je samozřejmě směšná. (Vezměme si oxymóron absurdně pojmenovaného daňového úřadu \enquote{Internal Revenue Service,} který nedělá nic jiného, než že každoročně okrade stovky milionů lidí o biliony dolarů). Většina státních žoldnéřů, od úředníčků po nájemné vrahy, místo aby vůbec uvažovala o tom, že jejich pravidelná činnost -- účast na systému agrese a nátlaku -- je nemorální a necivilizovaná, prostě říká, že \enquote{jen dělají svou práci,} a představuje si, že je to zbavuje veškeré osobní odpovědnosti za jejich činy a jejich výsledky.

To je především příčinou úpadku lidské společnosti. Většina zla a nespravedlnosti, které lidé páchají, není důsledkem chamtivosti, zloby nebo nenávisti. Je to důsledek toho, že lidé dělají, co se jim řekne, že plní rozkazy, že \enquote{dělají svou práci.} Stručně řečeno, většina nelidskosti člověka vůči člověku je přímým důsledkem víry v autoritu. Škody způsobené pouhými poslušnými lidmi jsou stejně skutečné a stejně ničivé, jako kdyby je každý z nich páchal z osobní zloby. Zda starou paní okrade ozbrojený pouliční násilník, nebo dobře oblečený a vzdělaný \enquote{výběrčí daní,} v tom není z morálního ani praktického hlediska žádný rozdíl. Zda rodinu v Iráku zabijí vojáci Saddáma Husajna, nebo vojáci Spojených států, je morálně i prakticky jedno. Ať už něčí osobní volby násilně ovládá násilník ze sousedství, nebo \enquote{policie,} z morálního nebo praktického hlediska to nehraje žádnou roli.

Jediný rozdíl je v tom, že autoritářský násilík v důsledku své bludné víry v mýtickou entitu zvanou \enquote{stát} odmítá přijmout osobní odpovědnost za své činy. Jeho víra v nejnebezpečnější pověru mu znemožňuje rozpoznat zlo jako zlo. Ve skutečnosti se bude cítit \emph{hrdý} na svou loajální poslušnost vůči svým pánům, když bude den za dnem působit strádání a utrpení nevinným lidem, protože ho celý život učili, že když se zlo stane \enquote{zákonem,} přestane být zlem a stane se dobrem.

Ve skutečnosti, pokud je něco hříchem, pak je to slepá poslušnost \enquote{autoritě.} Působit jako vymahtel \enquote{státu} se rovná duchovní sebevraždě -- ve skutečnosti \emph{horší} než sebevražda fyzická, protože každý autoritářský \enquote{vymahatel} nejenže si vypíná svobodnou vůli a schopnost úsudku, které ho činí člověkem (čímž \enquote{zabíjí} svou vlastní lidskost), ale také ponechává své tělo nedotčené, aby ho tyrani mohli použít jako nástroj útlaku. Být \enquote{strážcem zákona} znamená dobrovolně se změnit z člověka v robota -- robota, který je pak předán některým z nejhorších lidí na světě, aby ho použili k ovládnutí a podmanění lidské rasy. Nošení uniformy vojáka nebo odznaku \enquote{strážce zákona} není důvodem k hrdosti; mělo by být důvodem k velkému studu za to, že člověk opustil vlastní lidskost ve prospěch toho, aby se stal loutkou utlačovatelů.

\chapter{Účinky na cíle}

\section{Hrdost být okraden}

Jedním z nejbizarnějších důsledků víry v autoritu je, že způsobuje, že se oběti \enquote{státní} agrese cítí \emph{povinny} být obětí a že se cítí špatně, pokud se \emph{vyhnou} tomu, aby se stali obětí. Ukázkovým příkladem je občan, který prohlašuje, že je hrdý na to, že platí \enquote{daně.} I když člověk věří, že část toho, čeho se vzdává, je použita na financování užitečných věcí (silnice, pomoc chudým atd.), být hrdý na to, že mu bylo vyhrožováno a že byl k financování takových věcí donucen, je stejně divné. Hrdost na to, že je člověk \enquote{zákona dbalým daňovým poplatníkem,} není důsledkem toho, že pomohl lidem, což mohl udělat mnohem účinněji na dobrovolné bázi; hrdost pramení z toho, že věrně poslouchal příkazy domnělé \enquote{autority.} Analogicky může mít člověk dobrý pocit z toho, že dobrovolně obdaroval někoho v nouzi, ale nebyl by hrdý na to, že se nechal okrást chudákem. Pravděpodobně jediná situace, kdy se někdo chlubí tím, že byl k něčemu donucen, nastává v souvislosti s člověkem, který se domnívá, že je povinen poslouchat domnělou \enquote{autoritu.}

Lidé, kteří byli vycvičeni k tomu, aby považovali poslušnost za ctnost, se chtějí cítit dobře, když odevzdávají \enquote{státu} to, co si vydělali. A tak s pomocí politické propagandy halucinují, že jejich \enquote{příspěvky} vlastně pomáhají společnosti jako celku. Mluví, jako by placení \enquote{daní} znamenalo \enquote{vracení společnosti} nebo \enquote{investování do země.} Taková rétorika, jakkoli je běžná, je logicky nesmyslná, protože předpokládá, že každý z jednotlivců, kteří tvoří \enquote{společnost} a \enquote{zemi,} nějak dluží skupině jako celku, ale není jí nic dlužen. To, co lidé ve skutečnosti dělají, když platí \enquote{daně,} je, že dávají peníze nikoli \enquote{společnosti} nebo \enquote{zemi,} ale politikům, kteří tvoří vládnoucí třídu, aby je utratili, jak se jim zlíbí. Z toho plyne, jakkoli je to zvláštní, že \enquote{lid} může mít prospěch jako celek, když je každý z \enquote{lidu} okrádán jednotlivě. Představa, že \enquote{společnému dobru} lépe poslouží, když politici utratí peníze všech, než kdyby každý člověk utratil své vlastní peníze, je přinejmenším podivná. V poslední době se lež o tom, že \enquote{daně} slouží společnému dobru, stává průhlednější, protože \enquote{státy} utrácejí astronomické částky za věci, které zjevně slouží elitě na úkor společnosti a lidstva. Patří sem mimo jiné válečné štvaní, přímé mnohamiliardové přerozdělovací programy ve prospěch nejbohatších lidí na světě (\enquote{bailouts}) a \enquote{státní} převzetí různých segmentů ekonomiky (např. zdravotnictví).

Ve skutečnosti neexistuje téměř nic, co by průměrný člověk mohl finančně podpořit a co by bylo pro společnost a lidstvo obecně méně užitečné než placení \enquote{daní.} Ať už člověk považuje za užitečné cokoli -- školy, silnice, obranu, pomoc chudým atd. -- by mohl stejně snadno podpořit i bez politiků a \enquote{státu.} Přesto mnoho lidí výslovně vyjadřuje hrdost na to, že odevzdali plody své práce svým pánům, že \enquote{zaplatili daně.} Uvažte, jak by se pohlíželo na někoho, kdo by hrdě prohlásil: \enquote{Lhal jsem ve svém daňovém přiznání, vyhnul jsem se odevzdání 3 000 dolarů státu a místo toho jsem ty tři tisíce dolarů věnoval opravdu dobré charitě.} Mnozí lidé by takového člověka stále odsuzovali za jeho \enquote{zločinnou} neloajalitu vůči pánům, i když by jeho činy posloužily lidstvu lépe, než kdyby \enquote{zaplatil daně.} Je to proto, že hrdost, kterou mnozí lidé vyjadřují, nepramení z pomoci lidstvu, ale z poslušnosti \enquote{vrchnosti.}

Je jen malá nebo žádná šance, že by někdo dobrovolně přispěl svým vlastním majetkem na každý z programů a schémat, které jsou nyní financovány prostřednictvím \enquote{státu.} A pokud odevzdává peníze jen proto, že ho k tomu donutil nějaký \enquote{zákon} nebo jiná \enquote{autorita,} a pak vyjadřuje hrdost na to, že tak učinil, v podstatě se chlubí tím, že byl násilně ovládán, přesně tak, jak může být důkladně indoktrinovaný otrok hrdý na to, že dobře slouží svému pánovi. Je velký rozdíl mezi dobrým pocitem z toho, že dobrovolně podpořil nějakou záslužnou věc, a hrdostí na to, že se podřídil. Namísto toho, aby byly pohoršeny urážkou a nespravedlností, že jsou násilně ovládány a vykořisťovány -- vlastně namísto toho, aby to vůbec uznaly za nespravedlnost -- cítí mnohé oběti \enquote{státního} útlaku hlubokou loajalitu ke svým vládcům.

\section{Hrdost být ovládán}

Pokud lze otroka přesvědčit, že by měl být otrokem, že jeho zotročení je správné a legitimní, že je právoplatným majetkem svého pána a že má povinnost produkovat pro svého pána co nejvíce, pak není třeba ho fyzicky utlačovat. Jinými slovy, zotročení mysli činí zotročení těla zbytečným. A to je přesně to, co víra v autoritu dělá: učí lidi, že je morálně ctnostné, aby svůj čas, úsilí a majetek, stejně jako svobodu a kontrolu nad vlastním životem odevzdali vládnoucí třídě.

Mnoho lidí vyjadřuje hrdost na to, že jsou \enquote{zákona dbalými daňovými poplatníky,} což znamená pouze to, že dělají to, co jim politici nařídí, a dávají politikům peníze. Když jsou konfrontováni s myšlenkou, že není správné, aby byli násilně zbavováni plodů své práce, i když se tak děje \enquote{legálně,} tito lidé často vehementně hájí ty, kteří je nadále okrádají, a trvají na tom, že takové okrádání je pro lidskou civilizaci nezbytné. (Samozřejmě nepoužívají k popisu situace výraz \enquote{okrádání,} ačkoli jsou si dobře vědomi toho, co by se jim stalo, kdyby odmítli platit.) Stejně tak, když někdo vznese námitky proti výši daní nebo jinému násilnému ovládání, které nad ním \enquote{stát} vykonává, ostatní, kteří jsou také utlačováni, často odsoudí toho, kdo vznáší námitky, a řeknou mu, že pokud se mu nelíbí, jak se s ním zachází, měl by opustit zemi. Znevažování kolegy, který je obětí nátlaku, za to, že si na něj stěžuje, je neklamnou známkou toho, že se člověk ve skutečnosti pyšní svým vlastním zotročením.

Frederick Douglass, bývalý otrok, byl svědkem a popsal přesně tento jev mezi svými spoluotroky, z nichž mnozí byli hrdí na to, jak tvrdě pracovali pro své pány a jak věrně dělali, co se jim řeklo. Z jejich pohledu byl otrok na útěku hanebným zlodějem, který se \enquote{ukradl} pánovi.

Douglass popsal, jak důkladně byli mnozí otroci indoktrinováni, a to do té míry, že skutečně věřili, že jejich vlastní zotročení je spravedlivé a oprávněné:

\enquote{\emph{Zjistil jsem, že k tomu, aby byl otrok spokojený, je třeba, aby byl nemyslící. Je třeba zatemnit jeho morální a duševní zrak a pokud možno zničit sílu rozumu. Musí být schopen nepostřehnout v otroctví žádné nesrovnalosti; musí být donucen cítit, že otroctví je správné; a k tomu ho lze přivést jen tehdy, když přestane být člověkem.}}

Ačkoli se otroctví již nepraktikuje otevřeně, mentalita loajální podřízenosti zůstává. Většina lidí dnes nevidí žádné nesrovnalosti v tom, že vládnoucí třída může násilím vydírat a ovládat všechny ostatní, a ve skutečnosti má pocit, že takové vydírání a útlak jsou správné, a to do té míry, že mnozí cítí skutečný stud, jsou-li přistiženi, že si nechávají to, co vydělají, a řídí svůj vlastní život. Jedna věc je cítit stud, když vás někdo přistihne při krádeži, podvodu nebo agresi. Ale něco jiného je, když někdo cítí stud za to, že udělal něco, co by nebýt nařízení (\enquote{zákonů}) politiků považoval za naprosto přípustné. Takový stud nepramení z nemorálnosti samotného činu; pramení pouze z pomyslné nemorálnosti neuposlechnutí \enquote{autority,} tj. z \enquote{porušení zákona.}

Když je například průměrný občan přistižen, jak \enquote{podvádí} na daních, nemá na autě registrační nálepku, kouří marihuanu nebo dělá některou z tisíců dalších věcí, které nepředstavují agresi vůči někomu jinému, ale které přesto vládnoucí třída prohlásila za \enquote{nezákonné,} obvykle se v něm objeví pocit viny. Bez pocitu povinnosti uposlechnout by se na přistižení a potrestání agenty \enquote{státu} pohlíželo stejně jako na pokousání psem: jako na nepříjemný následek, kterému je třeba se vyhnout, ale který v sobě nemá žádný morální prvek. Místo toho má většina lidí alespoň do určité míry pocit, že přistižení při páchání \enquote{zločinu} bez oběti svědčí o nějakém jejich morálním selhání, protože neudělali, co se jim řeklo. Touha mít souhlas \enquote{autority} je téměř u každého nesmírně silná, a to v míře, kterou si ani sami neuvědomují. Všudypřítomné poselství autoritářství má mnohem hlubší psychologický dopad, než si většina lidí dokáže představit, jak ukázaly Milgramovy experimenty. Téměř každý člověk prožívá dramatický emocionální stres a nepohodlí, kdykoli se dostane do konfliktu s \enquote{autoritou,} a vynaloží velké úsilí, bez ohledu na to, jakých zlých činů se musí dopustit, aby si získal souhlas svých pánů.

Dokonce i terminologie, kterou lidé používají, ukazuje, jak účinně byli vycvičeni, aby se cítili morálně povinni poslouchat \enquote{autoritu.} To se projevuje v takových jednoduchých frázích, jako je \enquote{To nesmíš} nebo dokonce \enquote{To nemůžeš,} když se mluví o nějakém chování, které bylo vládnoucí třídou prohlášeno za \enquote{nezákonné.} Takové fráze nevyjadřují pouze potenciální nepříznivý důsledek, ale také naznačují, že vzhledem k tomu, že nějaký čin byl pány zakázán, je jeho spáchání špatné, nedovolené, nebo dokonce nemožné (tj. \enquote{To nemůžeš!}).

Pohled na statistická fakta ukazuje sílu víry v autoritu. Ve Spojených státech vydírá asi 100 000 zaměstnanců IRS asi 200 000 000 obětí. Na každého zloděje tedy vychází asi dva tisíce okradených. Toho by nikdy nebylo možné dosáhnout pouhou hrubou silou; pokračuje to jen proto, že většina okradených cítí \emph{povinnost} být okradena a představuje si, že takové okrádání je legitimní a platné. Totéž platí o mnoha jiných \enquote{zákonech,} které jsou obecně dodržovány, přestože jejich vymahatelé čelí obrovské přesile těch, které se snaží ovládat. Vysoká míra dodržování zákonů nepramení ani tak ze strachu z trestu, jako spíše z pocitu ovládaných, že mají morální povinnost spolupracovat na svém podrobení.

\section{Dobro platí zlo}

I když se jedinec nikdy nestane osobní obětí \enquote{orgánů činných v trestním řízení,} nikdy se nedostane do střetu s policií a \enquote{stát} má na jeho každodenní život jen malý nebo vůbec žádný přímý vliv, mýtus autority má stále dramatický dopad nejen na jeho vlastní život, ale také na to, jak jeho existence ovlivňuje svět kolem něj. Například miliony poddajných poddaných, kteří cítí povinnost odevzdávat státu část toho, co vydělají, platit svůj \enquote{spravedlivý podíl} na \enquote{daních,} neustále financují nejrůznější snahy a aktivity, které by tito lidé jinak nefinancovali -- které by jinak nefinancoval téměř nikdo, a které by tedy jinak neexistovaly. Prostřednictvím \enquote{daní} ti, kdo se prohlašují za \enquote{stát,} zabavují milionům obětí téměř nepochopitelné množství času a úsilí a přeměňují je na palivo pro agendu vládnoucí třídy. Například miliony lidí, kteří jsou proti válce, jsou nuceni ji financovat prostřednictvím \enquote{daní.} Produkt jejich času a úsilí je použit k umožnění něčeho, proti čemu morálně vystupují.

Totéž platí pro státem řízené programy přerozdělování bohatství (např. \enquote{sociální dávky}), Ponziho schémata (např. \enquote{sociální zabezpečení}), takzvanou \enquote{válku proti drogám} atd. Většina těchto \enquote{státních} programů by neexistovala, kdyby mezi obyvatelstvem nepanovala víra v morální povinnost platit \enquote{daně.} Dokonce i \enquote{státní} programy, které údajně mají ušlechtilé cíle -- jako je ochrana veřejnosti a pomoc chudým -- se stávají nafouklými, neefektivními a zkorumpovanými monstry, která by téměř nikdo dobrovolně nepodporoval, kdyby neexistoval \enquote{zákon,} který by to vyžadoval.

Kromě plýtvání, korupce a destruktivních věcí, které \enquote{stát} dělá se zabaveným bohatstvím, je tu také méně zřejmá otázka, co by lidé se svými penězi dělali jinak. Tím, že \enquote{stát} bere bohatství producentům, aby sloužilo jejím vlastním cílům, zbavuje zároveň producenty možnosti prosazovat své vlastní cíle. Někdo, kdo odevzdá 1 000 dolarů na \enquote{daních} vládnoucí třídě, může nejen financovat válku, proti níž morálně vystupuje, ale je také zbaven možnosti dát 1 000 dolarů do úspor nebo věnovat 1 000 dolarů nějaké charitě, kterou považuje za užitečnou, nebo zaplatit někomu 1 000 dolarů za nějaké terénní úpravy. Škoda způsobená mýtem autority je tedy dvojí: nutí lidi financovat věci, o nichž nejsou přesvědčeni, že jsou dobré pro ně samotné nebo pro společnost, a zároveň jim brání financovat věci, které považují za hodnotné. Jinými slovy, podřízenost \enquote{autoritě} nutí lidi jednat způsobem, který je v té či oné míře v přímém rozporu s jejich vlastními prioritami a hodnotami.

Dokonce i lidé, kteří si představují, že jejich \enquote{daně} konají dobro tím, že staví silnice, pomáhají chudým, platí policii a podobně, by téměř jistě nefinancovali \enquote{státní} verzi těchto služeb, přinejmenším ne ve stejné míře, kdyby se k tomu necítili nuceni. Jakákoli soukromá charitativní organizace, která by měla takovou neefektivitu, korupci a záznamy o zneužívání jako AFDC, HUD, Medicare a další \enquote{státní} programy, by rychle ztratila všechny své dárce. Jakákoli soukromá společnost, která by byla stejně drahá, zkorumpovaná a neefektivní jako \enquote{státní} programy v oblasti infrastruktury, by ztratila všechny své zákazníky. Jakákoli soukromá ochranná služba, která by byla tak často přistižena při zneužívání, napadání a dokonce zabíjení neozbrojených, nevinných lidí, by neměla žádné zákazníky. Jakákoli soukromá společnost, která by tvrdila, že poskytuje obranu, ale svým zákazníkům by tvrdila, že potřebuje každý týden miliardu dolarů na vedení dlouhodobé války na druhém konci světa, by měla jen málo přispěvatelů, pokud vůbec nějaké, a to i mezi těmi, kteří nyní takové vojenské operace verbálně podporují.

Pocitu povinnosti platit \enquote{daně} zřejmě jen málo brání skutečnost, že \enquote{stát} je notoricky známý svou nehospodárností a neefektivitou. Zatímco miliony \enquote{daňových poplatníků} se snaží vyjít s penězi a přitom platí svůj \enquote{spravedlivý podíl} na \enquote{daních,} politici plýtvají miliony na směšně hloupé projekty -- od zkoumání kravských prdů přes stavbu mostů do nikam až po placení zemědělcům, aby \emph{nepěstovali} určité plodiny, a tak dále do nekonečna -- a další miliardy se prostě \enquote{ztratí,} aniž by se zjistilo, kam se poděly. Mnoho z toho, co lidé umožňují platbou \enquote{daní,} je však nejen promrháno, ale pro společnost zcela destruktivní. Zřejmým příkladem je \enquote{válka proti drogám.} Kolik lidí by dobrovolně přispělo soukromé organizaci, jejímž deklarovaným cílem je odvléct miliony nenásilných jedinců od jejich přátel a rodin a zavřít je do klecí? Dokonce i mnozí Američané, kteří dnes uznávají, že \enquote{válka proti drogám} je naprostým selháním, nadále prostřednictvím \enquote{daní} poskytují finanční prostředky, které umožňují pokračovat v ničení doslova milionů životů.

Dokonce i ti nejhlasitější kritici nejrůznějšího zneužívání ze strany stále rostoucího policejního státu jsou často mezi těmi, kteří toto zneužívání umožňují tím, že na něj poskytují finanční prostředky. Ať už se jedná o zjevný útlak, korupci nebo pouhou byrokratickou neefektivitu, každý může poukázat alespoň na několik věcí, ve kterých se \enquote{státem} nesouhlasí. A přesto, protože byl vycvičen k poslušnosti vůči \enquote{autoritě,} bude se i nadále cítit povinen poskytovat finanční prostředky, které umožňují tytéž neúspěšné, zkorumpované a utlačovatelské \enquote{státní} aktivity, které kritizuje a proti kterým vystupuje. Málokdy si někdo všimne zřejmého rozporu, který je vlastní někomu, kdo se cítí \emph{povinen} financovat věci, které považuje za špatné.

Lidé pracující pro neautoritářské organizace mohou být samozřejmě také neefektivní nebo zkorumpovaní, ale když vyjde najevo, co dělají, jejich zákazníci je mohou jednoduše přestat financovat. To je přirozený korekční mechanismus v lidské interakci, který je však vírou v autoritu zcela poražen. Kolik je lidí, kteří nejsou v současné době nuceni financovat nějaký \enquote{stántí} program nebo činnost, s nimiž morálně nesouhlasí? Velmi málo, pokud vůbec někdo. Proč tedy tito lidé stále financují věci, které jsou podle nich pro společnost destruktivní? Protože jim to nařizuje \enquote{autorita} a protože věří, že je dobré \enquote{autoritu} poslouchat. V důsledku toho se nadále vzdávají plodů své práce, aby poháněli stroj na utlačování -- stroj, který by jinak neexistoval a nemohl existovat. \enquote{Státy} žádné bohatství nevytvářejí; to, co utratí, musí nejprve někomu jinému vzít. Každý \enquote{stát,} včetně těch nejutlačovatelštějších režimů v dějinách, byl financován placením \enquote{daní} loajálními, produktivními poddanými. Díky víře v autoritu bude bohatství vytvořené miliardami lidí i nadále využíváno nikoli k tomu, aby sloužilo hodnotám a prioritám lidí, kteří na jeho vytvoření pracovali, ale aby sloužilo plánům těch, kteří především touží po nadvládě nad svými bližními. Třetí říši umožnily miliony německých \enquote{daňových poplatníků,} kteří cítili povinnost platit. Sovětské impérium umožnily miliony lidí, kteří cítili povinnost dát státu vše, co požadoval. Každá invazní armáda, každé dobyvačné impérium bylo vybudováno z bohatství, které bylo odebráno produktivním lidem. Ničitelé byli vždy financováni tvůrci; zloději byli vždy financováni výrobci; díky víře v autoritu byly programy zla vždy financovány úsilím dobra. A tak tomu bude i nadále, dokud nebude tato nejnebezpečnější pověra odstraněna. Až výrobci přestanou cítit morální povinnost financovat parazity a uzurpátory, ničitele a vládce, tyranie uvadne, protože bude vyhladovělá. Do té doby budou dobří lidé nadále dodávat zdroje, které zlí lidé potřebují k uskutečňování svých destruktivních plánů.

\section{Kopou si vlastní hrob}

Je smutné, že víra v autoritu dokonce vede k tomu, že se lidé cítí povinni napomáhat svému zotročení, útlaku a někdy i smrti. Ve skutečnosti je jen malé procento \enquote{státního} donucení prováděno vymahateli \enquote{autority;} většinu z něj provádějí její \emph{oběti}. Vládnoucí třída lidem pouze říká, že jsou povinni dělat určité věci, a většina lidí se podřizuje, aniž by docházelo ke skutečnému vynucování. Jako jeden z působivých příkladů lze uvést, že desítky milionů Američanů každoročně vyplňují zdlouhavé a nepřehledné formuláře známé jako \enquote{daňové přiznání,} čímž se v podstatě vydírají. Pokud by oběti IRS souhlasily s tím, že zaplatí, pouze v případě, že \enquote{stát} zjistí jejich údajné daňové povinnosti, systém by se zhroutil. Každé daňové přiznání je v podstatě podepsaným udáním, přičemž oběť vyděračského systému nejenže prozradí vše o svých financích -- v podstatě vyslýchá sama sebe -- ale dokonce si spočítá i částku, která bude ukradena, aby zloději nemuseli.

Všechny neproduktivní a nemilé nepříjemnosti a byrokratické obtíže, kterým se lidé vystavují jen proto, že jim bylo řečeno, že to \enquote{zákon} vyžaduje, však nejsou ničím ve srovnání s vážnějšími příznaky víry v autoritu. Na základě mytologie o \enquote{službě vlasti} a \enquote{zákonů} ukládajících vojenskou brannou povinnost (\enquote{mobilizaci}) se miliony lidí v průběhu dějin staly \emph{vrahy} pro stát. Jen malá část z nich (tzv. \enquote{zbězi}) se kdy postavila na odpor a obvykle se jim dostalo opovržení ze strany jejich krajanů, protože byli zbabělci nebo jim chyběl \enquote{patriotismus.}

V případě mnoha \enquote{zákonů} může být obtížné rozlišit mezi lidmi, kteří poslouchají z prostého strachu z trestu, a těmi, kteří poslouchají z pocitu morální povinnosti podřídit se příkazům politiků (\enquote{zákonu}). U vojenské branné povinnosti je však snadné rozlišit, protože \enquote{dodržování} je obvykle mnohem \emph{nebezpečnější} než jakýkoli trest, kterým \enquote{stát} hrozí těm, kdo se odmítají podřídit. Pokud je na výběr \enquote{vyhovět} a případně zemřít strašlivou smrtí na nějakém bojišti na druhém konci světa, nebo neuposlechnout a případně jít do vězení, je nepravděpodobné, že by samotná hrozba byla důvodem, proč se tolik lidí \enquote{registruje} a dostaví se do \enquote{služby,} když jsou povoláni. Stručně řečeno, míra dodržování \enquote{branné povinnosti,} přinejmenším v minulosti, zcela jasně ukazuje, že většina lidí raději spáchá vraždu nebo zemře, než aby se nepodřídila \enquote{autoritě.} Těžko by se našel lepší důkaz toho, jak mocná je pověra autority: že tisíce a tisíce jinak civilizovaných, mírumilovných lidských bytostí opustí domov, někdy cestují přes půl světa, aby zabíjely nebo umíraly jen proto, že jim to nařídila příslušná vládnoucí třída.

Každý voják je vymahatelem i obětí pověry autority, ať už se přihlásil dobrovolně, nebo byl odveden. Bojovat na obranu nevinných proti agresorům je ušlechtilý cíl a často je záměrem těch, kteří vstoupí do armády. V hierarchickém vojenském režimu se však voják stává spíše nástrojem mašinérie než odpovědným jedincem. Místo aby se řídil vlastním svědomím, je zcela ovládán rozkazy, které dostává prostřednictvím velitelského řetězce. A pokaždé, když ho jeho poslušnost vede k něčemu nemorálnímu (což se stává poměrně často), ubližuje nejen svým obětem, ale i sám sobě. Jako jeden z příkladů lze uvést, že po válce ve Vietnamu se mnoho amerických vojáků vrátilo domů s nepoškozeným tělem, ale s hlubokými psychickými problémy. Těžko říci, nakolik bylo jejich psychické poškození důsledkem toho, že byli svědky krveprolití, a nakolik důsledkem toho, že krveprolití osobně \emph{vytvářeli}. Dlouhodobý strach z bezprostřední smrti může samozřejmě způsobit vážné psychické problémy, stejně jako způsobení smrti druhým.

Násilné konfrontace mohou být značně stresující, a to i tehdy, když se jedinec cítí zcela oprávněně, například když brání svou rodinu před útočníkem. Ale zapojit se do smrtelného boje, kdy nikdo, včetně bojujících, zřejmě nemá jasnou představu o účelu nebo ospravedlnění konfliktu, jako tomu bylo ve Vietnamu, zřejmě přidává další stupeň psychického traumatu. Jak potvrdili mnozí vojáci v bojích, jakmile se ocitnou ve válečném pekle, jakýkoli nejasný, ale ušlechtilý důvod nebo ospravedlnění boje je obvykle zapomenut a jediné, co zůstane, je touha zůstat naživu a pomoci svým přátelům zůstat naživu -- oběma těmto cílům mnohem lépe poslouží, když se vrátí domů nebo když do armády vůbec nevstoupí. A přesto je počet lidí, kteří jednoduše odejdou, poměrně malý, a to z jednoho prostého důvodu: protože by to představovalo akt neposlušnosti vůči domnělé \enquote{autoritě.} A průměrný voják, i když má odvahu a sílu vrhnout se do smrtelného boje, nemá odvahu a sílu neposlechnout domnělou \enquote{autoritu.} Stejně jako v mnoha jiných případech autoritářského nátlaku oběti vojenské branné povinnosti téměř vždy výrazně převyšují počet těch, kteří se ji snaží zavést. I když je lidem \enquote{legálně} přikázáno obětovat svůj rozum a tělo ve prospěch tyranských válek o území, prostá pasivní neposlušnost jakékoli významné části \enquote{branců} by válečnou mašinérii zastavila. Jaký trest by byl horší než výsledek podřízení se? Obvyklým výsledkem boje ve válce je dlouhodobý teror, fyzická a duševní bolest a utrpení, zmrzačení nebo smrt. Přesto se i poté, co se na vlastní oči přesvědčí o hrůzách války, jen málokdo dokáže přimět k tomu, aby neuposlechl \enquote{autoritu,} svlékl uniformu a odešel.

O síle víry v autoritu svědčí dobře zdokumentovaná (i když zřídka diskutovaná) skutečnost, že zvěrstva páchaná na německých Židech nacisty byla často prováděna ve spolupráci a za pomoci \emph{židovské} policie, jako například ve varšavském ghettu. V jejich kultuře, stejně jako v téměř každé jiné kultuře, byli lidé tak důkladně přesvědčeni, že poslušnost je ctnost, že i když \enquote{vládl} někdo nový, stále se cítili povinni dělat, co se jim řekne, i kdyby to znamenalo násilné utlačování vlastních příbuzných. Co je však možná ještě více znepokojující (ale nesporné), je skutečnost, že mnoho milionů lidí v historii napomáhalo svému \emph{vlastnímu} vyhlazování, protože jim to \enquote{autorita} nařídila. Například během holocaustu mnoho set tisíc Židů z vlastní moci nastoupilo do dobytčích vagónů právě těch vlaků, které je měly odvézt na smrt, aniž by se pokusili skrýt, utéct nebo klást odpor. Proč? Protože jim to nařídili ti, kteří se vydávali za \enquote{autoritu.} I když je nepochybné, že ne všichni věděli, co přesně je na druhém konci čeká, přesto se vydali do péče mašinérie, která jim zjevně chtěla ublížit.

Člověk má určitý pocit pohodlí a bezpečí, když se podřizuje a poslouchá. Víra, že věci jsou v rukou někoho jiného, a důvěra, že někdo jiný dá věci do pořádku, je způsob, jak se vyhnout odpovědnosti. Autoritářská indoktrinace zdůrazňuje myšlenku, že ať se děje cokoli, pokud prostě budete dělat, co se vám řekne, a dělat to, co dělají všichni ostatní, bude všechno v pořádku a ti, kdo jsou za to zodpovědní, vás odmění a ochrání. Počty mrtvých při jednom \enquote{státním} zvěrstvu za druhým ukazují, jak scestné takové přesvědčení skutečně je. Kdyby oběti \enquote{legálního} útlaku a vraždění prostě \emph{nepomáhaly}, i kdyby neprovedly ani špetku násilného odporu, svět by dnes vypadal úplně jinak. Kdyby nacisté museli každého Žida, živého či mrtvého, fyzicky odnést do plynových komor či krematorií, míra vraždění by byla dramaticky nižší. Kdyby každý otrok prodaný do otroctví odmítl pracovat, brzy by neexistoval obchod s otroky. Kdyby musel berňák vypočítat dlužnou daň a pak ji přímo vybrat od každého \enquote{daňového poplatníka,} neexistovalo by už žádné federální \enquote{zdanění.} Zkrátka, kdyby \emph{oběti} autoritářského vydírání, obtěžování, sledování, přepadávání, únosů a vražd jednoduše přestaly \emph{napomáhat} svému vlastnímu útlaku, tyranie by se zhroutila. A kdyby lidé šli ještě o krok dál a násilím se postavili na odpor, tyranie by se zhroutila ještě rychleji. Ale odpor, ať už pasivní nebo násilný, vyžaduje, aby se lidé nepodřídili vnímané \enquote{autoritě,} a toho většina lidí není psychologicky schopna. Nakonec je to víra v autoritu mezi \emph{obětmi} útlaku, dokonce více než víra vládnoucí třídy a jejich vymahatelů, která umožňuje, aby tyranie a nelidskost člověka vůči člověku pokračovala v tak velkém měřítku.

\section{Účinky na skutečné zločince}

Paradoxně v situacích, kdy by poslušnost skutečně zlepšila lidské chování, nemá \enquote{autorita} žádný účinek. Například ti jedinci, kterým jejich vlastní svědomí nebrání v okrádání nebo napadání bližních, protože se nestarají o obvyklé normy dobra a zla, se také nestarají o to, co jim \enquote{autorita} nařizuje. Pouze ti, kteří se snaží být dobří, se někdy cítí nuceni poslouchat \enquote{autoritu.} Víra v autoritu je vírou v morálku -- je to představa, že poslušnost je morálně dobrá. Na ty, kteří se nestarají o to, co je považováno za \enquote{dobré} -- právě na lidi, jejichž svědomí nestačí k tomu, aby se chovali civilizovaně -- nemá mýtus autority žádný vliv. Jinak řečeno, pouze ti, kteří nepotřebují být ovládáni -- tj. ti, kteří se již snaží žít morálně -- cítí nějakou povinnost poslouchat ovládající. Mezitím ti, kteří představují skutečnou hrozbu pro mírumilovnou společnost, stejně necítí žádnou morální povinnost poslouchat jakoukoli \enquote{autoritu.} Obecně řečeno, všechny příkazy \enquote{autority,} včetně ve své podstatě ospravedlnitelných příkazů, jako je \enquote{nekrást} a \enquote{nevraždit,} jsou vždy buď zbytečné (pokud jsou zaměřeny na dobré lidi), nebo neúčinné (pokud jsou zaměřeny na špatné lidi). Těžko si lze představit situaci, kdy by se jedinec jinak bez výčitek svědomí dopustil krádeže, přepadení nebo vraždy, ale cítil by se provinile, kdyby porušil \enquote{zákony,} které takové jednání zakazují.

Zde je třeba rozlišovat mezi morální povinností a strachem z odplaty. Zloděj, který necítí morální povinnost zdržet se krádeže, nebude cítit ani morální povinnost dodržovat \enquote{zákony} proti krádeži. Pokud však bude vnímat ohrožení vlastní bezpečnosti, ať už ze strany \enquote{policie} nebo kohokoli jiného, může ho to od okradení odradit. Tento odstrašující účinek však vychází výhradně z hrozby násilí, nikoli z tvrzené \enquote{autority,} která je v pozadí této hrozby. To znamená, že domnělá \enquote{autorita} nikdy nezabrání skutečným zločinům a že účinný odstrašující systém vůbec nevyžaduje \enquote{autoritu.} O tom se podrobněji hovoří níže.

\chapter{Účinky na pozorovatele}

\section{Hřích nevzdoru}

Je zřejmé, že víra v autoritu ovlivňuje vnímání a jednání \enquote{strážců zákona} a ovlivňuje také vnímání a jednání těch, proti kterým jsou \enquote{zákony} uplatňovány. Ale i vnímání a jednání \emph{přihlížejících}, tedy těch, kterých se to přímo netýká, hraje při určování stavu lidské společnosti obrovskou roli. Přesněji řečeno, obrovský vliv má \emph{nečinnost} pozorovatelů, kteří tiše dovolují, aby byl na druhých vykonáván \enquote{legální} nátlak. Historie je plná příkladů, které dokazují, že Edmund Burke měl pravdu, když říkal, že k vítězství zla stačí, aby dobří lidé nedělali nic.

Masové vraždění páchané režimy Stalina, Maa, Hitlera a mnoha dalšími bylo umožněno nejen ochotou \enquote{vymahatelů} plnit jejich příkazy, ale také pomyslnou povinností jejich obětí poslouchat \enquote{autoritu} a přesvědčením téměř všech přihlížejících, že by neměli zasahovat do výkonu \enquote{práva.} Pachatelé masového bezpráví, včetně masových vražd, mají vždy obrovskou početní převahu nad svými oběťmi, a když k tomu připočteme počet pozorovatelů -- všech těch lidí, kteří mohli zasáhnout -- je zřejmé, jak významné může být jednání (nebo nečinnost) pouhých \enquote{pozorovatelů.}

Samozřejmě, že někteří lidé nezasáhnou v dané situaci prostě ze strachu. Svědek přepadení, který se neodváží zasáhnout, svou nečinností přepadení neschvaluje. Jednoduše si cení přínosu pro svou vlastní bezpečnost, který plyne z nečinnosti, více než jakéhokoli přínosu, který by podle něj mohl být pro oběť, kdyby zasáhl. Existuje však mnoho případů, kdy víra v autoritu nutí lidi váhat se zapojit do konfliktu, a to nejen ze strachu, ale i z hluboké psychologické nechuti jít proti \enquote{autoritě.} To může způsobit, že pozorovatelé nečinně přihlížejí \enquote{zákonnému} bezpráví páchanému na někom jiném, a to dvěma způsoby: 1) pozorovatel může věřit, že bezpráví je vlastně dobré, protože je to \enquote{zákon,} nebo 2) pozorovatel může nesouhlasit, ale jeho ochota skutečně vystoupit proti \enquote{strážcům zákona,} nebo dokonce vystoupit proti \enquote{autoritě,} je potlačena jeho vycvičenou podřízeností. Ať tak či onak, výsledek je stejný: pozorovatel neudělá nic, aby nespravedlnost zastavil. Oběma jevům se však budeme věnovat odděleně.

\section{\enquote{Legální} zlo jako dobro}

Existují doslova miliony příkladů, na nichž lze demonstrovat, jak dramaticky ovlivňuje vnímání široké veřejnosti víra v autoritu. Stačí se zamyslet nad tím, jak průměrný člověk vnímá a posuzuje nějaký čin, když ho spáchá ten, kdo se prohlašuje za \enquote{autoritu,} na rozdíl od toho, jak vnímá a posuzuje naprosto stejný čin, když ho spáchá kdokoli jiný. Zde je několik příkladů:

1) Situace A: Americký voják v cizí zemi chodí dům od domu, vykopává dveře, nosí samopal a míří jím na zcela cizí lidi, rozkazuje jim a vyslýchá je, přičemž pátrá po \enquote{povstalcích.} Situace B: Obyčejný občan ve své vlastní zemi chodí dům od domu, vykopává dveře, nosí samopal a míří jím na úplně cizí lidi, rozkazuje jim a vyslýchá je, přičemž hledá lidi, které nemá rád. Toho prvního většina lidí považuje za statečného a ušlechtilého vojáka \enquote{sloužícího své zemi,} zatímco toho druhého za strašně nebezpečného, pravděpodobně psychicky narušeného jedince, kterého je třeba za každou cenu odzbrojit a zkrotit.

2) Situace A: \enquote{Strážce zákona} obsluhuje \enquote{kontrolní stanoviště} nebo hraniční přechod a zastavuje každého, aby se ho zeptal, zda je v zemi \enquote{legálně} nebo zda nepil alkohol, případně aby jinak zjistil, zda nenajde nějaký náznak nebo důkaz \enquote{trestné} činnosti. Situace B: Muž bez odznaku zastavuje každé auto, které projíždí jeho ulicí, každého řidiče se ptá, zda je Američan, zda nepil alkohol, a prohlíží si jeho auto, zda v něm není něco podezřelého. Policista, který se dopouští takového dotěrného, nepříjemného obtěžování, zadržování, výslechů a prohlídek, je mnohými považován za statečného \enquote{strážce zákona,} který dělá svou práci, zatímco kdokoli jiný, kdo by se takto choval, by byl považován jako psychicky nemocný a nebezpečný.

3) Situace A: Pracovník \enquote{ochrany dětí} obdrží spis a na základě anonymního udání se dostaví do domu, aby vyslechl majitele domu s cílem rozhodnout, zda jsou vhodnými rodiči, nebo zda by jim stát měl násilím odebrat děti. Situace B: Obyčejný člověk se na základě pověsti, kterou slyšel od cizího člověka, dostaví do domu jiných cizích lidí, klade jim otázky a vyhrožuje, že jim odebere děti, pokud tazatel nebude s odpověďmi spokojen.

Opět je \enquote{státní} pracovník představován jako ten, kdo jen \enquote{dělá svou práci,} zatímco průměrný jedinec, který dělá totéž, je považován za nebezpečného, pravděpodobně psychicky labilního člověka. Tím nechci říci, že by nikdy nemohla nastat situace, kdy by dítě mělo být rodičům odebráno kvůli jeho vlastní ochraně, ale takové záležitosti by bral mimořádně vážně každý jedinec, který by musel nést osobní odpovědnost za své činy. Naproti tomu byrokrat, který působí pouze jako kolečko ve stroji \enquote{státu,} bude takové věci dělat s mnohem menším zaváháním a s menším ospravedlněním, protože si bude představovat, že za vše, co udělá, je odpovědné výhradně něco, čemu se říká \enquote{zákon.}

4) Situace A: Pilot Letectva Spojených států po obdržení rozkazu odlétá na správné souřadnice a doručí svůj náklad na zamýšlený cíl. Výsledkem je, že je zabito několik žoldáků jiné \enquote{autority} a několik civilistů, kteří se náhodou nacházeli v oblasti. Situace B: Americký občan, jednající na vlastní pěst, naloží letadlo podomácku vyrobenou výbušninou, přeletí nad budovou ve městě, kde, jak známo, sídlí zákeřný pouliční gang, a shodí munici. Výsledkem je smrt několika členů gangu a tuctu nevinných kolemjdoucích, kteří náhodou procházeli kolem.

Průměrný Američan považuje civilní oběti prvního scénáře za nešťastné, ale přičítá je rizikům války. Vojenský pilot je považován za hrdinu, protože sloužil své zemi, a dostane medaili. V druhém případě však průměrný Američan považuje pilota za šílence, teroristu a vraha a požaduje, aby byl do konce života uvězněn.

To, zda byl čin politiky formálně prohlášen za \enquote{legální} a zda je prováděn na příkaz \enquote{autority,} má obrovský vliv na vnímání morálnosti a legitimity činu. Ve zcela reálném smyslu nejsou ti, kdo plní příkazy \enquote{autority,} ani považováni za lidi, protože jejich chování a jednání je posuzováno podle tak drasticky odlišných měřítek než chování a jednání běžných lidských bytostí. Jako další příklad lze uvést, že spousta lidí by byla znepokojena zprávou, že se v jejich sousedství pohybuje \enquote{muž se zbraní,} pokud by se nedozvěděla, že tento muž má odznak.

Lidé posuzují chování převážně na základě toho, zda bylo \enquote{autoritou} povoleno nebo zakázáno, a ne na základě toho, zda je dané chování ve své podstatě legitimní. Když jsou například občané předvoláni k autoritářskému soudu jako porotci v \enquote{trestním} procesu, je běžné, že \enquote{soudce} porotě sdělí, že se \emph{nemá} zabývat tím, zda obviněný udělal něco \emph{špatného}; má pouze rozhodnout, zda jeho jednání bylo v souladu s tím, co \enquote{soudce} prohlásí za \enquote{zákon.} Za zmínku stojí, že ti, kteří jsou u moci, v průběhu let záměrně a metodicky odbourávali starou tradici známou jako \enquote{porotní anulování,} kdy porota mohla v podstatě zrušit to, co považovala za špatný \enquote{zákon,} tím, že vynesla verdikt \enquote{nevinen,} i když se domnívala, že obviněný skutečně porušil \enquote{zákon.} Tuto pravomoc má stále každá porota, ale autoritářští soudci dělají vše pro to, aby si ji porotci neuvědomovali.

I když nejsou v porotě, většina lidí stále posuzuje ostatní skrze autoritářsky zabarvené brýle a dobrotu druhého posuzuje z velké části na základě toho, zda se podřizuje příkazům politiků -- tedy zda je \enquote{zákona dbalý daňový poplatník.} Porovnejte, jak by průměrný občan nahlížel na dvě níže popsané osoby.

\textbf{Osoba A} nemá řidičský průkaz, pracuje \enquote{na černo,} aby nemusela platit \enquote{daně,} nikdy se nezaregistrovala k \enquote{vojenské službě,} vlastní neregistrovanou střelnou zbraň bez licence, občas kouří trávu, občas hraje hazardní hry (\enquote{nelegálně}) a bydlí v chatě, kterou vlastní, ale nemá ji \enquote{zkolaudovanou,} a která má vzadu terasu, kterou si postavil, aniž by si předtím vyřídil stavební povolení.

\textbf{Osoba B} má řidičský průkaz, platí daně z toho, co vydělá, je registrována k odvodu, vlastní registrovanou střelnou zbraň, občas pije pivo, občas hraje státní loterii a bydlí ve \enquote{státem} prověřeném a schváleném domě s \enquote{státem} prověřenou a schválenou terasou vzadu.

Oba žijí jinak podobný život, oba jsou produktivní a ani jeden z nich nikoho neokrádá ani nenapadá. Jejich chování, volby a životní styl jsou si téměř ve všech ohledech velmi podobné, až na to, že proti jednání Osoby A existují \enquote{zákony,} ale proti jednání Osoby B nikoli. Už jen tato skutečnost, bez jakéhokoli jiného podstatného rozdílu v tom, co dělají nebo jak se chovají k ostatním lidem, způsobí, že mnoho lidí bude na Osobu A pohlížet s jistým opovržením, zatímco na Osobu B s respektem a souhlasem. Ve skutečnosti, pokud by byla Osoba A obviněna, zadržena, a dokonce fyzicky napadena (např. taserována, zbita a spoutána) \enquote{strážci zákona,} i kdyby nikdy nikoho neohrožovala ani nikomu neublížila, mnoho věřících ve stát by vyjádřilo názor, že si to \enquote{zasloužila,} že si \emph{zasloužila} být napadena a zavřena do klece za to, že neuposlechla příkazů politiků.

Tendence přihlížejících obviňovat \emph{oběti} autoritářského násilí je neuvěřitelně silná. Člověk, který přijme pověru o autoritě -- myšlenku, že někteří jedinci mají právo násilně ovládat druhé a že tito druzí mají povinnost se podřídit -- bude předpokládat, že pokud \enquote{autorita} používá násilí proti člověku, musí to být oprávněné, a proto oběť takového násilí musela udělat něco špatného. Tento vzorec se projevuje v různých situacích. Když například američtí vojáci zabijí civilisty v nějaké cizí zemi, mnozí Američané jsou zoufalí, a proto automaticky předpokládají, aniž by měli jediný důkaz, že mrtvé oběti museli být \enquote{povstalci,} kolaboranti nebo alespoň sympatizanti \enquote{nepřítele.} Jako další příklad lze uvést, že když byli Davidiáni z větve u texaského Waco podrobeni vojenskému útoku, po němž následovalo dlouhodobé fyzické a psychické mučení a poté hromadná likvidace, mnoho Američanů rychle usoudilo, že každý, komu to \enquote{stát} udělal, si to musel zasloužit. Američtí tyrani tento postoj podporovali tím, že vymýšleli různé fámy a obvinění, aby démonizovali oběti tohoto násilného, fašistického útoku na nenásilné lidi. Ve skutečnosti byl incident výsledkem reklamního triku ATF, založeného na fámách, že někteří lidé ve skupině vlastnili \enquote{nelegální} součástky ke zbraním. Mnoho lidí předpokládá, že pokud byl někdo napaden, stíhán nebo uvězněn agenty \enquote{autority,} pak musel udělat něco špatného a musel si zasloužit to, co mu bylo provedeno. Tento předpoklad může pramenit z toho, že lidé odmítají vzít v úvahu možnost, že \enquote{stát,} na jehož ochranu spoléhají, je ve skutečnosti agresorem, nebo může pramenit z toho, že nechtějí vzít v úvahu možnost, že kdokoli, včetně jich samotných, se může stát další bezmocnou obětí autoritářského násilí, i když se ničeho špatného nedopustil. Bez ohledu na příčinu je konečným výsledkem to, že když je ve jménu \enquote{práva} pácháno zlo, mnozí pozorovatelé okamžitě nenávidí oběti a radují se z bolesti a utrpení, které jsou jim působeny.

\section{Povinnost jednat špatně}

Zatímco každý ví, že existují \enquote{zákony} proti loupežím a vraždám (s výjimkou případů, kdy jsou spáchány ve jménu \enquote{autority}), průměrný člověk vůbec nezná desítky tisíc stran dalších zákonů, pravidel a předpisů -- federálních, státních i místních. Ale i když mají jen velmi malou představu o tom, co přesně \enquote{zákon} dovoluje a nedovoluje, většina lidí stále zastává obecné přesvědčení, že \enquote{dodržování zákona} je dobrá věc a že \enquote{porušování zákona} je špatná věc. Ve skutečnosti, i když je člověk silně proti určitému \enquote{zákonu} a domnívá se, že je nespravedlivý, může stále zastávat obecné, rozporuplné přesvědčení, že \enquote{zákony} by se měly dodržovat a že je oprávněné trestat ty, kteří je nedodržují. Tento psychologický paradox je ve skutečnosti docela běžný, mnoho lidí vehementně lobbuje za změnu \enquote{zákonů,} které považují za špatné, a zároveň podporuje myšlenku, že dokud je to zákon, lidé by ho měli dodržovat.

Takové mentální rozpory jsou běžné v kontextu víry v autoritu, ale mimo ni jsou vzácné. Nikdo by například netvrdil, že je morálně špatné pokusit se ukrást kabelku staré paní, ale zároveň je morálně špatné, aby se stará paní nenechala svou kabelku ukrást. Pojem \enquote{špatný zákon} se však v mysli člověka, který věří v autoritu, redukuje na podobný paradox: \emph{špatný} příkaz, který je zároveň \emph{špatné} neuposlechnout. Pozorovatel, který věří v autoritu, může považovat určitý příkaz, uzákoněný pány a prováděný vymahately, za nedůležitý, zbytečný, kontraproduktivní, nebo dokonce hloupý či nespravedlivý, a zároveň věří, že lidé mají stále morální povinnost tento příkaz dodržovat, prostě proto, že je to \enquote{zákon.} Příkladů důsledků takového pohledu je mnoho, od všedních až po děsivé. Zde je několik z nich.

1) Ve dvě hodiny ráno na otevřené, rovné a prázdné silnici vedoucí neobydlenou zemědělskou krajinou řidič zpomalí, ale nezastaví na znamení STOP na křižovatce. Policista na motocyklu, který se skrývá o sto metrů dál za křovím, rozsvítí světla. Téměř každý by se vzhledem k těmto skutečnostem shodl na tom, že řidič nikoho nepoškodil ani neohrozil, ani ničí majetek, a přesto by většina lidí souhlasila s tím, že policista by měl právo požadovat po řidiči peníze prostřednictvím \enquote{pokuty.} Jinými slovy, i když by připustili, že jediné \enquote{špatné} na tom, co řidič udělal, bylo to, že to bylo technicky \enquote{nezákonné,} domnívají se, že už jen to ospravedlňuje násilné okradení řidiče. Pokud bychom šli ještě o krok dál, pak pokud by se řidič pokusil z místa odjet, místo aby přijal \enquote{pokutu,} většina pozorovatelů by souhlasila s tím, že policista by měl právo řidiče pronásledovat, chytit a uvěznit.

2) \enquote{Státní} inspektor ze státní \enquote{Zdravotní rady} provádí kontrolu restaurace. Restaurace je dokonale čistá a uspořádaná a inspektor nezjistí žádné známky toho, že by v ní cokoli představovalo pro někoho zdravotní riziko. Přesto však zjistí několik technických porušení místního \enquote{předpisu} pro restaurace. V důsledku těchto porušení -- ne proto, že by někoho ohrožovala, ale proto, že jsou \enquote{proti pravidlům} -- je majitel restaurace pokutován stovkami dolarů. Opět platí, že i když majitel restaurace nikomu neublížil ani nikoho neohrozil, většina lidí by považovala za legitimní, aby byl majitel okraden těmi, kdo jednají jménem \enquote{státu.} A pokud by se majitel pokusil takovému okradení bránit -- ať už tím, že by se pokusil technické \enquote{porušení} utajit, nebo tím, že by \enquote{inspektora} podplatil, nebo tím, že by odmítl pokutu zaplatit -- většina lidí by ho považovala za nemorálního a vymahatele by viděla jako lidi, kteří mají právo použít jakékoli prostředky k dosažení dodržování \enquote{zákona.}

3) Muž veze svého přítele domů z večírku. Protože věděl, že bude muset řídit, neměl žádný alkohol k pití, ačkoli jeho přítel ano. Vysadí svého přítele a jede domů. Všimne si, že před ním policie provádí dopravní kontrolu, a vzpomene si, že jeho kamarád nechal v autě zpola plnou láhev piva. Protože ví, že mít v autě otevřenou nádobu s alkoholem je \enquote{nezákonné,} zakryje ji. Nikomu neublížil ani nikoho neohrozil a ve skutečnosti se zachoval docela zodpovědně, když působil jako určený řidič, aby se ujistil, že jeho přítel dorazí v pořádku domů. Přesto však \enquote{porušil zákon} (i když omylem) tím, že řídil auto s otevřenou lahví piva, a pak se snažil důkazy o této skutečnosti skrýt. Kdyby byl při tom přistižen a zatčen, málokdo by v této situaci považoval policistu za toho špatného.

4) Muž prodává brokovnici s hlavní o čtvrt palce kratší, než povoluje \enquote{zákon.} Zbraň není o nic smrtonosnější než brokovnice o čtvrt palce delší a nikdo ze zúčastněných nikomu nevyhrožoval ani nepoužil násilí. Muž, který byl s \enquote{nelegálním} předmětem přistižen, je však vystaven polovojenské invazi na svůj pozemek, po níž následuje ozbrojená přestřelka, při níž je zabito několik lidí.

Tento příklad bohužel není hypotetický. Stalo se to Randymu Weaverovi v Ruby Ridge v roce 1992. A nebyl pouze \enquote{přistižen} při prodeji \enquote{nelegální} brokovnice; byl k tomu zlákán tajnými \enquote{strážci zákona.} Výsledkem ozbrojeného vniknutí na Weaverův pozemek a následné přestřelky a obléhání bylo, že manželka a syn pana Weavera byli zabiti a on a jeho přítel zraněni. Ačkoli by bylo absurdní, kdyby někdo tvrdil, že existuje morální rozdíl mezi držením brokovnice s 18palcovou hlavní a držením brokovnice se 17¾palcovou hlavní, a i když toto tvrzení bylo celým \enquote{právním} odůvodněním ozbrojeného útoku a konfrontace, mnoho pozorovatelů by přesto obviňovalo Randyho Weavera a považovalo by ho za toho špatného, protože se nechal přemluvit k porušení arbitrárního, zcela iracionálního (a dokonce protiústavního) \enquote{zákona.} To je síla víry v autoritu: může vést k tomu, že mnoho lidí považuje bandu sadistických, vraždících násilíků za ty dobré a jejich \emph{oběti} za ty špatné.

Pro většinu lidí má \enquote{porušení zákona,} aniž by bylo uvedeno, jakého zákona, automaticky negativní konotaci. Neuposlechnutí \enquote{autority} považují nejen za nebezpečné, ale i za nemorální. Pro věřící ve \enquote{stát} je otevřené neuposlechnutí představitele \enquote{autority} něco mnohem horšího než spáchání drobného \enquote{trestného činu} bez oběti.

Průměrný pozorovatel, který pozoruje interakci mezi představitelem \enquote{autority} a kýmkoli jiným, se často dívá s opovržením na každého, kdo okamžitě a bezvýhradně neodpovídá na jakékoli otázky a nevyhoví jakýmkoli požadavkům muže s odznakem. Dokonce i v případě, že osoba vyhoví, ale projeví vůči \enquote{autoritě} nějaký \enquote{postoj} -- jakýkoli jiný postoj než pokornou podřízenost -- mnoho pozorovatelů rychle odsoudí toho, kdo se nepokloní. A na toho, kdo před policií utíká, i kdyby se původně ničeho nedopustil, se většina lidí dívá s opovržením. A když je někdo, kdo utíká, schovává se nebo odmítá spolupracovat, \enquote{strážci zákona} zmlácen, mučen nebo dokonce zavražděn, mnoho pozorovatelů vysloví názor, že oběť měla udělat to, co jí policie nařídila. A když se někdo aktivně \emph{vzpírá} \enquote{autoritě,} málokdo má odvahu postavit se na jeho stranu za jakýchkoli okolností, dokonce i pouhými slovy. Stejně jako dobře vycvičený pes nekousne svého pána, ani když je sadisticky týrán, tak i ti, kteří byli vycvičeni k tomu, aby se podřizovali \enquote{autoritě,} se obvykle psychicky nedokážou přinutit k tomu, aby hnuli prstem na obranu sebe, natož někoho jiného, před jakoukoli agresí páchanou ve jménu \enquote{zákona,} \enquote{státu} a \enquote{autority.} Většina lidí totiž v důsledku své autoritářské indoktrinace své bližní spíše odsoudí, než aby se společně se svými bližními postavila na odpor proti tyranii.

Je samozřejmě rozdíl mezi tvrzením, že není chytré, aby někdo něco dělal, a tvrzením, že je nemorální něco dělat. Jedna věc je říci, že je hloupé, když někdo \enquote{držkuje} před policistou, a druhá věc je říci, že je to vlastně nemorální a že ten, kdo to udělá, si proto \emph{zaslouží} jakékoli zneužití nebo trest. Věřící v autoritu často vyjadřují druhý názor na každého, kdo se \enquote{policii vzpírá,} ať už je důvod jakýkoli.

Představa, že by průměrní lidé vnucovali spravedlnost svéhlavým \enquote{strážcům zákona,} etatisty existenčně děsí, a to i v případě, že se \enquote{strážce zákona} dopustil něčeho tak závažného, jako je spáchání vraždy. V očích dobře indoktrinovaných lidí je jediným \enquote{civilizovaným} postupem v takové situaci prosit nějakou \emph{jinou} \enquote{autoritu,} aby věci napravila, ale nikdy ne \enquote{vzít zákon do vlastních rukou.} Lidé si mohou stěžovat na \enquote{zákonné} bezpráví a odsuzovat je, ale jen málokdo je schopen vůbec uvažovat o možnosti zapojit se do promyšleného, \enquote{nezákonného} odporu, a to i v případě, že agenti \enquote{státu} páchají krutou brutalitu na neozbrojených, nenásilných cílech. A pokud lze dlouhodobým vymýváním mozků učinit lidi psychicky neschopnými klást odpor útlaku páchanému ve jménu \enquote{autority,} pak je jedno, zda tito lidé mají fyzické prostředky k odporu. Moderní tyrani a jejich vymahatelé vyždy čelí mnohonásobné přesile (a často i ve zbrani) svých obětí. Přesto se tyrani stále udržují u moci nikoli proto, že by lidé neměli \emph{fyzickou} schopnost klást odpor, ale proto, že jim v důsledku hluboce vštípené víry v autoritu chybí \emph{mentální} schopnost klást odpor. Jak řekl Stephen Biko: \enquote{Nejsilnější zbraní v rukou utlačovatele je mysl utlačovaného.}

\section{Dvojí metr na násilí}

Dvojí metr v myslích těch, kteří byli indoktrinováni k autoritářství, pokud jde o použití fyzické síly, je obrovský. Když je například \enquote{strážce zákona} zachycen na videu, jak brutálně napadá neozbrojeného, nevinného člověka, obvykle se mluví o tom, zda by měl být policista pokárán, nebo snad dokonce přijít o práci. Pokud naopak nějaký občan napadne \enquote{policistu,} téměř všichni budou nadšeně požadovat -- často aniž by se zajímali nebo ptali, proč to dotyčný udělal -- aby byl dotyčný na mnoho let zavřen do klece. A pokud se někdo uchýlí k použití smrtící síly proti údajnému představiteli \enquote{autority,} málokdo se vůbec obtěžuje zeptat, proč to udělal. V jejich myslích, bez ohledu na to, co agent \enquote{autority} udělal, není nikdy v pořádku zabít zástupce boha zvaného \enquote{stát.} Pro věřící v autoritu není nic horšího než \enquote{vrah policistů,} bez ohledu na to, proč to udělal.

Ve skutečnosti je použití smrtící síly proti tomu, kdo předstírá, že jedná jménem \enquote{autority,} morálně totožné s použitím smrtící síly proti komukoli jinému. Akt agrese se nestává legitimnějším nebo spravedlivějším jen proto, že je \enquote{legalizován} a spáchán těmi, kdo tvrdí, že jednají jménem \enquote{autority.} A použití jakékoli síly, která je nezbytná k zastavení nebo zabránění aktu agrese, ať už je agrese \enquote{legální,} nebo ne, a ať už je agresor \enquote{strážcem zákona,} nebo ne, je oprávněné. (Samozřejmě, že rizika spojená s odporem proti \enquote{legální} agresi jsou často mnohem vyšší, ale to neznamená, že je to méně morální nebo oprávněné.) Mnohé z důvodů, které dnes \enquote{strážci zákona} používají k násilnému zadržování lidí -- jako je účast na pokojných veřejných demonstracích bez \enquote{povolení} nebo fotografování \enquote{strážců zákona} či \enquote{státních} budov nebo nepodrobení se náhodnému zastavení a výslechu od \enquote{strážců zákona} -- nemají bez mýtu \enquote{autority} žádné opodstatnění. Odporovat takovému fašistickému násilnictví, i když je k tomu zapotřebí smrtící síly, je proto morálně ospravedlnitelné, i když nesmírně nebezpečné. Většina lidí však není doslova schopna o takové myšlence ani uvažovat. Dokonce i když rozpoznají nespravedlivý útlak, představují si, že \enquote{civilizovanou} reakcí je nechat bezpráví dojít a později prosit nějakou jinou \enquote{autoritu} o nápravu.

Když čelí \enquote{legální} agresi a útlaku, existují pouze dvě možnosti: buď jsou lidé povinni \emph{umožnit} \enquote{strážcům zákona,} aby na nich páchali nejrůznější nespravedlnost a útlak (a pak si stěžovat), nebo mají právo použít jakoukoli míru síly, která je nezbytná k zastavení takové nespravedlnosti a útlaku. Bylo by zbytečné říkat, že někdo má \enquote{právo} na to, aby mu \enquote{státní} agenti bezdůvodně neprohledávali a nezabavovali věci (podle čtvrtého dodatku), kdyby oběť takové tyranie byla povinna v danou chvíli umožnit, aby k tomu došlo, a později si na to stěžovat. Mít \enquote{právo} na svobodu od takového útlaku logicky znamená právo použít jakoukoli sílu, která je nutná k tomu, aby k takovému útlaku vůbec nedošlo, i kdyby to vyžadovalo zabití policistů. Ale už samotná myšlenka děsí ty, kteří byli vycvičeni k tomu, aby se vždy podřizovali \enquote{autoritě.} Většina těch, kteří hovoří o \enquote{nezcizitelných} právech, se stále vzpírá při pomyšlení, že by tato práva měla být násilím bráněna proti autoritářským útokům.

Říkat, že někdo má na něco \enquote{právo,} a zároveň tvrdit, že by nebyl oprávněn takové právo násilně bránit proti \enquote{státním} zásahům, je rozpor. Popravdě to, co většina lidí nazývá \enquote{právy,} ve skutečnosti vnímá jako \enquote{státem} udělená privilegia, která, jak doufají, jim jejich páni povolí, ale která nemají v úmyslu násilím bránit, pokud by ta \enquote{práva} \enquote{stát} \enquote{zakázal.} Například mít nezadatelné právo říkat své názory (právo na svobodu projevu) znamená, že dotyčný má také právo použít jakoukoli míru násilí, až po smrtící sílu, na obranu proti agentům \enquote{státu,} kteří se ho snaží umlčet. Ačkoli tento bod působí loajálním věřícím v autoritu velmi nepříjemně, samotný koncept osoby, která má nezadatelné právo něco dělat, znamená také právo, pokud vše ostatní selže, zabít všechny \enquote{strážce zákona,} kteří se jí v tom pokusí zabránit. Ve skutečnosti však neexistuje téměř nic, co by \enquote{stát} mohl udělat, ať už jde o cenzuru, napadení, únos, mučení, nebo dokonce vraždu, proti čemu by průměrný etatista obhajoval násilný, \enquote{nezákonný} odpor.

(Čtenář je vyzván, aby si vyzkoušel hloubku své vlastní loajality k mýtu autority tím, že se zamyslí nad otázkou, co by se muselo stát, aby se sám cítil oprávněn zabít \enquote{strážce zákona.})

\enquote{Strážci zákona} neustále eskalují neshody na úroveň násilí, pokaždé když se snaží někoho zatknout, vniknout do něčího domu nebo někomu násilím sebrat majetek. A autoritářští vymahatelé pak budou úroveň použitého násilí zvyšovat tak dlouho, dokud nedosáhnou svého. Výsledkem je, že lidé, pokud nejsou ochotni se zapojit do otevřené revoluce proti celému systému, se dříve či později podvolí vůli vládnoucí třídy, nebo budou zabiti. A přestože žoldáci státu vždy používají sílu nebo hrozbu síly, aby si podmanili a podrobili průměrné lidi, v okamžiku, kdy jejich zamýšlené oběti odpoví na násilí násilím, většina pozorovatelů okamžitě identifikuje \emph{oběť} agrese -- toho, kdo použil sílu pouze k \emph{obraně} proti útoku -- jako \enquote{toho špatného.} Tento do očí bijící dvojí metr -- představa, že je v pořádku, když se \enquote{autorita} pravidelně dopouští násilných agresivních činů, ale je strašně špatné, když obyčejní lidé někdy reagují obranným násilím -- ukazuje, jak drasticky může víra v autoritu pokřivit lidské vnímání reality.

Je ironií, že když vezmeme v úvahu jiná místa a jinou dobu, téměř všichni akceptují a dokonce chválí použití \enquote{nezákonného} násilí, včetně smrtícího násilí, proti zástupcům \enquote{státu.} Málokdo by ještě trval na tom, že Židé, kteří žili v Německu v roce 1940, se měli nadále snažit \enquote{pracovat v rámci systému} tím, že by volili a podávali petice za spravedlnost Třetí říši. Místo toho jsou ti, kteří se \enquote{ilegálně} skrývali, utíkali nebo dokonce kladli násilný odpor (jako tomu bylo ve varšavském ghettu), dnes téměř všemi považováni za oprávněné, přestože technicky vzato šlo o \enquote{zločince,} \enquote{porušovatele zákona,} a dokonce \enquote{vrahy policistů.} Autoritáři však ve své době a ve své zemi nejenže nadále odsuzují každého, kdo se \enquote{nezákonně} snaží vyhnout útlaku nebo se mu bránit, ale vesele se škodolibě radují z utrpení takových lidí, když jsou \enquote{státem} potrestáni. Radovat se například z potrestání \enquote{daňového podvodníka,} jak to dělají mnozí Američané, je podobné, jako kdyby měl otrok radost z bičování spoluotroků, kteří se pokusili o útěk.

Může v tom být aspekt prosté závisti: pocit, že pokud byl jeden subjekt obětí, není \enquote{spravedlivé,} že jiný takovému utrpení unikl. To přispívá k tomu, že \enquote{daňoví poplatníci} -- tj. ti, kteří byli násilím vydíráni vládnoucí třídou -- často vyjadřují nelibost vůči každému, kdo se podobnému vydírání vyhnul. Je zvláštní, že oběti \enquote{legální} loupeže si často představují, že jsou ctnostní, protože byli okradeni, a dívají se křivě na ty, kteří z jakéhokoli důvodu okradeni nebyli.

\section{Nebezpečí nečinnosti}

Člověk, který považuje \enquote{porušování zákona} za špatné ze své podstaty, bez ohledu na to, o jaký \enquote{zákon} se jedná, může rychle nahlásit \enquote{autoritám} jakoukoli \enquote{nezákonnou} činnost, o které ví, i když se jedná o činnost bez oběti a nepředstavuje ani násilí, ani podvod. Stejně tak ti, kdo zasedají v porotách \enquote{státních} soudních síní, pokud si představují neposlušnost vůči \enquote{autoritě} (\enquote{porušení zákona}) jako něco ze své podstaty nemorálního, pravděpodobně dají požehnání tomu, aby byl někdo potrestán, někdy i dost přísně, za něco, co nikomu neublížilo a nepředstavovalo ani podvod, ani násilí. V případě \enquote{udavače} a porotce však takové jednání vyřazuje člověka z role pouhého pozorovatele a posouvá ho do role \emph{spolupracovníka} útlaku.

Škody způsobené vírou v autoritu u pozorovatelů útlaku vznikají častěji jejich nečinností, než jejich konáním. Útlak -- velký i malý -- je páchán znovu a znovu přímo pod nosem v zásadě dobrých lidí, kteří s tím nic nedělají. Do jisté míry je to důsledek prostého pudu sebezáchovy: člověk se může vyhýbat tomu, aby se zapojil, jednoduše proto, že se bojí o svou vlastní bezpečnost. Milgramovy experimenty však zcela jasně ukázaly, že i bez jakéhokoli skrytého ohrožení sebe sama se většina lidí cítí neodolatelně nucena poslouchat \enquote{autoritu,} i když vědí, že to, co se jim říká, je špatné a škodlivé pro ostatní. A pokud je pro ně obtížné neuposlechnout domnělou \enquote{autoritu,} bude pro ně ještě obtížnější, ne-li nemožné, přimět se k zásahu, když \enquote{autorita} uplatňuje svou vůli na někom jiném.

Výsledek toho, že \emph{pozorovatelé} byli vycvičeni k pasivitě, poslušnosti a nekonfliktnosti, lze vidět na mnoha případech, kdy po celém světě a v historii desítky, stovky nebo dokonce tisíce pozorovatelů stojí jako zombie a přihlížejí, jak zástupci \enquote{autority} napadají nebo vraždí nevinné lidi. Dokonce i ve Spojených státech, údajné \enquote{zemi svobody a domově statečných,} se stále objevují videa zachycující policejní brutalitu přímo před zraky davů přihlížejících, kteří jen stojí a koukají, aniž by hnuli prstem, aby ochránili své bližní před zlem páchaným ve jménu \enquote{autority.}

\chapter{Účinky na zastánce}

\section{\enquote{Legalizovaná} agrese}

Ačkoli si většina lidí pravděpodobně představuje, že jsou \enquote{pozorovateli} autoritářského útlaku a bezpráví, ve skutečnosti je téměř každý z nich v té či oné podobě zastáncem \enquote{státního} násilí. Každý, kdo volí bez ohledu na kandidáta, nebo dokonce verbálně podporuje nějakou \enquote{politiku} či \enquote{státní program,} schvaluje zahájení násilí vůči svým bližním, i když si to jako takové neuvědomuje. Je tomu tak proto, že \enquote{právo} není o přátelských návrzích nebo zdvořilých žádostech. Každý takzvaný \enquote{zákon} vydaný politiky je příkazem, který je podpořen \emph{hrozbou násilí} vůči těm, kdo se nepodřídí. (Jak řekl George Washington: \enquote{\emph{Stát není rozum, není to výmluvnost, je to síla}.})

Většina lidí se ve svém každodenním životě velmi zdráhá použít proti svým bližním výhrůžky nebo fyzickou sílu. Pouze nepatrný zlomek z mnoha osobních neshod, které se vyskytnou, vede k násilným konfliktům. Kvůli své víře ve stát však téměř každý obhajuje rozšířené násilí, aniž by si to uvědomoval. A necítí se při tom nijak provinile, protože hrozby a nátlak vnímají jako neodmyslitelně legitimní, když se jim říká \enquote{vymáhání práva.}

Každý ví, co se stane, když někoho přistihnou při \enquote{porušování zákona.} Může to být jen \enquote{pokuta} (požadavek na zaplacení pod pohrůžkou násilí), nebo to může být \enquote{zatčení} (násilné odvedení někoho do zajetí), nebo to může dokonce vyústit v to, že \enquote{strážci zákona} zabijí někoho, kdo se nadále brání. Ale \emph{každý \enquote{zákon} je hrozbou, za níž stojí schopnost a ochota použít smrtící sílu proti těm, kdo neuposlechnou}, a každý, kdo o této myšlence poctivě uvažuje, tuto skutečnost uzná.

Víra v autoritu však vede ke zvláštnímu rozporu v tom, jak lidé vidí svět. Téměř každý je zastáncem toho, aby se \enquote{právo} používalo k donucování druhých k určitým věcem nebo k financování určitých věcí. Tito zastánci však při obhajobě takového jednání, s plným vědomím důsledků pro každého, kdo bude přistižen při neuposlechnutí, nedokážou rozpoznat, že to, co obhajují, je násilí. Existují například miliony lidí, kteří se považují za mírumilovné, civilizované lidi -- někteří dokonce hrdě nosí nálepku \enquote{pacifisté} -- a přitom obhajují ozbrojené přepadení všech, které znají, i milionů cizích lidí. Nevidí v tom žádný rozpor, protože loupež dostává eufemismus \enquote{zdanění} a provádějí ji lidé, kteří si představují, že mají \emph{právo} loupit, a to ve jménu \enquote{státu.}

Úroveň zapírání, kterou víra v autoritu vytváří, je hluboká. Když lidé obhajují \enquote{politické} násilí, nepřijímají žádnou odpovědnost za jeho výsledky. Ti, kdo například žádají o \enquote{dávky,} žádají o to, aby dostali kořist násilně ukradenou svým sousedům prostřednictvím \enquote{daní.} Stejně tak žádost o \enquote{státní} zaměstnání se rovná žádosti, aby byli sousedé \emph{vynuceni} k placení jejich platu. Ať už dotyčný dostane přímou platbu, nebo nějakou službu, program či jinou výhodu, obvykle ukradený majetek přijme bez sebemenšího náznaku studu či viny. Jinak se může chovat naprosto sousedsky k lidem, o jejichž okradení stát požádal. V žádné jiné situaci nedochází k tak podivnému mentálnímu odpojení, a to nejen u toho, kdo akt agrese obhajuje, ale ani u jeho \emph{oběti}. Kdyby například jeden člověk zaplatil ozbrojenému zloději, aby se vloupal do domu jeho souseda a ukradl mu nějaké cennosti, a soused by věděl, že to udělal, asi by takoví sousedé nebyli v přátelském vztahu (mírně řečeno). Když se však totéž děje s využitím \enquote{autority,} prostřednictvím voleb a následné \enquote{legislativní} krádeže, zloděj ani oběť na tom obvykle nevnímají nic špatného.

(\emph{Osobní poznámka autora: Už jsem ztratil přehled o tom, kolik lidí vyjádřilo soucit se mnou a mou ženou, protože jsme byli uvězněni za to, že jsme se nepodřídili daňovému úřadu. Ale naše neanarchistické známé zřejmě nikdy nenapadne, že nás do klece zavřeli právě ti, které volili, za to, že jsme neuposlechli příkazů, které prosazovali. Pokud vím, nikdo z našich známých etatistů si ani nevšiml schizofrenie a pokrytectví, když aktivně podporují masové vydírání (\enquote{zdanění}) a pak srdečně soucítí s oběťmi téhož vydírání.})

Nadpřirozenou podstatu \enquote{autority} lze spatřovat v tom, že mezi lidmi, kteří budou ochotně hlasovat pro to, aby jejich sousedé byli \enquote{legálně} vydíráni a okrádáni, by jen málokdo žádal nebo platil obyčejným smrtelníkům, aby dělali totéž. Málokdo by se cítil oprávněn najmout pouliční gang, aby okradl jeho sousedy, aby zaplatil školní docházku vlastního dítěte, ale mnoho milionů lidí obhajuje totéž, když schvaluje \enquote{majetkové daně} na financování \enquote{veřejných} škol. Proč jim tyto dvě věci připadají morálně tak odlišné? Protože ti, kdo věří ve stát, věří, že se skládá z něčeho víc než z lidí v ní. Představují si, že má práva, která nemá žádný obyčejný smrtelník.

Z pohledu etatisty má žádost, aby \enquote{stát} něco udělal, mnohem více společného s modlitbou k bohům, aby něco udělali, než s žádostí, aby něco udělali lidé. Etatista, který požaduje určitou \enquote{legislativu,} by byl zděšen a uražen, kdyby se nějaká skupina průměrných lidí nabídla, že poskytne podobné služby. Představte si, že by nějaký pouliční gang učinil místnímu obyvateli následující nabídku:

\enquote{\emph{Oloupíme tvé sousedy a z toho, co získáme, zaplatíme věci, které chceš -- školu pro tvé dítě, opravu silnic a podobně. Část si samozřejmě musíme nechat sami. A řekněte nám, jak byste si přáli, aby se vaši sousedé chovali, a my zajistíme, aby se tak chovali. Když nebudou dělat, co jim řekneme, sebereme jim věci nebo je zavřeme do klece}.}

Kdyby takovou nabídku učinili průměrní lidé, byli by odsouzeni za pokus o násilnictví. Když však totéž navrhne v předvolebním projevu někdo, kdo se uchází o \enquote{státní} funkci, a když se takové věci dějí ve jménu vágních politických abstrakcí, jako je \enquote{obecné blaho} nebo \enquote{vůle lidu,} jsou považovány nejen za přípustné, ale i za ušlechtilé a ctnostné. Když politik řekne: \enquote{Musíme zajistit odpovídající financování vzdělání našich dětí a musíme investovat do naší infrastruktury,} mluví doslova o násilném odebrání peněz lidem (prostřednictvím \enquote{daní}) a jejich utracení způsobem, o němž si myslí, že by měl být utracen. Taková agrese je přijímána jako ospravedlnitelná, pokud je prováděna ve jménu \enquote{autority,} ale je uznávána jako nemorální, pokud ji provádějí obyčejní smrtelníci. To ukazuje, že v mysli etatisty je \enquote{stát} něco víc než soubor lidských bytostí. Paradoxně bude etatista trvat na tom, že vše, co \enquote{stát} smí dělat a čím je, pochází od \enquote{lidu.} Veškerá víra ve stát vyžaduje absurdní, sektářské přesvědčení, že prostřednictvím pseudonáboženských politických dokumentů a rituálů (ústavy, voleb, jmenování, legislativy atd.) může skupina obyčejných smrtelníků vykouzlit entitu, která má nadlidská práva -- práva, která nemá nikdo z lidí, kteří ji vytvořili. A jakmile si lidé existenci takové věci halucinují, budou tuto věc dychtivě prosit, aby mohla násilím ovládat a vydírat své sousedy. Lidé uznávají, že obyčejní smrtelníci nemají právo dělat takové věci, ale upřímně věří, že božstvo zvané \enquote{stát} má na takové věci plné právo.

\section{Výmluvy pro agresi}

Ačkoli je \enquote{demokracie} často vychvalována jako vrchol civilizace, spolupráce a vzájemného \enquote{vycházení,} je pravým opakem. Hlasování je aktem agrese a láska k \enquote{demokracii} se rovná lásce k rozsáhlému násilí a neustálým konfliktům. Politické volby nejsou o sounáležitosti, jednotě nebo toleranci; jsou o dohadování se o tom, jak by se všichni měli \emph{přinutit} chovat a co by všichni měli \emph{přinutit} finančně podporovat prostřednictvím řídící mašinérie zvané \enquote{stát.} Množství předvolebních nápisů na zahrádkách před každými volbami není znakem osvícené a svobodné společnosti; je to znak duševně i fyzicky zotročené společnosti, která se hádá o to, který otrokář chce držet bič. Každý, kdo volí (demokraty, republikány nebo jinou stranu), se snaží dostat k moci lidi, kteří budou provádět rozsáhlé vydírání (\enquote{zdanění}), aby financovali různé \enquote{státní} programy. Každý kandidát, který by navrhl úplné zrušení takového okrádání -- zrušení všech \enquote{daní} -- by byl zesměšněn jako extrémistický blázen. Všichni voliči se pokoušejí posílit postavení bandy, o níž vědí, že bude páchat masové loupeže, ale nikdo z těchto voličů za to nepřijímá žádnou odpovědnost. Vědí, co jejich kandidáti udělají, pokud se dostanou k moci, vědí, jaké důsledky budou mít pro každého, kdo pak neuposlechne příkazů těchto politiků, ale víra v autoritu způsobuje, že voliči nejsou psychologicky schopni rozpoznat, že to, co dělají, je obhajoba rozsáhlého násilí.

Ve skutečnosti, bez ohledu na tradiční mytologii a rétoriku, nikdo, kdo věří ve stát, ve skutečnosti \emph{nechce}, aby byla spravována s takzvaným \enquote{souhlasem občanů.} Kdyby se tak skutečně dělo prostřednictvím skutečného souhlasu, znamenalo by to, že politické preference každého člověka by byly vnucovány pouze jemu samotnému, pokud by ostatní náhodou nezastávali přesně stejný program. Je zřejmé, že cílem voliče není donutit se finančně podporovat věci, které se mu líbí, ani ovládat vlastní volby a chování; cílem každého voliče je vždy využít mechanismus \enquote{státu} k tomu, aby donutil \emph{ostatní} lidi k určitým volbám, financování určitých věcí a chování určitým způsobem.

Jednotlivý etatista má totiž někdy dosti laxní pohled na svou \emph{vlastní} povinnost podřídit se nesčetným politickým příkazům (\enquote{zákonům}), neboť má pocit, že je kompetentní spoléhat se na svůj vlastní zdravý rozum a úsudek bez ohledu na \enquote{zákon,} a zároveň má pocit, že všichni ostatní musí být ovládnáni a řízeni \enquote{autoritou.} Věří, že on sám je důvěryhodný a morální a může se rozhodovat sám, a že účelem \enquote{zákona} je udržet všechny ostatní v souladu.

Míra, do jaké různí voliči chtějí, aby \enquote{autorita} ovládala ostatní, se výrazně liší. Konstitucionalista chce, aby federální \enquote{vláda} nutila ostatní financovat pouze ty věci, které jsou Ústavou USA výslovně označeny za federální záležitosti. Zatímco \enquote{progresivista} chce, aby \enquote{stát} nutil ostatní financovat nejrůznější věci, od umění, přes obranu, péči o chudé, vzdělávání, důchodové programy atd. Ale i když se oba typy voličů liší v míře a druzích agrese, kterou podporují, v principu se neliší: oba přijali předpoklad, že \enquote{autorita} má právo násilím vymáhat peníze na funkce \enquote{státu,} které jsou považovány za nezbytné; liší se pouze v tom, co se považuje za \enquote{nezbytné.}

Myšlení téměř každého etatisty je paradoxní. Na jedné straně etatisté vědí, že každý \enquote{zákon,} který schvalují, je příkaz podpořený hrozbou násilí. Jsou si plně vědomi toho, co se děje každému \enquote{porušovateli zákona,} který se nechá chytit, ale průměrný etatista na dotaz vehementně \emph{odmítne}, že by schvaloval zahájení násilí vůči svým bližním. V praktické rovině etatista ví, že jakákoli \enquote{politická} agenda, kterou podporuje, bude v případě jejího uzákonění spravována jakoukoli mírou zastrašování nebo hrubé síly, která je nezbytná k získání podřízení se ze strany lidí. Průměrný etatista, ačkoli si to plně uvědomuje, však zároveň projeví obrovskou logickou nesoudržnost a odmítá připustit, že otevřeně a přímo obhajuje násilné vydírání a ovládání milionů nevinných lidí. Důvodem je to, že etatista věří, že entita zvaná \enquote{autorita} má \emph{právo} vládnout, a v důsledku toho, když se dopouští násilí, nepovažuje to \emph{za násilí}.

Dokud násilí páchají ti, kteří se prohlašují za \enquote{autoritu} a kteří si představují, že mají výjimku z obvyklých pravidel morálky (nekrást, nenapadat, nevraždit atd.), mohou si i ti, kteří jsou nejhorlivějšími zastánci různých \enquote{daní} a jiných \enquote{zákonů,} nadále představovat, že jsou mírumilovní, soucitní a nenásilní lidé. Někteří si dokonce představují, že jsou pacifisté. (Protože vše, co \enquote{stát} dělá, se děje prostřednictvím síly nebo hrozby silou, nic takového jako etatistický pacifista neexistuje a ani existovat nemůže. I když samozřejmě ne všichni anarchisté jsou pacifisté, všichni skuteční pacifisté jsou anarchisté). Existuje mnoho způsobů -- několika z nich se budeme věnovat níže -- jak jinak slušní a ctnostní lidé schvalují agresi a napadání, zastrašování a loupeže, protože věří, že je naprosto přípustné, aby se nadlidské, mýtické božstvo známé jako \enquote{stát} dopouštělo takových činů, a proto věří, že je naprosto morální a ctnostné, aby \emph{požádali} \enquote{stát} o spáchání takových činů.

\section{Charita pomocí násilí}

Typický etatista je hluboce schizofrenní, protože si je zároveň zcela vědom a zcela si neuvědomuje, že osobně obhajuje rozsáhlé používání násilí proti ostatním. Dramatickým příkladem mohou být ti, kteří se považují za milující a soucitné, protože podporují \enquote{státní} programy na pomoc chudým. To, co prostřednictvím podpory \enquote{sociálních} programů doslova obhajují, je masivní vydírání, v němž jsou mnohé miliony lidských bytostí okrádány o miliardy dolarů prostřednictvím hrozby zavření do klece. Zastánci takové \enquote{charity prostřednictvím násilí} si představují, že jsou ctnostní a starostliví kvůli tomu, co mohou potřební dostat, a přitom se zcela distancují od vyhrožování, zastrašování, obtěžování, násilného zabavování a věznění, o nichž vědí, že k nim dochází a které jsou, jak vědí, nezbytné pro každý program \enquote{sociální péče.} Díky tomuto bizarnímu selektivnímu popírání si mohou být ti, kdo věří ve vládu, zcela vědomi hrubé síly, kterou jsou takové \enquote{zákony} prováděny, a přitom si zdánlivě neuvědomují, že oni sami takovou hrubou sílu \emph{podporují}, když takové \enquote{zákony} požadují.

Právě víra v autoritu umožňuje tento podivný psychologický rozpor, neboť přesvědčuje zastánce systémů přerozdělování bohatství, že oběti \enquote{legálního} vydírání mají \emph{povinnost} spolupracovat, a že použití násilí proti těm, kdo neplatí \enquote{své daně,} je proto oprávněné. V důsledku toho je základní měřítko morálky a ctnosti postaveno zcela na hlavu, přičemž zastánci \enquote{blahobytu} se považují za soucitné, protože obhajují násilnou krádež, zatímco na každého, kdo se snaží tomuto násilí vyhnout nebo se mu bránit, pohlížejí jako na zavrženíhodného zločince.

Stejně tak zastánci \enquote{sociálního zabezpečení,} Ponziho systému přerozdělování bohatství, si představují, že jsou starostliví a soucitní. Zaslepeni svou vírou ve vládu si neuvědomují, že nejenže nutí lidi do něčeho, co je (falešně) prezentováno jako \enquote{státem} řízený důchodový systém, ale také přidávají urážku k újmě tím, že naznačují, že lidem nemůžeme a neměli bychom věřit ohledně plánování jejich vlastní budoucnosti. Někdo, kdo vehementně podporuje nucení lidí k účasti na \enquote{investičním} programu, který neinvestuje do ničeho, nemá žádný majetek a jehož výnos je mnohem horší než u většiny skutečných investic (a ve skutečnosti ani žádný výnos nezaručuje), a pak se cítí ušlechtilý a dobročinný, protože lidi do takového programu nutí, musí být opravdu odtržen od reality. (Nejenže neexistuje žádný \enquote{účet} sociálního zabezpečení -- ani individuální, ani kolektivní -- na který by se \enquote{platilo,} ale Nejvyšší soud USA (ve věci \emph{Flemming v. Nestor}, 363 U.S. 603) jasně stanovil, že nikdo nemá smluvní nárok na žádné \enquote{benefity} sociálního zabezpečení, bez ohledu na to, kolik do systému \enquote{zaplatil,} a že Kongres může kdykoli jakékoli nebo všechny \enquote{benefity} přerušit.)

\section{Zastánci brutality}

V dějinách se často stávalo, že lid podporoval odporný útlak, částečně proto, že nebyl schopen rozpoznat zlo jako zlo, když bylo pácháno ve jménu \enquote{práva} a \enquote{autority.} Pokud lidé skutečně věří, že \enquote{stát} má právo vládnout, čemuž dnes věří téměř všichni, bude většina lidí podporovat nebo alespoň pasivně přijímat nejrůznější autoritářská \enquote{řešení.} Například mnozí Němci ve čtyřicátých letech 20. století, kteří by se sami nikdy nedopustili ani neschvalovali soukromé zastrašování nebo napadení, natož vraždu, přesto horlivě podporovali myšlenku \enquote{legislativního,} \enquote{státem} schváleného a \enquote{státem} spravovaného \enquote{řešení} takzvaného \enquote{židovského problému} (jak ho nazýval Hitler). Bylo to oficiálně schváleno a provedeno prostřednictvím \enquote{zákona,} takže si lidé představovali, že jsou bez viny za cokoli, co se stalo, i když se toho vehementně zastávali.

Dnešní Američané, trpící selektivním popíráním, rychle spravedlivě odsuzují to, co dělaly \emph{jiné} násilné, utlačovatelské režimy, ale pomalu si uvědomují, že v důsledku své vlastní víry v autoritu i oni schvalují rozšířenou drakonickou brutalitu ve jménu \enquote{zákona.} I když útlak přesahuje pouhé hrozby a zastrašování a vede k neustálému, rozšířenému, otevřenému násilí a brutalitě, většina lidí v důsledku své víry v autoritu stále není schopna rozpoznat, že jde o zlo.

Zřejmým příkladem je válka. Nacionalismus, který je u autoritářů tak silný, je zaslepuje před absolutním zlem, které ve jménu \enquote{národní obrany} schvalují a podporují. V mnoha případech je tato slepota záměrná. Politici i konzervativní voliči si stěžují, když se Američanům ukazuje tupá realita války. Chtějí mávat vlajkou a fandit svému týmu, nadšeně se účastnit stádní mentality, ale nechtějí, aby museli skutečně \emph{vidět} reálné výsledky toho, co podporují. Lze je přesvědčit, aby hrdě \enquote{podporovali vojáky} a abstraktně věřili v údajně spravedlivou válku, pokud jsou chráněni před nutností vidět krveprolití -- krev, vnitřnosti a části těl -- které jejich \enquote{vlastenectví} způsobuje.

Ačkoli je láska k \enquote{vlasti} stále vydávána za velkou ctnost, pravdou je, že vrazi na obou stranách každé války, včetně těch, kteří bojovali za nejbrutálnější a nejkrutější režimy v dějinách, byli motivováni pocitem spravedlnosti, který jim dává nacionalistická smčková mentalita. K válce by vůbec nemohlo dojít, kdyby vojáci nestavěli oddanost a loajalitu k vlastní tlupě, kmeni nebo \enquote{zemi} nad konání toho, co je správné. \enquote{Vlastenectví} a víra v autoritu jsou dvě klíčové složky války. Nejjednodušší způsob, jak oklamat v podstatě dobré lidi, aby páchali zlo, je vydávat akty agrese a dobývání za \enquote{boj za vlast.}

Zatímco vládci již dlouho praktikují ovládání mysli svých poddaných, v mnoha případech si ovládání mysli těch, kteří věří v autoritu, způsobují sami. Chtějí věřit ve \enquote{svou zemi} a v nějaký spravedlivý, abstraktní princip, nějaký ideál, nějakou ušlechtilou věc (např. \enquote{šíření demokracie}), aniž by museli přemýšlet o tom, co se děje, v jednoduchých, doslovných termínech. Je snazší podporovat masové vraždění, když se mu říká \enquote{válka,} a ještě snadněji, když se mu říká \enquote{obrana země.} Když je zahalena do autoritářské smečkové terminologie, umožňuje to jejím zastáncům -- a těm, kteří ji skutečně provádějí -- představovat si, že podporují něco statečného a spravedlivého. I když jednotliví vojáci mohou skutečně věřit, že bojují za ušlechtilou věc, není možné být \enquote{dobrým člověkem} a zároveň válčit s celou zemí, jak bylo řečeno dříve. Způsob, jakým \enquote{státy} vedou válku, není \emph{nikdy} ospravedlnitelný a \emph{nikdy} morální, protože vždy zahrnuje rozsáhlé násilí vůči nevinným. Ale to je fakt, který nacionalisté, levicoví i pravicoví, odmítají vidět.

Dalším příkladem moderní drakonické brutality, \enquote{legálně} páchané ve \enquote{svobodném světě,} je násilná kampaň známá jako \enquote{válka proti drogám.} Ve jménu snahy vymýtit návyk -- nikoli násilí, krádež nebo podvod, ale pouhý \emph{návyk} -- byly napadeny, terorizovány a zavřeny do klecí miliony nenásilných, mírumilovných a produktivních lidských bytostí. Prosazování \enquote{protidrogových zákonů} probíhá obzvláště brutálním a krutým způsobem, kdy jsou běžné polovojenské invaze do soukromých domů a hojně se vyskytují mnohaleté tresty odnětí svobody za \enquote{zločiny} bez obětí. A obhájci \enquote{války proti drogám} jsou si dobře vědomi nejen otevřeně násilných donucovacích opatření, ale také skutečnosti, že jedinými měřitelnými účinky jsou vyšší ceny některých látek měnících mysl, více trestných činů páchaných kvůli zaplacení těchto látek, násilné konflikty mezi konkurenčními prodejci těchto látek a více finančních prostředků, zbraní, moci a \enquote{legislativních} povolení k obtěžování a napadání nevinných lidí pro ty, kteří nosí nálepku \enquote{autority.} I kdyby to skutečně fungovalo a eliminovalo nebo výrazně omezilo užívání některých drog, byla by taková brutalita naprosto neopodstatněná a nemorální. Ale přestože se naprosto nepodařilo přiblížit se ani o píď k vytčenému cíli, mnozí \enquote{konzervativci} nadšeně fandí dalšímu obtěžování, terorismu a násilí. (Aby se k fašismu přidalo i pokrytectví, většina těchto \enquote{konzervativců} pije alkohol: což je čin morálně totožný s chováním, které chce \enquote{autorita} násilně potlačit).

A zatímco miliony životů jsou nadále ničeny tímto brutálním, drakonickým křížovým tažením, mnozí etatisté horlivě obviňují oběti prohlášením, že \enquote{porušily zákon,} a proto si zaslouží vše, co se jim děje. Takže pro údajně morální a zodpovědné \enquote{konzervativce,} i když člověk nikomu neublížil a nedopustil se ani násilí, ani podvodu, pokud prostě neuposlechl svévolných nařízení svých pánů, \emph{zaslouží si} být napaden, zavřen do klece nebo zabit. A takoví \enquote{konzervativci} samozřejmě považují za neodpustitelné, pokud se některý z cílů takového fašistického násilnictví rozhodne bránit. Ze zvráceného, bludného pohledu zbožného nacionalistického autoritáře je ušlechtilé a ctnostné, když státní žoldáci násilně přepadnou a pokusí se unést a zavřít do klece produktivního, mírumilovného kuřáka trávy, ale odporně zlé, když tento kuřák použije násilí, aby se proti takové agresi bránil. Takové je šílenství způsobené pověrou autority.

\section{Nucené benefity}

Etatisté často obhajují \enquote{zdanění} argumentem, že násilná konfiskace bohatství \enquote{státem} se stává zpětně oprávněnou, když je část zkonfiskovaných peněz vynaložena způsobem, který prospívá tomu, komu byly peníze odebrány, nebo alespoň prospívá společnosti obecně. Etatista může například tvrdit, že pokud někdo jezdí po silnici, která byla zčásti financována z peněz, jež byly této osobě odebrány, nebo nepřímo těží z toho, že ostatní mohou tuto silnici používat, pak by si tato osoba neměla stěžovat na to, že byla \enquote{zdaněna,} aby ji mohla financovat. Etatisté ignorují skutečnou povahu situace a nesprávně ji označují za pouhé placení za služby. Nikdo by však podobně neargumentoval, kdyby se nejednalo o \enquote{autoritu.} Předpokládejme například, že restaurace doručí jídlo někomu, kdo si ho neobjednal, a pak na něj pošle ozbrojené násilníky, aby od něj vybrali sto dolarů. Pokud by se tato osoba poté, co byla takto vydírána, rozhodla jídlo sníst, žádný rozumný člověk by netvrdil, že by to činilo jednání restaurace morálně přijatelným. Přesto je to přesně analogické obvyklému názoru etatistů: pokud někdo využívá \enquote{státních} služeb, neměl by si stěžovat na \enquote{daně.} Nevysloveným předpokladem je, že \enquote{legální} loupež je naprosto legitimní, pokud \enquote{autorita} poté poskytuje nějaký prospěch tomu, kdo byl okraden. A zdá se, že etatistům je celkem jedno, zda je takový \enquote{prospěch} pouze nepřímý, nebo je strašlivě drahý, nebo je kombinován s nejrůznějšími dalšími věcmi, které dotyčnému vůbec neprospívají, nebo se jim dotyčný morálně brání (např. financování války, potratů nebo nějaké náboženské či protináboženské agendy). Je tomu tak proto, že etatisté věří, že v konečném důsledku je výsadou těch, kdo mají \enquote{autoritu,} nikoli těch, kdo peníze vydělali, rozhodovat o tom, jak má být bohatství utraceno, a že dokud vládnoucí třída tvrdí, že okrádá a ovládá lidi pro jejich vlastní dobro, nemají rolníci právo vzdorovat jakémukoli nátlaku a násilí, které páni považují za nezbytné.

\section{Útok na obranu}

Odnoží představy, že \enquote{stát} poskytující \enquote{výhody} zpětně ospravedlňuje krádeže a vydírání, je zjevně směšný argument, že je nutné, aby lidé byli násilně ovládáni a okrádáni, aby je \enquote{stát} mohl \emph{ochránit} před zlými lidmi, kteří by je jinak mohli násilně ovládat a okrádat. Tato absurdní, překroucená racionalizace je zcela běžná, ať už se diskuse týká autoritářské armády nebo domácích \enquote{orgánů činných v trestním řízení.} A etatisté se spoléhají na vyvolávání strachu, aby takové šílenství podpořili, a vyslovují hrozivé předpovědi o všech nepříjemných věcech, které by podle jejich teorie nastaly, kdyby lidé nebyli \emph{násilně} okrádáni prostřednictvím masivního autoritářského vydírání.

Opět platí, že takové hloupé argumenty se nikdy nepoužívají v situacích, kdy se nejedná o \enquote{autoritu.} Nikdo by nepřijal tvrzení, že je v pořádku, když restaurace nutí někoho zaplatit za jídlo, které si neobjednal, s odůvodněním, že jinak by dotyčný mohl umřít hlady. Nikdo by nepřijal tvrzení, že je v pořádku, aby stavitel nutil někoho zaplatit za stavbu, kterou si neobjednal, s odůvodněním, že jinak by dotyčný mohl zůstat bez domova. Ještě směšnější by však bylo tvrdit, že je v pořádku, když jeden pouliční gang provozuje \enquote{ochranářské} kšefty, aby měl prostředky na to, aby všechny ostatní nebezpečné pouliční gangy udržel mimo své město. Přitom právě to je pokus o ospravedlnění veškerého \enquote{státu:} musí mít dovoleno páchat agresi proti všem, aby je mohl \emph{ochránit} před ostatními, kteří by mohli páchat agresi proti nim. Zastánci silných policejních složek nebo početné armády -- obojí financováno z násilně zabaveného bohatství -- přijali předpoklad, že je nejen v pořádku, ale dokonce nezbytné, aby byli lidé \enquote{státem} utlačováni, ovládáni a vydíráni, pokud se tak děje pro jejich vlastní dobro. A skutečnost, že autoritářští \enquote{ochránci} nejenže nedokážou zabránit zločinu nebo válce, ale obojí dramaticky \emph{zvyšují} prostřednictvím válečného štvaní a vytváření \enquote{nelegálních} trhů, zřejmě zůstává bez povšimnutí těch, kdo obhajují obranu prostřednictvím \enquote{státu.} Znovu opakuji, že někdo vůbec může přijít s nesmyslným argumentem, že je správné zahájit násilí proti lidem za účelem jejich \enquote{ochrany,} jen proto, že si \enquote{autorita} představuje, že má právo páchat agresi.

\section{Násilí jako standard}

Mnohdy lidé dokonce obhajují násilně zavedený autoritářský plán jen proto, že si nejsou jisti, co by se stalo, kdyby jej neobhajovali, nebo si nejsou jisti, jak by se něčeho dosáhlo, kdyby se lidem ponechala svoboda. Například pokud si někdo těžko dokáže představit, jak by fungoval zcela soukromý silniční systém, bude obvykle obhajovat \enquote{státní} plán, financovaný z donucení. Pokud si není jistý, jak dobře by se svobodní lidé mohli bránit bez stálé armády, bude pravděpodobně obhajovat autoritářské vojenské řešení, financované násilnými \enquote{daněmi.} Ti, kdo věří ve stát, standardně obhajují násilí. Stačí trocha nejistoty a neznalosti, aby průměrný člověk začal obhajovat násilný \enquote{státní} plán téměř na cokoli.

Takto se lidé ve svém každodenním životě nechovají. Průměrný člověk nechodí a neiniciuje násilí proti každému, koho potká, protože si není jistý, že se jinak každý, koho potká, bude chovat správně a činit správná rozhodnutí. Ale právě to dělá většina etatistů prostřednictvím \enquote{státu:} prosazují rozsáhlé násilné ovládání milionů lidských bytostí jen proto, že si nejsou zcela jisti, že lidé, pokud by jim byla ponechána svoboda, by utráceli své peníze tak, jak by měli, chovali se k ostatním tak, jak by měli, nacházeli mírumilovná a účinná řešení problémů atd. Prostřednictvím pověry o autoritě mohou etatisté pohodlně obhajovat násilné podmanění svých bližních jen proto, že si nejsou zcela jisti, jak by se jejich bližní jinak chovali.

A ti, kdo touží po moci, této skutečnosti využívají ve svůj prospěch. Jediné, co politik musí udělat, aby získal podporu pro autoritářské uchopení moci, je sdělit veřejnosti, že by věci nemusely fungovat příliš dobře, kdyby nechal lidi na svobodě. Nemusí ani čekat, až někdo skutečně provede něco nečestného, zlomyslného, nedbalého nebo jinak destruktivního. Stačí, když naznačí možnost, že pokud budou lidé ponecháni na svobodě, \emph{mohou se stát špatné věci}. Protože zastánci \enquote{státního} násilí neuznávají \enquote{zákon} jako násilí, je práh, při kterém podpoří autoritářské, vynucené \enquote{řešení,} velmi nízký. Ti, kdo touží po moci, mohou jednoduše navrhnout, že nějaký \enquote{plán} by mohl někomu někde pomoci, a mnoho lidí bude schvalovat \enquote{legální} násilí jen na základě tohoto předpokladu.

Mnoho \enquote{státního} násilí je založeno na odhadech toho, co by lidé \emph{mohli} udělat a co by se v důsledku toho \emph{mohlo} stát. Například velká část státního nátlaku prováděného ve jménu \enquote{environmentalismu} je založena na myšlence, že stát musí násilím ovládat volby každého člověka, protože jinak by lidé mohli učinit rozhodnutí, která by přispěla ke globálnímu oteplování, zániku deštných pralesů, vyhynutí zvířat atd. Jen málo lidí, jednajících na vlastní pěst, by se dopustilo agrese na základě domněnky o možných nepřímých důsledcích nezáměrného a nenásilného jednání druhých. Přesto je to ve \enquote{státní} politice běžné.

Jako další příklad obhajoby \enquote{státního} násilí jako standardního principu uveďme praxi násilného bránění cizincům vkročit kamkoli do celé \enquote{země} bez písemného souhlasu vládnoucí třídy této \enquote{země.} Takové imigrační \enquote{zákony} vytvářejí něco podobného válečné mentalitě, kdy je celá demografická kategorie lidí kriminalizována a démonizována a podrobena agresi na základě obav z toho, co by \emph{někteří} z těchto lidí \emph{mohli} udělat.

Lidé tvrdí, že mnozí \enquote{ilegálové} jsou zločinci nebo přicházejí do země jen proto, aby získali \enquote{dávky.} Bez ohledu na to, jak často jsou taková tvrzení přesná, výsledkem je, že všichni \enquote{ilegálové} -- kdokoli, kdo je v zemi bez povolení politiků -- jsou násilně ovládáni. To je výsledek smečkové mentality skupinové viny. Mělo by být samozřejmé, že použití násilí proti jednomu člověku jen proto, že je stejné rasy nebo ze stejné země nebo je nějakým jiným způsobem podobný někomu jinému, kdo skutečně způsobil škodu, je naprosto neoprávněné. Za zmínku stojí, že \enquote{státní} pokusy potlačit \enquote{nelegální imigraci} vedou také k agresi páchané na mnoha \enquote{legálních} obyvatelích (stejně jako na \enquote{nelegálech}) na kontrolních stanovištích \enquote{pohraniční hlídky,} z nichž mnohá ani nejsou na hranicích. Zastavovat a vyslýchat každého, kdo jede po silnici, protože tam někdo \emph{možná} je \enquote{nelegálně,} je přesně ten druh neoprávněné agrese, kterého se běžně dopouštějí \enquote{státní} agenti a zřídkakdy kdokoli jiný.

Toto násilí se projevuje také při vtíravých prohlídkách a výsleších každého, kdo se pokusí letět letadlem v \enquote{zemi svobody.} To, že majitel letadla klade podmínky každému, kdo chce letět jeho letadlem (a to by platilo i pro vlak, auto nebo cokoli jiného), se velmi liší od toho, když třetí strana násilím zabrání komukoli v jízdě jakýmkoli letadlem kdekoli v celé zemi, pokud se potenciální cestující nejprve nepodrobí výslechu, prohlídce svých zavazadel, a dokonce i osobní prohlídce ze strany agentů třetí strany. Lidé by nikdy nestrpěli, aby se takto choval jakýkoli soukromník (s přístupem \enquote{raději budu všem ostatním vnucovat svou vůli, pro jistotu}), ale pro agenty \enquote{autority} je tato taktika běžná. A lidé si ji představují jako legitimní. Ve skutečnosti často \emph{požadují}, aby \enquote{autorita} takové věci dělala.

V každodenním životě je nenásilí pro většinu lidí \enquote{standardním} typem chování. I když občas dochází k fyzickým konfliktům, většina lidí vynakládá velké úsilí, aby se jim vyhnula, a to nejen tím, že se snaží nezačínat rvačky, ale také tím, že se snaží napjaté situace zmírnit. I když k hádce dojde, obě strany obvykle nakonec odejdou. Miliardy lidí každý den nacházejí způsoby, jak pokojně koexistovat, i když mají výrazně odlišné názory, přesvědčení a postoje. Ale to je v jejich \emph{osobním} životě. Pokud jde o \enquote{politiku,} je násilí standard. Každý volič v té či oné míře usiluje o to, aby jeho názory a představy byly násilím vnuceny všem ostatním, a to prostřednictvím mechanismu \enquote{státu.} Standardním principem není nechat ostatní \enquote{dělat si své věci} nebo se snažit vycházet v míru; Standardním principem je prosazovat agresi vůči naprosto všem, a to prostřednictvím autoritářského nátlaku zvaného \enquote{zákon.}

Mezi tím, co průměrný člověk považuje za \enquote{civilizované chování} na individuální bázi, a tím, co považuje za legitimní a civilizované, pokud jde o jednání \enquote{autority,} je ohromný rozpor. Těžko si lze představit, že by se někdo v osobním životě choval tak, jak se chovají voliči, když jde o \enquote{politiku.} Takový člověk by neustále okrádal ostatní -- přátele i cizí lidi -- o obrovské částky peněz na financování věcí, které považuje za důležité, a také by používal výhrůžky, fyzickou sílu, a dokonce i únosy, aby donutil ostatní učinit rozhodnutí, která by podle něj byla nejlepší, ať už pro jeho oběti, nebo pro společnost obecně. Stručně řečeno, každý, kdo by ve svém soukromém životě jednal tak, jak jednají \emph{všichni} etatisté na \enquote{politické} scéně, by byl okamžitě rozpoznán jako násilník, zloděj a šílenec. Ale dělat přesně totéž prostřednictvím \enquote{státu,} obhajovat masové vydírání a násilnictví, je většinou přijímáno jako něco, co by normální, civilizovaní lidé \emph{měli} dělat. Ve skutečnosti někdy označují hlasování za povinnost, jako by bylo vlastně nemorální \emph{neobhajovat} násilné ovládání svých bližních. Kupodivu a ironicky, \emph{jediní} lidé, kteří \emph{neobhajují} neustálé rozsáhlé násilí a donucování prostřednictvím \enquote{státu} -- známí jako voluntaristé nebo anarchisté -- jsou většinou považováni za podivíny, necivilizované a nebezpečné.

\section{Jak mýtus poráží ctnost}

Téměř všichni rodiče běžně vysílají svým dětem dvě zcela protichůdné zprávy: 1) krást, bít, šikanovat atd. je od přírody špatné a 2) je dobré poslouchat \enquote{autority.} Téměř všechno, co \enquote{autorita} dělá, je šikana: používá násilí nebo hrozbu násilí k ovládání chování druhých a k tomu, aby jim vzala jejich majetek. Každá \enquote{autorita,} od učitele ve škole až po diktátora země, nejenže pravidelně násilně ovládá své podřízené, ale také mluví a jedná, jako by na to měla absolutní a nezpochybnitelné právo. Učitel tedy vždy násilím vnucuje svou vůli žákům a zároveň jim říká, že není správné, aby \emph{oni} násilím vnucovali svou vůli ostatním. Je to dokonalý příklad pokryteckého poselství \enquote{Dělej, jak řeknu, ne jak dělám.}

Kdyby byly děti vychovávány v tom smyslu, že krást, bít, šikanovat atd. je ze své podstaty špatné, proč by bylo nutné, aby se ve společnosti učily také \enquote{úctě k autoritě?} Vychovává je to pouze k tomu, aby byly snáze manipulovatelné a ovladatelné, což je výhodné pro ty, kteří nad nimi chtějí vládnout (ať už jde o rodiče, učitele nebo politiky), ale nevychovává je to k větší civilizovanosti, soucitu nebo lidskosti. Působí přesně opačně, jak ukázaly Milgramovy experimenty. Stručně řečeno, děti se učí, jak být civilizovanými lidskými bytostmi, a pak se učí šílené pověře, která překonává a činí zastaralým vše, co je učili o civilizovanosti. Tento bizarní paradox můžeme v moderní společnosti pozorovat všude.

Průměrný člověk by cítil stud a vinu, kdyby ukradl sto dolarů svému sousedovi, ale bez skrupulí se zasazuje o to, aby \enquote{stát} prostřednictvím hlasování vzala témuž sousedovi mnoho tisíc dolarů. Průměrný člověk podrží dveře cizímu člověku, ale zároveň se bude zasazovat o to, aby stejný cizí člověk měl většinu svého života pod násilnou nadvládou prostřednictvím \enquote{zákona.} Povrchní zdvořilost a ohleduplnost, kterou většina lidí projevuje, ztrácí smysl a hodnotu v důsledku masivního státního nátlaku a agrese, které prosazují. Dokonce i nacisté se u stolu chovali slušně, říkali \enquote{prosím} a \enquote{děkuji} (německy), projevovali náležitou etiketu a byli obecně zdvořilí, pokud zrovna nepáchali masové vraždy.

Existuje dramatický kontrast mezi tím, jak se téměř všichni etatisté chovají k ostatním ve svém osobním životě, a tím, jak prosazují, aby \enquote{stát} zacházel s ostatními prostřednictvím \enquote{zákona.} Miliony lidí, kteří by se velmi zdráhali fyzicky udeřit jiného člověka, přesto hrdě schvalují násilné podmanění nebo přímo vraždu tisíců lidí. Říkají tomu \enquote{podpora armády.} Někteří etatisté dokonce tvrdí, že jsou proti válce, ale podporují armádu. To je srovnatelné s tvrzením, že někdo je proti znásilnění, ale podporuje násilníky. Protože \enquote{státní} armáda \emph{vždy} používá nátlak a násilí vůči nevinným lidem, a to navíc k jakékoli obranné síle, \enquote{podpora armády} nutně znamená podporu útlaku. Kvůli smečkové mentalitě a citovému poutu ke svým krajanům se však mnoho lidí snaží oddělit \enquote{armádu} od toho, co \enquote{armáda} dělá.

Dalším příkladem toho, jak víra v autoritu zkresluje vnímání, je skutečnost, že mnozí příjemci \enquote{sociálních dávek} otevřeně přiznávají, že mají-li na výběr mezi přijetím dobrovolně darovaných darů od lidí, které znají, a přijetím něčeho, co \enquote{stát} násilím vzal úplně cizímu člověku, dávají přednost té druhé možnosti, protože je podle nich \emph{méně} hanebná. Skutečnost, že by někdo někdy dal přednost přijetí ukradeného majetku před přijetím soucitu a štědrosti, ukazuje, jak hluboce víra v autoritu deformuje lidský smysl pro morálku.

Stručně řečeno, každý etatista -- každý, kdo věří ve stát -- si namlouvá, že je dobrým člověkem, který podporuje dobré věci a staví se proti nespravedlnosti, halucinuje v sobě úctu k bližnímu a zároveň obhajuje, aby byl jeho bližní násilím ovládán, vydírán, vězněn, nebo dokonce zabíjen. Pověra o autoritě se zaryla do mysli mas tak hluboko, že mohou obhajovat zlo na masové, téměř nepochopitelné úrovni, a přitom si ještě představovat, že jsou dobročinní a soucitní. Požadují, aby \enquote{stát} dělal věci, které by je samotné ani nenapadlo dělat. Představují si, že jsou nenásilné, civilizované a osvícené bytosti, a přitom běžně obhajují, aby všichni jejich bližní byli okradeni a násilně ovládáni a v případě odporu zavřeni do klecí nebo zabiti. Ve skutečnosti je povrchní lidská dobročinnost, soucit a zdvořilost jen krutým vtipem ve srovnání s tím, co téměř každý udělá nebo o co požádá druhé ve jménu \enquote{autority.}

Mnoho rodičů a učitelů pravidelně opakuje snad nejzákladnější pravidlo lidskosti, někdy nazývané \enquote{zlaté pravidlo:} Chovej se k druhým tak, jak chceš, aby se chovali k tobě. Nikdo z učitelů a téměř nikdo z rodičů, kteří toto pravidlo hlásají, se jím však ve skutečnosti neřídí, protože připouštějí, aby \enquote{autorita} násilím ovládala své podřízené ve škole i mimo ni. \enquote{Zlaté pravidlo} je v podstatě receptem na anarchii: pokud se někomu nelíbí, že je podmaňován a násilně ovládán, neměl by obhajovat, aby ostatní byli podmaňování a násilně ovládáni. Pokud chce být někdo ponechán v klidu, měl by nechat v klidu i ostatní. Pokud někdo touží po svobodě řídit svůj vlastní život, měl by nechat svobodu dělat totéž i ostatním. Řečeno na rovinu, obhajovat agresi vůči druhým, a to i prostřednictvím jakékoli formy \enquote{státu,} je naprosto neslučitelné s tím být dobročinným, ohleduplným, soucitným, laskavým, slušným a milujícím člověkem. A jediný důvod, proč tolik jinak dobrých lidí nadále obhajuje široce rozšířenou neustálou agresi prostřednictvím \enquote{státu,} je ten, že byli oklamáni a přijali lež, že existuje tvor zvaný \enquote{autorita,} který není vázán morálními normami platnými pro lidské bytosti.

\section{\enquote{Liberální} zbabělost}

Na rovinu, lidé chtějí, aby \enquote{autorita} existovala, protože jsou sami nevyzrálí zbabělci. Chtějí všemocnou entitu, která by ostatním vnucovala jejich vůli. To má různé podoby v různých odrůdách politického prosazování, ale základní motivace je vždy stejná. Například \enquote{liberál} má odpor k realitě. Nechce svět, v němž je možné utrpení a nespravedlnost. Ale místo toho, aby jako člověk dělal, co může, chce, aby to za něj udělal \enquote{stát.} Chce, aby nějaká magická entita zajistila, že všichni, včetně jeho samotného, budou mít co jíst, kde bydlet a o koho bude postaráno, bez ohledu na to, jak jsou líní nebo nezodpovědní. Místo aby věřil lidem, že se o sebe navzájem postarají, chce nadlidskou \enquote{autoritu,} která všem zaručí bydlení, jídlo, zdravotní péči a nejrůznější další věci. Chce to tak moc, že odmítá přijmout zjevnou pravdu, že žádná taková záruka není nikdy možná a že pokud se obyčejní smrtelníci nepostarají sami o sebe a o sebe navzájem, nepostará se o ně nic jiného.

Liberál vnímá svět jako pokračování školní třídy, kde je vždy nějaká \enquote{autorita,} která třídu vede a má ji na starosti, a která se postará o to, aby byly hodné děti odměněny a chráněny před zlobivými. Každému dítěti je řečeno, co má dělat, a je o něj postaráno, a jediné, co se od něj žádá, je, aby dělalo, co se mu řekne. Neočekává se od něj, že by neslo jakoukoli odpovědnost za své vlastní blaho, kromě poslušnosti vůči \enquote{autoritě.} Sám si nezajišťuje jídlo, přístřeší, ochranu ani nic jiného. Jednoduše věří, že se o něj \enquote{autority} (např. učitelé a rodiče) postarají. Je vychováváno v prostředí, které se vůbec nepodobá realitě, a je učeno, že všechny své potřeby musí hledat u \enquote{autorit.}

A přesně v tom liberál pokračuje i dlouho poté, co opustí školu. Mluví o tom, že každý člověk má \enquote{právo} na bydlení, jídlo, zdravotní péči a další věci, jako by nějaká obrovská zubní víla byla povinna zajistit, aby se takové věci zázračně objevily pro každého. Podstata reality, ačkoli mu denně hledí do tváře, je pro něj příliš znepokojující, než aby si ji připustil, protože se tolik liší od světa, v němž vyrůstal a kde za všechno odpovídala \enquote{autorita.} \enquote{Státní} programy, které podporují \enquote{liberálové,} jsou projevem jejich vlastní klamné hrůzy z reality a odmítání vidět svět takový, jaký je. Bojí se nejistoty natolik, že se snaží halucinovat existenci nadlidské entity (\enquote{státu}), který dokáže nějakým způsobem překonat všechny nejistoty reality a vytvořit vždy bezpečný, vždy předvídatelný svět. A když tento mytologický spasitel nejenže nedokáže svět napravit, ale všechno ještě mnohem zhorší (jako se to stalo v případě kolektivistických režimů v Sovětském svazu, na Kubě, v Číně a mnoha dalších), \enquote{liberál} se stále odmítá vzdát své slepé víry ve vševědoucího a všemocného boha zvaného \enquote{stát.}

Jednoduchá analogie způsobí, že se zhroutí celá \enquote{liberální} politická teorie. Kdyby stovka lidí ztroskotala na ostrově, co by vůbec znamenalo tvrdit, že každý z nich má \enquote{právo} na jídlo, nebo že každý má \enquote{právo} na zdravotní péči, nebo \enquote{právo} na práci, nebo \enquote{právo} na \enquote{životní minimum?} Pokud má například někdo \enquote{právo} na bydlení a bydlení vzniká pouze díky znalostem, dovednostem a úsilí jiných lidí, znamená to, že jeden člověk má právo \emph{vynutit si}, aby mu jiný člověk postavil dům. Přesně to se děje v širším kontextu, když \enquote{liberálové} prosazují, aby byli někteří lidé násilně okrádáni prostřednictvím \enquote{zdanění,} aby mohli poskytovat \enquote{výhody} jiným. Představa, že lidé mají z titulu své pouhé existence nárok na nejrůznější věci -- věci, které vznikají pouze jako výsledek lidského poznání a úsilí -- je bludná. Logickým důsledkem tohoto údajně láskyplného a soucitného pohledu je násilí a otroctví, protože pokud něčí \enquote{potřeba} opravňuje k něčemu, znamená to, že to musí být násilím odebráno komukoli jinému, kdo to má nebo může vyrobit, pokud to neposkytne dobrovolně.

Skutečnost, že se takový krátkozraký, zvířecí postoj (\enquote{kolektivismus}) vydává za \enquote{pokrokovou,} soucitnou filozofii, nic nemění na tom, že je ve skutečnosti k nerozeznání od \enquote{filozofie} krys a švábů: bez ohledu na to, kdo něco vyrobil, pokud to někdo jiný chce (nebo tvrdí, že to \enquote{potřebuje}), má si to násilím vzít. (Komunistický manifest to vyjadřuje slovy \enquote{od každého podle jeho schopností, každému podle jeho potřeb}). Je ovšem zásadní rozdíl mezi návrhem, že lidé, kteří mají bohatství nazbyt, by měli \emph{dobrovolně} pomáhat těm méně šťastným, a obhajobou toho, že by se mělo použít \emph{násilí}, aby se věci staly \enquote{spravedlivými.} \enquote{Státní} programy nikdy nejsou o tom, že by se lidé \emph{žádali}, aby si navzájem pomáhali; vždy se jedná o použití hrozeb a agrese k tomu, aby se lidé \emph{přinutili} dělat určité věci a chovat se určitým způsobem. Mýtus autority však umožňuje \enquote{liberálům} obhajovat rozsáhlé, neustálé násilí a zastrašování, a přitom si stále představovat, že jsou starostliví a soucitní. Političtí \enquote{levičáci} v podstatě chtějí vševědoucí, všemocnou \enquote{maminku,} která by \emph{přinutila} lidi, aby se dělili a hráli si na hodné, a ignorují skutečnost, že nic takového neexistuje a že představa něčeho takového jen přidává společnosti násilí, utrpení a bídu.

\section{\enquote{Konzervativní} zbabělost}

Stejně jako političtí \enquote{liberálové} chtějí obrovský stát jako matku, která by všechny chránila a starala se o ně, političtí \enquote{konzervativci} chtějí obrovský stát jako tátu, který by dělal totéž. Výsledky jsou trochu jiné, ale základní klam je stejný: touha po všemocné \enquote{autoritě,} která by lidstvo chránila před realitou. \enquote{Pravicový} blud se méně zaměřuje na mateřské hýčkání a držení za ruku a více se soustředí na otcovskou ochranu a disciplínu. \enquote{Konzervativci} chtějí, aby \enquote{autorita} sloužila k vytvoření velké, mocné ochranné mašinérie a k pevnému vnucení morálky obyvatelstvu, kterou si představují jako nezbytnou pro přežití lidstva. Jejich popírání reality je stejně silné jako u levičáků. Analogie s ostrovem to opět dobře demonstruje. Kdyby stovka lidí ztroskotala na ostrově, kdo by si představoval, že nutit většinu z nich sloužit a poslouchat \enquote{ochránce} by bylo nutné nebo užitečné? A kdo by si představoval, že když necháme jednoho nebo dva z nich násilím vnucovat svou morálku ostatním, bude taková skupina ctnostnější?

Konzervativní \enquote{tatínkovská} forma \enquote{státu} je ekvivalentem disciplinovaného otce, který působí jako ochránce rodiny před vnějšími silami (ekvivalent \enquote{státní} armády) a ochránce každého člena rodiny před ostatními v rodině (ekvivalent vnitrostátního \enquote{vymáhání práva}) a ten, kdo drží \enquote{nežádoucí osoby} mimo rodinu (ekvivalent imigračních \enquote{zákonů}), a také vymahatel morálky, který trestá členy rodiny, kteří pravidla nedodržují. Tento poslední bod se rovná \enquote{zákonům} proti pornografii, prostituci, hazardním hrám, užívání drog a dalším návykům a chováním, které sice vůči nikomu nevyžadují násilí ani podvod, ale mnozí je považují za destruktivní -- fyzicky, morálně nebo duševně -- pro ty, kdo se jich dopouštějí.

Snaha násilně vnutit morálku je však škodlivější než samotné chování. Kromě toho, že nikdo nemá právo násilím ovládat nenásilné volby druhého, je také strašně nebezpečné vytvořit precedens, že je v pořádku používat násilí k potlačení nevhodného nebo nevkusného chování. Jakmile bude takový předpoklad v zásadě přijat, lidská společnost se stane neustálou válkou všech proti všem. Nikdy nenastane doba, kdy by všichni sdíleli stejné hodnoty a názory. Mír a svoboda nemohou existovat, pokud každý rozdíl v názorech a každý rozdíl v životním stylu nebo chování povede k násilnému konfliktu prostřednictvím \enquote{státního} donucení. Civilizace, stav mírového soužití, není výsledkem toho, že všichni věří ve stejné věci, ale toho, že se lidé dohodnou, že nebudou iniciovat násilí, a to ani vůči lidem, kteří \emph{nevěří} ve stejné věci. \enquote{Konzervativní} etatismus, stejně jako jeho \enquote{liberální} verze, zaručuje věčné spory a konflikty, protože se snaží přehlušit svobodnou vůli a individuální úsudek takzvanou morálkou vládnoucí třídy, jejíž první zásadou je vynucená konformita a stejnost. Násilí samozřejmě nemůže vytvořit ctnost, i když někdy vytváří poslušnost, takže všechny pokusy \enquote{autority} donutit lidi k morálce a ctnosti jsou odsouzeny k nezdaru a v konečném důsledku nepřinášejí nic jiného než zvýšení míry násilí a konfliktů ve společnosti.

\section{Opravdová tolerance}

Víra v autoritu je tak silná, že mnoho lidí automaticky spojuje jakýkoliv nesouhlas s tím, že chtějí, aby to \enquote{stát} učinil \enquote{nezákonné.} V soukromém životě by většinu lidí ani ve snu nenapadlo uchýlit se k násilí vůči každému člověku, který má nějaký zvyk nebo životní styl, jenž jim není příjemný.

Téměř každý běžně toleruje rozhodnutí a chování druhých, které neschvaluje. \enquote{Tolerovat} něco ovšem znamená pouze umožnit tomu existenci (tj. nesnažit se to násilím vymýtit); neznamená to, že to schvalujeme nebo že to akceptujeme. Skutečná tolerance umožňuje mírové soužití lidí s různými názory a systémy víry.

Je ironií, že \enquote{tolerance} je etatisty často používána jako záminka k netoleranci. Pokud se například zaměstnavatel rozhodne s někým neobchodovat na základě jeho rasy, náboženství, sexuální orientace nebo jiné obecné charakteristiky, někteří to nazývají \enquote{netolerancí} (což to není) a pak obhajují, aby \enquote{autorita} použila sílu \enquote{zákona} a donutila zaměstnavatele zaměstnat toho, koho si \enquote{autorita} myslí, že by měl. A to \emph{je} netolerance, protože se to rovná odmítnutí umožnit člověku, aby se sám rozhodl, s kým se bude stýkat a s kým bude obchodovat.

To je jen jeden z mnoha příkladů toho, jak víra v autoritu prohlubuje rozdíly a zavádí násilí tam, kde by k němu jinak nedošlo. Existuje několik nenásilných způsobů, jak mohou lidé odrazovat od chování, které neschvalují. Vezměme si příklad majitele podniku, který odmítá zaměstnávat černochy (což, jakkoli je to odporné, \emph{není} aktem agrese). Ti, kteří takovou politiku považují za urážlivou, by mohli bojkotovat jeho podnik nebo vystupovat proti jeho praktikám či přesvědčením. Místo toho je běžnou reakcí na takovou situaci to, že etatisté podávají petice těm, kdo mají \enquote{autoritu,} aby \emph{vnutili} údajně spravedlivé a osvícené rozhodnutí všem.

Totéž platí pro mnoho dalších společenských problémů. Boj o to, zda mají být manželství osob stejného pohlaví \enquote{legálně} uznána, nebo \enquote{zakázána,} není ničím jiným než soutěží v netoleranci z obou stran. Není oprávněné násilím bránit dvěma lidem, aby o sobě říkali, že jsou manželé, ani není oprávněné nutit kohokoli jiného, aby takový vztah uznal jako \enquote{manželství.} Představa, že všichni musí mít stejnou představu o tom, co je manželství (nebo cokoli jiného), je příznakem konformistického fašismu. Stejně tak se zákony o \enquote{obscénnosti} snaží násilně omezit to, co lidé mohou číst nebo sledovat. Zákony o \enquote{narkotikách,} stejně jako většina toho, co dělá Úřad pro kontrolu potravin a léčiv, představují pokusy násilně omezit látky, které mohou lidé požívat. \enquote{Zákony o minimální mzdě} se snaží násilně ovládat, na čem se mohou dva lidé dohodnout. \enquote{Antidiskriminační} zákony se pokoušejí přinutit lidi k dohodám a sdružením, které nechtějí uzavírat. \enquote{Zákony,} jako je \enquote{zákon o zdravotně postižených Američanech,} se ve jménu \enquote{spravedlnosti} pokoušejí násilím ovládat, jaké služby mohou lidé nabízet, například zavřít podnik, pokud si jeho majitel nemůže dovolit instalovat rampu pro vozíčkáře. Všechny takové \enquote{zákony,} všechny takové akty \enquote{autority} a \enquote{státu} jsou akty agrese, přesný opak tolerance. Je absurdní snažit se lidi nutit, aby byli milí, spravedliví nebo soucitní, a to nejen proto, že agrese je ze své podstaty špatná, ale také proto, že nikdy nebude existovat pouze \emph{jediná} představa o tom, co je milé, spravedlivé a soucitné. Miliony lidí, kteří neustále bojují o meč \enquote{autority} a každý z nich doufá, že všem ostatním násilím vnutí svůj pohled na \enquote{dobro,} byly přímou příčinou většiny násilí a útlaku v dějinách. Ačkoli se to může zdát neintuitivní, tento fakt je historicky nezpochybnitelný: většina zla spáchaného v dějinách pochází z pokusů využít \enquote{autoritu} k dosažení \emph{dobrých} věcí.

Například ústava Sovětského svazu popisovala \enquote{autoritu,} která měla zacházet se všemi stejně, bez ohledu na rasu nebo náboženství, povolání nebo pohlaví, a zachovávat individuální práva všech občanů v jejich ekonomickém, politickém a sociálním životě. Mezi \enquote{práva} vyjmenovaná v sovětské ústavě patřila mimo jiné svoboda slova a náboženského vyznání, právo na práci, právo na odpočinek a volný čas, právo na bydlení, právo na vzdělání, právo na zdravotní péči a právo občanů na péči ve stáří. Reálným výsledkem tohoto ušlechtile znějícího experimentu však byly neustálé násilné represe, pronásledování a zastrašování, ekonomické zotročování, násilné potlačování myšlenek a názorů, všeobecná chudoba a vyvraždění desítek milionů lidských bytostí, z nichž mnohé byly záměrně vyhladověny. Ústava Čínské lidové republiky je velmi podobná ústavě Sovětského svazu a podobné byly i její výsledky: rozsáhlé násilné represe, tyranie a masové vraždění. (Pokus čínských \enquote{autorit} použít státní násilí ke snížení populačního růstu měl obzvláště strašlivé a žalostné výsledky).

Tyrani vždy vyznávali ty nejušlechtilejší úmysly. Ale i dobré úmysly, pokud se k nim přidá víra v autoritu, vždy vedou k nemorálnímu násilí, někdy až v nepochopitelné míře. I bez všech historických příkladů by mělo být zřejmé, že snaha dosáhnout soucitu a spravedlnosti, lásky a ctnosti, spolupráce a bratrství cestou autoritářské \emph{agrese a násilí} je šílená a že \enquote{stát} jako nástroj násilné nadvlády ze své podstaty nikdy nemůže vést a nikdy nepovede ke spravedlnosti, míru a harmonii.

Stojí také za povšimnutí, že politická levice \emph{i pravice} jsou obě zamilované do konceptu \enquote{rovnosti,} přičemž politická pravice prosazuje \enquote{rovnost před zákonem} a levice prosazuje rovnost výsledků. Ale ani jedna z nich ve skutečnosti nechce skutečnou rovnost, protože obě \emph{vyjímají} vládnoucí třídu z této \enquote{rovnosti.} Skutečná rovnost vylučuje jakýkoli \enquote{stát,} protože vládce a poddaný si zjevně nikdy nemohou být rovni. Etatisté ve skutečnosti chtějí rovnost mezi otroky, ale obrovskou nerovnost mezi otroky a pány. To opět ukazuje, že \enquote{stát} považují za nadlidi, protože je nikdy nenapadne, když prosazují \enquote{rovnost pro všechny,} že by se rovnost měla týkat i politiků a policie.

\section{Velký, malý, levý, pravý, stále zlý}

Každý, kdo obhajuje \enquote{stát} v jakékoli podobě -- ať už liberální, konzervativní, minimální, nezávislý, komunistický, fašistický, konstitucionalistický nebo jakoukýkoli jiný -- věří, že představitelé \enquote{autority} by se měli ve velkém měřítku dopouštět činů, které by, kdyby je dělal kdokoli jiný, byly všeobecně uznávány jako nespravedlivé a nemorální. Všichni etatisté věří, že lidé, kteří tvoří \enquote{stát,} mají \emph{výjimku} ze základní lidské morálky a nejenže \emph{mohou} dělat věci, na které ostatní nemají právo, ale měli by a \emph{musí} takové věci dělat pro (domnělé) dobro společnosti. Typ a míra agrese se liší, ale \emph{všichni} etatisté agresi obhajují.

V etatistické mytologii jsou politická \enquote{levice} a politická \enquote{pravice} protiklady. Ve skutečnosti jsou to dvě strany téže mince. Rozdíl spočívá pouze v tom, v co různí voliči doufají, že ti, kdo jsou u moci, s touto mocí naloží. V praxi se však všichni \enquote{levicoví} i \enquote{pravicoví} politici věnují přerozdělování bohatství, válečnému štvaní, centralizovanému řízení obchodu a četným násilným omezením chování svých poddaných. Jak se \enquote{pravicové} a \enquote{levicové} státy blíží k úplné moci, stávají se od sebe naprosto nerozlišitelnými. Hitlerův údajně \enquote{krajně pravicový} režim a Stalinův údajně \enquote{krajně levicový} režim byly prakticky totožné. Ať už byl původní deklarovaný účel obou z nich jakýkoli, konečným výsledkem byla naprostá moc a nadvláda politiků a naprostá bezmoc a zotročení všech ostatních. Možnost volby mezi politickou \enquote{levicí} a politickou \enquote{pravicí} poskytuje lidem přesně tolik moci a svobody, jako když jim umožníte volit mezi smrtí oběšením a smrtí zastřelením. A přidání nezávislé třetí strany přidává pouze možnost smrti na elektrickém křesle. Dokud se lidé budou hádat jen o to, \emph{který} gang má všechny zotročit (známé také jako \enquote{demokracie}), lidé zůstanou zotročeni.

Je ironií, že etatisté všech politických směrů naříkají nad vlivem \enquote{lobbistů} a \enquote{prosebníků} na politiky, přičemž ignorují skutečnost, že \emph{každý} volič je prosebník a každý přispěvatel na kampaň je lobbista. Jakmile lidé přijmou předpoklad, že \enquote{stát} má právo násilně řídit společnost, je nevyhnutelná neustálá soutěž mezi skupinami, z nichž každá hází peníze a laskavosti politikům, aby se pokusila dosáhnout svého. Je hloupé obhajovat autoritářskou nadvládu (\enquote{stát}) jen proto, abychom si pak stěžovali na nevyhnutelný \emph{důsledek} autoritářské nadvlády: lidé se snaží koupit si vliv. Politici se dají koupit jen proto, že mají moc na prodej, a tu mají jen proto, že lidé státu věří. Státní moc bude vždy použita k tomu, aby sloužila agendě jednoho člověka na úkor druhého (jak jinak by se dalo použít donucení?), takže myšlenka \enquote{státní korupce} je zbytečná. Každý etatista \emph{chce} \enquote{stát,} aby násilím vnucoval jeho vůli ostatním, ale nazývá ji \enquote{korupcí,} pokud zvítězí něčí \emph{jiná} agenda. Pokrytectví je ohromující.

Stejně tak konzervativní vědátoři v rozhlase i jinde svatouškovsky peskují liberály za to, že obhajují nucené přerozdělování bohatství, zatímco tito vědátoři dělají přesně totéž za trochu jinými účely. Kritizovat sociální dávky a zároveň podporovat dotace pro podniky, kritizovat pokusy o uzákonění \enquote{férovosti} a zároveň podporovat \enquote{válku proti drogám} nebo kritizovat liberální plány na znárodnění průmyslu a zároveň podporovat obří, násilně financovanou \enquote{státní} armádu (což se rovná znárodnění ochranného průmyslu) svědčí o naprosté absenci filozofických zásad. Zároveň je stejně pokrytecké, když liberálové spravedlivě odsuzují \enquote{pravicové} válečné štvaní a zároveň podporují obří, vtíravý a krutý vyděračský systém (\enquote{daně}) nebo si stěžují na \enquote{netoleranci} \enquote{pravice} a zároveň obhajují nejrůznější autoritářské ovládání chování. Ve skutečnosti není mezi filozofickými zásadami jednoho etatisty a druhého žádný skutečný rozdíl, protože oba přijímají předpoklad, že vládnoucí třída s právem ovládat a okrádat obyvatelstvo je nezbytná a legitimní. Jediný spor poté už není principiální, ale jde prostě o debatu o tom, jak by se měla kořist rozdělovat a jaká rozhodnutí by měla být rolníkům vnucena. Neexistuje nic takového jako tolerantní liberál nebo tolerantní konzervativec, protože ani jeden z nich \emph{netoleruje}, aby lidé utráceli své vlastní peníze a řídili své vlastní životy.

Je pravda, že míra zla a druhy obhajované nemorální agrese se liší v závislosti na různých stylech etatismu. Například konstitucionalisté obhajují relativně nízkou míru loupeže a vydírání (\enquote{zdanění}) a prosazují, aby byly prostřednictvím hrozeb a nátlaku ovládány pouze určité, omezené činnosti a chování (\enquote{regulace}). Ale každá pravomoc, kterou jakákoli ústava předstírá, že přiznává nějakému zákonodárci, je pravomocí, kterou \emph{nemá} pouhý smrtelník. Kdo by se obtěžoval psát do ústavy řádek předstírající, že deleguje \emph{určitým} lidem právo, které již mají všichni ostatní? Všechna taková \enquote{udělení moci} a jakýkoli dokument, který se tváří, že vytváří \enquote{stát} nebo zmocňuje jakýkoli \enquote{zákonodárný sbor} k čemukoli, jsou pokusy o vydání licence k páchání zla. Jak by však mělo být zjevně samozřejmé, žádný člověk ani skupina lidí -- bez ohledu na to, jaké dokumenty vytvářejí nebo jaké rituály provádějí -- nemůže někomu jinému udělit morální povolení páchat zlo. A kladení domnělých \enquote{omezení} na takové povolení jej nečiní o nic rozumnějším či legitimnějším. Stručně řečeno, obhajovat \enquote{stát} znamená vždy obhajovat zlo.

Jak liberálové, tak konzervativci trvají na tom, že někdo musí \enquote{velet,} protože v takové realitě byli vychováni: jediné, co se od nich vyžadovalo, bylo, aby byli poslušní autoritám. Z této výchovy vyplývá, že nemají téměř žádnou představu o tom, co mají dělat, pokud jsou ponecháni sami sobě, pokud jim nikdo neříká, co mají dělat. Proto odmítají dospět a snaží se halucinovat do existence nadlidské \enquote{autority.} Paradoxní je, že ačkoli nad lidmi není žádný pozemský druh, snaží se tuto nadlidskou entitu vyrobit z ničeho jiného než z lidí a pak se jí snaží propůjčit nadlidské vlastnosti, práva a ctnosti.

Celá tato představa je bludná, ale sdílí ji naprostá většina lidí na celém světě, kteří odmítají přijmout skutečnost, že neexistuje žádná zkratka k určení dobra a zla, že neexistuje žádný kouzelný trik, díky němuž by automaticky zvítězila pravda a spravedlnost, že neexistuje žádný \enquote{systém,} který by mohl zaručit bezpečnost nebo spravedlnost, a že každodenní smrtelné lidské bytosti se všemi svými nedostatky a vadami jsou tou nejlepší a jedinou nadějí pro civilizaci. Neexistuje žádná zoubková víla, Ježíšek ani magická entita zvaná \enquote{stát,} která by dokázala přimět nemorální živočišný druh k morálnímu chování nebo přimět skupinu nedokonalých lidí k dokonalému fungování. A víra v takovou entitu, místo aby byla pouze zbytečná a neúčinná, drasticky \emph{zvyšuje} celkový konflikt, nespravedlnost, netoleranci, násilí, útlak a vraždění v lidské společnosti. Nicméně většina lidí indoktrinovaných k uctívání \enquote{státu} raději lpí na svých známých, strašlivě destruktivních, odporně zlých a hluboce antihumánních pověrách, než aby dospěla a přijala fakt, že nad nimi nikdo není, že neexistuje žádná obrovská maminka nebo tatínek, kteří by je zachránili, že oni jsou na vrcholu a že každý z nich je osobně zodpovědný za to, že se rozhodne, co má dělat, a pak to udělá. Je smutné, že raději budou trpět peklem věčné války a totálního zotročení, než aby čelili nejistotě a odpovědnosti, která přichází se svobodou.

Víra v autoritu popírá a přehlušuje téměř všechny pozitivní účinky náboženského a morálního přesvědčení. To, co většina lidí nazývá svým \enquote{náboženstvím,} je prázdná výkladní skříň a to, co většina lidí uvádí jako svou morální ctnost, je irelevantní, pokud věří v mýtus autority. Křesťané se například učí věci jako \enquote{Když tě někdo uhodí, nastav druhou tvář,} \enquote{Miluj svého bližního} (a dokonce \enquote{Miluj svého nepřítele}) a \enquote{Čiň druhým tak, jak chceš, aby oni činili tobě.} Každý takzvaný křesťan, který věří ve stát, však tyto zásady neustále opouští a prostřednictvím kultu \enquote{státu} obhajuje neustálou agresi vůči všem -- příteli i nepříteli, sousedovi i cizinci. Dělat ze sebe zbožného, nábožného, soucitného, milujícího a ctnostného člověka a přitom \enquote{volit} partu, která slibuje, že bude násilím ovládat jednání všech, které znáte, je vrcholem pokrytectví. Zdržet se osobního okrádání bližního a zároveň prosazovat, aby to dělal někdo \emph{jiný}, je zbabělé a pokrytecké. Přesto téměř každý křesťan (a každý příslušník jiného náboženství) takové věci běžně dělá, a to formou \enquote{politické} propagace.

Jak už bylo řečeno, víra ve stát je čistě náboženská víra. Naprostá většina těch, kteří nosí nálepku \enquote{ateista,} tak ve skutečnosti ateisty nejsou, protože věří v boha zvaného \enquote{stát.} Jako náboženskou víru ji samozřejmě neuznávají, ale jejich víra v onoho nadpozemského, nadlidského zachránce lidstva (\enquote{autoritu}) je stejně hluboká a založená na víře jako jakákoli jiná náboženská víra. Je ironií, že ateisté často rychle poukazují na zkázu, která byla v dějinách spáchána ve jménu náboženství, ale nevšímají si hrůzných výsledků boha, \emph{kterému se klaní}: \enquote{státu.} Ateisté naprosto správně poukazují na to, že v době, kdy církve byly uznávanou \enquote{autoritou} -- organizacemi, o nichž se myslelo, že mají právo násilně ovládat ostatní -- se mnohé z nich dopouštěly rozsáhlých, ohavných teroristických činů, mučení a vražd. Většina moderních ateistů si však navzdory jasným důkazům, které jim hledí do tváře, neuvědomuje, že jsou členy nejničivější církve v dějinách, církve \enquote{státu,} které se podařilo způsobit spoušť, smrt a zkázu na daleko vyšší úrovni, než jaké se dopouštěly i ty nejkrutější církve v minulosti. Například během dvou set let bylo v náboženských válkách známých jako \enquote{křížové výpravy} zabito kolem jednoho až dvou milionů lidí. Pro srovnání, za polovinu této doby ve dvacátém století bylo \enquote{pokrokovou politikou} kolektivistických \enquote{státu} zabito více než \emph{stokrát} více lidí. Velkou roli v nárůstu počtu mrtvých nepochybně sehrál technologický pokrok, ale podstatné je, že ať už masku \enquote{autority} nosí církev nebo stát, pověra je strašlivě nebezpečná a její důsledky strašlivě ničivé. Skutečnost, že tolik ateistů horlivě odsuzuje jednu formu pověry, zatímco ji vehementně \emph{obhajuje} v jiné formě, svědčí o úžasné míře selektivní slepoty. Často ti, kdo nejvíce kritizují útlak prostřednictvím \enquote{náboženství,} jsou jedni z nejoddanějších \enquote{pravých věřících} v boha zvaného \enquote{stát.}

\section{Žádný objektivní standard}

V očích těch, kteří věří ve stát, je opět velký rozdíl mezi přijatelným chováním jednotlivce a přijatelným chováním \enquote{státu.} Když jednotlivec ukradne 100 dolarů, je to považováno za nemorální zločin; když ti ve \enquote{státu} ukradnou \emph{triliony} dolarů ročně, je to považováno za přijatelné. Pokud si průměrný jednotlivec vytiskne vlastní stodolarovku, vyjde ven a utratí ji, je to považováno za podvod a padělání -- nemorální čin podobný krádeži. Když \enquote{stát} dá \enquote{legální} povolení Federálnímu rezervnímu systému, aby dělal totéž, ale s \emph{triliony} fiatních \enquote{dolarů} z čistého vzduchu je to považováno za přijatelné, dokonce za užitečné a nezbytné. Zatímco různé \enquote{státy} prohlásily, že běžný člověk \enquote{nesmí} vlastnit střelné zbraně, žoldáci \enquote{státu} mají povoleno mít zbraně, bomby, stíhačky, tanky, rakety, dokonce i jaderné hlavice.

Ironií je, že tyto zbraně -- s výjimkou jaderných zbraní -- běžně dostávají do rukou ti samí lidé, kteří \emph{předtím}, než se stali žoldáky státu, měli držení střelných zbraní zakázáno. Jinými slovy, když tito jedinci použijí \emph{vlastní} úsudek, někteří politici je prohlásí za příliš nedůvěryhodné a příliš nebezpečné pro společnost, než aby jim mohli svěřit pětiranný revolver. Když však ti samí lidé slepě plní rozkazy, poslouchají řetězec velení, ti samí politici prohlašují, že jim lze svěřit útočné pušky, odstřelovací pušky, granáty, těžké kulomety, tanky, stíhačky, bombardéry, těžké dělostřelectvo a nespočet dalších nástrojů rozsáhlého ničení.

Kromě obrovské propasti mezi tím, co masy vnímají jako přijatelné chování jednotlivců, a přijatelným chováním \enquote{státu} se zdá, že veřejný smysl pro to, kdy \enquote{stát} zašel \enquote{příliš daleko,} je téměř náhodný. Měřítka, podle nichž jsou průměrní jednotlivci posuzováni, jsou jednoduchá a neměnná: pokud kradou, podvádějí, přepadávají nebo vraždí, je to špatné. Ale měřítko správného a špatného pro \enquote{stát} se zdá být do značné míry libovolné. Například se dnes všeobecně uznává, že \enquote{postavit mimo zákon} alkohol by bylo neoprávněné, ale \enquote{postavit mimo zákon} marihuanu -- a používat k prosazení tohoto zákazu rozsáhlé a neustálé násilí -- je legitimní. Ještě bizarnějším rozporem je, že většina lidí by se oprávněně pohoršovala, kdyby se \enquote{stát} pokoušel donutit každého, aby sbíral odpadky ve svém vlastním sousedství, ale většina lidí přijímá jako legitimní, když \enquote{stát} prostřednictvím \enquote{branné povinnosti} nutí lidi, aby šli do jiné země buď zabíjet lidi, nebo zemřít. Bizarní je, že tento nejodpornější příklad nucených prací -- nutit lidi, aby šli na druhý konec světa vraždit úplně cizí lidi -- spáchal dokonce \enquote{stát,} jehož vlastní pravidla (tj. třináctý dodatek) zakazují \enquote{nevolnictví.}

Je zřejmé, že hranice toho, co smí \enquote{stát} dělat, pokud jde o širokou veřejnost, nejsou založeny na žádném principu. Jedním z důvodů, proč se lidé na celém světě a v celé historii tak pomalu brání tyranii, je to, že dokud je růst tyranie pomalý a stálý, tyrani nejsou nikdy považováni za ty, kteří \enquote{překročili hranici.} Je tomu tak proto, že bez jakýchkoli základních principů, podle kterých by se dalo posuzovat, co je správné a co špatné, nelze žádnou hranici překročit. Víra v autoritu je zcela neslučitelná s \emph{jakýmikoli} morálními principy právě proto, že podstatou této víry je představa, že ti, kdo mají \enquote{autoritu,} nejsou vázáni stejnými pravidly chování jako jejich poddaní. Jak by z logiky věci mohlo být vůbec ospravedlnitelné, aby poddaní diktovali normy chování svým pánům? Pokud se \enquote{zdanění} (nucená konfiskace majetku) zvýší z 62 \% na 63 \%, jak by mohl kterýkoli etatista z principu prohlásit, že byla překročena jakákoli hranice nebo že \enquote{stát} překročil své meze? Nemůže existovat žádná principiální námitka proti loupeži, pokud to není námitka proti jakékoli míře loupeže, i když je \enquote{legální.} Je-li 1 \% násilné konfiskace majetku \enquote{státem} z principu legitimní, pak je legitimní i 99 \%. Buď vládci vlastní lid a mají právo si vzít, kolik se jim zlíbí, nebo lid vlastní sám sebe a vládci nemají právo mu cokoli násilně vzít. Nikde mezi tím nemůže být žádný princip. Jak by mohl existovat? Na jakém racionálním základě by bylo možné zastávat názor, že 46\% otroctví je dobré, ale 47\% otroctví je špatné? Jak by mohla existovat nějaká \emph{principiální} hranice kdekoli mezi 0 \% a 100 \%?

Když se násilí \enquote{státu} stane příliš rozšířeným, příliš svévolným a příliš krutým, začnou o něm pomalu pochybovat i oddaní etatisté. Neexistují však žádné skutečné zásady, jimiž by se řídili při posuzování spravedlnosti jednání vládnoucí třídy. Jakmile je jednou přijato, že jedna skupina lidí má přirozené právo páchat akty agrese proti ostatním, neexistuje žádný objektivní standard, jak takové právo omezit. Jestliže \enquote{stát} může vyžadovat, aby lidé měli \enquote{povolení} na jízdu autem do obchodu na rohu, proč by nemohl vyžadovat, aby lidé měli \enquote{povolení} na chůzi po ulici? Jestliže je legitimní, aby \enquote{zákonodárci} požadovali registraci a regulaci soukromých střelných zbraní, proč není legitimní také požadovat, aby byly registrovány a regulovány všechny formy projevu a vyjadřování? Pokud je v pořádku, aby politici vytvořili vynucený \enquote{státní} monopol na doručování dopisů (jako má Poštovní služba Spojených států), proč není v pořádku, aby vytvořili vynucený \enquote{státní} monopol na telefonní služby?

Důvodem, proč je \enquote{stát} vždy šikmá plocha, která se neustále vzdaluje od svobody a směřuje k totalitě, je to, že jakmile někdo přijme předpoklad vládnoucí třídy, neexistuje vůbec žádný objektivní základ pro uplatňování jakýchkoli omezení pravomocí této vládnoucí třídy. Nemůže existovat žádný racionální morální standard, který by říkal, že určitá osoba má právo páchat agresivní činy -- krádeže, zastrašování, napadání a nátlak -- ale že se takových činů smí dopouštět jen v určité míře nebo jen tehdy, když je to \enquote{nutné.} Připustit, že otroci jsou právoplatným majetkem někoho jiného, a pak tvrdit, že existují omezení toho, co s nimi jejich majitelé mohou dělat, je logický rozpor. Stejně tak to, že poddaný přijme jakéhokoli pána (včetně toho, který se nazývá \enquote{stát}), a pak si představuje, že on -- poddaný -- bude rozhodovat o rozsahu pánových pravomocí, odporuje logice a realitě. Přesto se o to všichni věřící v \enquote{zastupitelskou demokracii} snaží.

Stručně řečeno, ti, kdo věří v autoritu, přijali na té nejzákladnější úrovni, že jsou \emph{vlastněni} někým jiným: lidmi, kteří tvrdí, že mají \enquote{autoritu.} Poté, co tuto myšlenku přijali, pokračují v žebrání o laskavosti svých pánů. Tím však lidé neustále posilují myšlenku, že nakonec záleží na pánech, co se s poddanými bude dít. V celém \enquote{politickém procesu} se neustále ozývá jedno poselství: \enquote{Tady jsou věci, které my, lid, žádáme vás, vládce, abyste nám \emph{povolili} dělat.} Implicitním poselstvím, které je základem veškerého politického dění, je, že jediná moc, kterou lid má, je moc fňukat a prosit, a že nakonec vždy záleží na pánech, co se stane. Prosazovat jakoukoli změnu \enquote{zákona} znamená akceptovat, že \enquote{zákon} je legitimní.

Naproti tomu, kdyby ozbrojeného řidiče přepadl zloděj s nožem v ruce, necítil by řidič potřebu lobbovat u agresora a prosit ho, aby mu dal \emph{povolení}, aby si řidič mohl ponechat své vlastní auto. Pokud by řidič měl prostředky k násilnému odražení útočníka, měl by na to plné právo. \emph{Prosit} o něco znamená smířit se s tím, že rozhodnutí je na druhé osobě. Požádat ty, kdo jsou ve \enquote{státu,} o trochu více svobody, znamená připustit, že je na \emph{nich}, zda lidé mohou být svobodní, nebo ne. Jinými slovy, žádat o svobodu znamená nebýt svobodný, ale přijmout své podřízení někomu jinému. Uvažte, jakým oxymóronem je, když člověk tvrdí, že má \enquote{nezadatelné právo} něco dělat, a pak žádá politiky o legislativní \emph{povolení} k tomu, aby mohl tuto věc dělat. Víra v autoritu nakonec vede k tomu, že i ti, kdo se představují jako horliví zastánci svobody, \emph{sami sobě podsouvají} vlastní podřízenost. Bez ohledu na to, jak hlasitě \enquote{požadují,} aby politici změnili nějaký \enquote{zákon,} ti, kdo tvrdí, že milují svobodu, a přitom stále trpí pověrou o autoritě, pouze posilují legitimitu nadvlády vládnoucí třídy nad nimi tím, že implicitně souhlasí s tím, že lidé potřebují \enquote{legislativní} \emph{povolení} vládnoucí třídy, aby měli právo něco dělat.

\section{Účinky na zastánce svobody}

\enquote{Stát} sám o sobě neškodí, protože je to fiktivní entita. Ale \emph{víra} ve \enquote{stát} -- představa, že někteří lidé mají morální právo vládnout ostatním -- způsobila nezměrnou bolest a utrpení, nespravedlnost a útlak, zotročení a smrt. Zásadní problém nespočívá v žádném souboru budov, v žádné skupině politiků ani v žádné bandě vojáků či vymahatelů. Základní problém není organizace, kterou lze vyloučit, svrhnout nebo \enquote{reformovat.} Základním problémem je samotná víra -- blud, pověra a mýtus autority -- která sídlí v myslích několika miliard lidských bytostí, včetně těch, které kvůli této víře trpěly nejvíce. Je ironií, že víra v autoritu dramaticky ovlivňuje vnímání a jednání i těch, kteří proti danému režimu aktivně bojují. Tato pověra drasticky mění a omezuje způsoby, jakými odpůrci \enquote{bojují} proti útlaku, a činí téměř veškeré jejich úsilí bezmocným. Navíc při vzácných příležitostech, kdy je určitý tyran svržen, je jedna forma útlaku téměř vždy nahrazena jinou -- často ještě horší než ta předchozí.

Namísto boje proti neexistující bestii by \enquote{bojovníci za svobodu} měli uznat, že není skutečná, že neexistuje, že nemůže existovat, a pak podle toho jednat. Samozřejmě, pokud tuto pověru překoná jen několik lidí, budou pravděpodobně zesměšněni, odsouzeni, napadeni, uvězněni nebo zavražděni těmi, kteří v mýtus stále pevně věří. Když však i významná menšina lidí pověru překoná a změní podle toho své chování, svět se drasticky změní. Až lidé budou skutečně \emph{chtít} skutečnou svobodu, dosáhnou jí, aniž by k tomu potřebovali nějaké volby nebo revoluci.

Problém je v tom, že téměř nikdo si ve skutečnosti nepřeje, aby lidstvo bylo svobodné, a téměř nikdo se zásadně nestaví proti útlaku. Působení mýtu autority zůstává nedotčeno i v myslích většiny lidí, kteří se považují za rebely, nonkonformisty a volnomyšlenkáře. V období dospívání prochází mnoho lidí obdobím zdánlivého rebelství, které spočívá většinou v tom, že dělají vše, co jim \enquote{autority} nařizují, že \emph{nemají} dělat: kouří, nabízí sexuální služby, užívají drogy, nosí jiné oblečení nebo účesy, nechávají si dělat tetování nebo piercingy a podobně. Jejich jednání je tak stále řízeno, i když obráceně, mýtem autority. Místo aby poslouchali kvůli poslušnosti, neposlouchají kvůli neposlušnosti, ale stále nejeví známky toho, že by byli schopni samostatně myslet. Místo spokojených dětí se chovají jako vzteklé děti, ale stále se nechovají jako dospělí. A ve většině případů jejich přirozená touha přetrhnout okovy \enquote{autority} netrvá dlouho, \enquote{vyrostou} ze svých antiautoritářských sklonů a postupně se opět promění ve \enquote{vzorné občany,} tj. poslušné poddané.

Například údajně radikální antiautoritářští hippies z 60. let se s nástupem Billa Clintona do prezidentského úřadu ve Spojených státech víceméně \emph{stali} novou \enquote{vládou.} Dokonce i \enquote{pacifisté,} jejichž mantrou bylo \enquote{žít a nechat žít,} když dostali příležitost stát se novou \enquote{autoritou,} se rozhodli násilně zasahovat do životů ostatních stejně nebo více než jejich předchůdci, a to i prostřednictvím vojenských výbojů. Stejně tak ti z \enquote{Generace X,} davu \enquote{MTV} a podobně vždy soustředili své úsilí na to, aby se k moci dostali lidé, kteří s nimi souhlasí, místo aby se snažili o skutečné dosažení svobody. Je zásadní rozdíl mezi tím, když si stěžujete na určitou vládnoucí třídu, a tím, když z principu rozpoznáte šílenství \enquote{autority} a postavíte se proti němu. Stručně řečeno, ve všech různých společenských projevech takzvané vzpurnosti a nekonformity se téměř žádný z nich ve skutečnosti nevyhnul mýtu autority. Místo toho se pouze pokusily vytvořit novou \enquote{autoritu,} novou vládnoucí třídu, nový \enquote{stát,} nový centralizovaný donucovací stroj, jehož prostřednictvím by si mohly násilím podmanit a ovládat své bližní. Stručně řečeno, téměř všichni takzvaní \enquote{rebelové} jsou falešní, kteří předstírají, že vzdorují \enquote{pánovi,} ale ve skutečnosti chtějí jen \emph{být} \enquote{pánem.}

A to se dalo očekávat. Pokud vycházíme z předpokladu, že by měla a musí existovat \enquote{autorita} a že \enquote{stát} vládnoucí nad obyvatelstvem je legitimní situace, proč by někdo nechtěl být tím, kdo vládne? Každý člověk z definice chce, aby svět byl takový, jaký si myslí, že by měl být, a jakým lepším způsobem by toho mohl každý člověk dosáhnout než tím, že se stane králem? Pokud někdo přijme názor, že autoritářská moc je oprávněná, proč by ji nechtěl využít k tomu, aby se pokusil vytvořit svět takový, jaký chce, aby byl? Proto jediní lidé, kteří skutečně obhajují svobodu v principu, jsou anarchisté a voluntaristé -- lidé, kteří chápou, že násilné ovládání druhých není legitimní, a to ani tehdy, když se nazývá \enquote{zákonem,} a dokonce ani tehdy, když se děje ve jménu \enquote{lidu} nebo \enquote{obecného dobra.} Je velký rozdíl mezi snahou o nového, moudřejšího a ušlechtilejšího pána a snahou o svět rovných, kde nejsou žádní páni a žádní otroci.

Stejně tak je velký rozdíl mezi otrokem, který věří v princip svobody, a otrokem, jehož konečným cílem je stát se novým pánem. A to i v případě, že tento otrok má skutečně v úmyslu být laskavým a velkorysým pánem. Dokonce i ti, kdo obhajují relativně omezený, dobrotivý typ \enquote{státu,} se staví proti svobodě. Dokud budou lidé věřit v mýtus autority, bude po každém pádu jednoho tyrana následovat vznik a růst tyrana nového.

Historie je plná příkladů, jako byli Fidel Castro a Che Guevera, kteří se vydávali za \enquote{bojovníky za svobodu} jen tak dlouho, než se stali novými utlačovateli. Nepochybně byli zcela upřímní ve svém zuřivém odporu proti útlaku, kterým oni a jejich přátelé trpěli, ale v zásadě nebyli proti autoritářskému útlaku, jak jasně ukázalo jejich chování, jakmile sami získali moc. Tento vzorec se v dějinách opakoval stále znovu a znovu, přičemž odpor k jednomu tyranskému režimu se stal zárodkem dalšího tyranského režimu. Dokonce i Hitlerův nástup k moci byl z velké části způsoben hněvem na vnímanou nespravedlnost a útlak, který byl Německu způsoben Versailleskou smlouvou. Samozřejmě, dokud budou vzbouřenci trpět povědomím o autoritě, bude jejich první prioritou, jakmile svrhnou jednu \enquote{vládu,} nastolit novou. Takže i činy velké odvahy a hrdinství u těch, kteří stále věří ve stát, nepřinesly o mnoho víc než nahrazení jednoho tyrana jiným. Mnozí dokázali rozpoznat konkrétní tyranské činy konkrétních režimů a postavit se proti nim, ale jen málokdo si uvědomil, že základním problémem není to, kdo sedí na trůně; problémem je, že na trůně vůbec někdo sedí.

Stejné selhání v rozpoznání skutečného problému se objevuje i v případě přízemnějších, relativně mírumilovných \enquote{reforem.} Například ve Spojených státech je velká část obyvatelstva dokonale schopna vidět nespravedlnosti vyplývající z \enquote{války proti drogám,} globálního válečného štvaní a dalšího porušování občanských práv ze strany republikánských tyranů. Protože však neuznávají víru v autoritu jako skutečný problém, řešení, které navrhují ti, kdo takové nespravedlnosti uznávají, spočívá v tom, že místo toho předají otěže \enquote{státu} demokratickým tyranům. Mezitím je další velká část obyvatelstva dokonale schopna vidět nespravedlnosti vyplývající z vysokého \enquote{zdanění,} \enquote{státního} mikromanagementu průmyslu, systémů přerozdělování bohatství, odzbrojování občanů (\enquote{regulace zbraní}) atd. Protože však neuznávají víru v autoritu jako skutečný problém, řešení, které navrhují ti, kdo tyto nespravedlnosti uznávají, je vrátit otěže \enquote{státu} republikánským tyranům. A tak se dekádu za dekádou mění stroj útlaku, zatímco svoboda jednotlivce ve všech oblastech života se stále zmenšuje. A přesto jediné, o čem může většina Američanů vůbec uvažovat jako o řešení, jsou další volby, další politická strana nebo další lobbistické úsilí v naději, že se podaří vládnoucí třídu uprosit, aby byla moudřejší nebo benevolentnější.

Někteří lidé, kteří vidí katastrofu způsobenou systémem dvou stran, viní z negativních účinků \enquote{státu} \enquote{extremismus.} Domnívají se, že kdyby lidé podpořili jen takovou formu násilné nadvlády, která by byla někde uprostřed mezi \enquote{krajní levicí} a \enquote{krajní pravicí,} věci by se zlepšily. Takoví lidé o sobě tvrdí, že jsou nezávislí, otevření a umírnění, ale ve skutečnosti jsou pouze obecnými zastánci útlaku, místo aby byli zastánci určité příchuti útlaku. \enquote{Levice} a \enquote{pravice} jsou pouze dvě masky, které nosí jedna vládnoucí třída, a vytvoření nové masky, která by byla kompromisem mezi oběma, nebude mít na povahu bestie ani na zkázu, kterou způsobuje, žádný vliv. Zaujmout pozici na půli cesty mezi \enquote{levicovou} a \enquote{pravicovou} tyranií nevede ke svobodě, ale k dvoustranické tyranii.

Mezi těmi, kdo volí demokraty nebo republikány -- nebo jakoukoli jinou stranu -- nikdo nerozpozná základní problém, a v důsledku toho se nikdo nikdy nepřiblíží k řešení. Zůstávají otroky, protože jejich myšlenky a diskuse se omezují na nesmyslnou otázku, \emph{kdo} by měl být jejich pánem. Nikdy neuvažují -- a ani se neodvažují uvažovat -- o možnosti, že by neměli mít vůbec žádného pána. V důsledku toho se soustřeďují výhradně na politické akce toho či onoho druhu, které jsou všechny založeny na víře v autoritu, jež je sama problémem. Úsilí etatistů tedy je a vždy bude odsouzeno k nezdaru. To platí i pro méně mainstreamová, údajně svobodě více nakloněná \enquote{politická hnutí,} včetně konstitucionalistů, libertariánské strany a dalších. Dokud budou myslet a jednat v mezích hry na \enquote{stát,} jejich úsilí nejenže je zcela neschopné problém vyřešit, ale ve skutečnosti jej ještě zhoršuje, neboť nechtěně legitimizuje systém nadvlády a podmanění, který nese označení \enquote{stát.}

\section{Pravidla hry}

Dokonce i většina lidí, kteří tvrdí, že milují svobodu a věří v \enquote{nezcizitelná} práva, dovoluje pověře autority, aby drasticky omezila jejich účinnost. Většina toho, co tito lidé tak či onak dělají, spočívá v tom, že \emph{žádají} tyrany, aby změnili své \enquote{zákony.} Ať už aktivisté vedou kampaň za konkrétního kandidáta, nebo proti němu, nebo lobbují za konkrétní \enquote{zákon,} nebo proti němu, pouze posilují předpoklad, že poslušnost vůči autoritě je morálním imperativem.

Když se aktivisté snaží přesvědčit politiky, aby snížili \enquote{daně} nebo zrušili nějaký \enquote{zákon,} tito aktivisté nepřímo přiznávají, že ke svobodě potřebují svolení svých pánů. A člověk, který \enquote{kandiduje na úřad} a slibuje, že bude bojovat za lid, tím také naznačuje, že je na těch, kteří řídí \enquote{stát,} aby rozhodli, co bude rolníkům dovoleno. Jak to vyjádřil Daniel Webster: \enquote{\emph{Ve všech dobách jsou lidé, kteří chtějí dobře vládnout, ale chtějí vládnout; slibují, že budou dobrými pány, ale chtějí být pány.}} Aktivisté vynakládají obrovské množství času, peněz a úsilí na to, aby prosili své pány, aby změnili své příkazy. Mnozí z nich dokonce vystupují z cesty, aby zdůraznili skutečnost, že \enquote{pracují v rámci systému} a že neprosazují nic \enquote{nezákonného.} To svědčí o tom, že bez ohledu na svou nespokojenost s těmi, kdo jsou u moci, stále věří v mýtus autority a budou spolupracovat s \enquote{legálním} bezprávím, dokud se jim nepodaří přesvědčit pány, aby změnili pravidla -- aby \enquote{legalizovali} spravedlnost. I když zamýšleným poselstvím nespokojenců může být, že nesouhlasí s tím, co páni dělají, skutečným poselstvím, které veškeré politické akce vysílají těm, kdo jsou u moci, je: \enquote{\emph{Přejeme si, abyste změnili své příkazy, ale my budeme nadále poslouchat, ať už to uděláte, nebo ne.}} Pravdou je, že ten, kdo se snaží dosáhnout svobody tím, že žádá ty, kdo jsou u moci, aby mu ji dali, již neuspěl, a to bez ohledu na odpověď. Prosit o požehnání \enquote{autority} znamená smířit se s tím, že volba je pouze na \emph{vládci}, což znamená, že dotyčný je již z definice otrokem.

Ten, kdo prosí o nižší \enquote{daně,} nepřímo souhlasí s tím, že záleží na politicích, kolik si člověk může ponechat z toho, co vydělal. Ten, kdo prosí politiky, aby ho neodzbrojovali (prostřednictvím \enquote{regulace zbraní}), tím připouští, že je na pánovi, zda nechá člověka ozbrojeného, nebo ne. Ve skutečnosti ti, kdo lobbují za to, aby politici respektovali jakákoli \enquote{nezcizitelná práva} lidí, v nezcizitelná práva vůbec nevěří. Práva, která vyžadují souhlas \enquote{státu,} nejsou nezcizitelná a nejsou ani právy. Jsou to privilegia, která jsou udělována nebo odepírána podle rozmaru pána. A ti, kdo zastávají mocenské pozice, vědí, že se nemusí bát lidí, kteří nedělají nic jiného, než že žalostně prosí o svobodu a spravedlnost. Jakkoli hlasitě nespokojenci mluví o tom, že se \enquote{dožadují} svých práv, ve skutečnosti vysílají toto poselství: \enquote{Souhlasíme, pane, že záleží na vás, co smíme a co nesmíme dělat.}

Toto základní poselství lze spatřovat v nejrůznějších aktivitách, které jsou mylně považovány za formy odporu. Lidé se například často zapojují do protestů před \enquote{státními} budovami, nosí nápisy, skandují hesla, někdy se dokonce dopouštějí násilí, aby vyjádřili svou nespokojenost s tím, co páni dělají. I takové \enquote{protesty} však většinou nedělají o moc víc, než že \emph{posilují} autoritářství. Pochody, stávky vsedě, protesty atd. mají za cíl vyslat pánům zprávu, jejímž cílem je přesvědčit pány, aby změnili své zlé způsoby. Toto poselství však stále předpokládá, že záleží na pánech, co smí lidé dělat, což se stává sebenaplňujícím se proroctvím: když se lidé cítí zavázáni \enquote{autoritě,} jsou zavázáni \enquote{autoritě.} Ti, kdo jsou ve \enquote{státu,} odvozují veškerou svou moc z toho, že si jejich poddaní \emph{představují}, že mají moc.

\section{Legitimizace útlaku}

Čím usilovněji se lidé snaží pracovat v rámci jakéhokoli politického systému, aby dosáhli svobody, tím více ve své mysli a v mysli všech, kdo se dívají, posilují, že \enquote{systém} je legitimní. Petice politikům za změnu jejich \enquote{zákonů} naznačují, že na těchto \enquote{zákonech} záleží a že by se měly dodržovat. Nic neukazuje sílu víry v autoritu lépe než podívaná, kdy sto milionů lidí prosí několik set politiků o nižší \enquote{daně.} Kdyby lidé skutečně chápali, že plody práce člověka jsou jeho vlastní, nikdy by se do takového šílenství nepouštěli; jednoduše by přestali odevzdávat svůj majetek politickým parazitům. Jejich vycvičená touha mít souhlas \enquote{autority} v nich vytváří myšlení podobné myšlení otroka: doslova se cítí špatně, když si nechávají své vlastní peníze a rozhodují se sami, aniž by k tomu nejprve dostali svolení od pána. I když jim svoboda náleží, etatisté se nadále plazí u nohou megalomanů a prosí o svobodu, čímž si zajišťují, že \emph{nikdy} nebudou svobodní.

Pravdou je, že člověk nemůže věřit v autoritu a být svobodný, protože přijmout mýtus státu znamená přijmout vlastní povinnost poslouchat pána, což znamená přijmout vlastní zotročení. Je smutné, že mnoho lidí věří, že prosit pána prostřednictvím \enquote{politické akce} je to jediné, co mohou dělat. A tak se věčně zapojují do rituálů, které vztah s otrokářem a pánem pouze legitimizují, místo aby se tyranům jednoduše nepodřídili. Myšlenka, že neposlouchají \enquote{autoritu,} \enquote{porušují zákon} a jsou \enquote{zločinci,} je pro ně znepokojivější než představa, že jsou otroky.

Ti, kdo chtějí podstatně nižší míru autoritářské nadvlády a donucování, jsou někdy obviňováni z toho, že jsou \enquote{protistátní,} což většina z nich vehementně popírá a tvrdí, že nejsou proti \enquote{státu} jako takovému, ale že chtějí pouze \emph{lepší} \enquote{stát.} Svými vlastními slovy však přiznávají, že \emph{nevěří} ve skutečnou svobodu, ale stále věří v božské právo politiků a v myšlenku, že vládnoucí třída může být dobrá a legitimní věc. Pouze někdo, kdo stále cítí trvalou povinnost poslouchat příkazy politiků, by se chtěl vyhnout označení \enquote{protistátní.} Protože \enquote{stát} \emph{vždy} spočívá v agresi a nadvládě, nelze být skutečně pro svobodu \emph{bez} toho, aby člověk byl proti \enquote{státu.} Skutečnost, že tolik aktivistů odmítá tuto nálepku (\enquote{protistátní}), ukazuje, jak hluboce zakořeněná zůstává pověra o autoritě, a to dokonce i v myslích těch, kteří se představují jako horliví zastánci individuální svobody.

(Zde stojí za zmínku jeden obzvlášť fascinující jev. Rozhořčeni autoritářským bezprávím, ale přesto neochotni vzdát se v sobě pověry o autoritě, mnozí členové sílícího hnutí za svobodu/militarismus/\enquote{vlastenectví} nadále hledají nebo tvrdí, že našli nějaký \enquote{právní} prostředek, který tyrany přesvědčí, aby je nechali na pokoji. V průběhu let se objevuje jedna teorie za druhou, které tvrdí, že existuje nějaký tajný \enquote{státní} formulář nebo nějaký \enquote{právní} trik či úřední postup, který může jednotlivce osvobodit od nadvlády \enquote{státu.} To bohužel dokazuje pouze to, že tito lidé stále nedělají nic jiného, než že hledají způsob, jak získat \emph{povolení} ke svobodě. Cestou ke skutečné svobodě však nikdy nebyl a nebude nový politický rituál, nový \enquote{právní} dokument či argument nebo jakákoli jiná forma \enquote{politického} jednání. Jedinou cestou ke skutečné svobodě je pro jednotlivce zbavit se vlastní připoutanosti k pověře autority).

\section{Libertariánský protimluv}

Snad nejlepším příkladem toho, jak víra v autoritu deformuje myšlení a brání dosažení svobody, je skutečnost, že existuje \enquote{libertariánská} politická strana. Srdcem a duší libertariánství je princip neútočení: myšlenka, že iniciovat sílu nebo podvádět druhé je vždy špatné a že síla je ospravedlnitelná pouze tehdy, je-li použita v \emph{obraně} proti agresi. Tento princip je naprosto správný, ale snaha o jeho uskutečnění prostřednictvím jakéhokoli \emph{politického} procesu je naprosto rozporuplná, protože \enquote{stát} a neagrese jsou naprosto neslučitelné. Kdyby organizace zvaná \enquote{stát} přestala používat jakékoli hrozby nebo násilí, s výjimkou obrany proti agresorům, přestala by být \enquote{státem.} Neměla by právo vládnout, neměla by právo \enquote{zdaňovat,} neměla by právo \enquote{vydávat zákony,} neměla by monopol na ochranu a neměla by právo dělat nic, na co nemá právo kterákoli jiná lidská bytost.

Jednou z výmluv pro libertariánský politický aktivismus je tvrzení, že společnost se může transformovat ze svého současného autoritářského uspořádání ve skutečně svobodnou společnost pouze tehdy, pokud se tak stane pomalu a postupně. To se však nikdy nestalo a nikdy nestane, a to z velmi prostého důvodu: buď něco takového jako \enquote{autorita} existuje, nebo neexistuje. Buď existuje legitimní vládnoucí třída s právem vládnout všem, nebo každý jednotlivec vlastní sám sebe a je zavázán pouze svému svědomí. Obě paradigmata se navzájem vylučují. Je nemožné, aby existoval nějaký mezičlánek, protože kdykoli dojde ke konfliktu mezi tím, co přikazuje \enquote{autorita,} a tím, co diktuje individuální úsudek člověka, je nemožné podřídit se obojímu. Jedno musí mít přednost před druhým. Pokud \enquote{autorita} převáží nad svědomím, pak jsou všichni obyčejní lidé právoplatným majetkem vládnoucí třídy, a v takovém případě svoboda nemůže a neměla by existovat. Pokud je naopak svědomí nadřazeno \enquote{autoritě,} pak každý člověk vlastní sám sebe a každý se musí vždy řídit svým vlastním úsudkem o tom, co je správné a co špatné, bez ohledu na to, co mu přikazuje jakákoli samozvaná \enquote{autorita} nebo \enquote{zákon.} Nemůže dojít k \enquote{postupnému posunu} mezi oběma, ani ke kompromisu.

Snaha o přeměnu libertariánství v politické hnutí vyžaduje znetvořený, zvrácený hybrid obou možností: myšlenku, že k dosažení individuální svobody lze použít systém nadvlády (\enquote{stát}). Kdykoli \enquote{libertarián} lobbuje za zákony nebo kandiduje na úřad, svým vlastním jednáním připouští, že \enquote{autorita} a člověkem vytvořené \enquote{právo} je legitimní. Kdyby však někdo skutečně věřil v princip neútočení, pochopil by, že příkazy politiků (\enquote{zákony}) nemohou tento princip přebít a jakýkoli \enquote{zákon,} který je v rozporu s tímto principem, je nelegitimní. To platí i pro myšlenku \enquote{nezcizitelných práv.} Pokud má jednotlivec přirozené právo něco dělat, pak k tomu z definice nepotřebuje žádné povolení od tyranů. Nepotřebuje lobbovat za změnu \enquote{legislativy} a nemusí se snažit zvolit nějakého pána, který se rozhodne jeho práva respektovat.

Každý, kdo skutečně věří v princip neútočení -- základní premisu libertariánství -- musí být anarchistou, protože je logicky nemožné vystupovat proti zahájení násilí a zároveň podporovat jakoukoli formu \enquote{státu,} která není ničím jiným než násilím. A libertariáni nemohou být konstitucionalisty, protože ústava zcela jasně (v článku I, oddílu 8) tvrdí, že některým lidem uděluje právo iniciovat násilí, mimo jiné prostřednictvím \enquote{zdanění} a \enquote{regulace.} Princip libertariánství logicky vylučuje \emph{veškerý} \enquote{stát,} dokonce i ústavní republiku. (Každý, kdo se pokusí popsat \enquote{stát,} který se nedopouští žádných agresivních činů, popíše v nejlepším případě soukromou bezpečnostní agenturu). Nicméně tolik lidí bylo tak důkladně vyškoleno v autoritářském myšlení, že i když vidí zjevnou morální nadřazenost života podle principu neútočení (základ libertarianismu), stále se odmítají vzdát absurdní představy, že právo vládnout (\enquote{autorita}) může být použito jako nástroj svobody a spravedlnosti. Je zásadní rozdíl mezi dohadováním o tom, co by měl pán dělat -- v čemž spočívá veškerá \enquote{politika} -- a prohlášením, že pán nemá právo vládnout vůbec. Být libertariánským kandidátem znamená snažit se o obě tyto protichůdné věci. Je zřejmé, že legitimizuje funkci, kterou se kandidát snaží zastávat, a to i přesto, že kandidát prohlašuje, že věří v principy neútočení a sebevlastnictví, které zcela \emph{vylučují} možnost jakékoli legitimní \enquote{veřejné funkce.}

Stručně řečeno, pokud je cílem svoboda jednotlivce, je \enquote{politická akce} nejen bezcenná, ale i nesmírně kontraproduktivní, protože to hlavní, čeho dosahuje, je legitimizace moci vládnoucí třídy. Jediný způsob, jak dosáhnout svobody, je nejprve dosáhnout mentální svobody, a to tak, že si uvědomíme, že nikdo nemá právo vládnout druhému, což znamená, že \enquote{stát} není nikdy legitimní, nikdy není morální, nikdy není ani skutečný. Ti, kdo si to dosud neuvědomili a nadále se snaží žádat \enquote{systém,} aby je učinil svobodnými, hrají tyranům do karet. Ani petice za nižší míru \enquote{zdanění} nebo \enquote{státních} výdajů nebo žádosti o \enquote{legalizaci} či \enquote{deregulaci} věcí nebo prosby o jiné omezení \enquote{státní} nadvlády nad lidmi stále nijak neřeší opravdový problém a ve skutečnosti k němu přispívají tím, že nevědomky opakují a posilují myšlenku, že pokud lidé chtějí svobodu, musí ji mít \enquote{legalizovanou.} Politická akce ze své podstaty vždy posiluje postavení vládnoucí třídy a zbavuje lid jeho práv.

Pokud si dostatečný počet lidí uvědomí a opustí mýtus o autoritě, není třeba žádných voleb, politických akcí ani revoluce. Kdyby si lidé nepředstavovali, že mají povinnost poslouchat politiky, politici by byli doslova ignorováni do bezvýznamnosti. Ve skutečnosti víra v \enquote{demokracii} dramaticky \emph{snižuje} schopnost lidí bránit se tyranii tím, že omezuje způsoby, jakými se jí brání. Kdyby například 49 \% obyvatelstva chtělo nižší úroveň \enquote{zdanění,} ale zachovalo si víru v autoritu, nemohlo by prostřednictvím \enquote{demokracie} dosáhnout přesně ničeho. Na druhou stranu, pokud by i 10 \% obyvatelstva nechtělo vůbec žádné \enquote{zdanění} a uniklo by mýtu o autoritě (včetně té \enquote{demokratické}), mohli by svého cíle snadno dosáhnout prostým nepodřízením se. Na příkladu USA, kdyby dvacet milionů lidí -- méně než 10 \% amerických \enquote{daňových poplatníků} -- otevřeně odmítlo spolupracovat s pokusy IRS o jejich vydírání, vládnoucí třída by byla bezmocná a nechvalně proslulý daňový úřad by se spolu s masivním vyděračským systémem, který spravuje, zastavil. Bylo by naprosto nemožné, aby 100 000 zaměstnanců IRS neustále okrádalo miliony Američanů, kteří necítí žádnou povinnost platit. Ve skutečnosti by bylo nemožné, aby jakýkoli úřad prosazoval jakýkoli \enquote{zákon,} který by bez pocitu studu nebo viny mohl nedodržovat byť jen zlomek veřejnosti. Pouhou hrubou silou by nebylo možné dosáhnout dodržování zákonů.

Jakoukoli velkou populaci lidí, která by sama o sobě nevnímala poslušnost jako ctnost a necítila žádnou přirozenou povinnost poslouchat příkazy těch, kdo si nárokují právo vládnout, by bylo naprosto nemožné utlačovat. K válkám dochází proto, že lidé cítí povinnost jít do boje, když jim to \enquote{autorita} nařídí. (Jak se říká: \enquote{Co kdyby uspořádali válku a nikdo by nepřišel?}) Dokud bude možné lidi obelhávat a neustále je prosit o \enquote{uzákonění} svobody, bude snadné si je podmanit a ovládat. Dokud bude vnímaná povinnost člověka poslouchat \enquote{autoritu} nadřazena jeho vlastnímu individuálnímu úsudku, jsou jeho přesvědčení a hodnoty prakticky vzato irelevantní. Dokud není zastánce svobody ochoten neuposlechnout pána -- \enquote{porušit zákon} -- je jeho údajná láska ke svobodě lží a ničeho nedosáhne.

\section{Stejný jako starý pán}

Mnozí tvrdí, že společnost bez vládců není možná, protože jakmile se jeden \enquote{stát} zhroutí nebo je svržen, okamžitě vznikne nový \enquote{stát.} V jistém smyslu je to pravda. Pokud se lidé budou i nadále držet mýtu autority, po jakémkoli převratu určitého režimu si jednoduše vytvoří novou sadu pánů, která nahradí tu starou. Důvodem však není ani nutnost \enquote{státu,} ani základní povaha člověka. Téměř všichni \enquote{bojovníci za svobodu,} kteří brojí proti tyranii a útlaku, si neuvědomují, že základním problémem nikdy nejsou konkrétní lidé u moci. Základní problém spočívá v myslích utlačovaných lidí, včetně myslí většiny \enquote{bojovníků za svobodu.} Dokud lidé přijímají mýtus o autoritě, ani otevřená revoluce z dlouhodobého hlediska nijak nepomůže omezit útlak. Když padne jedna skupina vládců a vykořisťovatelů, lidé jednoduše nastolí jinou. (Ačkoli si to možná málokdo z těch, kdo mávají vlajkami na \enquote{Den nezávislosti,} uvědomuje, míra útlaku za krále Jiřího III. těsně před americkou revolucí byla zanedbatelná ve srovnání se současnou mírou \enquote{zdanění,} \enquote{regulace} a dalších autoritářských zásahů, nátlaku a šikany, k nimž dnes v USA běžně dochází).

Pro lidi je snadné vidět konkrétní nespravedlnosti páchané ve jménu určitého režimu, ale pro tytéž lidi je mnohem obtížnější rozpoznat, že hlavní příčinou těchto nespravedlností je \emph{systém víry} široké veřejnosti. Historické knihy jsou plné příkladů dlouhých, krvavých vlád tyranů, po nichž nakonec následovala krvavá revoluce a po ní pomazání nového tyrana. Typ tyrana se může změnit -- monarchu nahradí komunistický režim, \enquote{pravicového} tyrana nahradí \enquote{levicový} tyran, utlačovatelskou teokracii nahradí utlačovatelský \enquote{populistický} režim a tak dále -- ale dokud zůstane víra v autoritu, zůstane i útlak.

I ty nejodpornější příklady lidské nelidskosti vůči člověku, spáchané ve jménu \enquote{autority,} jen zřídkakdy někoho přimějí zpochybnit myšlenku \enquote{autority} jako takové. Místo toho je to vede pouze k tomu, aby se postavili proti \emph{konkrétní} skupině tyranů. Jako odstrašující příklad lze uvést, že hlavní opozice vůči nacistům v Německu vycházela od komunistů, kteří sami prosazovali stejně krutou a ničivou formu útlaku jako Hitlerův režim. Kvůli svému autoritářskému smýšlení neměli Němci šanci dosáhnout míru nebo spravedlnosti, protože celá jejich národní debata se týkala pouze toho, \emph{který} druh všemocných vládců by měl být u moci, aniž by se byť jen náznakem uvažovalo o tom, že by takovou moc neměl mít \emph{nikdo}. Podobně probíhala veřejná diskuse po většinu světa a po většinu času, kdy se soustředila na to, \emph{kdo} by měl vládnout, namísto toho, aby se ptala, zda by vůbec měli existovat vládci.

\section{Směs moudrosti a šílenství}

Na konci 18. století se stalo něco velmi neobvyklého, něco, co jako by mohlo přerušit nekonečný cyklus sériových tyranů. Tou událostí byl podpis Deklarace nezávislosti. Neobvyklost této události nespočívala v tom, že se lidé vzbouřili proti tyranovi -- k tomu došlo nesčetněkrát předtím -- ale v tom, že povstalci vyjádřili některé základní filozofické principy, odmítli nejen konkrétní režim, ale odmítli útlak z principu. Téměř.

Americká revoluce byla výsledkem směsice protichůdných myšlenek, z nichž některé podporovaly suverenitu jednotlivce, jiné vládnoucí třídu. Deklarace nezávislosti a ústava, která následovala o několik let později, byly kombinací hlubokého porozumění a do očí bijících protimluvů. Pozitivní je, že tehdejší diskuse nebyla jen o tom, kdo bude vládnout, ale silně se soustředila na koncept individuálních práv a omezení moci \enquote{státu.} Deklarace nezávislosti zároveň mylně tvrdila, že \enquote{stát} může mít ve společnosti legitimní úlohu: chránit práva jednotlivců. To však v praxi nikdy neplatilo a nemůže to platit ani teoreticky. Jak bylo vysvětleno výše, organizace, která by nedělala nic jiného než hájila práva jednotlivců, by nebyla \enquote{státem} v žádném smyslu slova. Deklarace také hovořila o nezcizitelných právech a tvrdila, že \enquote{všichni lidé jsou stvořeni sobě rovni} (pokud jde o jejich práva). Autoři si však neuvědomili, že takové pojmy zcela vylučují jakoukoli možnost legitimní vládnoucí třídy, a to i velmi omezené. Samotné zásady, které vyjádřili, pak byly okamžitě popřeny jejich snahou vytvořit protektorský \enquote{stát.} Jeden den prohlašovali, že \enquote{\emph{všichni lidé jsou stvořeni sobě rovni}} (Deklarace nezávislosti), a druhý den prohlašovali, že někteří lidé, kteří si říkají \enquote{Kongres,} mají právo okrádat (\enquote{zdaňovat}) všechny ostatní (Ústava USA, čl. I, odst. 8, čl. 1). Kromě toho Deklarace tvrdí, že když se jakákoli \enquote{vláda} stane destruktivní pro práva jednotlivce -- což se stane vždy, jakmile vznikne -- lid má povinnost ji změnit nebo zrušit. Ústava však tvrdí, že Kongresu dává pravomoc \enquote{potlačovat vzpoury} (Ústava USA, čl. I, odst. 8, bod 15). To znamená, že lid má právo vzdorovat útlaku \enquote{vlády,} ale že \enquote{vláda} má právo jej násilně potlačit, když tak učiní. Stručně řečeno, díla \enquote{otců zakladatelů} se skládají z kombinace hluboké moudrosti a naprostého šílenství. Na některých místech docela dobře popsali koncept sebevlastnictví, jinde se snažili vytvořit vládnoucí třídu. Zdá se, že si nevšimli, že tyto dva záměry jsou navzájem naprosto neslučitelné.

Výsledkem jejich snažení byl v jistém smyslu obrovský neúspěch. Režim, který vytvořili, se rozrostl mnohem více, než federalisté i antifederalisté chtěli. Deklarace a ústava naprosto selhaly v tom, že nedokázaly udržet \enquote{státní} moc omezenou. Slib \enquote{státu,} který bude služebníkem lidu, bude chránit jeho práva, ale jinak ho nechá na pokoji, přerostl v největší a nejmocnější autoritářské impérium, jaké kdy svět poznal, včetně největšího a nejvlezlejšího vydírání, jaké kdy bylo známo, největší a nejmocnější válečné mašinérie v dějinách a nejvlezlejší a nejinvazivnější byrokracie v dějinách.

Ve skutečnosti byla tato myšlenka od počátku odsouzena k zániku. Snad nejcennější věcí, které \enquote{velký americký experiment} dosáhl, bylo prokázání, že \enquote{omezená vláda} je nemožná. Nemůže existovat pán, který se zodpovídá svým otrokům. Nemůže existovat pán, který slouží svým poddaným. Nemůže existovat vládce, který by byl nad lidmi a zároveň jim byl podřízen. Bohužel je stále mnoho těch, kteří se odmítají poučit z této lekce a místo toho trvají na tom, že neselhala Ústava, ale lid -- tím, že to neudělal správně, že nebyl dostatečně bdělý nebo nějakým jiným zanedbáním či zkažeností. Kupodivu je to stejná výmluva, jakou komunisté uvádějí, proč se \emph{jejich} chybná filozofie, je-li uvedena do praxe v reálném světě, vždy změní v násilný útlak. Pravdou je, že \emph{jakákoli} forma autoritářské nadvlády -- jakýkoli typ \enquote{státu,} ať už ústavní, demokratický, socialistický, fašistický nebo jakýkoli jiný -- vyústí v to, že skupina pánů násilně utlačuje skupinu otroků. To je to, co \enquote{autorita} je -- vše, čím kdy byla a čím kdy může být, bez ohledu na to, kolik vrstev eufemismů a líbivé rétoriky se používá ve snaze to zakrýt.

\section{Mýtus o smlouvě}

Mytologie kolem ústavy tvrdí, že sloužila jako jakási smlouva mezi lidem a jeho novými \enquote{služebníky} v Kongresu. Na tom však není ani špetka pravdy. Podpisem smlouvy nelze zavázat někoho \emph{jiného} k \enquote{dohodě.} Představa, že by několik desítek bílých, mužských a bohatých vlastníků půdy mohlo uzavřít dohodu jménem více než dvou milionů dalších lidí, je absurdní. Kromě toho žádná smlouva nemůže vytvořit právo, které nemá žádný z účastníků, což je to, co všechny \enquote{státní} ústavy předstírají. Forma dokumentu jasně ukazuje, že nešlo o skutečnou smlouvu, ale o pokus vyfabrikovat ze vzduchu právo vládnout, jakkoli \enquote{omezené} mělo být. Skutečná dohoda na základě smlouvy je zásadně odlišná věc od jakéhokoli dokumentu, který má údajně vytvářet \enquote{stát.} Kdyby například tisíc amerických kolonistů podepsalo smlouvu, v níž by stálo: \enquote{Souhlasíme s tím, že výměnou za ochranné služby od Ochranné společnosti George Washingtona odevzdáme desetinu toho, co vyprodukujeme,} mohli by být takovou smlouvou morálně vázáni. (Uzavření dohody a její porušení je formou krádeže, podobně jako když jdete do obchodu a vezmete si něco, aniž byste za to zaplatili.) Nemohli by však touto smlouvou zavázat nikoho jiného a nemohli by ani takovou smlouvu použít k tomu, aby dali \enquote{Ochranné společnosti George Washingtona} právo začít okrádat nebo jinak ovládat lidi, kteří se smlouvou nemají nic společného. Ústava také sice předstírá, že \enquote{Kongres} zmocňuje k různým věcem, ale ve skutečnosti ho k ničemu \emph{nezavazuje}. Kdo se zdravým rozumem by podepsal smlouvu, která druhou stranu k ničemu nezavazuje? (Ve věci \emph{DeShaney v. Winnebago}, 489 U.S. 189, Nejvyšší soud oficiálně prohlásil, že \enquote{stát} nemá žádnou skutečnou povinnost chránit veřejnost.) Výsledkem je, že ústava, místo aby byla brilantní, užitečnou a platnou smlouvou, byla šíleným pokusem hrstky lidí jednostranně podřídit miliony dalších lidí nadvládě agresivní mašinérie, a to výměnou za žádnou záruku ničeho. Skutečnost, že se k tomu miliony konstitucionalistů zoufale snaží vrátit v naději, že to může zachránit jejich \enquote{zemi,} pokud se o to lidé znovu pokusí -- poté, co to naprosto selhalo poprvé -- svědčí o síle a šílenství pověry o autoritě.

\chapter{Život bez pověry}

\section{Řešení}

Téměř každý vidí alespoň nějaké problémy se \enquote{státem,} ve kterém žije, ať už se jedná o korupci, válečné štvaní, socialistické přerozdělování, dotěrnost policejních složek nebo jiný útlak. A mnozí se zoufale snaží najít řešení těchto problémů. A tak volí toho či onoho kandidáta, podporují to či ono politické hnutí nebo stranu, lobbují za nebo proti tomu či onomu zákonu a téměř vždy skončí zklamáni výsledkem. Snadno identifikují různé problémy a stěžují si na ně, ale skutečné řešení jim vždy uniká.

Důvodem jejich zklamání je to, že problém nespočívá v lidech, kterým se říká \enquote{stát,} ale v myslích jejich obětí. Pohrávání si se \enquote{státem} nemůže vyřešit problém, který nepochází ze \enquote{státu.} Nespokojený volič si neuvědomuje, že právě jeho \emph{vlastní} pohled na realitu, jeho \emph{vlastní} víra v autoritu je hlavní příčinou většiny problémů společnosti. Věří, že vládnoucí třída je přirozenou, nezbytnou a prospěšnou součástí lidské společnosti, a proto se veškeré jeho úsilí soustředí na hádky o to, kdo by měl vládnout a k čemu by měla být moc \enquote{státu} využita. Když přemýšlí o \enquote{řešeních,} přemýšlí uvnitř škatulky etatismu. V důsledku toho je od počátku bezmocný. Prosit pány, aby byli hodní, nebo žádat o nového pána nikdy nevede ke svobodě. Naopak, takové chování je jasným ukazatelem toho, že dotyčný není svobodný ani uvnitř své vlastní mysli. A člověk, jehož mysl není svobodná, nebude nikdy svobodný ani tělesně.

Lidé jsou natolik zvyklí účastnit se kultovních rituálů souhrnně označovaných jako \enquote{politika} (hlasování, lobbování, petice, kampaně atd.), že jakýkoli návrh, aby se neobtěžovali účastnit se takových nesmyslných a impotentních snah, se v jejich očích rovná návrhu, aby \enquote{nedělali nic.} Protože považují hlasování, kňučení a žebrání za celou škálu možností, které se jim v souvislosti se \enquote{státem} otevírají, nejsou schopni ani pochopit nic, co by mohlo skutečně přinést svobodu. Když tedy voluntarista nebo anarchista vysvětlí problém i cestu z něj, ale nepředloží nového kandidáta, kterého by bylo třeba volit, novou politickou stranu, kterou by bylo třeba podpořit, nebo nějaké nové hnutí či kampaň, za kterou by bylo třeba se postavit -- jinými slovy, nenavrhne nic, co by se shodovalo s pověrou o státu a autoritě -- průměrný etatista si bude stěžovat, že mu nebylo nabídnuto žádné řešení. Z jejich pohledu každý, kdo nehraje \enquote{politickou hru} v rámci pravidel stanovených vládnoucí třídou, \enquote{nic nedělá.} Nadšeně prohlašují: \enquote{Musíte být aktivní!.} Neuvědomují si, že aktivita ve hře vytvořené a řízené tyrany \emph{je} \enquote{neděláním ničeho} -- přinejmenším ničeho užitečného.

Ve skutečnosti skutečné řešení -- jediné řešení problémů týkajících se \enquote{státu} -- spočívá spíše v tom, že se určité věci nebudou dělat a že se určité věci nebudou dít. V jistém smyslu neexistuje žádné pozitivní, aktivní řešení \enquote{státu.} Konečné řešení je negativní a pasivní:

\emph{Přestaňte obhajovat agresi vůči svým sousedům. Přestaňte se účastnit rituálů, které schvalují iniciaci násilí a posilují představu, že někteří lidé mají právo vládnout. Přestaňte myslet, mluvit a jednat způsobem, který posiluje mýtus, že normální lidé by měli být a musí být zavázáni nějakému pánovi a měli by ho poslouchat, místo aby se řídili vlastním svědomím.}

Až se lidé přestanou klanět před oltářem \enquote{státu,} přestanou hrát hry tyranů, přestanou respektovat svévolná pravidla napsaná megalomany, problém zmizí sám od sebe. Jelikož je \enquote{autorita} mýtickou entitou, není třeba ji svrhávat, volit nebo \enquote{reformovat.} Lidé si pouze musí přestat \emph{představovat} něco, co není a nikdy nebylo. Kdyby lidé přestali dovolovat, aby iracionální pověra pokřivila jejich vnímání, jejich \emph{jednání} by se okamžitě a dramaticky zlepšilo. Většina agresivity, která se nyní odehrává ve jménu \enquote{autority,} by ustala. Nikdo by nevydával příkazy, nevynucoval by je a necítil by povinnost příkazy plnit, pokud by samotné příkazy nebyly vnímány jako ze své podstaty oprávněné na základě situace, nikoliv na základě toho, kdo příkaz vydává, nebo jeho údajné \enquote{autority.} To samo o sobě by odstranilo naprostou většinu krádeží, vydírání, zastrašování, obtěžování, nátlaku, terorismu, napadání a vražd, kterých se dnes lidé dopouštějí jeden na druhém. Když lidé žádného pána neuznávají a nepřijímají, nebudou mít žádného pána. Nakonec jejich otroctví a prostředky, jak z něj uniknout, existují výhradně v jejich vlastní mysli.

Lidská společnost nepotřebuje k vyřešení většiny svých problémů nic přidávat, ani nepotřebuje instituci nějakého nového \enquote{systému} nebo realizaci nějakého nového generálního plánu. Místo toho potřebuje, aby byla ze společnosti \emph{odstraněna} jedna věc -- jedna všudypřítomná, nesmírně destruktivní věc: víra v autoritu a stát. To, co \enquote{zajistí, aby věci fungovaly,} není žádný centralizovaný plán, žádný autoritářský program, ale vzájemná dobrovolná interakce mnoha jednotlivců, z nichž každý slouží svým vlastním hodnotám a řídí se svým vlastním svědomím. To se samozřejmě vůbec neslučuje s tím, jak byli téměř všichni naučeni myslet: že společnost potřebuje generální plán s \enquote{vůdci,} kteří ho uskuteční. Ve skutečnosti společnost nejvíce potřebuje naprostou \emph{neexistenci} generálního plánu a naprostou \emph{neexistenci} autoritářských \enquote{vůdců,} kterým musí lidé odevzdat svou svobodnou vůli a úsudek. Řešením není přidat do společnosti nějakou novou věc, ale jednoduše pochopit a rozptýlit nejnebezpečnější pověry.

\section{Skutečnost je anarchie}

Mnoho lidí se stalo anarchisty -- zastánci dobrovolné společnosti bez vládnoucí třídy -- poté, co dospěli k závěru, že společnost by byla prosperující a mírumilovnější a že by se těšila větší spravedlnosti a bezpečnosti bez jakéhokoli \enquote{státu.} To je však něco podobného, jako kdyby se člověk po pečlivé analýze rozhodl, že Vánoce budou lépe fungovat bez Ježíška. Pokud Ježíšek není skutečný, je zbytečné vést debatu o tom, zda je \enquote{potřeba,} aby Vánoce \enquote{fungovaly.} Pokud Vánoce vůbec fungují, pak už fungují \emph{bez} Ježíška. A tak je to i s obvyklou debatou mezi \enquote{státem} a \enquote{anarchií.} \enquote{Stát} neexistuje. Nikdy neexistoval a nikdy existovat nebude, což lze dokázat pomocí logiky, která vůbec nezávisí na morálním přesvědčení jednotlivce.

Pro rychlý přehled: \emph{lidé nemohou delegovat práva, která nemají}, což znemožňuje, aby kdokoli získal právo vládnout (\enquote{autoritu}). Rovněž \emph{lidé nemohou měnit morálku}, což způsobuje, že \enquote{státní zákony} postrádají jakoukoli přirozenou \enquote{autoritu.} Ergo, \enquote{autorita} -- právo vládnout -- logicky nemůže existovat. Samotný pojem je vnitřně rozporný, podobně jako pojem \enquote{militantní pacifista.} Lidská bytost nemůže mít nadlidská práva, a proto nikdo nemůže mít přirozené právo vládnout. Člověk nemůže mít morální povinnost ignorovat svůj vlastní morální úsudek, a proto nikdo nemůže mít přirozenou povinnost poslouchat druhého. A právě tyto dvě složky -- právo vládce poroučet a povinnost subjektu poslouchat -- jsou jádrem a duší pojmu \enquote{autorita,} bez nichž nemůže existovat.

A bez \enquote{autority} neexistuje \enquote{stát.} Pokud je nadvláda, kterou gang zvaný \enquote{stát} vykonává nad ostatními, bez legitimity, není to \enquote{stát} a jeho příkazy nejsou \enquote{zákony.} Bez \emph{práva} vládnout a současné morální \emph{povinnosti} mas poslouchat není organizace zvaná \enquote{stát} ničím jiným než bandou násilníků, zlodějů a vrahů. \enquote{Stát} je nemožný; prostě nepřichází v úvahu, stejně jako nepřichází v úvahu Ježíšek. A nic na této skutečnosti nezmění trvání na tom, že je \enquote{nutný,} i když neexistuje a ani existovat nemůže, nebo předpovídání zkázy a zmaru, pokud tuto mýtickou entitu mít nebudeme. Tvrdit, že lidé \emph{potřebují} mít právoplatného vládce, který má morální právo násilně ovládat všechny ostatní a kterého jsou všichni ostatní povinni poslouchat, nic nemění na tom, že nic takového neexistuje a ani existovat nemůže.

Účelem této závěrečné kapitoly proto není pouze tvrdit, že společnost by fungovala lépe bez fikce zvané \enquote{stát,} ale představit čtenáři způsoby, jakými budou lidé jinak vnímat realitu, jinak myslet, jinak se chovat a jinak komunikovat -- a to velmi odlišně -- jakmile se vzdají nejnebezpečnější pověry: víry v autoritu. Anarchie, tedy absence \enquote{státu,} \emph{je to, co je}. Vždycky byla a vždycky bude. Až lidé tuto pravdu přijmou a přestanou halucinovat o bytosti zvané \enquote{autorita,} přestanou se chovat tak iracionálně a destruktivně jako nyní.

Téměř každý, alespoň na začátku, má problém o takovém pojmu jasně uvažovat. Protože každý politik a každý \enquote{stát} neustále navrhuje \enquote{řešení,} která se zabývají tím, jak bude společnost organizována, řízena a ovládána prostřednictvím centralizovaného, autoritářského \enquote{systému,} většina lidí ani neví, jak myšlenku naprosté \emph{neexistence} jakéhokoli násilně vnuceného \enquote{systému} mentálně zpracovat. Instinktivně se ptají na věci jako \enquote{Kdo by stavěl silnice?} nebo \enquote{Jak bychom se bránili?.} Pravdou je, že nikdo nemůže vědět, jak by vše fungovalo nebo co přesně by se stalo. Jednotlivci mohou navrhovat, jak by věci měly fungovat, nebo předpovídat, jak by mohly fungovat, ale nikdo nemůže vědět, jak by vše fungovalo nejlépe. Navzdory obrovské míře nejistoty, kterou to vytváří, jsou historické výsledky lidí žijících ve svobodě mnohem lepší než jakékoliv centralizované a řízené \enquote{řešení.}

Etatisté však byli vyškoleni, aby se děsili tohoto nekonečně složitějšího typu společnosti, kde neexistuje jeden generální plán, ale miliardy individuálních plánů, které se navzájem ovlivňují nesčetnými různými způsoby. To pro ně znamená chaos. A v jistém smyslu to chaos je, protože neexistuje jediná vůdčí myšlenka a jediná řídící entita. To však neznamená, že se lidé nemohou dohodnout, spolupracovat nebo hledat kompromisy. Naopak to znamená, že každý člověk se bude na život dívat jako dospělý, místo aby zahodil svou svobodnou vůli a odpovědnost a slepě následoval program někoho jiného.

Jen na okraj, i bez pověr o autoritě by stále existovali vůdci a následovníci. Obvykle by však šlo o skutečné vůdcovství, kdy by jeden člověk šel příkladem, projevoval by určitou úroveň inteligence, soucitu nebo odvahy, která by inspirovala ostatní k podobnému chování. To je zcela odlišný jev od toho, čemu se dnes obvykle říká \enquote{vůdcovství.} Když se mluví o \enquote{vůdcích} zemí, mluví se o lidech, kteří násilím ovládají miliony jiných. Výraz \enquote{vůdce svobodného světa} je nepřesný a rozporuplný sám o sobě, když se jím označuje \enquote{státní} zástupce. Politici nejdou příkladem. Když už, tak jdou příkladem toho, jak být nečestný, zákeřný, narcistický a mocichtivý. Říkají to, co lidé chtějí slyšet, aby si je mohli podmaňovat a ovládat je. Nazývat takové lidi \enquote{vůdci} je stejně směšné jako nazývat zloděje \enquote{výrobci} nebo vrahy \enquote{léčiteli.} Při absenci víry ve stát by se mohli objevit skuteční vůdci: lidé, kteří si nenárokují žádné právo vládnout, žádné právo nutit někoho jiného, aby je následoval, ale jejichž ctnosti a činy ostatní uznávají jako hodné následování.

Nikdo nemůže předvídat a nikdo nebude určovat, co se stane ve světě bez mýtu státu. Následující text proto nemá být úplným vysvětlením toho, jak by fungovala každá součást lidské společnosti, až by mýtus autority zmizel. Místo toho je úvodem k několika způsobům, jak by lidské bytosti mohly přestat dovolovat, aby iracionální pověra narušovala jejich myšlení a zvrátila jejich chování, a mohly se začít chovat jako racionální, svobodné bytosti, které se řídí vlastní svobodnou vůlí a individuálním úsudkem, jak by měly.

\section{Strach ze svobody}

Většina lidí žije svůj život obklopena autoritářskou hierarchií, od rodiny, přes školy, podniky až po všechny úrovně \enquote{státu.} V důsledku toho si většina lidí jen těžko dokáže představit civilizaci \enquote{bez vůdců,} společnost rovných, existenci bez vládců, svět bez \enquote{zákonodárců} a jejich \enquote{zákonů.} Už jen tato myšlenka vyvolává v myslích většiny lidí představu chaosu a zmatku.

Lidem více vyhovuje to, na co jsou zvyklí, a bojí se neznámého. Lidé jsou natolik připoutáni k tomu, co je jim důvěrně známé, že i ti, kteří žijí ve válečných zónách nebo v oblastech s velmi vysokou kriminalitou, jen zřídka opouštějí svět, který znají, a hledají něco lepšího. Stejně tak je dobře zdokumentováno, že někteří dlouhodobí vězni si vypěstují strach z propuštění, a když se tak stane, páchají další trestnou činnost s úmyslem dostat se zpět do vězení. Dokonce i otroci mohou projevovat strach z propuštění. Je to proto, že život vězně nebo otroka, ačkoli není pravděpodobně naplňující, je předvídatelný a představa nového, drasticky změněného života, na cizím místě, mezi cizími lidmi, se všemi souvisejícími nejistotami -- Jak budu jíst? Kde budu žít? Jaké to bude? Budu v bezpečí? -- děsí téměř každého. Stejně tak je to, když většina lidí uvažuje o lidské společnosti bez vládnoucí třídy. Tato představa je tak cizí všemu, co kdy znali a o čem kdy přemýšleli, a všemu, co je učili, že je nutné a dobré, že si to jen stěží dovedou začít představovat. Dokonce i samotný náš jazyk ilustruje náš strach ze života ve společnosti svobodných a sobě rovných, protože takový stav je definován jako \enquote{anarchie} -- což je termín, který se také používá pro označení chaosu a destrukce. Tak jsme si zvykli na mentální klec, kterou kolem každého z nás vytvořil mýtus autority, že většinu z nás děsí představa života \emph{bez} této klece. Doslova se bojíme vlastní svobody.

A někteří lidé se snaží tento strach posilovat. Ti, kteří z mýtu autority těží nejvíce -- ti, kteří touží po nadvládě nad ostatními a po nezaslouženém bohatství a moci, kterou jim dává -- neustále vnucují poselství, že život bez jejich vlády by znamenal neustálou bolest a utrpení pro všechny. Prostě všeho, čeho se lidé mohou bát -- zločinu, chudoby, nemocí, invaze, ekologické katastrofy atd. -- používají tyrani k tomu, aby lidi zastrašili a přinutili je k podřízenosti. Detaily se liší, ale šablona poselství tyranů je vždy stejná: \enquote{Pokud nám nad vámi nedáte moc, abychom vás mohli ochránit, budete strašně trpět.} Vždy se jedná o to, že tyrani se snaží zastrašit lidi, kteří mají strach, aby se jim něco nestalo. Toto poselství v kombinaci s vrozeným strachem člověka z neznámého umožnilo nepochopitelnou míru útlaku, krádeží a otevřeného vraždění, trvajícího generaci za generací, po celém světě. Ironií je, že právě prázdný příslib ochrany před utrpením a nespravedlností přiměl tolik lidí k přijetí toho, co způsobilo více utrpení a nespravedlnosti než cokoli jiného v dějinách: víry ve stát.

Zdá se podivné, že by každý myslící člověk nebyl přirozeně otevřený a vnímavý k myšlence, že je sám sobě vlastníkem a že by měl být zodpovědný za svůj vlastní život, aniž by mu v tom bránila jakákoli lidská \enquote{autorita.} Průměrný člověk, který takové poselství uslyší, se však na posla často oboří, trvá na tom, že skutečná svoboda, svět bez pánů a poddaných, by znamenala chaos a zkázu, a pak vehementně obhajuje pokračující zotročování celého lidstva, včetně sebe sama. Nečiní tak na základě racionálního uvažování, důkazů či zkušeností, ale na základě své hluboce zakořeněné existenciální hrůzy z neznámého -- neznámým je v tomto případě společnost rovných místo pánů a poddaných. Nikdy ji neviděl v akci ve velkém měřítku, nikdy o ní nepřemýšlel, nedokáže si ji představit, a proto se jí bojí. A ti, kdo touží po nadvládě nad druhými, tento strach v těch, které si chtějí podrobit, neustále posilují a podporují.

\section{Jiný pohled na svět}

Když se někdo, kdo byl indoktrinován kultem \enquote{autority,} konečně odpoutá od pověr, první, co se stane, je, že uvidí drasticky odlišnou realitu. Když pozoruje účinky pověry o autoritě, které pronikají téměř do všech aspektů života většiny lidí, vidí věci tak, jak skutečně jsou, a ne tak, jak si je dříve představoval. Když vidí takzvané \enquote{vymáhání práva} v akci, většinou poznává, že jde o surové, nelegitimní a nemorální násilnictví, které se používá k vydírání a ovládání lidí, aby sloužili vůli politiků. (Výjimkou jsou případy, kdy policie použije sílu k zastavení jiných osob, které se skutečně dopouštějí agresivních činů -- paradoxně právě těch, které policie běžně páchá ve prospěch vládnoucí třídy). Když zotavující se etatista sleduje různé politické rituály, ať už prezidentské volby, legislativní debatu v Kongresu nebo místní územní radu, která schvaluje nějakou novou \enquote{vyhlášku,} vidí, co to je: jednání lidí, kteří byli indoktrinováni do zcela iracionálního kultu, plné bludů a halucinací. Jakékoli diskuse v médiích o tom, jaká by měla být \enquote{veřejná politika,} kteří \enquote{zástupci} by měli být zvoleni nebo jaké \enquote{zákony} by měly být přijaty, se člověku, který unikl pověrám, jeví přesně tak užitečné a racionální, jako když dobře oblečení, atraktivní a slušně vypadající lidé vážně diskutují o tom, jak by měl Ježíšek zvládnout příští Vánoce.

Pro toho, kdo unikl mýtu autority, se předpoklad, na němž stojí \emph{veškerá} politická diskuse, rozpadá a každý kousek rétoriky, který z pověry vychází, je rozpoznán jako naprosto šílený. Neindoktrinovaný jedinec vnímá každý projev v kampani, každý politický argument, každou diskusi ve zprávách o čemkoli politickém, každé vysílání CNN o další debatě na půdě Sněmovny reprezentantů o nějakém novém \enquote{zákonu} jako projev příznaků hlubokých bludů způsobených slepým přijetím naprosto nesmyslného, sektářského dogmatu. Veškeré hlasování, volební kampaň, psaní \enquote{kongresmanovi,} podepisování petic se najednou nejeví o nic racionálnější a užitečnější než modlitba k bohu sopky, aby udělil kmeni své požehnání. Člověk, který byl deprogramován, vidí nejen marnost veškerého \enquote{politického} konání, ale vidí, že takové akce, bez ohledu na jejich zamýšlené cíle, ve skutečnosti \emph{posilují} pověru. Stejně jako všichni v kmeni, kteří se modlí k bohu sopky, posilují představu, že bůh sopky \emph{existuje}, tak žebrání u politiků o laskavost posiluje představu, že existuje právoplatná vládnoucí třída, že její příkazy jsou \enquote{zákonem} a že poslušnost těmto \enquote{zákonům} je morálním imperativem.

Ti, na které dnes většina lidí pohlíží s velkou úctou a kteří jsou často označováni za \enquote{čestné,} jsou těmi, kdo unikli mýtu autority, považováni za pomatené, bohem zakomplexované blázny. Neindoktrinovaný člověk by se neštítil podat ruku \enquote{prezidentovi} o nic víc než kterémukoli jinému psychotickému, narcistickému masovému vrahovi. Muži, kteří nosí černé šaty, třímají dřevěná kladiva a označují se za \enquote{soud,} jsou vnímáni jako šílenci, jimiž jsou. Ty, kteří nosí odznaky a uniformy a představují si, že jsou něčím jiným než pouhými lidskými bytostmi, nepovažují deprogramovaní za ušlechtilé bojovníky za \enquote{právo a pořádek,} ale za zmatené duše trpící něčím, co je jen o málo víc než duševní poruchou.

Ti, kdo se vzdali pověrčivosti vůči autoritám, se samozřejmě stále mohou obávat škod, které jsou megalomani a jejich žoldáci -- vojáci a policisté -- schopni napáchat, ale činy žoldáků už nejsou považovány za jakkoli legitimní, racionální nebo morální. Ti, kdo unikli mýtu, začínají chápat, že ti, jejichž jednání je ovlivněno jejich \enquote{oficiálním} odznakem, jsou stejně nebezpeční jako lidé, kteří jsou sjetí PCPčkem, a to ze stejného důvodu: protože mají halucinace o realitě, která neexistuje, což je vede k násilnému jednání, neomezenému racionálním myšlenkovým procesem. Ti, kdo se vymanili z pověry o autoritě, se při konfrontaci s \enquote{policistou} mohou stále chovat stejně, jako kdyby byli konfrontováni se vzteklým psem: mluvit tiše, chovat se poddajně a nedělat prudké pohyby. Není to však z úcty ani k \enquote{strážci zákona,} ani ke vzteklému psovi; je to ze strachu před nebezpečím, které představuje mozek, jenž špatně funguje, protože je nakažen destruktivní nemocí, ať už je to vzteklina, nebo víra v autoritu. Když se věřící v autoritu dopouštějí agresivních činů a představují si, že takové činy jsou spravedlivé, protože se nazývají \enquote{zákonem,} jejich cíle mají jen málo možností. Když se \enquote{výběrčí} daní, policista nebo jiný vymahatel vůle politiků pokouší vydírat, obtěžovat, ovládat nebo napadat ty, kteří se vymykají mýtu autority, mohou se cíle \enquote{legální} agrese buď smířit s tím, o čem vědí, že je to nespravedlnost, nebo se mohou pokusit \enquote{legální} agresory nějak obejít či se před nimi skrýt, nebo se mohou agresorům násilím postavit na odpor. Je smutné, že poslední možnost je někdy nutná, protože ačkoli je použití obranné síly morálně ospravedlnitelné (i když je \enquote{nezákonné}), je smutné, že by jeden dobrý člověk musel někdy použít násilí proti jinému \emph{dobrému} člověku jen proto, že ten druhý má své vnímání dobra a zla pokřivené a zvrácené iracionální pověrou. I vraždící násilníci nejbrutálnějších režimů v dějinách si díky své víře v mýtus autority mysleli, že konají svou povinnost; na určité úrovni si mysleli, že jejich činy jsou ušlechtilé a spravedlivé, jinak by se jich nedopustili. Taková bezmyšlenkovitá loajalita k \enquote{autoritě} často ponechává zamýšleným obětem pouze dvě možnosti: podrobit se tyranii, nebo zabít oklamané \enquote{strážce zákona.} Pro všechny by bylo mnohem lepší, kdyby se dříve, než bude nutné klást násilný odpor, podařilo žoldáky státu z jejich bludu deprogramovat, aby nebylo nutné je zastrašovat, zraňovat nebo dokonce zabíjet, aby se jim zabránilo v páchání zla.

(\emph{Osobní poznámka autora: To nejhezčí, co můžete udělat pro každého, kdo se nechal oklamat a funguje jako pěšák utlačovatelské mašinérie zvané \enquote{stát,} je udělat vše, co můžete, abyste ho přesvědčili, aby přehodnotil svou loajalitu k mýtu autority. Pokud vše ostatní selže, dejte mu výtisk této knihy. Jakkoli to může být nepříjemné, možná tím prokážete velkou laskavost mnoha jeho potenciálním budoucím obětem a možná tím dokonce prokážete velkou laskavost i samotnému vymahateli, protože tím popřete nutnost, aby ho některá z jeho budoucích zamýšlených obětí zmrzačila nebo zabila.})

\section{Svět bez vládců}

Člověk, který byl deprogramován, se podívá na svět a místo hierarchie různých vládnoucích tříd v rámci různých jurisdikcí vidí svět rovných -- samozřejmě ne co do talentu, schopností nebo bohatství, ale co do práv. Vidí svět, v němž každý člověk vlastní sám sebe, a dochází k poznání, že nemá žádného právoplatného pána, že nad ním nikdo nestojí a že to platí i pro všechny ostatní. Není podřízen žádnému \enquote{státu,} žádné \enquote{zemi} ani žádnému \enquote{zákonu.} Je suverénní bytostí. Je vázán svým vlastním svědomím a ničím jiným.

Takové uvědomění je neuvěřitelně osvobozující, ale také může být značně znepokojující pro ty, kteří své chování vždy měřili podle toho, jak dobře poslouchají ostatní. Poslušnost je nejen snadná, protože umožňuje, aby o všem rozhodoval někdo jiný, ale také umožňuje tomu, kdo slepě poslouchá, představovat si, že za následky, ať už jsou jakékoli, je vždy zodpovědný někdo jiný. Muset si \emph{vyjasnit}, co je správné a co ne, a vědět, že za svá rozhodnutí a činy jste zodpovědní jen vy sami, může působit jako obrovské břemeno. Ztráta víry v autoritu v podstatě znamená dospět, což má své výhody i nevýhody. Odnaučený člověk už nemůže čelit světu jako bezstarostné, nezodpovědné dítě, ale zároveň bude disponovat takovou mírou svobody a moci, jakou si dříve nedokázal představit.

Etatisté mají často hluboce zakořeněnou hrůzu ze světa, v němž se každý člověk sám rozhoduje, co má dělat. Naneštěstí pro ně je to vše, co kdy existovalo a bude existovat. Každý člověk se již nyní sám rozhoduje, co bude dělat.

Tomu se říká \enquote{svobodná vůle.} Mnozí předpokládají, že pokud jedinec není vázán žádnou \enquote{autoritou} a má postoj \enquote{mohu si dělat, co chci,} bude se chovat jako sobecké zvíře. Někteří si dokonce představují, že by se sami stali zvířaty, kdyby nebyli řízeni pánem. Takové přesvědčení znamená, že lidé cítí silnou morální povinnost dělat, co se jim řekne, ale jinak nemají vůbec žádný morální kompas. Většina lidí však \enquote{zákon} dodržuje, protože věří, že je \emph{dobré} tak činit. Není důvod si myslet, že bez podřízenosti pánovi by ti samí lidé přestali dbát na to, aby byli dobří. Přesto si mnozí stále představují lidi jako hloupé divochy, které drží na uzdě pouze vládci. Očekávají tedy, že kdyby nebyli omezováni vírou v autoritu, většina lidí by se chovala jako utržená z řetězu.

Ti, kdo se vzdali klamu autority, už vědí jak to je. Činy mají samozřejmě své důsledky, ať už s \enquote{autoritou,} nebo bez ní. Kromě morálních otázek se většina lidí obvykle rozhoduje chovat tak, aby nevyvolala hněv druhých. I kdyby nikdo nevěřil v dobro a zlo, být obvyklým zlodějem nebo vrahem by bylo nebezpečné a hledání způsobů mírového soužití je prospěšné pro jednotlivce \emph{i} pro skupinu. Ale kromě toho se většina lidí snaží být dobrá. Ve skutečnosti právě proto dodržují \enquote{zákony:} protože je učili, že je to dobré. Problém není v tom, že by lidé nechtěli být dobří; problém je v tom, že jejich úsudek o tom, co je dobré a co špatné, je strašlivě pokroucený a zvrácený vírou v autoritu. Učí je, že financovat a poslouchat bandu násilníků je ctnost a odporovat je hřích. Učí se, že žádat tyto násilíky, aby okrádali a ovládali své sousedy (prostřednictvím \enquote{legislativy}), je naprosto morální a legitimní. Stručně řečeno, pokud jde o \enquote{autoritu,} učí se, že dobro je zlo a zlo je dobro. Iniciování násilí prostřednictvím \enquote{zákona} je považováno za dobro a bránění se takovým útokům (\enquote{porušování zákona}) je považováno za zlo.

Bez mýtu autority by lidé stále měli neshody a někteří lidé by stále byli zlomyslní nebo nedbalí a stále by dělali hloupé nebo nepřátelské věci. Hlavní rozdíl v tom, jak by spolu lidé komunikovali bez pověry o autoritě, je poměrně jednoduchý: \textbf{Pokud by se někdo necítil oprávněný něco udělat sám, necítil by se oprávněný požádat o to někoho jiného, ani by se necítil oprávněný udělat to sám za někoho jiného.} Tento koncept je tak jednoduchý, až to zní téměř triviálně, ale vedl by k obrovské změně v lidském chování. Pokud by se například někdo necítil oprávněný platit za vzdělání svých dětí násilným okrádáním sousedů, necítil by se oprávněný ani při \enquote{hlasování} o tom, aby místní \enquote{stát} zavedl \enquote{daň z nemovitosti} na placení \enquote{veřejných} škol. A pokud by se někdo necítil ospravedlněn krádeží majetku svého souseda za účelem financování školy, necítil by se ospravedlněn ani tehdy, kdyby dostal odznak a bylo mu řečeno, aby tak učinil ve jménu \enquote{zákona.} Jiný příklad: pokud by se někdo necítil oprávněný vykopnout někomu dveře, odvléct ho a zavřít na léta do klece za to, že vlastnil rostlinu s psychoaktivními účinky, pak by se necítil oprávněný ani při podpoře \enquote{protidrogových zákonů.} Stejně tak by se najednou necítil ospravedlněn k tomu, aby se dopouštěl takového vniknutí na cizí pozemek, napadení a únosu jen proto, že mu nějaká \enquote{autorita} dala odznak a nařídila mu to ve jménu nějakého \enquote{zákona.} Jako další příklad: pokud by se někdo necítil oprávněný použít násilí, aby zabránil úplně cizímu člověku vkročit kamkoli do celé \enquote{země,} pak by se stejně necítil oprávněný to udělat, kdyby mu někdo dal odznak ICE, ani by se necítil oprávněný podporovat \enquote{imigrační zákony,} které to nařizují jiným.

Ve společnosti bez mýtu autority by stále existovali zloději, vrazi a další agresoři. Rozdíl je v tom, že všichni lidé, kteří považují krádeže a vraždy za nemorální, by \emph{neobhajovali} a \emph{nepodporovali} \enquote{legální} krádeže a vraždy, což dnes dělá \emph{každý} etatista. Opět platí, že obhajovat jakýkoli \enquote{zákon} znamená obhajovat použití jakékoli úrovně autoritářské síly, až po smrtící sílu, aby se dosáhlo jeho dodržování. A lidé, kteří vnímají krádež a vraždu jako nemorální, by se takových činů nedopouštěli jen proto, že jim to nařizuje nějaká \enquote{autorita} nebo \enquote{zákon.}

Kolik z toho, co policie denně dělá, by dělali sami od sebe, aniž by jim to nařizoval nějaký \enquote{zákon} nebo \enquote{stát?} Velmi málo. Kolik z toho, co běžně dělají \enquote{vojáci,} by dělali sami od sebe, aniž by jim to nařídil autoritativní vojenský vůdce? Velmi málo. Kolik z toho, co dnes dělají \enquote{výběrčí daní,} by dělali sami od sebe, aniž by jim to nařizoval nějaký \enquote{stát?} Nic z toho. Všechno \emph{dobré}, co nyní dělají lidé, kterým se říká \enquote{strážci zákona} -- tj. snaží se zabránit skutečně nepřátelským, destruktivním lidem, aby ubližovali nevinným -- by mohli dělat i nadále bez mýtu autority. A mohli by tak činit z dobroty srdce nebo jako placenou kariéru v pravděpodobném případě, že by jim za to ostatní lidé chtěli dobrovolně platit. Zároveň by vše \emph{špatné}, co nyní \enquote{strážci zákona} a vojáci dělají -- např. terorizování nebo střílení do lidí, o kterých nic nevědí, agresi proti těm, kteří se dopouštějí \enquote{zločinů} bez obětí, zadržování, vyslýchání a napadání zcela cizích lidí -- většina z nich přestala dělat.

Jen velmi málo lidí bylo napadeno, mučeno a zavražděno obyvatelstvem Německa jako celku nebo obyvatelstvem Ruska jako celku nebo obyvatelstvem Číny jako celku \emph{předtím}, než příslušné \enquote{státy} těchto zemí za Hitlerova, Stalinova a Maova režimu přijaly \enquote{zákony} předstírající \emph{legitimizaci} takových zvěrstev. Kolik zvěrstev však bylo spácháno \emph{po} tom, co \enquote{autorita} vydala příkazy nařizující jejich páchání? Čísla jsou ohromující: desítky milionů zavražděných, stovky milionů napadených, utlačovaných nebo mučených. Je zřejmé, že obyvatelé těchto zemí (a téměř všech ostatních zemí) byli mnohem méně ochotni páchat agresivní činy na vlastní pěst než na příkaz imaginární \enquote{autority.}

Je ironií osudu, že tváří v tvář konceptu čistě dobrovolné společnosti, v níž je každá služba, dokonce i obrana a ochrana, financována dobrovolnými zákazníky namísto násilných \enquote{daní,} mnozí etatisté předpovídají, že soukromé bezpečnostní firmy by se vyvinuly v nové, zneužívající, utlačovatelské \enquote{státy} nebo že by konkurenční bezpečnostní firmy skončily ve vzájemných věčných násilných konfliktech. Takové předpovědi si neuvědomují, že většina lidí \emph{nechce} napadat a okrádat své sousedy a nechce být sama napadána a okrádána, a pouze díky víře v autoritu se většina lidí cítí v pořádku, když obhajuje okrádání prostřednictvím \enquote{daní,} nebo se cítí povinna souhlasit s tím, aby byla sama napadána a okrádána prostřednictvím \enquote{dodržování zákonů.} Bez představy, že \enquote{stát} má práva, která jednotlivci nemají, by žádná zákeřná, agresivní soukromá bezpečnostní firma nikdy neměla podporu veřejnosti. Kdyby byly vnímány pouze jako soukromí zaměstnanci obyčejných lidí, nikdo ze zúčastněných -- ani zákazníci, ani jejich najatí ochránci -- by si nepředstavoval, že zaměstnanci mají právo krást, obtěžovat, terorizovat nebo dělat cokoli, na co nemá právo kdokoli jiný.

Podívejte se na to z jiného úhlu pohledu, a aby to bylo osobnější, představte si, že byste žili ve světě, kde by nikdo z vašich sousedů neměl právo prosazovat, abyste byli \enquote{zdaněni} za účelem financování věcí, které vám vadí. Představte si, že by každá věc, každý plán, každý program, každá myšlenka, každé navrhované řešení nejrůznějších problémů bylo něco, co byste mohli buď \emph{dobrovolně} podpořit, nebo ne. Představte si, že byste žili ve světě, kde by nikdo z vašich sousedů neměl pocit, že má právo vám násilím vnucovat své myšlenky, volby a životní styl. Cítili by se oprávněně (jako to již dělají) použít sílu, aby vás zastavili, pokud byste se rozhodli je napadnout nebo okrást, ale jen málokdo by se cítil dobře, kdyby se vůči vám dopustil jakéhokoli druhu agrese.

Na rozdíl od toho, co se většina lidí domnívá, přesně takto by vypadal \enquote{svět bez pravidel.} Každý člověk by se řídil svým vlastním svědomím -- což by se dalo považovat za \enquote{pravidla} nebo \enquote{sebeřízení} -- a i když by se někteří lidé jednající na vlastní pěst stále rozhodovali hloupě nebo zlovolně a dopouštěli se agresivních činů, nikoho by už nenapadlo, že když něco nazveme \enquote{zákonem} nebo \enquote{pravidlem,} může se z ve své podstatě neoprávněného činu stát něco dobrého. A pokud byste se takovému aktu agrese \emph{vzepřeli}, vaši sousedé by vás za to chválili, místo aby vás odsoudili jako \enquote{zločince,} což by dnes udělali téměř všichni, pokud byste se vzepřeli aktu agrese, který by byl zrovna \enquote{legální.}

\section{Myslet jinak, mluvit jinak}

Mnoho termínů, které lidé používají, a diskusí, které denně vedou, vychází z předpokladu, že \enquote{autorita} může existovat. Tím, že neustále slyšíme a opakujeme dogmata založená na pověrách, téměř každý nevědomky posiluje tento mýtus, a to jak ve své mysli, tak v mysli těch, s nimiž hovoří. Autoritářská propaganda je tak všudypřítomná, že masám vůbec nepřipadá jako \enquote{myšlenka;} připadá jim jako \enquote{mluvení o tom, co je.}

Většina každé učebnice dějepisu se zabývá tím, kdo kdy vládl v jaké oblasti, který autoritářský režim si podrobil jiný autoritářský režim, které osoby nebo politické strany se dostaly k moci, jaké formy \enquote{vlády} a typy \enquote{veřejné politiky} měly různé říše atd. Mluví se o volbách, o tom, kdo měl v zákulisí moc, jaké \enquote{zákony} byly přijaty, jaké \enquote{daně} byly uvaleny a co si lidé mysleli o svých \enquote{vůdcích.} Základní premisa, která se ozývá hlasitě a jasně, i když není nikdy otevřeně vyslovena, je, že je nevyhnutelné a legitimní, aby existovala vládnoucí třída -- jakási odrůda vládce s právem násilně ovládat všechny ostatní.

Toto poselství je i nadále stálým základním tématem téměř všeho, co se píše v novinách nebo vysílá v rozhlase či televizi. Ve zprávách, ať už místních nebo celostátních, se hovoří o tom, jaké \enquote{zákony} schválili \enquote{zastupitelé} nebo \enquote{kongresmani,} co ten den udělali \enquote{strážci zákona,} kteří kandidáti se ucházejí o \enquote{veřejnou funkci,} jakou \enquote{veřejnou politiku} podporují atd. Způsob, jakým se o tom všem informuje, je silně poznamenán pověrou o autoritě. Způsob, jakým lidé myslí, samozřejmě ovlivňuje i to, jak mluví, a každý člověk neustále vyjadřuje své základní přesvědčení, a to i ve zdánlivě banálních diskusích.

Porovnejte, jak by o stejné situaci a událostech pravděpodobně referoval nejprve ten, kdo věří v autoritu, a pak ten, kdo v ni nevěří:

S pověrou: \enquote{\emph{Dnes místní samospráva Springfieldu zavedla čtyřprocentní zvýšení místních poplatků za stavební povolení, jejichž výtěžek je určen na financování programu poskytování určité lékařské pomoci starším lidem.}}

Bez pověry: \enquote{\emph{Dnes místní zločinecký syndikát vydal oficiální výhrůžku všem, kdo ve Springfieldu provádějí stavební práce nebo rekonstrukce, a požaduje o čtyři procenta vyšší poplatky, než skupina od takových lidí požadovala dříve. Zloději tvrdí, že část zabavených peněz hodlají věnovat starším lidem.}}

Když se někdo vymaní z pověry autority, jeho myšlenkové vzorce, a tedy i způsob řeči, se dramaticky změní. Nepoužívá eufemistické výrazy, které \enquote{legálnímu} násilí přisuzují legitimitu. Popisuje \enquote{výběrčí daní} jako to, čím ve skutečnosti jsou: profesionální vyděrači. \enquote{Strážce zákona} popisuje jako to, čím ve skutečnosti jsou: nájemnými násilníky politiků. Popisuje \enquote{zákony} jako to, čím ve skutečnosti jsou: výhrůžky politiků. Sám sebe hrdě neoznačuje za \enquote{zákona dbalého daňového poplatníka,} protože si uvědomuje, co tento termín ve skutečnosti znamená: člověk, který se nechává okrádat a ovládat megalomany toužícími po moci.

Většina etatistů si jen těžko dokáže představit svět, v němž neexistuje centralizovaná mašinérie, která se snaží ovládat všechny ostatní. Pro některé je však stejně obtížné představit si svět, v němž oni sami nejsou násilně ovládáni. Představa, že se dívají na svět a necítí se být nikomu zavázáni, necítí se povinni dodržovat \enquote{zákony} druhých, je jim naprosto cizí v čemkoli, o čem kdy uvažovali. Je to smutné, ale pro mnoho lidí je velmi těžké si vůbec představit svět, ve kterém neexistuje nikdo, komu by se museli klanět, žádná legislativa, které by se museli podřizovat, žádný \enquote{zákon} nebo \enquote{pravidlo,} které by kdy mohlo převýšit jejich vlastní svědomí. Takové představy jsou na míle vzdálené tomu, čemu se téměř všichni učili věřit, a přijetí tak drasticky odlišného pohledu na realitu je pro ně hlubokým existenciálním probuzením. Ten, kdo unikl mýtu, si říká něco takového:

\enquote{\emph{Má někdo nebo nějaká skupina lidí právo požadovat ode mě platbu za něco, o co jsem nežádal a co nechci financovat? Samozřejmě, že ne. Pokud se nedopouštím vůči nikomu agrese (prostřednictvím násilí nebo podvodu), má někdo právo mě nutit k rozhodnutí, které si přeje, abych učinil? Samozřejmě že ne. Mám právo se takové agresi bránit? Samozřejmě, že mám. Má nějaká osoba nebo skupina osob nějaká práva, která já nemám? Samozřejmě že ne. (Jak a odkud by taková práva získali?) Mám kdykoli, kdekoli a za jakýchkoli okolností povinnost dělat něco jiného, než co mi diktuje mé vlastní svědomí? Existuje vůbec nějaká situace, kdy by mě nařízení nebo 'zákony' nějaké domnělé 'autority' mohly jakýmkoli způsobem a do jakékoliv míry zavazovat k tomu, abych se vzdal své svobodné vůle nebo ignoroval svůj vlastní smysl pro dobro a zlo? Jistěže ne.}}

\section{Výuka morálky vs. výuka autority}

Obecně se má za to, že pokud se děti nenaučí respektovat a poslouchat \enquote{autoritu,} budou jako divoká zvířata, která kradou, napadají atd. Poslušnost však sama o sobě znamená pouze to, že místo toho, aby se jedinec řídil vlastním úsudkem, bude se podřizovat úsudku těch, kteří usilují o mocenské pozice a získávají je -- jedněch z nejnemorálnějších, nejzkorumpovanějších, nejbezcitnějších, nejzlomyslnějších a nejnepoctivějších lidí na světě. Výchova lidí k pouhé poslušnosti zabraňuje zvířecímu chování pouze tehdy, pokud údajná \enquote{autorita} sama \emph{neschvaluje} a \emph{nenařizuje} krást a útočit, jak to dělal ve jménu \enquote{daní} a \enquote{vymáhání práva} každý \enquote{stát} v dějinách.

Je zřejmé, že učení poslušnosti civilizaci nepomůže, pokud ti, kdo dávají příkazy, přikazují právě takové chování, které společnosti škodí: akty agrese vůči nevinným. Myšlenka, že všeobecná poslušnost je pro společnost dobrá, se opírá o zjevně falešný předpoklad, že lidé v mocenských pozicích jsou morálně nadřazeni všem ostatním. Mělo by být samozřejmé, že když většina lidí nedbá na své svědomí a místo toho svěří rozhodování za sebe politikům, společnost se nestane bezpečnější ani ctnostnější. Místo toho to pouze legitimizuje činy, které \emph{brání} mírovému lidskému soužití.

Vezměme si analogii s robotem, který je naprogramován tak, aby dělal vše, co mu jeho majitel přikáže, ať už produktivní nebo destruktivní, civilizované nebo agresivní. Je to podobné, jako když se dítě učí respektovat \enquote{autoritu.} Zda se poslušný robot nebo dítě nakonec stanou nástrojem destrukce a útlaku, závisí pouze na tom, kdo nakonec vydává příkazy. Pokud se naopak děti učí principu sebevlastnictví -- myšlence, že každý jedinec patří sám sobě a neměl by být okrádán, ohrožován, napadán nebo zabíjen -- pak je údajná ctnost poslušnosti zcela zbytečná. Zvažte, co z následujícího by pravděpodobněji vedlo ke spravedlivé a mírumilovné společnosti: zda miliardy lidí, kteří se učí základům toho, jak být morálními lidmi (např. principu neagrese), nebo miliardy lidí, kteří se učí pouze poslouchat v naději, že těch několik málo lidí, kteří se nakonec dostanou do vedení, bude náhodou vydávat dobré rozkazy. Pokud je obtížné si představit, co by se v těchto dvou scénářích stalo, stačí se podívat do historie, abyste viděli, co se \emph{stalo}.

Dokonce i náhodně vybraní \enquote{vládci,} pokud dostanou povolení násilně ovládat všechny ostatní, budou rychle zkorumpováni a stanou se tyrany. Průměrní, slušní lidé však nejsou těmi, kdo touží po moci nad ostatními. Ti, kdo usilují o moc a získávají ji, jsou obvykle již narcisté a megalomani, lidé s nekonečnou touhou po moci, kteří milují myšlenku ovládat druhé. A touha po nadvládě není nikdy vedena snahou pomoci ovládaným, ale vždy touhou posílit postavení ovládajícího na úkor těch, které ovládá. Přesto lidé stále opakují tvrzení, že průměrný člověk, pokud by se řídil pouze svým vlastním svědomím, by byl méně důvěryhodný, méně civilizovaný a méně morální, než když odloží své vlastní svědomí stranou a jen slepě dělá vše, co mu tyrani tohoto světa nařídí. Kdyby se každý člověk spoléhal na svůj vlastní úsudek, byla by to podle definice \enquote{anarchie,} zatímco všeobecná poslušnost autoritářským tyranům představuje podle definice \enquote{právo a pořádek.} Všimněte si drastického kontrastu mezi obvyklými konotacemi těchto pojmů -- \enquote{anarchie} zní děsivě a násilně, \enquote{zákon a pořádek} zní civilizovaně a spravedlivě -- a skutečnými \emph{výsledky} následování svědomí oproti následování vládců. Míra zla páchaného jednotlivci, kteří jednají na vlastní pěst, je naprosto trpasličí ve srovnání s mírou zla páchaného lidmi, kteří se podřizují domnělé \enquote{autoritě.}

Ačkoli si mnozí představují, že učení poslušnosti \enquote{autoritám} je synonymem pro učení o tom, co je správné a co ne, ve skutečnosti jde o dva protiklady. Učit děti respektovat práva každého člověka a učit je, že páchání agrese je ze své podstaty špatné, je velmi důležité. Ale když je učíme, že poslušnost je ctnost a že \enquote{respektování autority} je morální imperativ, vyrostou z nich buď zastánci rozsáhlé agrese, nebo \emph{účastníci rozsáhlé agrese}. Každý etatista dělá jedno nebo druhé (nebo obojí). Ve skutečnosti učení poslušnosti dramaticky brzdí sociální a duševní vývoj dětí. Poté, co vyrostly v situaci, kdy byly ovládány druhými, odměňovány za poslušnost a trestány za neposlušnost, pokud se z této situace někdy vymaní, budou mít jen malý nebo žádný tréning a malou nebo žádnou zkušenost či praxi v tom, jak myslet a jednat na základě morálky a zásad. Budou jako cvičené opice, které sice unikly, ale nemají žádnou možnost, jak se vyrovnat s životem na svobodě, protože nikdy neuplatňovali svůj individuální úsudek a osobní odpovědnost a věděli jen to, jak dělat, co se jim řekne. Pokud byla jejich výchova formována především vládnoucími \enquote{autoritami,} lidé se stanou existenčně ztracenými, pokud tato nadvláda zmizí. Zkrátka, lidé vychovaní k poslušnosti \enquote{autoritám} neumějí být nezávislými, suverénními a zodpovědnými lidskými bytostmi, protože byli celý život záměrně a cíleně vychováváni k tomu, aby se \emph{neřídili} vlastním svědomím a \emph{nepoužívali} vlastní úsudek. Takže když indoktrinovaní uniknou z jednoho institucionalizovaného ovládaného prostředí (\enquote{školy}), halucinují jinou \enquote{autoritu,} která má nastoupit na její místo: \enquote{stát.} Uprchlé opice si prostě postaví novou klec a dychtivě do ní skočí, protože to je vše, co znají a co kdy znaly.

Ve světě bez mýtu o autoritách by se naopak děti mohly učit morálce, a ne pouze poslušnosti. Mohly by se učit respektovat lidi, místo aby respektovaly nelidské, násilnické monstrum zvané \enquote{stát.} Mohly by se učit, že je na nich, aby nejen dělaly správné věci, ale aby samy přišly na to, co \enquote{správné věci} jsou. Díky tomu by z nich mohli vyrůst zodpovědní, přemýšliví a schopní dospělí, členové mírumilovné a produktivní komunity, místo aby z nich vyrostl jen dobytek na farmách tyranů.

\section{Žádný generální plán}

Kdyby se zítra nějakým zázrakem všichni na světě zbavili víry v autoritu, naprostá většina krádeží, přepadení a vražd ve společnosti by okamžitě ustala. Skončily by všechny války, přestaly by všechny loupeže ve jménu \enquote{daní,} přestal by veškerý útlak prováděný ve jménu \enquote{práva.} Lidé jako celek -- včetně pachatelů, obětí a pozorovatelů útlaku -- by již nepovažovali takové akty agrese za legitimní.

Došlo by však i k další, méně bezprostřední změně. Víra v autoritu je v podstatě psychologická klec. Vychovává lidi k přesvědčení, že nemusí sami posuzovat, co je správné a co špatné; že nemusí sami na sebe brát odpovědnost za nápravu společnosti; že jediné, co se od nich vyžaduje, je, aby \enquote{hráli podle pravidel} a dělali, co se jim řekne, a přitom vzhlíželi k \enquote{vůdcům} a \enquote{zákonodárcům,} aby se vypořádali s problémy společnosti. Víra v autoritu zkrátka vychovává lidi k tomu, aby nikdy nedospěli, aby se na svět vždy dívali tak, jak se na něj dívají děti: jako na nepochopitelně složité místo, za které je a vždy bude zodpovědný někdo jiný. Ať už se jedná o jakýkoli problém -- chudobu, kriminalitu, nemoci, ekonomické nebo ekologické potíže -- indoktrinovaní etatisté vždy hledají nějakého nového \enquote{vůdce,} kterého by zvolili a který by slíbil, že věci napraví. V jistém smyslu funguje svět autoritářů přesně tak, jako třída v mateřské škole: pokud se něco pokazí -- pokud se objeví cokoli, co se vymyká předvídatelnému, předem naplánovanému a centrálně řízenému programu -- \enquote{děti} volají \enquote{učitele,} aby vše napravil. Celé autoritářské prostředí školní třídy učí děti, že nikdy nejsou ve vedení; nikdy není na nich, aby rozhodly, co mají dělat. Ve skutečnosti jsou důrazně odrazovány od toho, aby kdy přemýšlely nebo jednaly samostatně. Koneckonců, kdyby jim bylo dovoleno přemýšlet a rozhodovat se samy, první rozhodnutí, které by většina z nich učinila, by bylo odejít ze třídy.

Stejně tak se dospělým autoritářům neustále říká, že by člověk neměl \enquote{brát zákon do svých rukou.} Lidé jsou vyškoleni, aby volali \enquote{autority,} kdykoli dojde ke konfliktu nebo jinému problému, a pak pokorně udělali vše, co jim \enquote{státní} vymahatelé řeknou. Pokud dojde k nějakému sporu mezi lidmi, je lidem řečeno, že mají vždy běžet za pány, ať už zavoláním \enquote{policie,} nebo tím, že se obrátí na autoritářské \enquote{soudy,} aby neshody vyřešily. Při diskusích o společenských problémech hovoří dobře vyškolení poddaní státu v termínech jako např: \enquote{Měli by přijmout zákon...} nebo \enquote{Měli by vytvořit státní program....} Své životy vnímají jako součást obrovského centralizovaného generálního plánu, takže z toho logicky vyplývá, že pokud chtějí, aby se jejich život zlepšil, řešením je požádat plánovače o změnu plánu. Tento pohled je v masách natolik zakořeněný, že mnozí lidé doslova nedokážou pochopit myšlenku, že jednotlivci žijí své životy, aniž by byli součástí něčího generálního plánu. Dokládá to běžná reakce autoritářů na myšlenku společnosti bez vládců. Téměř bez výjimky začne etatista, který uvažuje o bezstátní společnosti, otázkou, jak budou věci \enquote{fungovat} bez vládnoucí třídy. Neptá se tak jen proto, že by ho zajímalo, jak by mohly fungovat silnice, obrana, obchod, řešení sporů a další věci bez \enquote{státu.} Ptá se tak proto, že byl vždy vycvičen nahlížet na lidskou existenci v rámci nějakého centralizovaného, násilím vnuceného generálního plánu a doslova není schopen uvažovat mimo toto paradigma. A tak se bude ptát, jak budou věci fungovat \enquote{v anarchii,} a bude o ní mluvit jako o \enquote{systému,} představovat si ji jako nový typ generálního plánu, který má být vnucen masám, i když jde samozřejmě o pravý opak: o naprostou \emph{neexistenci} centralizovaného, násilně vnuceného plánu. Celkový plán pro lidstvo je však to jediné, co etatista kdy uvažoval, a často je to to jediné, co dokáže pochopit. Představa, že \emph{nikdo} nebude \enquote{vládnout,} že \emph{nikdo} nebude určovat \enquote{pravidla} pro všechny ostatní, že \emph{nikdo} nebude plánovat ani řídit lidstvo jako celek a že nikdo nebude etatistovi říkat, co má dělat, je prostě něco, co si většina autoritářů nikdy ani nepředstavila. Tento pojem je jim natolik neznámý, že ho ani neumějí zpracovat, a tak se zoufale snaží napasovat myšlenku \enquote{anarchie} (bezstátní společnosti) do formy generálního plánu.

(Takové rozporuplné myšlení jen posilují ti, kdo nosí nálepku \enquote{anarchokomunisty.} Tento termín naznačuje, že by neexistovala žádná vládnoucí třída \emph{a} že by společnost byla organizována do kolektivistického systému. Samozřejmě, pokud si nějaká skupina osobuje právo násilím vnutit takový systém všem ostatním, jedná se o autoritářství, a tak by se na ni ta \enquote{anarcho} část termínu nevztahovala. Další možností je, že ti, kdo si říkají \enquote{anarchokomunisté,} pouze \emph{doufají}, že při neexistenci vládnoucí třídy se každý jednotlivec na planetě svobodně rozhodne pro účast v komunách nebo kolektivech -- což by se samozřejmě nestalo. Jako poslední možnost by možná \enquote{anarchokomunisté} pro sebe zvolili účast v komuně, ale ostatním by umožnili zvolit si jiné uspořádání. Nakonec termín \enquote{anarchokomunista} nedává příliš smysl a je vlastně \emph{symptomem} autoritářství: i když někteří lidé obhajují bezstátní společnost, automaticky si představují, že musí existovat nějaký zastřešující systém nebo plán, nějaké velké schéma, nějaká forma řízení společnosti, která musí být vnucena lidstvu jako celku).

Pravdou je, že ať už s mýtem autority, nebo bez něj, nikdo nemůže zaručit spravedlnost nebo prosperitu, ani předvídat vše, co by mohlo nastat, ani znát všechny problémy, které by mohly nastat, ani způsob, jak je všechny vyřešit. Rozdíl je v tom, že ti, kdo věří v autoritu, navzdory neustálým zdrcujícím důkazům o opaku nadále předstírají, že autoritářský systém nadvlády může zaručit bezpečí, jistotu, prosperitu, spravedlnost a férovost. Zatímco ti, kdo se vzdali nejnebezpečnější pověry, již nepředstírají, že je možné ovládat všechno a všechny prostřednictvím jakéhokoli \enquote{systému.} Je bizarní, že navzdory téměř nepochopitelné míře ekonomických katastrof, lidského utrpení a masového útlaku, které víra ve stát opakovaně způsobila, zastánci autoritářství stále trvají na tom, že ti, kdo se staví proti etatismu, musí být schopni do nejmenších podrobností přesně popsat, jak by vše ve společnosti fungovalo v případě neexistence \enquote{státu,} aby se nemohlo stát nic špatného. A pokud to nedokážou -- což samozřejmě nikdo nedokáže -- etatisté to prohlásí za důkaz, že \enquote{anarchie nikdy nebude fungovat.}

Spíše než racionální závěr je taková myšlenka projevem hluboce zakořeněné mentální závislosti a strachu z neznámého. Etatisté chtějí příslib, že se o ně nějaká vševědoucí a všemocná entita postará a ochrání je před veškerým možným neštěstím a před všemi špatnými lidmi na světě. Skutečnost, že politici takové sliby dávají odjakživa a ani jednou takový slib ve skutečnosti nesplnili (protože je zjevně směšný), etatistům nebrání v tom, aby takový slib chtěli \emph{slyšet}. Bez ohledu na to, kolikrát autoritářská \enquote{řešení} strašlivě selžou, si většina lidí stále myslí, že jediným řešením je nějaký \emph{jiný} \enquote{státní} plán. Chtějí záruku, že nějaká všemocná entita mimo ně samotné se postará o to, aby jejich život byl pohodlný a bezpečný. Zdá se, že je nezajímá, nebo si dokonce nevšimli, že takové \enquote{záruky} se nikdy nenaplní a že každý, kdo si nárokuje moc takovou záruku poskytnout, je buď neuvěřitelně drzý lhář, nebo šílenec. Nicméně protože anarchisté a voluntaristé by nikdy nedali absurdní slib, že bez \enquote{státu} se nikdy nestane nic špatného, většina etatistů zůstává vyděšena myšlenkou bezstátní společnosti.

(\emph{Osobní poznámka autora: Zjistil jsem, že kdykoli se v mých diskusích s etatisty objeví téma bezstátní společnosti, téměř bez výjimky začnou klást otázky v pasivním hlase: jak se to udělá, jak se to vyřeší? Mluví, jako by i v případě vlastního života byli jen pozorovateli, kteří čekají, co se stane. Je to proto, že po mnoho let svého formování, zejména během \enquote{školní docházky,} byli jen o málo víc než pozorovateli. Scénáře jejich životů psali jiní; jejich osud určovala a rozhodovala \enquote{autorita,} nikoli oni sami. Když se mě tedy ve snaze přimět je, aby se vymanili z tohoto způsobu myšlení, zeptají na něco jako: \enquote{Jak se to bude za anarchie řešit?,} odpovídám: \enquote{Jak byste to řešili vy?.} Když se zeptají: \enquote{Co by se s tímto potenciálním problémem dělalo?,} zeptám se: \enquote{Co byste s tím udělali vy?.} A oni obvykle z hlavy vymyslí nápady, které jsou lepší než jakékoli autoritářské řešení. Problém není v tom, že by nebyli schopni být zodpovědní za sebe, svou budoucnost a vlastně i za budoucnost světa; problém je v tom, že je prostě nikdy nenapadlo, že oni už za sebe, svou budoucnost a budoucnost světa zodpovědní jsou.}).

Ten, kdo chápe, že \enquote{autorita} je mýtus, nemá povinnost vysvětlovat, jak by fungovaly všechny aspekty svobodné společnosti, stejně jako ten, kdo tvrdí, že Ježíšek není skutečný, nemá povinnost vysvětlovat, jak budou Vánoce fungovat bez něj. Etatisté však často trvají na tom, aby jim někdo jako podmínku toho, že vůbec uvažují o možnosti bezstátní společnosti, vysvětlil, jak bude každý aspekt života každého člověka fungovat bez \enquote{státu.} Nikdo samozřejmě neví -- ať už s mýtem státu, nebo bez něj -- co všechno by se mohlo stát, a je absurdní lpět na prokazatelně falešném, sám sobě odporujícím mýtu, který sám vedl k rozsáhlému vraždění, vydírání a útlaku, jen proto, že někdo nedokázal podrobně popsat dokonalý svět bez tohoto mýtu. Lidé mohou předkládat návrhy nebo předpovědi, jak by různé aspekty svobodné společnosti fungovaly bez zapojení \enquote{autority} -- a mnoho odborných pojednání přesně to dělá -- ale jakmile někdo skutečně pochopí šílenost, která je vlastní jakékoli víře v autoritu, nikdy se nevrátí k přijímání mýtu bez ohledu na to, co by se podle něj mohlo stát bez něj, stejně jako by se dospělý člověk nevrátil k víře v Ježíška, protože neví, zda by bez něj Vánoce fungovaly.

\section{Vládneš ty, vládnu já}

Pokud by neexistovala \enquote{autorita,} nikdo by podle definice neměl moc ani právo prohlásit: \enquote{Takhle se to bude dělat.} Přesto je to jediná myšlenková šablona, o které většina autoritářů vůbec uvažuje. Ti, kdo si uvědomují, že nemají ani možnost, ani právo ovládat celé lidstvo, neuvažují v rovině generálního plánu pro lidskou rasu. Místo toho přemýšlejí v termínech jediné věci, kterou mohou skutečně ovládat: svých vlastních činů. Přemýšlejí v rovině: \enquote{Co bych s tím měl udělat?} místo \enquote{Co bych měl požádat mistry, aby s tím udělali?.} Nejsou tak arogantní nebo iluzorní, aby si mysleli, že mají právo nebo schopnost rozhodovat za celé lidstvo. Rozhodují se sami a přijímají nevyhnutelnou skutečnost, že ostatní lidé se rozhodnou jinak.

Z praktického hlediska je absurdní očekávat, že systém centralizovaného řízení, v němž hrstka politiků se svými omezenými znalostmi a zkušenostmi vymyslí generální plán a pak ho vnutí všem ostatním, bude fungovat lépe než porovnávání a kombinování znalostí, vynalézavosti a odbornosti stovek milionů jednotlivců prostřednictvím sítě vzájemně dobrovolného obchodu a spolupráce. Ať už je cílem cokoli -- ať už jde o produkci potravin, stavbu silnic, ochranu před agresory nebo cokoli jiného -- nápady, které vzejdou z \enquote{chaosu} milionů lidí zkoušejících různé vynálezy a řešení, budou vždy lepší než nápady, s nimiž přijde hrstka politiků. To platí zejména ve světle skutečnosti, že zatímco politici vnucují své nápady všem prostřednictvím \enquote{zákona,} i když jsou to mizerné nápady, které se nikomu jinému nelíbí, nápady svobodného trhu musí být natolik dobré, aby je ostatní \emph{dobrovolně} podpořili.

Navzdory úžasnému blahobytu, který již vznikl díky relativně svobodnému, \enquote{anarchistickému} obchodu a vzájemné spolupráci, je myšlenka soužití lidí bez toho, aby byli všichni řízeni a regulováni nějakým generálním plánem, pro většinu etatistů stále nepochopitelná. Většina etatistů nikdy ani nezačala uvažovat o možnosti, že by skutečně řídili svůj vlastní život. Všechno v moderní autoritářské společnosti vychovává lidi k tomu, aby byli loajálními poddanými systému nadvlády, místo aby je vychovávalo k tomu, čím by měli být: suverénními subjekty, které si na věci přicházejí samy, komunikují s ostatními jako rovný s rovným a zodpovídají se především svému svědomí. Pro většinu lidí je představa světa, kde jsou to oni, kdo musí řešit problémy, urovnávat spory, pomáhat těm, kteří pomoc potřebují, chránit sebe i ostatní, aniž by se místo toho mohli utíkat k všemocné \enquote{autoritě,} zcela cizí a děsivá. Rádi prosazují autoritářská řešení, ale ve skutečnosti nechtějí být zodpovědní ani sami za sebe, natož aby byli osobně zodpovědní za to, jak společnost funguje. A právě víra v autoritu je to, čím se snaží vyhnout se této odpovědnosti a uniknout realitě života.

Život zvířete v kleci je v mnoha ohledech jednodušší než život ve volné přírodě. Stejně tak život nemyslícího lidského otroka může být předvídatelnější a může se zdát bezpečnější než život plný odpovědnosti. Ale stejně jako život ve volné přírodě činí zvířata silnějšími, chytřejšími a mnohem schopnějšími se o sebe postarat, tak i opuštění mýtu o autoritě donutí lidi být chytřejšími, kreativnějšími, soucitnějšími a morálnějšími. To neznamená, že všichni lidé budou bez víry ve stát moudří, laskaví a velkorysí. Kdyby však miliony jednotlivců pochopily, že je na nich osobně, aby učinily svět lepším, místo aby pouze poslušně hrály přidělenou roli v něčím generálním plánu a přitom volaly po tom, aby \enquote{stát} všechno napravil, uvolnilo by to takovou úroveň lidské tvořivosti, vynalézavosti a spolupráce, jakou si většina lidí ani nedokáže představit.

\section{Jiná společnost}

Dnes si většina lidí spojuje myšlenku, že \enquote{každý si dělá, co chce,} s chaosem a umíráním, a myšlenku, že každý je poslušný a \enquote{dodržuje zákony,} s řádem a civilizací. Bez mýtu autority by však lidé měli zcela jiné myšlení. Bez \enquote{autority,} kterou by slepě následovali a poslouchali, bez možnosti fňukat \enquote{mocným,} aby všechno napravili, by lidé museli sami přijít na to, co je správné a co ne, a jak řešit problémy. Někdo by mohl tvrdit, že lidé jsou příliš krátkozrací, líní a nezodpovědní na to, aby si sami řídili svůj život, ale právě víra v autoritu jim umožnila stát se tak línými a bezmocnými. Dokud věřili, že dělat věci správně není jejich práce, že řešení problémů není jejich práce a že jediné, co musí dělat, je poslouchat své pány a chovat se jako nemyslící pěšáci v něčím generálním plánu, neměli potřebu dospět. Odhození pověry však člověka nutí k tomu, aby si uvědomil, že nad ním na zemi nic není, což znamená, že je zodpovědný za své vlastní činy (nebo nečinnost); on je ten, jehož úkolem je učinit svět lepším místem; on je ten, kdo musí zajistit, aby společnost fungovala.

Je zřejmé, že existují etatisté, kteří se snaží o pozitivní změny, ale častěji než jindy jejich víra v autoritu mění jejich dobré úmysly ve zlé činy, jejich soucit převrací v násilí a jejich produktivitu mění v palivo útlaku. Například mnozí lidé, kteří vstupují do ozbrojených sil, začínají s ušlechtilým cílem bránit své krajany před nepřátelskými cizími mocnostmi a mnozí z těch, kteří se stávají \enquote{policisty,} tak činí s úmyslem pomáhat lidem a chránit dobré lidi před zlými. Jakmile se však stanou agenty mýtické bestie zvané \enquote{stát,} okamžitě přestávají být obhájci vlastních hodnot a vlastního vnímání dobra a zla a místo toho se stávají vymahateli svévolných rozmarů politiků. V každém \enquote{státě} v dějinách se ti, kteří se vydávali za \enquote{obránce,} rychle, ne-li okamžitě, změnili v agresory. Prvním činem téměř každého režimu je zavedení nějakého druhu \enquote{zdanění,} násilné okrádání poddaných, obvykle pod absurdní záminkou, že tak musí učinit, aby byl schopen chránit lid před lupiči. Je proto ironií, že tolik lidí přijímá myšlenku, že \enquote{stát} je jediným subjektem schopným chránit dobro před zlem. Ve skutečnosti mohou dobré úmysly rádoby ochránců a obránců skutečně sloužit lidstvu pouze v případě, že chybí pověra o moci.

Například soukromá domobrana, která vznikla za účelem obrany určitého obyvatelstva před cizími útočníky -- a taková domobrana podle představ svých členů ani nikoho jiného nemá žádnou zvláštní \enquote{autoritu} -- se bude řídit osobním svědomím každého jednotlivého člena. Taková organizace může být mimořádně účinným prostředkem k uplatnění oprávněné obranné síly, přičemž je imunní vůči obvyklé zkorumpovatelnosti autoritářských \enquote{ochranářských} kšeftů. Člen soukromé domobrany, který netrpí bludem autority, se nemůže a nikdy nebude vymlouvat na to, že \enquote{jen plní rozkazy,} aby se pokusil popřít odpovědnost za své vlastní činy. Pokud použije násilí, on i všichni v jeho okolí vědí, že se \emph{osobně} rozhodl a že je za to osobně zodpovědný a měl by za své činy nést osobní odpovědnost. Stručně řečeno, soukromá, neautoritářská domobrana by se mohla stát represivní pouze tehdy, kdyby se každý jednotlivec v ní osobně rozhodl takto jednat. Naproti tomu \enquote{státní} milice se mohou stát utlačovatelskými v důsledku i \emph{jedné} skutečně zlomyslné osoby v řetězci velení, pokud ti pod ní byli účinně vycvičeni k věrnému plnění rozkazů.

Bez mýtu autority se ne každý bude chovat zodpovědně nebo dobročinně. Když však každý člověk přijme, že je odpovědný sám za sebe, je mnohem méně pravděpodobné, že dobří lidé budou plnit příkazy zlých lidí, jak se to nyní prostřednictvím víry v autoritu neustále děje. Etatisté se často obávají toho, co by mohli někteří jedinci udělat, kdyby je \enquote{stát} neomezoval. Měli by se však obávat toho, co tito jednotlivci mohou udělat, pokud se \emph{stanou} \enquote{státem.} Množství škod, které může napáchat jeden nepřátelský, zlovolný jedinec sám o sobě, není ničím ve srovnání se škodami, které může napáchat jeden nepřátelský, zlovolný \enquote{autoritář} prostřednictvím poslušných, ale jinak dobrých lidí. Jinak řečeno, kdyby zlo páchali pouze zlí lidé, svět by byl mnohem lepším místem, než je dnes, kdy v podstatě \emph{dobří} lidé neustále páchají zlé činy, protože jim to nařídila domnělá \enquote{autorita.}

\section{Jiný druh pravidel}

Bez víry ve stát by si komunity téměř jistě vytvořily \enquote{pravidla,} která by na první pohled mohla připomínat to, čemu se dnes říká \enquote{zákony.} Byl by tu však zásadní rozdíl. Je legitimní a užitečné sepsat a pro všechny zveřejnit prohlášení o důsledcích jednání v určitých věcech. Lidé v jednom městě mohou například dát na vědomí, že pokud vás v jejich městě chytí při krádeži, budete podrobeni nuceným pracím, dokud své oběti třikrát nevrátíte to, co jste ukradli. Nebo lidé v nějaké čtvrti mohou dát najevo, že pokud vás tam přistihnou, jak řídíte opilí, seberou vám auto a srolují ho do jezera. Ale i když by taková nařízení představovala hrozby, zásadně by se lišila od toho, čemu se dnes říká \enquote{zákony,} a to z několika důvodů:

1) Ti, kdo skutečně vyhrožují -- ti, kdo rozhodují o tom, jakou odplatu budou považovat za oprávněnou vůči těm, kdo poškozují nebo ohrožují jejich sousedy -- by sami nesli odpovědnost za vyslovení a uskutečnění takových výhrůžek.

2) Výhrůžky by nevyžadovaly žádné volby ani konsensus. Jeden člověk nebo tisíc lidí společně by mohli vydat varování ve formě: \enquote{Když tě přistihnu při tomhle, udělám ti \emph{tohle}.} Výhrůžky by nebyly vnímány jako \enquote{vůle lidu,} ale pouze jako vyjádření záměrů těch, kteří varování skutečně vydávají.

3) Legitimita takových výhrůžek by se neposuzovala podle toho, kdo je vyslovil, ale podle toho, zda je hrozící následek (v očích pozorovatele) přiměřený spáchanému zločinu. Nikdo by necítil povinnost s takovou hrozbou souhlasit nebo se jí podřídit, pokud by ji považoval za nespravedlivou nebo neoprávněnou.

4) Taková varování by nepředstírala, že mění morálku, ani by nevymýšlela žádné nové \enquote{zločiny,} ani by si nikdo nepředstavoval, že taková varování jsou legitimní jen proto, že byla vydána (tak, jak dnes lidé nahlížejí na autoritářské \enquote{zákony}). Namísto toho by taková varování jednoduše představovala prohlášení o tom, co ti, kdo vyhrožují, považují za oprávněné. Proto by varování namísto autoritářského vzorce \enquote{Tímto \emph{děláme} následující ilegálním} zapadala do této šablony: \enquote{Domnívám se, že pokud uděláte \emph{tohle}, mám právo na to reagovat \emph{takovýmto} způsobem.}

Mnozí lidé, kteří byli vychováni k uctívání \enquote{autorit,} by se takového necentralizovaného způsobu interakce mezi lidmi, který by byl \enquote{volný pro všechny,} děsili. \enquote{Ale co když,} zeptá se etatista, \enquote{někdo napíše výhrůžku, že když se mi nebude líbit tvoje náboženství, tvůj účes nebo tvůj výběr stravy, zabiju tě?.} Zkoumání této otázky v kontextu společnosti, která stále trpí pověrou o autoritě, a v kontextu společnosti \emph{bez} takové víry ukazuje, jak nebezpečná pověra o autoritě ve skutečnosti je. Je pravda, že při absenci víry ve stát by jednotlivec stále mohl v neodůvodněných situacích vyhrožovat násilím. Nejde o to, že by všichni automaticky mysleli a chovali se správně, kdyby neexistovali vládci, ale o to, že takové zlovolné tendence lidí by byly mnohem méně nebezpečné a destruktivní \emph{bez} víry v autoritu, která by je legitimizovala.

Srovnejte například, co se stane, když se někteří jedinci vehementně brání konzumaci alkoholu, a když ji \enquote{autorita} zakáže. Je možné (i když nepravděpodobné), že by jedinec v bezstátní společnosti mohl sám od sebe prohlásit: \enquote{Považuji konzumaci alkoholu za hřích, a pokud zjistím, že piješ, přijdu k tobě domů s pistolí, abych tě srovnal do latě.} Každý, kdo by tak učinil, by byl téměř jistě přesvědčen, když ne slušnou argumentací, tak hrozbou násilné odvety, že by svou hrozbu neměl uskutečnit a měl by s takovým vyhrožováním přestat.

Je zřejmé, že jeden člověk by sám o sobě nemohl způsobit útlak milionům pivařů. Dokonce i mezi dalšími, kteří také považují pití alkoholu za hřích, i kdyby jich byla většina, by se jen málokdo cítil oprávněn pokoušet se násilím vnucovat své názory ostatním. Ať už by si uvědomovali, že taková agrese je neoprávněná, nebo by se prostě báli toho, co by se jim mohlo stát, kdyby se o to pokusili, v každém případě by se násilnému konfliktu vyhnuli.

Představme si naopak, že by skupina lidí s nálepkou \enquote{stát} prohlásila alkohol za \enquote{nelegální} a vytvořila těžce ozbrojenou bandu vymahatelů, která by pronásledovala a věznila každého, kdo by byl přistižen při držení alkoholu. Protože se to skutečně stalo, není třeba teoretizovat o výsledcích. S příslibem nápravy většiny společenských neduhů a s podporou veřejnosti zavedla vládnoucí třída USA v roce 1920 prohibici alkoholu. Spotřeba alkoholu pokračovala, mírně se snížila, a okamžitě vznikl černý trh s výrobou a distribucí alkoholu. Nesmírně výnosný, ale \enquote{nelegální} trh vedl k násilným konfliktům, skokovému nárůstu organizovaného zločinu a další kriminality, k rozsáhlé korupci ve \enquote{státu} a k brutálním pokusům o potlačení obchodu s alkoholem. Když viděli skutečné výsledky prohibice, většina lidí se brzy postavila proti ní a požadovala zrušení osmnáctého dodatku, který prohibici na federální úrovni povoloval. A samozřejmě po skončení prohibice skončilo i \emph{veškeré} související násilí -- \enquote{státní} i soukromé.

Na tomto příkladu a nespočtu dalších je vidět, že většina lidí, ponechána sama sobě, se nebude snažit násilím vnucovat své preference ostatním, ale půjde z cesty, aby se vyhnula násilným konfliktům. Pokud však existuje \enquote{stát,} který mohou lidé využít k násilnému vnucování svých hodnot ostatním, rádi jej o to požádají a nebudou se za to stydět ani cítit provinile. Kdyby každý člověk, který vyslovil nebo se pokusil prosadit nějakou hrozbu (nebo \enquote{pravidlo,} jak by se to dalo nazvat), musel nést osobní odpovědnost za to, že tak učinil, a musel by sám nést riziko, jen velmi málo lidí by tak ochotně ohrožovalo své bližní. Ale vzhledem k tomu, že je k dispozici prostředek \enquote{autority,} každý, kdo věří ve stát, pravidelně vyhrožuje všem svým sousedům a nepřijímá za to žádnou odpovědnost a nepřebírá žádné riziko. Stručně řečeno, víra v autoritu dělá z každého, kdo v ni věří, násilníka \emph{a} zbabělce.

\section{Organizace bez \enquote{autority}}

Když už jsme zmínili, v čem by se lidská společnost změnila, kdyby neexistoval mýtus autority, je stejně důležité si všimnout věcí, které by se nezměnily. Z nějakého důvodu se někteří lidé domnívají, že \enquote{anarchie} -- společnost bez státu -- se rovná \enquote{každý sám za sebe,} kdy si každý člověk musí sám vypěstovat jídlo, postavit dům atd. Z takového přesvědčení vyplývá, že lidská spolupráce a obchod probíhají jen proto, že je někdo \enquote{ve vedení.} Tak tomu samozřejmě není a nikdy nebylo. Lidé obchodují a spolupracují pro vzájemný prospěch, jak je vidět na mnoha milionech obchodů a transakcí, které již probíhají bez jakéhokoli zapojení \enquote{státu.}

Supermarkety jsou příkladem vysoce organizovaného, úžasně efektivního způsobu distribuce potravin, na němž se podílí mnoho tisíc jednotlivců, z nichž nikdo není k účasti nucen, ale každý tak činí pro svůj vlastní prospěch. Každý, od zemědělců přes řidiče kamionů, skladníky, pokladní, manažery obchodů až po majitele celých obchodních řetězců, dělá to, co dělá, protože za to získává osobní prospěch. Nikdo není \enquote{legálně} povinen vyrobit jediné sousto jídla pro někoho jiného, a přesto jsou stovky milionů lidí nasyceny, a to dobře, velkým množstvím potravinářských výrobků vysoké kvality, ale za nízkou cenu, což je v podstatě anarchistický systém výroby a distribuce potravin.

Je to důsledek lidské povahy a jednoduché ekonomiky. Tam, kde je nějaký výrobek nebo služba potřeba, se na jejich poskytování dá vydělat. A tam, kde se dají vydělat peníze, bude o ně soupeřit řada lidí nebo skupin lidí, kteří se budou snažit vyrábět lepší a levnější produkty. Takový \enquote{systém} -- který ve skutečnosti žádným systémem není -- automaticky \enquote{trestá} ty, jejichž výrobky jsou horší nebo příliš drahé, a odměňuje ty, kteří najdou způsob, jak lidem poskytnout to, co chtějí, za lepší cenu. A vzdát se mýtu o autoritách by tomu ani v nejmenším nebránilo.

Ve skutečnosti pověra o autoritě neustále překáží lidem, kteří se snaží organizovat ve vzájemný prospěch, tím, že jim do cesty hází \enquote{daně,} licenční požadavky, předpisy, inspektory a další \enquote{právní} překážky. Dokonce i \enquote{zákony} údajně určené k ochraně spotřebitelů obvykle nedělají nic jiného, než že \emph{omezují} možnosti, které mají spotřebitelé k dispozici. Konečným výsledkem je, že mnoho podnikatelů, kteří by se jinak museli soustředit na výrobu lepšího výrobku za lepší cenu, se místo toho soustředí na lobbování u \enquote{státu,} aby dělal věci, které znevýhodňují nebo ničí konkurenční podniky. Protože mechanismus \enquote{státu} vždy spočívá v použití síly, nemůže nikdy pomoci hospodářské soutěži; může ji pouze brzdit. Jinými slovy, spíše než aby byl pro organizovanou společnost nezbytný, je mýtus autority největší \emph{překážkou} tomu, aby se lidé organizovali ve vzájemný prospěch.

\section{Obrana bez \enquote{autority}}

Ti, kdo trvají na tom, že \enquote{stát} je nezbytný, často vznášejí otázku obrany a ochrany a tvrdí, že společnost bez \enquote{státu} by znamenala, že by si každý mohl dělat cokoli, že by neexistovaly žádné normy chování, žádná pravidla, žádné důsledky pro ty, kdo se rozhodnou krást nebo vraždit, a že by se proto společnost zhroutila do neustálého násilí a chaosu. Takové obavy však vycházejí z hlubokého nepochopení lidské povahy a toho, co \enquote{stát} je a co není.

Obrana proti agresorům nevyžaduje žádnou zvláštní \enquote{autoritu,} žádnou \enquote{legislativu,} žádný \enquote{zákon} ani žádné \enquote{strážce zákona.} Obranná síla je ze své podstaty oprávněná bez ohledu na to, kdo ji provádí, a bez ohledu na to, co říká jakýkoli \enquote{zákon.} A mít formální, organizované prostředky k zajištění takové obranné síly pro komunitu také nevyžaduje \enquote{stát} nebo \enquote{zákon.} Každý jednotlivec má právo bránit se sám nebo bránit někoho jiného. Může se rozhodnout, že si najme někoho jiného, aby mu poskytl obranné služby, ať už proto, že není fyzicky schopen se bránit sám, nebo jen proto, že by za to raději zaplatil někomu jinému. A pokud se řada lidí rozhodne zaplatit si organizaci vycvičených bojovníků se zbraněmi, vozidly, budovami a dalšími prostředky potřebnými k obraně celého města, mají toto právo také.

V tomto okamžiku většina věřících ve stát zaprotestuje a řekne: \enquote{To je všechno, co stát je.} Ale tak to není. A právě zde se ukazuje rozdíl. To, na co jednotlivec \emph{nemá} právo -- na co nemá právo žádná skupina lidí, bez ohledu na to, jak je velká -- je najmout někoho jiného (jednotlivce nebo skupinu), aby udělal něco, na co jakýkoli průměrný jednotlivec \emph{nemá} právo. Nemohou si oprávněně najmout někoho, aby spáchal loupež, i když tomu říkají \enquote{zdanění,} protože průměrný jednotlivec nemá právo krást. Nemohou si oprávněně najmout někoho, kdo by špehoval a násilně ovládal volby a chování jejich sousedů, i když tomu říkají \enquote{regulace.} Lidé v bezstátní společnosti by se cítili oprávněni najmout si někoho, kdo by použil sílu, pouze ve velmi omezených způsobech a ve velmi omezených situacích, v nichž má \emph{každý} jednotlivec právo použít sílu: na obranu proti agresorům. Naproti tomu většina toho, co takzvaní \enquote{ochránci} ve \enquote{státu} dělají, je \emph{páchání} agresivních činů, nikoli obrana proti nim.

Část toho, co je dnes klasifikováno jako \enquote{práce policie} -- vlastně vše, co \enquote{policie} dělá a co je ve skutečnosti legitimní, ušlechtilé, spravedlivé a společnosti užitečné -- by existovalo i bez mýtu o autoritě. Vyšetřování přestupků a zadržování skutečných zločinců -- tedy lidí, kteří škodí ostatním, nikoliv pouze lidí, kteří neposlouchají politiky -- by pokračovalo i bez mýtu autority jako něco, co by téměř každý chtěl a za co by byl ochoten platit. Svědčí o tom skutečnost, že kromě \enquote{ochranných} služeb \enquote{státu,} které je nucen financovat každý, již existují soukromí detektivové a soukromé bezpečnostní agentury.

Rozdíl by byl jen jeden, i když zásadní: na ty, kdo vykonávají práci vyšetřovatele a ochránce, by se při absenci pověry o autoritě vždy pohlíželo jako na osoby, které mají naprosto stejná práva jako všichni ostatní. Ačkoli by pravděpodobně byli pro svou práci lépe vybaveni a kvalifikovanější než průměrný občan, jejich činy by byly posuzovány podle stejných měřítek, podle kterých by byly posuzovány činy kohokoli jiného, což v případě takzvaných \enquote{strážců zákona} vůbec neplatí. Soukromí poskytovatelé ochrany by také posuzovali své \emph{vlastní} činy nikoli podle toho, zda jim nějaká \enquote{autorita} nařídila něco udělat, nebo zda jejich činy byly \enquote{státem} považovány za \enquote{legální,} ale podle toho, zda tyto činy byly podle jejich vlastního osobního názoru ve své podstatě oprávněné. Nejenže by výmluva na to, že \enquote{jen plní rozkazy,} nepřesvědčila širokou veřejnost, ale ani sami agenti by se takovou výmluvou nemohli, dokonce ani ve své vlastní mysli, vyhnout odpovědnosti za své činy, protože by se k nim nikdo nehlásil jako k \enquote{autoritě.}

Na neautoritářskou \enquote{policii} -- pokud by se tak vůbec dala nazvat -- by se pohlíželo zcela jinak než na \enquote{státní} agenty nyní. Nebyli by vnímáni jako ti, kdo mají právo dělat něco, na co nemá právo kterýkoli jiný člověk. Mohli by chodit na místa, vyslýchat lidi, používat sílu nebo dělat cokoli jiného pouze v situacích, kdy by kdokoli jiný byl oprávněn dělat totéž. V důsledku toho by průměrný člověk neměl důvod pociťovat v jejich přítomnosti jakoukoli nervozitu nebo sebevědomí, jak to nyní činí většina lidí v přítomnosti \enquote{strážců zákona.} Lidé by necítili o nic větší povinnost podrobit se výslechu, prohlídce nebo čemukoli jinému, co by požadovali soukromí ochránci, než kdyby takové požadavky vznesl nějaký cizinec na ulici. A pokud by se soukromý ochránce stal hrubým, nebo dokonce násilným, jeho oběť by měla právo reagovat stejně, jako kdyby se takto choval kdokoli jiný. A co je ještě důležitější, jedinec, který by se bránil agresi ze strany soukromého ochránce, by v takovém případě měl podporu svých sousedů, protože jeho sousedé by si nepředstavovali žádnou povinnost se někomu klanět kvůli nějakému odznaku nebo nějakému \enquote{zákonu.}

Nejlepší ochranou proti tomu, aby se obranná organizace stala zkorumpovanou nebo \enquote{nekontrolovatelnou,} je nakonec možnost zákazníků jednoduše přestat platit. Je zřejmé, že nikdo nechce platit nějakému gangu za to, že ho utlačuje, ale většina lidí také nechce platit gangu za to, že utlačuje někoho jiného. Stejně jako si průměrný člověk přeje, aby byli zloději a vrazi dopadeni a zastaveni, chce také dohlédnout na to, aby nebylo ubližováno nevinným. Kdyby zákazníci nějaké soukromé ochranné společnosti zjistili, že jejich \enquote{ochránci} obtěžují a napadají nevinné lidi -- tedy přesně ten typ chování, kterému mají zabránit -- zákaznická základna by okamžitě zmizela a násilníci by byli bez práce. A kdyby se při absenci jakékoli deklarované \enquote{autority} násilníci rozhodli, že se pokusí \emph{přinutit} své bývalé zákazníky, aby platili dál, odezva ze strany lidí by byla rychlá a tvrdá, protože nikdo by necítil žádnou \enquote{zákonnou} povinnost nechat se utlačovat.

Neautoritářský systém ochrany by také postrádal další obzvláště směšný aspekt téměř všech \enquote{státních} forem \enquote{ochrany.} Pro \enquote{státy} je standardní nejen to, že nutí lidi financovat \enquote{obranné} systémy, ale že odmítají lidem dokonce sdělit, co všechno financují. Americká \enquote{vláda,} a zejména CIA (ačkoli mnoho dalších agentur se rovněž podílí na tajných operacích), vynaložila desítky let a \emph{biliony} dolarů, z nichž velká část stále zůstává nezúčtována, na operace, o nichž její \enquote{zákazníci} -- Američané -- nesmějí vědět. Každý, kdo by se pokusil Američanům sdělit, co všechno financují, by byl uvězněn -- nebo by dopadnul ještě hůř -- za porušení \enquote{národní bezpečnosti.}

S téměř neomezenou mocí, téměř neomezenými finančními prostředky a povolením konat všechny své činy v utajení je naprosto absurdní si představovat, že by armáda a CIA dělaly jen užitečné a spravedlivé věci. Američané se totiž stále častěji dozvídají, že CIA se po celá desetiletí zabývala obchodem s drogami a zbraněmi, mučením, vraždami, kupováním vlivu u zahraničních vlád, dosazováním loutkových diktátorů a nejrůznějšími dalšími destruktivními a zlými praktikami. Dokonce i prezident Harry Truman, který CIA vytvořil, později prohlásil, že by to nikdy neudělal, kdyby věděl, že se z ní stane \enquote{americké gestapo.} Jakákoli soukromá společnost, která by nabízela ochranné nebo obranné služby, by nezískala vůbec žádné zákazníky, kdyby její prodejní slogan zněl: \enquote{Když nám dáte obrovské sumy peněz, ochráníme vás; jen vám neřekneme, za co platíte, a neřekneme vám, co děláme a jak to děláme.} Jediný důvod, proč \enquote{stát} získává finanční prostředky na základě tak směšného předpokladu, je ten, že své peníze získává násilným donucením, nikoliv dobrovolným obchodem. Lidé nemají možnost volby, zda ji financovat, nebo ne.

Existuje ještě jeden absurdní aspekt \enquote{ochrany} prostřednictvím \enquote{státu,} který by u soukromých poskytovatelů obrany a ochrany nikdy nenastal. Autoritářské režimy pod rouškou \enquote{regulace zbraní} a jiných \enquote{zákonů} o zbraních často násilím brání lidem v možnosti bránit se, přičemž směšně tvrdí, že se tak děje pro bezpečnost právě těch lidí, kteří jsou odzbrojováni. Ti, kdo jsou u moci, dobře vědí, že odzbrojená veřejnost je bezmocná veřejnost, a to je přesně to, co tyrani chtějí. Představa, že člověku, kterému nevadí porušování \enquote{zákonů} proti krádeži nebo vraždě, bude vadit porušování \enquote{zákonů} o zbraních, je absurdní. Statistiky kriminality i zdravý rozum ukazují, že přijetí \enquote{zákona} proti soukromému držení zbraní bude mít dopad pouze na osoby \enquote{dodržující zákony,} přičemž výsledkem bude, že v zásadě dobří lidé budou nakonec méně schopni bránit se agresorům. A to je přesně to, co politici chtějí, protože mají největší a nejsilnější bandu agresorů v okolí. Není třeba dodávat, že pokud někdo hledá ochranu před agresory, nebude dobrovolně platit společnosti, aby mu násilím odebrala jeho vlastní prostředky sebeobrany.

Násilné střety mezi policií a civilním obyvatelstvem by se navíc zřejmě omezily nebo vůbec nevyskytovaly, kdyby lidé jednoduše přestali financovat všechny \enquote{ochránce,} kteří se stali agresory. Například velká část rasového napětí a násilí v USA bylo důsledkem toho, že bílí \enquote{ochránci zákona} utlačovali a zneužívali černošské civilisty. Místo toho, aby \enquote{zákon} působil jako civilizační vliv, byl používán jako záminka k násilné agresi. Kdyby si obyvatelé černošské čtvrti mohli vybrat, zřejmě by dobrovolně neplatili za to, že je bělošští sadističtí násilníci pravidelně zastrašují a napadají. Mnoho dalších násilných střetů v USA i jinde bylo také výsledkem toho, že lidé byli naštvaní na to, co jim jejich vládnoucí třída provádí. Patří sem například masakr tisíců protestujících na náměstí Nebeského klidu čínskou armádou v roce 1989, zabití několika protiválečných demonstrantů Národní gardou v Kent State v Ohiu v roce 1970 atd.

Ve Spojených státech stále častěji končí veřejné demonstrace a protesty proti \enquote{státní} politice autoritářskými zásahy proti demonstrantům, při nichž se používá slzný plyn, obušky, tasery, gumové projektily atd. Je zřejmé, že žádná skupina lidí by dobrovolně nezaplatila za gang, který by stejným lidem násilím zabránil říci svůj názor. Důležitější je, že motivací takových protestů je téměř vždy nespokojenost s tím, co představitelé \enquote{státu} dělají proti vůli lidu (alespoň části lidu). Kdyby každý člověk mohl utrácet své vlastní peníze, místo aby byl nucen financovat centralizovanou, autoritářskou agendu, nebyl by důvod, aby k většině protestů tohoto typu a k následným střetům vůbec docházelo.

Neautoritářský ochránce by dělal jen věci, které by on i jeho zákazníci považovali za oprávněné, což by pravděpodobně bylo zakotveno ve smlouvě, v níž by se ochránce zavázal poskytovat konkrétní služby za určitý poplatek. Srovnejte to se standardní \enquote{státní} verzí \enquote{ochrany:} \enquote{Násilím vám sebereme tolik peněz, kolik budeme chtít, a sami rozhodneme, co pro vás uděláme, pokud vůbec něco.}

Většina lidí chce zastavit agresory a ochránit nevinné. Na volném trhu může společnost uspět tak, že dá zákazníkům to, co chtějí. Na rozdíl od \enquote{státu,} kdyby se soukromá obranná společnost musela spoléhat na ochotné zákazníky, měla by obrovskou motivaci nebýt bezohledná, nehospodárná, zneužívající nebo zkorumpovaná. Kdyby lidé mohli odejít jinam, vždy by se soutěžilo o to, kdo dokáže nejefektivněji zajistit skutečnou spravedlnost. Aby soukromá ochranná společnost uspěla, musela by svým zákazníkům prokázat, že: 1) umí velmi dobře zjistit, kdo je vinen a kdo ne; 2) umí velmi dobře zajistit, že nevinní nebudou obtěžováni, napadáni nebo pomlouváni; 3) umí velmi dobře zajistit, že skutečně nebezpeční lidé budou dopadeni a bude jim zabráněno v dalším škodění; 4) umí velmi dobře zajistit, že oběti trestných činů dostanou jakoukoli přípustnou náhradu; a 5) umí velmi dobře zajistit, že ti, kteří udělali něco špatného, ale nepotřebují být zcela odstraněni ze společnosti, se dostanou do prostředí, kde se jejich přístup a chování může skutečně zlepšit.

Naproti tomu \enquote{státní} zástupci se specializují na to, aby vždy démonizovali obviněné, a vždy mají motivaci dosáhnout usvědčení (nebo vynuceného přiznání známého jako \enquote{dohoda o vině a trestu}) bez ohledu na vinu či nevinu obviněného; \enquote{státní} soudy neustále propouštějí lidi, kteří stále představují zjevné nebezpečí pro ostatní, zatímco miliony lidí, kteří nikomu neublížili, zůstávají pod zámkem; \enquote{státní} vězeňský systém kvůli tomu, jak jsou vězni ponižováni, zneužíváni a napadáni \enquote{dozorci} i ostatními vězni, dělá z frustrovaných a naštvaných lidí lidi ještě více frustrované a naštvané, z nevinných lidí dělá zločince a ze zločinců ještě horší zločince. A Američané jsou nuceni tento destruktivní systém financovat, ať už chtějí, nebo ne.

Dalším důležitým bodem je, že v případě soukromé ochranné společnosti, pokud jeden \enquote{ochránce} začne zneužívat, pověst a kariéra každého dalšího ochránce závisí na odhalení a odstranění násilníka. Naproti tomu je dnes všeobecně známo, že \enquote{státní} policejní složky budou chránit především své vlastní. Když je jeden policista přistižen při něčem korupčním, \enquote{nezákonném} nebo násilném, téměř bez výjimky mu všichni ostatní policisté pomohou tento čin ututlat nebo ho budou obhajovat. Fungují na základě mentality gangů, protože lidé, kteří jsou nuceni platit jejich platy, nejsou \emph{lidé}, kterým se ve skutečnosti musí zodpovídat. Stejně jako většina \enquote{státních} zaměstnanců se zodpovídají politikům a veřejnost považují za dobytek, nikoli za zákazníky. Naproti tomu široká veřejnost by soukromé ochránce považovala za své přátele, spojence a zaměstnance, a co je důležitější, za sobě rovné. Nevnímala by je jako \enquote{autoritu,} před kterou se musí plazit, ani jako stálou potenciální hrozbu, které je třeba se obávat. Všichni, včetně najatého ochránce, by uznali, že ochránce nemá více práv než kdokoli jiný. Všichni by věděli, že pokud by se najatý ochránce někdy dopustil krádeže, přepadení nebo vraždy, bylo by na něj pohlíženo a zacházeno s ním přesně tak, jak by bylo pohlíženo a zacházeno s kterýmkoli jiným násilníkem.

Skutečný ochránce, který brání svobodu a majetek, nejenže nevyžaduje víru v autoritu, ale vyžaduje \emph{neexistenci} této víry. Ten, kdo si představuje, že má právo násilně ovládat všechny ostatní -- i kdyby jen \enquote{omezeně} -- bude podle toho s lidmi jednat. \enquote{Strážce zákona,} který rozdává pokuty za obskurní přestupky, zadržuje a vyslýchá lidi bez oprávněného důvodu a zřejmě stále hledá důvod, proč zasahovat do každodenního života lidí, není ochránce a nezaslouží si úctu ani spolupráci. Neautoritářský ochránce by naproti tomu nebyl ničím jiným než normální lidskou bytostí se stejnými právy jako všichni ostatní, i když by byl možná častěji ozbrojen a lépe vycvičen pro fyzický střet než většina ostatních. Byl by vnímán spíše jako soused, kterého je třeba zavolat v případě potíží, než jako zástupce bandy násilníků, která v první řadě slouží vládnoucí třídě. A práce ochránce by bez jakékoli zvláštní \enquote{autority,} moci nebo postavení přitahovala hlavně ty, kdo chtějí skutečně chránit nevinné, ale nepřitahovala by ty, kdo chtějí jen možnost uplatňovat moc a nadvládu nad druhými -- což je lidský nedostatek, který práce moderních \enquote{strážců zákona} živí. Tím není řečeno, že by soukromí ochránci nikdy neudělali nic špatného. Stále by to byli lidé, schopní špatného úsudku, nedbalosti, a dokonce i zlého úmyslu, stejně jako všichni ostatní. Neměli by však \enquote{legální} \emph{povolení} ke špatnému jednání a neměli by žádný \enquote{systém,} žádný \enquote{zákon,} žádnou \enquote{autoritu,} kterou by mohli obviňovat ze svých činů nebo za kterou by se mohli schovat, aby se vyhnuli hněvu svých obětí. Pokud by se někdy chovali jako agresoři, odplata by byla jistá a rychlá. V populaci, která se vzdala pověrčivosti vůči autoritám, by jakákoli skupina ochránců, která by se rozhodla stát se skupinou vyděračů, násilníků a tyranů, nebyla \enquote{přehlasována,} zažalována ani regulována nějakou \enquote{autoritou.} Byla by zastřelena. Jediné, co umožňuje dlouhodobý a rozsáhlý útlak jakéhokoli ozbrojeného obyvatelstva, je víra v autoritu mezi \emph{obětmi} útlaku. Bez ní je nemožné si lidi dlouhodobě podmaňovat a ovládat je.

\section{Odstrašování a podněcování}

Někteří předpokládají, že nebýt \enquote{státu,} mohli by si podvodníci dělat, co se jim zlíbí, bez jakýchkoli následků. To opět ukazuje na hluboké nepochopení lidské povahy a toho, co je \enquote{stát.} Ve skutečnosti víra v autoritu nepřispívá k účinnosti žádného systému obrany a ochrany.

Lidé, kteří používají agresi vůči druhým, jako je přepadení, krádež a vražda, zjevně nejsou omezováni vlastní morálkou nebo respektem k sebevlastnictví svých obětí. Mohou se však rozhodnout, že se určitého trestného činu nedopustí, pokud si představují riziko, že by mohli ublížit sami sobě. Tomu se říká \enquote{odstrašování.} A odstrašování z definice nezávisí na apelování na svědomí útočníka, ale využívají útočníkova pudu sebezáchovy. Jednoduše řečeno, poselství, které působí na skutečné zločince, není \enquote{Nedělej to, protože je to špatné;} poselství je \enquote{Nedělej to, nebo se ti něco stane.} Údajná morální spravedlnost nebo \enquote{autorita} hrozby vůči potenciálnímu agresorovi je pro účinnost odstrašování irelevantní. Ať už se jedná o \enquote{policistu,} psa, rozzlobeného majitele domu, nebo dokonce jiného zloděje, jedinou otázkou v mysli útočníka je, zda pravděpodobně utrpí bolest nebo smrt, pokud se pokusí někoho okrást nebo napadnout.

Odstrašování od jiných typů špatného chování, které nejsou tak závažné nebo zjevné jako krádeže nebo napadení, také nevyžaduje \enquote{autoritu.} Někteří tvrdí, že bez \enquote{státních} inspektorů a regulačních orgánů by každý podnikatelský subjekt vydával nekvalitní a nebezpečné výrobky. Takové tvrzení však opět vychází z hlubokého nepochopení lidské povahy a ekonomie. Bez ohledu na to, jak chamtivý nebo sobecký může podnikatel být, nemůže být dlouhodobě úspěšný, pokud prodává výrobky, které se nelíbí jeho zákazníkům. Ten, kdo vědomě prodává vadný výrobek nebo zkažené potraviny, bude mít jen málo zákazníků, pokud vůbec nějaké. O tom svědčí mnoho velmi drahých \enquote{stahování} výrobků, která mnohé společnosti dobrovolně provádějí, a to i v případě relativně banálních závad nebo problémů. Na rozdíl od současné situace, kdy je moc \enquote{státu} využívána k podpoře a ochraně nezodpovědných a destruktivních korporací, by se na skutečně volném trhu s informovanými spotřebiteli a otevřenou konkurencí korupce a kriminalita nevyplácely a podniky by se nemohly chránit před důsledky své nezodpovědnosti.

\enquote{Státní} inspektoři a regulátoři jsou vedeni motivací ukládat lidem pokuty a prosazovat \enquote{zákony} a \enquote{nařízení} bez ohledu na to, zda mají nějaký smysl. Naproti tomu systém soukromých inspektorů, který se zodpovídá pouze lidem, kteří chtějí vědět, co je bezpečné, a který nemá žádnou donucovací pravomoc, nemá motivaci zasahovat do podnikání nebo si vymýšlet věci, na které by si mohli stěžovat. Podniky by mohly \emph{dobrovolně} přizvat soukromé kontroly svých výrobků nebo zařízení, jak to již dělají Underwriters Laboratories (\enquote{UL}), Consumer Reports a další, aby mohly veřejnosti ukázat nezaujatý názor na to, jak bezpečné a spolehlivé jejich výrobky jsou. Mnoho společností to dnes dělá navíc k tomu, že musí proskakovat všemi byrokratickými obručemi, které jim \enquote{státy} staví do cesty.

Podobným neautoritativním způsobem by se dalo řešit mnoho dalších záležitostí. Soukromí stavební inspektoři, které již využívá mnoho realitních společností, by měli za úkol jménem potenciálních kupujících zjistit, jak bezpečná a pevná je budova. Kromě soukromých inspektorů by restaurace mohly jednoduše pozvat potenciální zákazníky, aby si sami prohlédli jejich zařízení. Všechny tyto činnosti by byly dobrovolné. Podnik by se mohl rozhodnout, že neumožní žádné inspekce, a potenciální zákazníci by se mohli rozhodnout, zda tento podnik navštíví, či nikoli.

Skutečnost, že se předpokládá, že tolik věcí by měla řešit \enquote{autorita,} je známkou intelektuální lenosti. Zákazníci chtějí kvalitní výrobky a podnikatelé, kteří chtějí být úspěšní, musí kvalitní výrobky poskytovat. Je tedy v zájmu obou, aby byli schopni objektivně prokázat kvalitu nabízených výrobků. Na rozdíl od stereotypu zlého, chamtivého a ziskuchtivého podnikatele je ve svobodné společnosti možné zbohatnout tak, že se budou poskytovat výrobky a služby, které jsou pro zákazníka skutečně přínosem. Téměř všechna nekalá schémata, která jsou dlouhodobě zisková, jsou ta, která jsou násilně vytvořena nebo podporována \enquote{státem,} jako například podvod \enquote{frakčního bankovnictví,} \enquote{legální} padělání zvané \enquote{měnová politika,} kšeftování se soudními spory atd.

I bez \enquote{státu} by občas docházelo k vážným konfliktům. Předpokládejme například, že by továrna vypouštěla toxický odpad do řeky a zabíjela všechny ryby po proudu na cizím pozemku, což by představovalo formu nedovoleného vniknutí a ničení majetku. Absence \enquote{autority} by nebránila obětem, aby s tím něco dělaly; ve skutečnosti jim to může usnadnit. Místo žaloby u \enquote{státního} soudu, kde lze soudce podplatit, aby podpořili miliardový byznys, by reakcí mohlo být něco účinnějšího, i když to vypadá méně civilizovaně. Lidé, kteří žijí u řeky, mohou udělat něco tak jednoduchého, jako říci majiteli továrny, že pokud bude nadále nechávat své znečištění proudit na jejich pozemky, fyzicky zničí jeho továrnu.

Samozřejmě mohou existovat i slušnější a mírumilovnější způsoby řešení problému, jako je bojkot nebo medializace nekalého jednání. Ať tak či onak, lidé mohou vytvořit účinný odstrašující prostředek proti nevhodnému chování, \emph{obzvláště} pokud se nejedná o \enquote{stát,} který lze podplatit a zkorumpovat. Mnohé příspěvky na volební kampaně se dnes rovnají jen úplatkům, aby se \enquote{státní} regulátoři \enquote{dívali jinam.} Stejně tak \enquote{státní} soudy mohou snadno najít důvody pro zamítnutí téměř každé žaloby, čímž umožňují bohatým zločincům (těm, kteří mají skutečné oběti) prosperovat.

Klišé o chamtivém a zlém podnikateli často opomíjí skutečnost, že rozsáhlé zločiny jsou obvykle páchány ve spolupráci se \enquote{státními} úředníky. Bez ochrany ze strany \enquote{státu} by i ten nejchamtivější a nejbezcitnější podnikatel měl obrovskou motivaci nerozzlobit své zákazníky natolik, aby přestali kupovat jeho výrobky, nebo natolik, aby proti němu reagovali násilně.

Většina lidí by se většinou zdráhala použít násilí, protože by věděla, že odpovědnost i rizika s tím spojená ponesou jen oni sami. Existovala by obrovská motivace řešit spory a neshody mírovou cestou a vzájemnou dohodou. Když naopak převládá víra ve stát, neexistuje \emph{žádná} motivace řešit věci mírovou cestou, protože vítězství v \enquote{politickém} boji nepředstavuje pro zastánce násilí prostřednictvím \enquote{státu} žádné riziko. Bez vládnoucí třídy, která by mohla fňukat a legislativně všem vnucovat nějakou centrální agendu, by lidé byli nuceni jednat mezi sebou jako rozumní dospělí, a ne jako ufňukané nezodpovědné děti. Lidem by mnohem lépe posloužily pokusy o spolupráci a mírové kompromisy než hádky o to, kdo se chopí meče \enquote{státu.} Až přestanou být šikana a agrese uznávány jako legitimní formy lidské interakce, naučí se lidé \enquote{hrát fér} z nutnosti.

\section{Anarchie v praxi}

Ačkoli se mnozí lidé děsí představy \enquote{anarchie,} pravdou je, že téměř každý člověk zažívá \enquote{anarchii} pravidelně. Když jdou lidé nakupovat potraviny nebo si prohlížet zboží v obchodním centru, vidí výsledky neautoritářské, vzájemné spolupráce. Nikdo není nucen vyrábět žádný z nabízených produktů, nikdo není nucen nic prodávat a nikdo není nucen nic kupovat. Každý jedná ve svém vlastním zájmu a všichni zúčastnění -- výrobce, prodávající i kupující -- z této dohody profitují. Všichni jednotlivci z toho mají prospěch a obecně společnost z toho má prospěch, aniž by se na tom podílel jakýkoli nátlak nebo vládci. Existuje nespočet příkladů vzájemně dobrovolných, kooperativních, mírumilovných, efektivních a užitečných akcí a organizací, které nezahrnují \enquote{stát.} Nicméně i když existuje nespočet snadno dostupných příkladů toho, jak efektivní, organizovaná a produktivní je \enquote{anarchistická} interakce ve srovnání s téměř všemi \enquote{státními} počiny, lidé si stále představují, že by vzájemná interakce lidských bytostí jako rovných \emph{pořád} vedla k chaosu a zmatku.

Když se auta potkají na složité křižovatce nebo když se lidé míjejí na chodníku, je to \enquote{anarchie} v akci. Miliardykrát denně se lidé řadí, nechávají místo ostatním a tak dále, aniž by jim to nějaká \enquote{autorita} přikazovala. Někdy jsou lidé bezohlední, ale i tehdy jen velmi zřídka dojde k vážnému konfliktu -- k něčemu vážnějšímu než hrubému gestu nebo vzteklému výkřiku. K potenciálním konfliktům, od velmi drobných věcí až po vážnější záležitosti, dochází každý den miliardkrát a v naprosté většině případů se vyřeší bez násilí a bez účasti jakékoli \enquote{autority.} I v případě závažnějších problémů lidé často nacházejí způsoby, jak dosáhnout vzájemné dohody. Zatímco organizované, nevládní metody řešení sporů -- s využitím rozhodců, vyšetřování a vyjednávání -- mohou pokojně vyřešit i závažné neshody, většina střetů zájmů se tak daleko nikdy nedostane. Většina lidí se většinou snaží případným střetům s ostatními vyhnout nebo je rychle urovnat.

Často je tu ještě jiný faktor, ačkoli někteří lidé by jej označili za známku vrozené dobroty člověka. Většina lidí prostě nechce mít potíže a stres, které s sebou přinášejí konfrontace, a zejména nechce riskovat, že se objeví \emph{násilné} konfrontace. Mnoho lidí \enquote{nastavuje druhou tvář} poměrně často, ne nutně proto, že jsou trpěliví a milující, ale prostě proto, aby se nemuseli obtěžovat časově náročnými a marnými hádkami. Mnozí, když se setkají s někým, kdo dělá něco nepříjemného, to prostě \enquote{nechají plavat,} protože mají důležitější věci na starosti. Ve většině lidí je silná tendence \enquote{vyjít s někým,} i když jen kvůli vlastnímu prospěchu. A kdyby neexistovala žádná \enquote{autorita,} ke které by se dalo utéct -- žádný obrovský stát maminek a tatínků, u kterého by se dalo brečet -- lidé by řešili věci jako dospělí mnohem častěji než nyní. Tím nechci říci, že bez \enquote{autority} by každý názorový rozdíl skončil smírně a spravedlivě, ale \emph{dostupnost} obřího \enquote{státního} spolku je stálým pokušením pro každého, kdo cítí zášť nebo chce někomu ublížit či chce získat nezasloužené bohatství prostřednictvím \enquote{soudního sporu.} Kdyby to tak nebylo, méně lidí by protahovalo nebo stupňovalo neshody či spory. Ať už z dobročinnosti, ze zbabělosti nebo jen z touhy vyhnout se bolestem hlavy, které s sebou vleklý konflikt přináší, mnoho lidí -- i těch, kteří si na někoho oprávněně stěžují -- prostě nechá minulost minulostí a žije dál.

I bez takových příkladů je naprosto iracionální tvrdit, že by lidé nemohli \enquote{vycházet} bez \enquote{státu,} když všechno, co \enquote{stát} dělá, tedy používání násilí a hrozby násilí k ovládání lidí, je přesným \emph{opakem} \enquote{vycházení.} Představa, že mírové soužití vyžaduje agresi a nátlak, je logicky směšná. Jediné, co přidání \enquote{autority} do situace zaručuje, je to, že \emph{nebude} dosaženo nenásilného, mírového řešení věci. Když někdo popisuje společnost, kterou by chtěl vidět, téměř vždy bude popisovat stav nenásilí, vzájemné spolupráce a tolerance. Jinými slovy, to, co bude popisovat, je naprostým protikladem násilí a donucování \enquote{autority.} Lidé, kteří byli vychováni v představě, že \enquote{autorita} je důležitou a pozitivní součástí společnosti, se však stále snaží dosáhnout míru válkou, spolupráce donucováním, tolerance netolerancí a lidskosti brutalitou. Takové šílenství je přímým důsledkem toho, že jsou lidé učeni respektovat a poslouchat \enquote{autoritu.}

\section{Antiautoritářská výchova}

Výchova je tak často založena na autoritářství, že si mnozí ani nedokážou představit, jak by neautoritářská výchova měla vypadat. Je důležité si ujasnit, jaký vliv by na výchovu měla ztráta pověry o autoritě. Neznamenalo by to, že by rodiče nekladli žádná omezení na to, co mohou jejich děti dělat, ani by to nevylučovalo, že rodiče budou v mnoha situacích omezovat děti proti jejich vůli. Dramaticky by se však změnilo myšlení rodičů i dětí.

V dnešní době většina lidí považuje učit, co je správné a co špatné, a učit poslouchat za totéž. Rodič však může dítěti přikázat, aby udělalo něco špatného, stejně snadno jako mu může přikázat, aby udělalo něco správného. Na rozdíl od toho, co učí autoritářská výchova, skutečnost, že rodič vydal příkaz, neznamená, že je automaticky správný, a nezavazuje dítě k poslušnosti. Pokud například rodič svému dítěti přikáže, aby kradlo v obchodě, nemá dítě žádnou morální povinnost to udělat a neuposlechnutí by bylo zcela oprávněné (i když pravděpodobně riskantní). Dítě samozřejmě nemusí chápat, že krádež je špatná, pokud mu rodiče řekli, aby kradlo.

Na druhé straně může rodič uložit dítěti nezbytné a odůvodněné omezení, které se dítěti nelíbí a nepovažuje ho za oprávněné. V obou případech je dítě povinno dělat pouze to, co považuje za správné. Alternativou by bylo, že má morální povinnost dělat to, co považuje za nesprávné, což je nemožné. V tom spočívá rozdíl: autoritářský rodič učí dítě, že poslušnost sama o sobě je morálním imperativem bez ohledu na příkaz (např. \enquote{\emph{Protože jsem tvůj otec a řekl jsem to!}}). Neautoritářský rodič může dítěti také ukládat omezení, ale nevyžaduje, aby se to dítěti \emph{líbilo}, ani nepředstírá, že taková omezení jsou spravedlivá jen proto, že je rodič uložil. Jinými slovy, neautoritářský rodič může vidět potřebu vnutit dítěti určitá omezení (týkající se doby spánku, stravy atd.), protože dítě ještě nemá dostatečné znalosti nebo porozumění, aby bylo dostatečně kompetentní činit všechna svá vlastní rozhodnutí, ale netvrdí, že dítě má nějakou morální povinnost se bez otázek podřídit. Čím dříve se dítěti podaří vysvětlit \emph{důvod} nějakého \enquote{pravidla,} tím dříve může pochopit, proč je pro něj prospěšné dělat to, co mu rodič řekne. To samozřejmě není vždy možné, zejména pokud jsou děti velmi malé. Rodič, který dítěti brání sníst balíček bonbónů, prospívá dítěti, které ještě nemá dostatek pochopení ani sebeřízení, aby mohlo sloužit svým vlastním zájmům. Ale učit dítě, že by mělo cítit morální povinnost dodržovat pravidla, která považuje za nespravedlivá, nesmyslná, hloupá nebo dokonce ubližující, jen proto, že mu to řekla \enquote{autorita,} znamená dát dítěti tu nejnebezpečnější lekci, jaká může být: že je morálně povinno snášet nespravedlivé, nesmyslné, hloupé a ubližující věci, pokud je dělá \enquote{autorita.}

Aby se rodiče vyhnuli předávání pověr o autoritě, neměli by nikdy uvádět jako důvod, proč by dítě mělo něco udělat, větu \enquote{protože jsem to řekl.} Rodič by měl vyjádřit, že pro omezení existují racionální důvody, i když dítě tyto důvody ještě nedokáže pochopit. Jinými slovy, zdůvodnění \enquote{pravidel} nespočívá v tom, že rodiče mají právo násilím vnucovat dětem jakákoli pravidla, ale v tom, že rodiče mají (doufejme) o tolik více porozumění a znalostí než děti, že rodiče musí mnoho rozhodnutí dítěte učinit za něj, dokud se nestane kompetentním rozhodovat se samo.

Ještě důležitější je, jak rodič omezuje chování svého dítěte vůči ostatním. Je nesmírně důležité naučit dítě, že úmyslně ublížit jinému člověku je ze své podstaty špatné (s výjimkou případů, kdy je to nutné k obraně nevinného). Pokud však rodič místo této zásady učí \enquote{poslouchej mě} a pak dítěti přikáže, aby nebilo druhé, naučil dítě poslušnosti, ale \emph{ne} morálce. Pokud se dítě zdrží bití druhých ne proto, že by chápalo, že je to špatné, ale jen proto, že mu bylo řečeno, aby to nedělalo, pak funguje stejně jako amorální robot a nenaučilo se nic o tom, jak být člověkem. Krátkodobý praktický výsledek může vypadat stejně -- tj. dítě se zdrží bití druhých -- ale naučené lekce jsou velmi odlišné. Když dítě, které bylo pouze naučeno poslouchat, vyroste a nějaká jiná \enquote{autorita} mu řekne, že by \emph{mělo} ubližovat ostatním, téměř jistě to udělá, protože bylo naučeno dělat, co se mu řekne. Na druhou stranu dítě, které bylo naučeno respektovat práva druhých a bylo naučeno zásadám sebevlastnictví a neagrese, se těchto zásad jen tak lehce nevzdá jen proto, že mu to někdo, kdo se prohlašuje za \enquote{autoritu,} nařídí.

Děti se učí příkladem. Pokud dítě vidí své rodiče, jak se vždy chovají jako bezvýhradní poddaní vládnoucí třídy, naučí se být otrokem. Pokud naopak rodiče ve svém každodenním životě ukazují, jak používat a následovat vlastní srdce a rozum, dítě se naučí jednat stejně. Dítě musí pochopit, že jeho povinností není pouze dodržovat pravidla dobrého chování, ale že na to, jaká jsou pravidla dobrého chování,musí samo přijít. Normy, podle nichž \enquote{sebevlastník} žije, lze stále označit za \enquote{pravidla,} ale hodnota takových \enquote{pravidel} neplyne z toho, že je vydala nějaká \enquote{autorita,} ale z toho, že jedinec věří, že taková \enquote{pravidla} popisují ve své podstatě morální chování. To neznamená, že se všichni shodují na tom, co je morální, i když na některých základních principech panuje široká shoda. Ale i když se chování každého člověka řídí jeho vlastním nedokonalým, neúplným chápáním dobra a zla, celkové výsledky by se drasticky zlepšily ve srovnání s autoritářskou alternativou, kdy v podstatě dobří lidé dělají věci, o kterých \emph{vědí}, že jsou špatné, protože se cítí nuceni dělat to, co jim \enquote{autorita} nařídí (jak ukázaly Milgramovy experimenty).

Ačkoli se mnozí lidé mylně domnívají, že společnost bez centralizované \enquote{autority,} která by určovala pravidla, by znamenala \enquote{každý sám za sebe,} skupinová spolupráce a dohody nevyžadují \enquote{autoritu} a děti, které stráví svá formativní léta tím, že se naučí komunikovat s různými lidmi všech věkových kategorií na základě vzájemné dobrovolnosti, místo aby se učily slepě dělat to, co se jim řekne, jsou mnohem lépe vybaveny k vytváření vztahů a ke společnému úsilí založenému na dohodě, kompromisu a spolupráci. Taková dobrovolná interakce může probíhat mezi dvěma lidmi nebo mezi dvěma miliony. Dokonce i omezená svoboda, kterou zažili Američané, ukázala, že i extrémně složitá průmyslová odvětví mohou být založena výhradně na dobrovolné účasti a dobrovolné spolupráci všech zúčastněných. A historie také ukázala, že v okamžiku, kdy se použije metoda organizace založená na centralizovaném, násilném ovládání, jako je tomu v takzvaném \enquote{plánovaném hospodářství,} produktivita se zhroutí a objeví se chudoba a zotročení. Přesto je většina dětí stále vychovávána v autoritářském prostředí s tím, že je to nejlépe připraví na život v reálném světě. Ve skutečnosti je to připravuje pouze na celoživotní zotročení.

\section{Na půli cesty}

V jakékoli skupině lidí, která se vzdala mýtu o autoritě -- ať už jde o malou skupinu přátel, obyvatele města nebo obyvatele celého kontinentu -- bude četnost a závažnost násilných konfliktů a agresivních činů \emph{vnitř} této skupiny dramaticky nižší než jinde, kde většina lidí prostřednictvím \enquote{hlasování} a dalších \enquote{politických} akcí agresi pravidelně obhajuje a páchá. Nicméně i když by se jednotlivci v takové skupině nemuseli bát jeden druhého, stále by se pravděpodobně museli potýkat s akty agrese ze strany těch, kteří stojí mimo skupinu a kteří se stále drží víry ve stát. Jedinec, jehož mysl byla osvobozena, ale který stále žije ve společnosti sužované bludem autority, bude neustále vystaven riziku, že se stane terčem autoritářské agrese. Být svobodný ve své mysli -- chápat koncept sebevlastnictví -- ještě nutně neznamená být svobodný fyzicky. Může však znamenat obrovský pozitivní rozdíl, protože otevírá nespočet nových prostředků, jejichž prostřednictvím se lidé mohou pokusit vyrovnat se s autoritářskými pokusy o ovládání, vyhnout se jim nebo se jim dokonce bránit.

Jedinec, který se pyšní tím, že je \enquote{občanem dbalým zákona,} má pouze jediný způsob, jak se \emph{pokusit} dosáhnout svobody, který téměř nikdy není účinný: prosit své pány, aby změnili své \enquote{zákony.} Naproti tomu člověk, který chápe, že vlastní sám sebe, nedluží žádnému domnělému pánovi žádnou oddanost a nepotřebuje žádné \enquote{zákonodárné} povolení, aby byl svobodný, má mnohem více možností. A čím více je lidí, kteří se vymanili z pověry, tím snazší je vyhýbání se nebo odpor. Například i malý počet \enquote{sebevlastníků} může vytvořit obchodní kanály, které obcházejí obvyklé regulace a vyděračská schémata zavedená \enquote{státy.}

Je ironií, že tato zcela legitimní a morální forma dobrovolné interakce je často označována jako \enquote{černý trh} nebo jako \enquote{podpultové} podnikání \enquote{na černo,} zatímco obvyklý systém agrese, nátlaku a vydírání je věřícími ve stát považován za legitimní a spravedlivý. Ve skutečnosti legitimita jakéhokoli obchodu (nebo jakékoli jiné lidské interakce) nezávisí na tom, zda o něm ví a ovládá ho nějaká \enquote{autorita,} jak naznačuje pojem \enquote{černý trh,} ale závisí pouze na tom, zda to, k čemu dochází, je vzájemně konsensuální. Ti, kdo toto chápou, mohou najít mnoho způsobů, jak obejít nebo zmařit pokusy \enquote{státu} o násilnou nadvládu a vykořisťování.

Mnoha agresivním činům páchaným ve jménu \enquote{zákona} se může poměrně snadno vyhnout nebo je porazit relativně malý počet lidí, pokud necítí automatickou morální povinnost dělat, co se jim řekne. Samozřejmě tomu tak není vždy. Pokud gang zvaný \enquote{stát} něco dělá dobře, pak je to uplatňování hrubé síly, ať už v podobě vojenských akcí nebo vnitrostátního \enquote{vymáhání práva.} Téměř ve všech případech je však většina moci, kterou \enquote{stát} disponuje, výsledkem nikoliv zbraní, tanků a bomb, ale \emph{vnímání} jejich obětí. Pokud 99 \% obyvatelstva poslouchá vládnoucí třídu z pocitu povinnosti nebo závazku, že tak musí činit, zbývající 1 \% lze obvykle ovládat hrubou silou (se souhlasem 99 \%). Pokud však podstatnější procento obyvatelstva žádnou povinnost poslušnosti necítí, množství hrubé síly potřebné k jejich ovládnutí se stává obrovským. Například mnoho obyvatel Spojených států nyní odevzdává na různých úrovních \enquote{daní} asi polovinu toho, co vydělají, a většina z nich se cítí povinna tak činit; pokud by však cizí mocnost nějakým způsobem napadla a dobyla zemi, bylo by uvalení 50\% \enquote{daně} naprosto nemožné, protože lidé by necítili žádnou morální, právní ani vlasteneckou povinnost se podřídit. Dvě stě milionů pracujících by našlo dvě stě milionů způsobů, jak se vyhnout takovým pokusům cizích zlodějů o jejich zotročení pomocí úniků, podvodů, utajování nebo jak je porazit i pomocí přímého násilí.

Dnes existuje pouze jedna banda, která je schopna utlačovat americký lid: \enquote{vláda} USA. Je to proto, že je to jediný gang, o kterém si většina lidí představuje, že má \emph{právo} nutit a ovládat (\enquote{regulovat}) a okrádat a vydírat (\enquote{zdaňovat}) americký lid. Častou obavou etatistů je, že bez silného \enquote{státu,} který by je chránil, by prostě přišla nějaká cizí mocnost a převzala by vládu. Takové obavy však zcela přehlížejí, jak velkou roli ve schopnosti utlačovat hraje \emph{vnímání}. Území o velikosti USA, obývané \emph{stovkou milionů} držitelů zbraní (kromě dalších sta milionů lidí, kteří by se pravděpodobně \emph{stali} držiteli zbraní, kdyby došlo k invazi), by bylo nemožné obsadit a ovládnout pouze hrubou silou. Historie poskytuje mnoho příkladů (např. varšavské ghetto za druhé světové války, válka ve Vietnamu, následky války v Iráku), jak může být i obrovská, technologicky vyspělá stálá armáda donekonečna mařena relativně malým počtem ozbrojených \enquote{povstalců.} A země obývaná \enquote{sebevlastníky} má další obrovskou výhodu v tom, že je pro ně doslova \emph{nemožné} se kolektivně vzdát. Pokud neexistuje žádný \enquote{stát,} který by předstíral, že zastupuje obyvatelstvo, a nikdo, kdo by tvrdil, že mluví jménem lidu jako celku, neexistuje doslova žádný způsob, jak by se mohli \enquote{vzdát,} aniž by se vzdal každý jednotlivec.

Dobrým způsobem, jak pochopit realitu situace, je uvažovat o věci z pohledu vůdce útočníků. Jak by se vůbec mohl někdo pokusit o invazi a trvalé obsazení oblasti, v níž žije mnoho milionů obyvatel, kteří se mohou skrývat kdekoli a mohou zabít cokoli v okruhu nejméně sta metrů, jako každý schopný lovec? Ctižádostivý tyran by měl mnohem větší šanci získat moc nad lidmi tím, že by se ucházel o úřad, čímž by v myslích svých obětí získal domnělé právo jim vládnout a ovládat je.

Útlak ve velkém měřítku, zejména od nástupu střelných zbraní, závisí mnohem více na ovládání mysli než na ovládání těla. Ti, kdo touží po nadvládě, získávají mnohem větší moc tím, že své oběti přesvědčí, že neuposlechnout jejich příkazů je \emph{špatné}, než tím, že je přesvědčí, že neuposlechnout je pouze \emph{nebezpečné} (ale morální).

Bez ohledu na to, jak moc si lidé stěžují a protestují, dokud lidé poslouchají \enquote{zákon} (příkazy politiků), tyrani se nemají čeho bát. Dokud budou jejich pokusy o ovládání a vydírání považovány za \enquote{legální} akty \enquote{autority} a dokud se proto lidé budou cítit povinni se podřídit, pokud a dokud vládnoucí třída takové \enquote{zákony} nezmění, lidé zůstanou zotročeni tělem, protože zůstanou zotročeni duševně. Je ironií, že mnoho lidí stále věří, že silný \enquote{stát} je to jediné, co může ochránit lid jako celek, zatímco víra ve stát je ve skutečnosti to jediné, co může \emph{zotročit} lid jako celek. Samotná hrubá síla to ve velkém měřítku ani po delší dobu nedokáže. Dokonce ani banda s tanky, letadly, bombami a dalšími zbraněmi nemá sílu ovládat ozbrojený lid po dlouhou dobu, pokud nejprve lidi neoblafne, aby uvěřili, že má \emph{právo} je ovládat. Jinými slovy, dlouhodobý útlak a zotročování může projít pouze gangu, který si představuje, že je \enquote{autoritou.} Výsledkem je, že \enquote{stát} (nebo víra v něj), místo aby byl nezbytný pro ochranu práv jednotlivce, je nezbytný pouze pro dlouhodobé a rozsáhlé \emph{porušování} práv jednotlivce. Je ironií, že i většina těch, kteří dnes uznávají \enquote{stát} jako největší hrozbu pro svobodu, stále trvá na tom, že \enquote{stát} určitého typu je pro ochranu nezbytný. Víra v autoritu je tak silná, že dokáže přesvědčit jinak racionálně uvažující lidi, že právě to, co je běžně okrádá, nutí a napadá, je potřeba k \emph{ochraně} před okrádáním, nátlakem a napadáním. Skutečnost, že \enquote{stát} byl \emph{vždy} agresorem a \emph{nikdy} nebyl čistě ochráncem, a to kdekoli na světě a kdykoli v dějinách, jimi neotřese z jejich sektářské víry v magickou moc a ctnosti abstraktní, mýtické entity zvané \enquote{autorita.}

\section{Cesta ke spravedlnosti}

Mnoho rozsáhlých nespravedlností v dějinách by se rychle zhroutilo -- nebo by nikdy nezačalo -- nebýt \enquote{autority,} která tyto nespravedlnosti schvaluje a prosazuje. Například zlo otroctví se často svádí na rasismus a chamtivost, ale \enquote{autorita} hrála obrovskou roli v tom, že otroctví bylo ekonomicky proveditelné. Kdyby neexistovala obrovská organizovaná síť \enquote{strážců zákona,} kteří chytali uprchlé otroky a všechny, kdo jim pomáhali utéct, jak dlouho by otroctví pokračovalo? Kdyby osvobozování otroků nebylo \enquote{nezákonné,} a tudíž v očích autoritářů nemorální, o kolik větší a efektivnější by byla \enquote{podzemní železnice?} Ve skutečnosti by pravděpodobně nebyla známá jako \enquote{podzemní,} kdyby nebyla \enquote{nezákonná.}

\enquote{Abolicionistické} hnutí se skládalo z lidí, kteří si mysleli, že otroctví je nemorální, a chtěli, aby se změnily \enquote{zákony} a otroctví bylo oficiálně \emph{prohlášeno} za nemorální a \enquote{nezákonné.} Kdyby všichni abolicionisté místo petic za změnu \enquote{zákonů} aktivně osvobozovali otroky, obchod s otroky by se pravděpodobně zhroutil o desítky let dříve, pokud by k němu vůbec někdy došlo. Přeprava otroků přes půl světa by byla skutečně velmi riskantní záležitostí, pokud by v okamžiku přistání mohl být váš \enquote{náklad} násilně osvobozen. Problém je v tom, že většina lidí věří, že i nemorální a nespravedlivé \enquote{zákony} by se měly dodržovat, dokud se \enquote{zákon} nezmění, což ukazuje, že loajalita takových lidí k mýtu autority je silnější než jejich loajalita k morálce a dělat to, co jim pánové řeknou, je pro ně důležitější než dělat to, co vědí, že je správné. A lidstvo kvůli tomu velmi trpí.

Schopnost lidí vzdorovat tyranii závisí do značné míry na tom, zda přijmou mýtus autority, nebo ne. Ti, kteří vidí nespravedlnost páchanou \enquote{státem,} ale nadále věří, že musí \enquote{dodržovat zákon} a \enquote{fungovat v rámci systému,} nikdy nedosáhnou spravedlnosti. Na druhou stranu ti, kteří nepovažují politické megalomany za právoplatné vládce, ti, kteří necítí povinnost podřizovat se nemorálnímu \enquote{zákonu,} ti, kteří necítí potřebu chovat se k tomu, co je ve skutečnosti třídou parazitů -- bandou politických zlodějů a násilníků -- jako k nedotknutelnému, váženému a čestnému, mají mnohem větší šanci porazit \enquote{legální} tyranii. (A většina tyranie a útlaku, k nimž v dějinách docházelo, byla prováděna \enquote{legálně.})

Těm, kteří jsou ochotni \enquote{nelegálně} vzdorovat nespravedlnosti a tyranii, je k dispozici mnoho metod, od pasivního odporu přes nenásilné sabotáže až po atentáty a další formy násilného odporu. V závislosti na závažnosti útlaku a na vlastních hodnotách, svědomí a přesvědčení o tom, kdy (pokud vůbec) je použití násilí vhodné, lze zvolit libovolný počet způsobů, jak porazit tyranii. Někteří se prostě pokusí zůstat \enquote{pod pokličkou} a vyhnout se pozornosti státních orgánů. Někteří mohou zvolit občanskou neposlušnost, například velkou skupinu otevřeně kouřící marihuanu před policejní stanicí. Někteří mohou zvolit aktivnější, ale nenásilnou metodu, například prořezávání pneumatik policejních aut nebo ničení jiného majetku používaného k páchání státní agrese. Jiní mohou zvolit metodu otevřeně násilného odporu, k jakému došlo například během americké revoluce.

Analogicky se zamýšlená oběť loupeže (té \enquote{nestátní}) může pokusit zloději uniknout, přelstít ho, nebo dokonce zabít, pokud k tomu dojde -- cokoli, co je třeba, aby se vyhnula oběti. Stejně tak by ti, kdo si uvědomují, že \enquote{legální} zlo je stále zlem a odpor proti němu je stále oprávněný, neztráceli čas volbami a lobbováním u politiků za změnu legislativy; prostě by udělali vše, co by mohli, aby ochránili sebe a případně i ostatní před tím, aby se stali obětí takové \enquote{legální} agrese. Od určitého bodu platí, že čím více lidí se brání, tím \emph{méně} násilí je k tomu zapotřebí. Pokud má místní policie tucet \enquote{protidrogových policistů} -- lidí, jejichž hlavní náplní práce je páchat agresi na jiných, kteří nepoužili ani násilí, ani nepodvádí -- a několik set civilistů, nechť je známo, že se domnívají, že mají právo použít cokoli, co je třeba, včetně smrtící síly, aby zastavili pokusy o únosy, vloupání do domů nebo podobné agresivní činy páchané \enquote{protidrogovými policisty,} agresoři (policie), pokud by neměli žádnou větší autoritářskou partu, na kterou by se mohli obrátit o pomoc, by se jednoduše vzdali, aby nebyli vyhlazeni. Odstrašující účinek, který funguje proti soukromým zločincům, může stejně dobře fungovat i proti \enquote{státním} zločincům.

V Indii se Mahátma Gándhí a jeho stoupenci snažili pomocí rozsáhlé pasivní neposlušnosti podkopat britskou nadvládu nad touto zemí. Dalším příkladem nemorálního \enquote{zákona,} který byl v podstatě \emph{neuposlechnut}, je prohibice alkoholu ve Spojených státech. Vysoká míra neposlušnosti spolu s odmítnutím většiny porotců dát požehnání \enquote{legální} agresi, spolu s některými akty násilného odporu (např. posypávání \enquote{vymahatelů prohibice} peřím) učinily nemorální \enquote{zákon} nevymahatelným. Zákonodárci jej nakonec zrušili ve snaze zachovat si tvář, protože mít v zákonech nevymahatelný zákon znamená zničit legitimitu vládnoucí třídy v očích jejích obětí. Všude tam, kde lidé necítí morální povinnost vyhovět autoritářským požadavkům, lze jakékoli \enquote{legální} akty agrese ignorovat bez povšimnutí. Když je však počet vlastníků menší, je někdy k poražení \enquote{legálních} aktů agrese nutné násilí. (A pokud nelegitimitu \enquote{legálního} útlaku rozpozná jen několik lidí, násilný odpor se často vymstí.)

Kde je útlak, tam je vždy násilí. Obvykle je jednostranné, většinu nebo veškeré násilí páchají představitelé \enquote{autority.} Člověk, který pasivně spolupracuje a přitom tvrdí, že je proti násilí, ve skutečnosti \emph{odměňuje} násilí státu. Jakmile je spáchán akt agrese -- ať už ze strany \enquote{autority} nebo kohokoli jiného -- nenásilí z definice přestává být možností. Jedinou otázkou je, zda agresivní násilí zůstane bez odezvy, nebo zda proti němu bude použita obranná síla. V každém případě k násilí dojde.

Samozřejmě, že zloději, násilníci a vrazi, kteří své zločiny prohlásí za \enquote{legální} -- což udělal každý tyran v dějinách -- vždy označí každého, kdo se jim postaví na odpor, za zločince a teroristy. Pouze ti, kteří necítí stud za to, že jsou označováni za \enquote{zločince,} protože se zbavili mýtu autority a uvědomují si, že pojem \enquote{zákon} je často používán ve snaze charakterizovat něco zlého jako něco dobrého, mají vůbec nějakou šanci dosáhnout svobody. Opět je poněkud ironické, že čím více bude lidí, kteří chápou sebevlastnictví a mýtickou povahu \enquote{autority} a kteří jsou ochotni bojovat za to, co je správné, a bojovat proti tomu, co je \enquote{legální,} ale špatné, tím \emph{méně} násilná bude cesta ke skutečné civilizaci (mírovému soužití).

\section{Vedlejší účinky mýtu}

Když se ohlédneme zpět do historie, není nouze o příklady nelidského přístupu člověka k člověku, příklady útlaku a utrpení, násilí a nenávisti, situace a události, které nevrhají dobré světlo na lidskou rasu obecně. A přestože mnohé z nejkřiklavějších nespravedlností v dějinách byly zřejmým produktem víry ve stát, například války a zjevný útlak, mnoho dalších nespravedlností, které se obvykle nepřičítají působení \enquote{státu,} by také nebylo možné bez účasti \enquote{autority.}

Kromě příkladu, zda by otroctví mohlo existovat, kdyby nebylo \enquote{legálně} vynucováno (jak bylo zmíněno výše), lze podobné otázky klást i ohledně zacházení s americkými indiány. Kdyby nebylo autoritativních \enquote{státních} nařízení a státních žoldnéřů, kteří je prosazovali, došlo by k tak rozsáhlému a koordinovanému úsilí o vyhlazení nebo násilné vystěhování domorodců z území, které po generace obývali? Nepochybně by stále docházelo k menším konfliktům kvůli střetu kultur a požadavkům na zemědělskou a loveckou půdu, ale bylo by v něčím osobním zájmu pouštět se do rozsáhlých násilných bojů?

Po ukončení otevřeného otroctví ve Spojených státech (přibližně ve stejné době, kdy se poprvé objevilo \enquote{legální} otroctví, tzv. daň z příjmu) pokračovalo rasové napětí a násilné konflikty. Mnozí se domnívají, že pak přišel \enquote{stát} a zachránil situaci. Ve skutečnosti násilné konflikty mezi rasami \emph{podporovala} \enquote{autorita.} Po mnoho let byla rasová segregace násilně vnucována prostřednictvím \enquote{zákonů.} Ironií je, že rasové napětí později ještě více prohloubila \enquote{státem} nařízená integrace, která se snažila přinutit lidi různých ras a kultur, aby se mísili, ať už chtěli, nebo ne. Výsledkem bylo opět násilí. Během celého fiaska by některé školy a podniky, pokud by jim byla ponechána svoboda, zvolily segregaci a některé integraci. Kdyby nebylo \enquote{státu,} který se snažil všem násilím vnutit jednu \enquote{oficiální politiku,} rodiče by si jednoduše vybrali, do kterých škol budou posílat své děti (ať už do segregovaných, nebo ne), a zákazníci by si jednoduše vybrali, které podniky budou navštěvovat (ať už segregované, nebo ne). Nejenže většinu násilí páchaného na černoších prováděli přímo \enquote{státní} vymahatelé (\enquote{policie}), ale dokonce i velká část soukromě páchaného násilí byla výsledkem hněvu nad tím, že lidé byli \enquote{státem} nuceni jednat s lidmi jiné rasy a kultury. Je hloupé si myslet, že \emph{vynucování} lidí od sebe nebo \emph{vynucování} lidí k sobě učiní lidi šťastnějšími, milejšími nebo otevřenějšími a tolerantnějšími. Ani v jednom případě autoritářské zásahy neprospěly míru nebo bezpečnosti žádné z ras. I když nelze přesně říci, jak rozšířená nebo dlouhodobá by byla segregace a rasismus bez \enquote{státního} zásahu, je zdravým rozumem dáno, že pokud je lidem všech ras a náboženství umožněno svobodně si vybrat, s kým se budou stýkat, je přinejmenším \emph{možné}, aby velmi odlišné kultury pokojně koexistovaly. Ale když se do toho zapojí \enquote{stát} a debata se vede mezi tím, zda nutit rasy, aby zůstaly oddělené, nebo nutit rasy, aby se mísily, je zřejmé, že někdo bude naštvaný tak jako tak.

To neznamená, že každý názor je stejně platný. Jde o to, že lidé s velmi odlišnými názory na svět -- ať už jsou jejich názory jakkoli moudré nebo hloupé, otevřené nebo zaujaté, informované nebo nevědomé -- mohou obvykle koexistovat v míru, dokonce i v těsné blízkosti, \emph{pokud} se do toho nezaplete \enquote{stát.} Různí lidé se nemusejí mít rádi, nemusejí si navzájem schvalovat své názory a životní styl a ve skutečnosti mohou jiné kultury tvrdě kritizovat nebo odsuzovat. To však neznamená, že nemohou pokojně koexistovat, přičemž obě strany se zdrží násilné agrese. Ale kdykoli se do toho vloží \enquote{stát,} nátlak, který je vlastní každému \enquote{zákonu,} zaručuje, že lidé spolu \emph{nebudou} jen tak \enquote{vycházet.}

Dalším příkladem nepřímých, škodlivých účinků \enquote{státního} působení je skutečnost, že násilí spojené s \enquote{obchodem s drogami} (výrobou a distribucí \enquote{nelegálních} látek) existuje jen díky \enquote{protidrogovým zákonům.} Tím, že politici \enquote{postaví mimo zákon} určitou látku nebo chování, i když jsou všichni účastníci dobrovolně dospělí, vytvářejí černý trh, který má nejen obrovský potenciál zisku díky omezení nabídky, ale vytváří situaci, která cíleně zbavuje zákazníky a dodavatele jakékoli \enquote{zákonné} ochrany. Pokud je například drogový dealer okraden nebo napaden, ať už policií nebo kýmkoli jiným, je nepravděpodobné, že by si na pomoc zavolal \enquote{strážce zákona.} \enquote{Postavení} něčeho dobrovolného mimo zákon -- ať už jde o prostituci, hazardní hry nebo užívání drog -- téměř zaručuje, že trh bude ovládat ten gang, který je nejnásilnější nebo který nejvíc podplatil policisty a další úředníky. Dokonalým příkladem \enquote{před a po} byla opět prohibice alkoholu ve Spojených státech. Když se alkohol stal \enquote{ilegálním,} okamžitě jej ovládl organizovaný zločin, který proslul nejen svým násilím, ale také schopností podplácet \enquote{státní} agenty a úředníky. Když se alkohol stal opět \enquote{legálním,} veškeré násilí s ním spojené téměř okamžitě ustalo.

Navzdory tomuto naprosto jasnému příkladu strašlivých výsledků přijímání \enquote{zákonů} zakazujících \enquote{neřesti} většina lidí stále podporuje \enquote{zákony} proti chování a zvykům, které se jim příčí. Výsledkem je, že související násilí pokračuje. Místo aby bylo uznáno jako problém, který existuje \emph{kvůli} \enquote{státu} a jeho \enquote{zákonům,} je stále představováno jako problém, proti kterému musí \enquote{stát} bojovat. Totéž lze říci o nechvalně známém násilí lichvářů, kteří se zabývají \enquote{nelegálním} hazardem, a o násilí \enquote{pasáků} v místech, kde je prostituce \enquote{nelegální.} V takových případech je ještě lepší než srovnání \enquote{před a po} srovnání vedle sebe: vede hazard k většímu násilí v Atlantic City, kde je \enquote{legální,} nebo v místech, kde je \enquote{nelegální?} Představuje prostituce větší hrozbu pro všechny zúčastněné v Amsterdamu, kde je \enquote{legální,} nebo ve všech místech, kde je \enquote{nelegální?} Tím nechci říci, že prostituce, hazardní hry a drogy (včetně alkoholu) jsou dobré věci, ale že ať už je to dobře nebo špatně, zavedení \enquote{státního} donucení do situace tyto \enquote{neřesti} nezlikviduje, ale pouze je učiní \emph{nebezpečnějšími} pro všechny zúčastněné a často i pro lidi, kteří se jich \emph{neúčastní}.

Aby si nikdo stále nepředstavoval, že takové \enquote{mravnostní zákony} jsou výsledkem dobrých úmyslů, politici dobře vědí, že hazardní hry, prostituce a \enquote{nelegální} užívání drog se stále vyskytují ve \enquote{státních} \emph{věznicích}. Politici dobře vědí, že pokud ani neustálé zajetí, dohled, namátkové prohlídky a tvrdé tresty nedokážou zabránit takovému chování u lidí, kteří jsou drženi v přísně sledovaných klecích, \enquote{zákony} zřejmě nedokážou vymýtit takové chování z celé země. Mohou však poskytnout a poskytují tyranům pohotovou záminku pro stále se rozšiřující moc, a to je přesně důvod, proč \enquote{státy} přijímají \enquote{mravnostní} zákony: aby \emph{vytvořily} \enquote{zločin} tam, kde žádný nebyl, ve snaze ospravedlnit existenci autoritářské moci a nadvlády.

Ve světě bez mýtu autority by mnoho lidí (včetně autora knihy) stále silně nesouhlasilo s užíváním drog, prostitucí a dalšími \enquote{neřestmi,} ale pravděpodobně by nepodporovali snahy o násilné potlačení takového chování. Nejenže by se obvykle necítili oprávněni obhajovat násilí, kdyby neměli záminku \enquote{autority,} za kterou by se mohli schovat, ale pravděpodobně by ani nechtěli poskytnout miliardy dolarů potřebné k vedení rozsáhlé násilné kampaně proti takto rozšířeným aktivitám. I ten nejodsouzenější člověk by měl jak ekonomické, tak morální pobídky nechat ostatní na pokoji, stejně jako strach z odvety od kohokoli, proti komu by se rozhodl páchat agresivní činy. Otevřená kritika životního stylu a chování a pokusy přesvědčit lidi, aby změnili své způsoby, jsou samozřejmě zcela přijatelnou součástí lidské společnosti. Ve skutečnosti, kdyby se lidé museli snažit získat lidi na svou stranu pomocí rozumu a slovního přesvědčování, místo aby používali hrubou sílu \enquote{státu,} možná by cíle byly otevřenější naslouchat. Lidé by přinejmenším přestali z problému špatných návyků dělat problém krveprolití a brutality, jak se to děje nyní při všech pokusech o \enquote{uzákonění} morálky.

Odvrácenou stranou představy, že \enquote{když je to nelegální, musí to být špatné,} je \enquote{když je to legální, musí to být v pořádku.} Snad největším příkladem toho je skutečnost, že v roce 1913 \enquote{vláda} USA nejen \enquote{legalizovala} otroctví prostřednictvím \enquote{daně z příjmu,} čímž přímo a násilně zabavila plody lidské práce, ale také prostřednictvím Federálního rezervního zákona legalizovala takovou míru padělání a bankovních podvodů, že se nad tím až tají dech. Stručně řečeno, politici dali bankéřům \enquote{legální} povolení vymýšlet peníze ze vzduchu a půjčovat tyto falešné, vymyšlené \enquote{peníze} na úrok ostatním, dokonce i \enquote{státům.} Ačkoli si většina lidí neuvědomuje podrobnosti toho, jak k takovým obrovským podvodům a loupežím prostřednictvím \enquote{fiatních měn} a \enquote{bankovnictví částečných rezerv} dochází, mnoho lidí nyní instinktivně tuší, že \enquote{banky} dělají něco podvodného a zkorumpovaného. Neuvědomují si však, že to byl \enquote{stát,} který dal bankám \emph{povolení} podvádět a šidit veřejnost doslova o biliony dolarů.

Dalším obzvláště kontroverzním příkladem toho, jak může debata o \enquote{legálnosti} přebít debatu o faktech a morálce, je otázka potratů. Jedna strana lobbuje za \enquote{autoritu,} která by potraty \enquote{legalizovala,} a pak tuto praxi obhajuje na základě její \enquote{legálnosti.} Druhá strana prosazuje, aby byly potraty \enquote{zakázány,} a doufá, že bude použito násilí \enquote{autority,} aby se této praxi zabránilo. Z logického hlediska je jediná relevantní otázka, která je otázkou náboženskou/biologickou/filozofickou, nikoli \enquote{právní,} následující: V jakém okamžiku se plod považuje za osobu? Odpověď na tuto otázku určuje, zda se potrat rovná vraždě, nebo je ekvivalentem odebrání ledviny. Obě strany se však místo řešení jediné skutečně důležité otázky -- jakkoli složité a kontroverzní -- obvykle soustředí na snahu získat na svou stranu násilí \enquote{autority.}

Jako další příklad \enquote{legalizovaného} nespravedlnosti lze uvést, že téměř každý si je vědom toho, jak nehoráznými a iracionálními se staly \enquote{soudní procesy} (např. zločinci, kteří vnikli na cizí pozemek, úspěšně žalují majitele nemovitosti poté, co se při vloupání zranili), ale neuvědomují si, že jsou to právě nařízení \enquote{státem} jmenovaných \enquote{soudců,} která to vůbec umožňují. Kromě toho, že \enquote{stát} může \enquote{legálně} okrást jednu osobu, aby mohl obdarovat jinou, vytváří \enquote{stát} prostřednictvím současného systému soudních sporů také mechanismus, díky němuž může jedna osoba přímo a \enquote{legálně} okrást druhou.

\enquote{Zákony} ve jménu environmentalismu jsou také využívány k nemorálnímu uchvácení moci v obou směrech. S dostatkem peněz může společnost, která skutečně znečišťuje, a tím porušuje vlastnická práva ostatních, vyměnit \enquote{příspěvky na kampaň} za \enquote{legální} \emph{povolení} ke znečišťování. Současně mohou využívat \enquote{zákony} o životním prostředí k potlačení konkurence tím, že vytvoří a prosadí labyrint \enquote{předpisů} o životním prostředí -- mnohé z nich jsou zbytečné nebo kontraproduktivní, někdy až idiotské -- aby udržely menší podniky mimo trh. Kromě toho mohou politici využívat vágní hrozby environmentálního nebezpečí jako záminku k získání nadvlády nad soukromým průmyslem, k ovládání chování milionů lidí nebo k vylákání dalších peněz pro své vlastní účely.

V mnoha průmyslových odvětvích dnes úspěch nezávisí ani tak na poskytování hodnotných služeb za rozumnou cenu, jako spíše na získání zvláštních výhod a přednostního zacházení ze strany \enquote{státu.} To může mít podobu přímých dávek (např. grantů nebo dotací), politického obchodování (např. \enquote{veřejné} zakázky bez výběrového řízení), licenčních systémů (např. ve zdravotnictví), cel na mezinárodní obchod, regulování a zvýhodňování a mnoha dalších prostředků. Výsledky všech těchto opatření -- vyšší ceny, horší výrobky a služby, menší výběr atd. -- se často považují za důsledek nedostatků soukromého průmyslu, místo aby se uznalo, čím jsou: nepříznivými důsledky autoritářské nadvlády nad lidskou interakcí.

Dalším příkladem vedlejšího účinku autoritářství je skutečnost, že velké hospodářské krachy jsou \emph{vždy} důsledkem \enquote{státních} zásahů do obchodu, úvěrů a měn. Jediným způsobem, jak zničit celou ekonomiku, je zasahovat do prostředku směny, \enquote{peněz,} a to prostřednictvím \enquote{legalizovaného} padělání, vydáváním smyšlených úvěrů a emisí fiatní měny. Většina lidí, kteří neznají ani základy ekonomie, považuje inflaci a další ekonomické problémy za přirozené, nešťastné, ale nevyhnutelné jevy. Ve skutečnosti se jedná o příznaky rozsáhlého, \enquote{legalizovaného} podvodu a krádeže.

Imigrační \enquote{zákony} jsou dalším příkladem nepřímých škod a sekundárních problémů způsobených \enquote{státem.} Kromě zjevného přímého nátlaku způsobují tyto \enquote{zákony} další problémy, které by jinak neexistovaly: 1) probíhají lukrativní a často kruté kšefty s pašováním \enquote{ilegálů} do země; 2) \enquote{ilegálové} jsou snadným terčem obchodu s lidmi a dalších forem vykořisťování, protože se neodváží ozvat nebo vyhledat pomoc; 3) \enquote{ilegálové} mají problémy najít užitečné a výdělečné zaměstnání, a proto se uchylují ke krádežím, protože nemohou být \enquote{legálně} zaměstnáni; a 4) lidé jsou nuceni žít v tyranských režimech, protože nemohou fyzicky uniknout. Obecněji řečeno, protože \enquote{ilegálové} jsou klasifikováni jako \enquote{zločinci} a často jsou považováni za \enquote{nevítané} už jen proto, že jsou v určité zemi, a nedostává se jim od většiny občanů ani respektu, ani ochrany, mají menší motivaci snažit se přizpůsobit nebo se jinak chovat \enquote{podle zákona.}

Dokonce i mnohé problémy, které se zdají být nestátní povahy, existují kvůli nějakému \enquote{zákonu.} Samozřejmě existují a vždy budou existovat případy podvodů a krádeží, kterých se dopouštějí bezohlední jednotlivci jednající na vlastní pěst, ale většina lidí si vůbec neuvědomuje, kolik zdánlivě soukromých švindlů, schémat a kšeftů je \enquote{autoritou} nejen povoleno, ale i podporováno a \emph{odměňováno} přes \enquote{státní zákony,} ať už záměrně, nebo omylem. Protože nemají žádný skutečně svobodný trh, s nímž by mohli srovnávat, mnozí se nadále domnívají, že státní donucení je nezbytné, zatímco ve skutečnosti jen brzdí a brání lidské produktivitě a pokroku.

\section{Jaká by mohla být společnost}

Nelze si ani představit, v kolika ohledech by se dějiny změnily, kdyby se pověra o autoritě dávno zhroutila. Je zřejmé, že ke zvěrstvům nacistického Německa, Stalinova Ruska, Maovy Číny, Pol Potovy Kambodže a mnoha dalších by nikdy nedošlo. Navíc, i když by stále mohlo docházet k násilným regionálním kulturním nebo náboženským střetům, k rozsáhlým válkám by prostě nemohlo dojít a nedošlo by k nim, aniž by vojáci slepě poslouchali domnělou \enquote{autoritu.} Kdyby se obrovské množství prostředků, úsilí a vynalézavosti, které byly vynaloženy na masové ničení (válku), vložilo do něčeho produktivního, kde bychom dnes byli? Kdyby lidé namísto toho, aby věnovali tak obrovské množství času a úsilí boji o to, kdo má mít otěže moci a k čemu má být tato moc použita, strávili všechna ta léta vynalézavostí a produktivitou, jak by mohl svět vypadat dnes? Co kdyby každý člověk mohl podporovat to, co chce, místo toho, aby \enquote{stát} všechny okrádal a pak vedl nekonečné spory o to, jak by se s těmito \enquote{veřejnými prostředky} mělo nakládat? Co kdyby místo hádek o to, který centralizovaný autoritářský plán by měl být každému násilně vnucen, lidé žili své vlastní životy a plnili si své vlastní sny? Kdo si vůbec dokáže představit, kam až by lidstvo jako celek mohlo pokročit?

Tím nechci říci, že bez víry v autoritu by nikdy nevznikly osobní konflikty. Docházelo by k nim a někdy by končily násilím. Rozdíl je v tom, že s vírou ve stát \emph{vždy} končí násilím (nebo hrozbou násilí), protože nátlak je to jediné, co \enquote{stát} kdy dělá. Zatímco lidé, a to i lidé velmi odlišných názorů a prostředí, obvykle dokáží najít způsoby mírového soužití, jakákoli situace, do níž se \enquote{autorita} zapojí, je automaticky \enquote{řešena} silou.

Co kdyby se v otázce \enquote{stejnopohlavního manželství} místo neustálého sporu o to, jaké názory a volby by měly být vnucovány všem, mohl každý církevní hodnostář, každý zaměstnavatel a každý jiný jednotlivec sám rozhodnout, jak chce žít, co chce nazývat \enquote{manželstvím} a tak dále? Co kdyby v případě otázky \enquote{modliteb ve škole} místo toho, aby \enquote{stát} vytvářel nepřátelský konflikt násilným zabavováním peněz všem vlastníkům nemovitostí na financování jednoho velkého, homogenního \enquote{veřejného} školského systému, si každý člověk (křesťan, žid, muslim, ateista atd.) mohl vybrat, které školy chce podporovat, pokud vůbec nějaké chce? To neznamená, že by se lidé různých názorů měli rádi nebo že by nakonec věřili ve stejné věci. Znamená to však, že \emph{bez} toho, že by věřili ve stejné věci, by přesto mohli pokojně koexistovat -- což je situace, kterou \enquote{stát} neumožňuje. Co kdyby místo toho, aby \enquote{státní} agentury rozhodovaly o tom, jaké léky a léčebné postupy \enquote{legálně} dovolí lidem vyzkoušet a kteří lékaři budou mít \enquote{licenci} k jejich praktikování, mohli lidé rozhodovat sami? (V takovém případě by se dařilo podnikání, které by poskytovalo zákazníkům objektivní informace o různých produktech a službách.)

\enquote{Státní} řešení jsou vždy o tom, že politici rozhodnou, jak řešit různé situace, a pak své představy násilím vnucují všem ostatním. Není však ani morálně legitimní, ani prakticky účinné, aby politici rozhodovali za všechny ostatní. A to platí pro nejrůznější aspekty lidské společnosti. Jak by svět vypadal, kdyby lidé posledních sto let místo dohadování o tom, jak lidem násilně \emph{omezit} možnosti (což dělá každý \enquote{zákon}), věnovali svůj čas a úsilí zkoušení nových myšlenek a přicházeli s novými přístupy k problémům, přičemž by každý člověk mohl věnovat svůj vlastní čas, úsilí a peníze tomu, co se osobně rozhodl podporovat?

Co kdyby místo centralizovaného systému nuceného přerozdělování bohatství (\enquote{státní sociální péče}) bylo lidem ponecháno právo, aby se sami rozhodli, jak nejlépe a s největším soucitem pomoci potřebným? Místo systému, který odměňuje lenost a nepoctivost a pěstuje závislost, bychom mohli mít systém, který lidem skutečně pomáhá. Co kdyby místo toho, aby \enquote{stát} nutil podniky dělat to, co politici a byrokraté prohlásí za \enquote{bezpečné,} mohli lidé přicházet s novými nápady a vynálezy, sami si určovat priority a rozhodovat o tom, jak se nejlépe chránit? Co kdyby si lidé místo centralizované vládnoucí mašinérie, která se snaží lidi donutit k \enquote{férovosti,} mohli sami vybrat, s kým se budou stýkat, jaké obchody budou uzavírat a podobně?

Vše, co \enquote{stát} financuje, vytváří konflikt. Každý \enquote{veřejný} projekt -- od \enquote{grantů,} které rozdává \enquote{Národní nadace pro umění,} přes granty na určitá studia nebo podniky, školy, parky a všechno ostatní \enquote{veřejné} -- se rovná okradení tisíců nebo milionů lidí, aby se peníze daly několika málo lidem. Proč by někdo očekával, že se všichni v celé zemi -- nebo dokonce stovka lidí -- přesně shodnou na tom, jak by měly být jejich peníze utraceny? Co kdyby se místo mnoha \emph{trilionů} dolarů, které jsou každoročně odváděny a uloupeny na financování agendy politiků a jejich byrokracie, toto bohatství použilo na věci, na kterých lidem, kteří tyto peníze vydělali, skutečně záleží a které chtějí podporovat?

Co kdyby namísto ekonomiky, kterou neustále táhnou dolů daně, regulace a inflace způsobené manipulací s fiatní měnou, existovala skutečně svobodná směna, kdy by každý člověk investoval plody své práce do věcí, kterých si váží a které považuje za hodnotné? Je nemožné si vůbec představit, jaké úrovně technologie a prosperity by bylo možné dosáhnout, stejně jako si lidé před sto lety nedokázali představit všechno to bohatství, pohodlí a vymoženosti, které máme dnes. Stejně jako dnes mají \enquote{chudí} ve Spojených státech mnoho pohodlí a luxusu, které ještě před několika desetiletími neměla ani královská rodina, mohla by skutečně svobodná společnost rychle vést k takové úrovni všeobecného pohodlí a bezpečnosti, jakou si dnes málokdo dokáže představit. Práce tři hodiny denně místo osmi by se mohla stát normou. S růstem celkového bohatství by si lidé mohli vydělávat velmi pohodlně, a to i při velmi podřadných pracích. S potenciálem takové hojnosti by již nebylo problémem zajistit životní potřeby pro nemocné nebo starší lidi -- pro ty, kteří mají omezené produktivní schopnosti. S rostoucím bohatstvím celé společnosti si lidé mohou dovolit věnovat větší pozornost otázkám životního prostředí. (Naproti tomu, když se lidé snaží sehnat dostatek jídla, aby se uživili, těžko se budou starat o dlouhodobý blahobyt místní flóry a fauny.)

Množství času, úsilí a vynalézavosti, které si vládnoucí třídy na celém světě přivlastnily, vyvolává údiv. Různé \enquote{státy} ukradly biliony a biliony dolarů a utratily je za dobývání, podmaňování, zabíjení a ničení -- a vytvořily tak nejen nespravedlnost a utrpení, ale i gigantickou čistou ztrátu bohatství pro lidstvo. Dokonce i státní programy, které mají údajně lidem pomáhat, jsou notoricky známé svou neefektivitou, plýtváním, podvody a náchylností ke korupci.

Přibližně \emph{třetinu} veškerého bohatství, které se ve Spojených státech každoročně vytvoří, si bere federální vládnoucí třída (a ještě více si ho bere na státní, okresní a místní úrovni). Část z něj se sice vrátí veřejnosti, ale velkou část prostě spotřebuje byrokracie a \enquote{státní} mašinérie a nevytvoří přitom žádnou hodnotu. Ve skutečnosti, kdyby se třetina všeho, co se v USA vyrobí, okamžitě vyhodila na skládku, místo aby se odevzdala federálním parazitům, lidé by byli \emph{bohatší} než nyní. Je tomu tak proto, že stát nejenže plýtvá a spotřebovává produktivitu a bohatství, ale ve skutečnosti používá bohatství, které ukradne, k tomu, aby platil lidem za to, že \emph{nebudou} produktivní (např. sociální zabezpečení, AFDC, potravinové lístky a další \enquote{sociální dávky}), aby platil lidem za to, že budou vyrábět věci, které nikdo nechce (např. byrokracie a uměle vytvořené pracovní pozice), aby platil lidem za to, že ničí bohatství a majetek (např. armáda) a aby platil lidem za to, že násilně zasahují do schopnosti \emph{jiných} lidí být produktivní (prostřednictvím daní, regulací, licencí, povolení, územního plánování, minimální mzdy, cel a obchodních omezení, vynucených monopolů, věznění nenásilných, jinak produktivních lidí atd.) Když sečteme množství produktivity, kterou stát přímo ukradl, \emph{a} množství produktivity, které stát násilně brání, můžeme si udělat představu o úrovni prosperity, které by se celý svět těšil, kdyby nebylo obří ekonomické mrtvé váhy zvané \enquote{stát,} a o úrovni prosperity, které se bude těšit, až se pověra o autoritě zhroutí.

To, čeho by lidstvo mohlo dosáhnout, kdyby mu nebránil mýtus státu, ohromuje představivost. Zahrnovalo by to drastický skok v materiálním pohodlí a bohatství miliard lidí a znamenalo by to konec chudoby a hladu na celém světě. (Nebýt daní, regulací a dalších \enquote{legálních} překážek, už dnes máme prostředky, abychom snadno nasytili všechny obyvatele planety.) Znamenalo by to také konec zadlužování téměř pro každého. Bez \enquote{legalizovaných} bankovních podvodů (např. systém Federálních rezerv) a neustálého \enquote{zdanění} příjmů, majetku, obchodu a dědictví by lidé hromadili bohatství, místo aby sotva přešlapovali na místě a obohacovali politiky a bankéře. A hojnost materiálního bohatství by umožnila, aby lidé mohli věnovat většinu nebo veškerý svůj čas a úsilí věcem, které je baví, místo aby museli dlouhé hodiny pracovat na věcech, které je nebaví, jen aby měli co jíst a kde bydlet. Život průměrných lidí by mohl být mnohem příjemnější a naplněnější, zatímco nyní je často únavný a stresující.

Je ironií, že téměř utopické vize, které mnozí tyrani slibovali (ale nikdy nesplnili), mohou být a nakonec budou naplněny přesným opakem: skutečně svobodnou společností bez jakýchkoli vládců nebo centralizovaných kontrolorů. Nebýt pověrčivosti vůči autoritám, už bychom tam dávno byli. Co kdyby se posledních několik tisíc let každý člověk staral sám o sebe a nesnažil se pomocí \enquote{státu} vnucovat své představy a priority ostatním? Co kdyby místo toho, aby obří centralizované monstrum násilně omezovalo volby a možnosti každého, kreativitu a vynalézavost každého, snažilo se vnutit konformitu a stejnost a zároveň vysávalo nápady a bohatství výrobců, zkoušeli různí lidé a různé skupiny nové nápady a přicházeli na nejlepší způsoby řešení problémů a vytváření lepšího světa, vedeni vlastním přesvědčením a hodnotami?

Bohužel tato myšlenka stále děsí mnoho lidí, kteří si stále představují, že svět násilně ovládaný politiky by byl bezpečnější a civilizovanější než svět obývaný svobodnými lidskými bytostmi uplatňujícími svobodnou vůli a individuální úsudek. Faktem je, že lidé, kteří věří, že \enquote{stát} zařídí, aby věci fungovaly, ačkoli jich je zdaleka většina a ačkoli to možná myslí dobře, jsou problémem. V důsledku své indoktrinace kultem \enquote{autority} nadále věří a prosazují hluboce šílenou myšlenku, že jedinou cestou k míru, spravedlnosti a harmonické civilizaci je neustálý, široce rozšířený nátlak a násilná \enquote{státní} nadvláda, neustálý útlak a zotročování prováděné ve jménu \enquote{práva} a obětování svobodné vůle a morálky na oltář podmanění a slepé poslušnosti. Jakkoli to může znít krutě, je to základ \emph{veškeré} víry ve stát.

\section{Přijetí reality}

Etatisté často říkají: \enquote{Ukažte mi příklad, kdy anarchie (společnost bez státu) fungovala.} Protože ovšem mluví o společnostech složených téměř výhradně z důkladně indoktrinovaných autoritářů, o lidské společnosti bez vládnoucí třídy se málokdy vůbec uvažuje, natož aby se o to někdo pokoušel. Přesto etatisté používají skutečnost, že nikdy nezkusili skutečnou svobodu -- protože tento koncept je jejich způsobu myšlení zcela cizí -- jako důkaz, že společnost bez státu \enquote{by nefungovala.}

Bylo by to podobné, jako kdyby skupina středověkých lékařů, kteří všichni používají pijavice na všechny neduhy, argumentovala: \enquote{Ukažte mi jediný případ, kdy lékař vyléčil bolest hlavy bez použití pijavic.} Samozřejmě, kdyby nikdo z nich nikdy neuvažoval o jiné léčbě než pijavicemi, neexistoval by příklad, že alternativní metody \enquote{fungují.} To by však svědčilo o neznalosti lékařů, nikoli o neúčinnosti léčby, kterou nikdy nevyzkoušeli.

Důležitější však je, že \enquote{anarchie} -- absence \enquote{státu} -- \emph{je to, co je}. Pokud je údajná \enquote{autorita,} na níž celý koncept \enquote{státu} spočívá, pouhou iluzí (jak bylo prokázáno výše), pak tvrzení, že společnost nemůže existovat bez \enquote{státu,} je přesně tak rozumné, jako tvrzení, že Vánoce nemohou nastat bez Ježíška. Společnost \emph{už} existuje bez \enquote{státu,} a to od samého počátku. Byli to lidé, kteří si \emph{představovali} entitu s právem vládnout -- \emph{halucinovali} o existenci \enquote{autority} -- díky čemuž se příběh lidstva skládá převážně z útlaku, násilí, utrpení, vražd a chaosu.

Je ironií, že etatisté často poukazují na smrt a utrpení, k nimž dochází, když se dvě nebo více skupin hádají o to, kdo by měl \enquote{vládnout,} označují to za \enquote{anarchii} a uvádějí to jako důkaz, že bez \enquote{státu} by nastal chaos a smrt. Takové krveprolití a útlak jsou však přímým a zřejmým důsledkem víry v autoritu, nikoli důsledkem \emph{neexistence} \enquote{státu.} Je pravda, že ve srovnání s životem ve stabilním, zakořeněném autoritářském režimu může být život v zemi, kde se lidé přou o to, kdo má být novou \enquote{autoritou} (prostřednictvím povstání, občanských válek, dobývání jednoho národa druhým atd.), mnohem nebezpečnější a nepředvídatelnější. V důsledku toho si lidé žijící ve válkou zmítaných oblastech často přejí pouze to, aby konflikt skončil, aby jedna strana zvítězila a stala se novým \enquote{státem.} Pro tyto lidi může stabilní \enquote{stát} představovat relativní mír a bezpečí, ale základní příčinou útlaku, kterého se stabilní režimy dopouštějí \emph{a} krveprolití, k němuž dochází během bojů o moc, je víra v autoritu. Kdyby nikdo nevěřil v legitimní vládnoucí třídu, nikdo by nebojoval o to, kdo má vládnout. Kdyby neexistoval trůn, nikdo by o něj nebojoval. Všechny občanské války a téměř všechny revoluce spočívají na předpokladu, že by někdo měl vládnout. Bez pověry o autoritě by nebyl důvod, aby k takovým věcem vůbec docházelo.

\enquote{Stát} ze své podstaty nepřináší společnosti nic pozitivního. Nevytváří žádné bohatství ani ctnosti. Přidává pouze nemorální násilí a iluzi, že takové násilí je legitimní. Umožnění některým lidem násilně ovládat všechny ostatní -- což je vše, co \enquote{stát} kdy dělá -- nepřináší společnosti ani špetku talentu, schopností, produktivity, vynalézavosti, důvtipu, kreativity, znalostí, soucitu nebo jakékoli jiné pozitivní vlastnosti, kterou lidské bytosti mají. Místo toho všechny tyto věci neustále dusí a omezuje svými násilnými \enquote{zákony.} Je destruktivní a šílené přijmout představu, že civilizace vyžaduje násilné omezování možností a násilné omezování lidské mysli a ducha -- že občanská společnost může existovat pouze tehdy, když je síla a ctnost každého jednotlivce násilně překonána a potlačena tlupou pánů a vykořisťovatelů -- že průměrnému člověku nelze věřit, že si bude vládnout sám, ale že politikům lze důvěřovat, že budou vládnout všem ostatním -- že jedinou cestou, jak může morálka a ctnost lidstva vyniknout, je potlačit svobodnou vůli a sebeurčení miliard lidských bytostí a všechny je proměnit v nemyslící, poslušné loutky vládnoucí třídy a zdroj moci pro tyrany a megalomany -- že cestou k civilizaci je \emph{zničení} svobodné vůle, soudnosti a sebeurčení jednotlivce.

To je základ, srdce a duše pověry zvané \enquote{autorita.} Až budou lidé připraveni rozpoznat tuto ohavnou lež takovou, jaká je, a začnou přijímat osobní odpovědnost za své činy i za stav společnosti -- a ani o chvíli dříve -- pak může začít skutečná lidskost. Lidé si mohou zoufale přát \enquote{mír na zemi,} dokud nezčernají, ale nikdy se ho nedočkají, pokud a dokud nebudou ochotni zaplatit cenu tím, že se vzdají jedné ohrané, staré pověry. Řešením většiny neduhů společnosti je, abys ty, milý čtenáři, rozpoznal mýtus autority jako to, čím je, vzdal se ho v sobě a pak začal usilovat o deprogramování a probuzení všech lidí, které znáš a kteří v důsledku své indoktrinace kultem uctívání \enquote{autority} a navzdory svým ctnostem a ušlechtilým úmyslům nadále podporují a podílejí se na násilné, protilidské, destruktivní a zlé mašinérii útlaku a agrese známé jako \enquote{stát.}

\section{Pointa znovu a naposled}

V rozporu s tím, čemu se téměř všichni naučili věřit, není \enquote{stát} pro civilizaci nezbytný. Není pro civilizaci prospěšný. Ve skutečnosti je protikladem civilizace. Není to spolupráce, týmová práce ani dobrovolná interakce. Není to mírové soužití. Je to nátlak, je to síla, je to násilí. Je to zvířecí agrese, která je maskována pseudonáboženskými rituály podobnými kultu, jež mají vyvolat dojem legitimity a spravedlnosti. Je to hrubé násilnictví, které se tváří jako souhlas a organizace. Je to zotročení lidstva, podmanění svobodné vůle a zničení morálky, které se maskuje jako \enquote{civilizace} a \enquote{společnost.} Problém není jen v tom, že \enquote{autorita} \emph{může} být použita ke zlu; problém je v tom, že ve své nejzákladnější podstatě \emph{je} zlem. Ve všem, co dělá, poráží svobodnou vůli lidských bytostí a ovládá je prostřednictvím nátlaku a strachu. Vytlačuje a ničí morální svědomí a nahrazuje jej bezmyšlenkovitou slepou poslušností. Nemůže být použita k dobru, stejně jako bomba nemůže být použita k uzdravení těla. Je to vždy agrese, vždy nepřítel míru, vždy nepřítel spravedlnosti. V okamžiku, kdy přestane být útočníkem, přestane odpovídat definici \enquote{státu.} Ze své podstaty je to vrah a zloděj, nepřítel lidstva, jed pro lidstvo. Jako pán a kontrolor, vládce a utlačovatel nemůže být ničím jiným.

Údajné právo vládnout, ať už v jakékoli míře a formě, je \emph{protikladem} lidskosti. Iniciace násilí je \emph{protikladem} harmonického soužití. Touha po nadvládě je \emph{protikladem} lásky k lidstvu. Skrývání násilí pod nánosy složitých rituálů a vnitřně rozporných racionalizací a označování hrubého násilnictví za ctnost a soucit na této skutečnosti nic nemění. Tvrzení o ušlechtilých cílech, že násilí je \enquote{vůlí lidu} nebo že je pácháno \enquote{pro obecné dobro} či \enquote{pro naše děti,} nemůže změnit zlo v dobro. \enquote{Legalizace} zla z něj neudělá dobro. Násilné podmanění jednoho člověka druhým, ať už je popsáno jakkoli nebo provedeno jakkoli, je necivilizované a nemorální. Zkázu, kterou způsobuje, nespravedlnost, kterou vytváří, škodu, kterou působí každé duši, jíž se dotkne -- pachatelům, obětem i pozorovatelům -- nelze odčinit tím, že ji nazveme \enquote{zákonem} nebo že tvrdíme, že byla nezbytná. Zlo, ať už se jmenuje jakkoli, je stále zlem.

Konečné poselství je velmi jednoduché. Křičí ho celá zaznamenaná historie, ale jen málokdo si ho až dosud dovolil slyšet. To poselství je následující:

\textbf{Pokud milujete smrt a ničení, útlak a utrpení, nespravedlnost a násilí, útlak a mučení, bezmoc a zoufalství, věčné konflikty a krveprolití, pak naučte své děti respektovat \enquote{autoritu} a naučte je, že poslušnost je ctnost.}

\textbf{Pokud si naopak ceníte mírového soužití, soucitu a spolupráce, svobody a spravedlnosti, pak naučte své děti principu sebevlastnictví, naučte je respektovat práva každé lidské bytosti a naučte je rozpoznat a odmítnout víru v autoritu takovou, jaká je: nejneracionálnější, nejrozporuplnější, nejprotilidštější, nejzlovolnější, nejničivější a nejnebezpečnější pověru, jakou kdy svět poznal.}

\end{document}